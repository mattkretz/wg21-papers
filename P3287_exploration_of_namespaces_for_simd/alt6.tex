\subsection{Alternative 6: Place everything but obvious overloads into a
single namespace}

The preceding alternative probably went too far with moving \code{<cmath>}
overloads and algorithms like \code{min}, \code{clamp}, etc. into the
\std\code{simd} namespace.
So let's keep all functions that are a clear overload (\code{f(simd<T>)}) from
an existing function (\code{f(T)}) directly in the \code{std} namespace.
This is the “namespace equivalent” to the status-quo approach of whether a
\code{simd_} prefix is needed or not.

\medskip\begin{lstlisting}[style=Vc]
namespace std::simd {

template<class T, class Abi = /*...*/>
  class basic_simd;

template<class T, @\simdsizetype@ N = /*...*/>
  using simd = basic_simd<T, @\deducet@<T, N>>;

template<class V, class G>
  V
  generate(G&& gen);

template<class V = void, class It, class... Flags>
  conditional_t<is_same_v<V, void>, simd<iter_value_t<It>>, V>
  load_from(It first, simd_flags<Flags...> f = {});

template<class Rg, std::integral Idx, class AbiIdx, class... Flags>
  simd<ranges::range_value_t<Rg>, basic_simd<Idx, AbiIdx>::size()>
  gather_from(const Rg&& in, const basic_simd<Idx, AbiIdx>& indexes,
                   simd_flags<Flags...> f = {});

template<size_t SizeSelector = 0, class T, class Abi, class PermuteGenerator>
  simd<T, output-size>
  permute(const basic_simd<T, Abi>& v, PermuteGenerator&& fn);

template<size_t Bytes, class Abi, class T, class U>
  auto
  select(const basic_simd_mask<Bytes, Abi>& c, const T& a, const U& b)
    -> decltype(simd-select-impl(c, a, b));

template<size_t Bs, class Abi>
  bool
  all_of(const basic_simd_mask<Bs, Abi>&);

template<class T>
  concept integral = /*...*/;

template<class T>
  concept generic_integral = std::integral<T> or std::simd::integral<T>;

}

namespace std {

template<class T, class Abi>
  simd::basic_simd<T, Abi>
  exp(const simd::basic_simd<T, Abi>& x);

template<class T, class Abi>
  simd::basic_simd<T, Abi>
  min(const simd::basic_simd<T, Abi>& x, const simd::basic_simd<T, Abi>& y);

}
\end{lstlisting}

Usage example:
\medskip\begin{lstlisting}[style=Vc]
void f(std::simd::simd<float> vf, const std::vector<int>& data) {
  auto iota = std::simd::generate<std::simd::simd<int>>([](int i) { return i; });
  auto chunk = std::simd::load_from(data.begin());
  auto chunk_swapped = std::simd::gather_from(data, iota ^ 1);
  auto chunk_swapped2 = std::simd::permute(chunk, [](int i) { return i ^ 1; });
  assert(std::simd::all_of(chunk_swapped == chunk_swapped2));

  vf = std::simd::select(vf > 1.f, 1.f, vf);
  vf = std::exp(vf);
  auto lo = std::min(iota, chunk);
}
\end{lstlisting}

When relying on ADL, nothing changes compared to the example in the preceding
section.
However, if we now create a namespace alias and call everything fully
qualified, the necessary qualifications could be considered slightly
incoherent:

\medskip\begin{lstlisting}[style=Vc]
namespace smd = std::simd;

void f(smd::simd<float> vf, const std::vector<int>& data) {
  auto iota = smd::generate<smd::simd<int>>([](int i) { return i; });
  auto chunk = smd::load_from(data.begin());
  auto chunk_swapped = smd::gather_from(data, iota ^ 1);
  auto chunk_swapped2 = smd::permute(chunk, [](int i) { return i ^ 1; });
  assert(smd::all_of(chunk_swapped == chunk_swapped2));

  vf = smd::select(vf > 1.f, 1.f, vf);
  vf = std::exp(vf);
  auto lo = std::min(iota, chunk);
}
\end{lstlisting}

At this point all functions already work for \simdgeneric code (or can be made
to work with suitable overloads in the \std\code{simd} namespace).
If LEWG were to adopt this naming style, then we need to decide on a per
function basis, whether the function is “SIMD-only” or whether an overload for
the value-type is useful on its own.
For the latter, the function goes into \code{std} otherwise it needs to go
into \std\code{simd}.

The \simdgeneric programming example from previous sections now looks like
this:
\medskip\begin{lstlisting}[style=Vc]
template<std::integral T>
T scalar_only(T a, T b) {
  return 2 * std::min(a, b);
}

template<std::simd::integral T>
T simd_only(T a, T b) {
  return 2 * std::min(a, b);
}

template<std::simd::generic_integral T>
T generic(T a, T b) {
  return 2 * std::min(a, b);
}
\end{lstlisting}

\begin{description}
  \item[pros]
    \begin{itemize}
      \item We are free to grab names out of the new namespace.
      \item ADL works.
      \item Fairly consistent.
      \item[$\Rightarrow$] Users need to learn: “If it's in the \std\code{simd}
        namespace then it works for \code{simd}s.
        When searching for a function for \code{simd}, if the same function
        exists / could exist for scalars look for it in \code{std}, otherwise
        look in the \std\code{simd} namespace.”
      \item \simdgeneric programming is straightforward to provide and use.
    \end{itemize}

  \item[cons]
    \begin{itemize}
      \item The class template name \std\code{simd::simd} is a bit awkward.

      \item We have a mix of non-member functions in \code{std} and
        \std\code{simd}.
    \end{itemize}
\end{description}

\myrating{acceptable; but not much different from the status quo --- not convinced this is actually \emph{better}}
