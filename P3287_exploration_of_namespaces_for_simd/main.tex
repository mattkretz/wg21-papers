\newcommand\wgTitle{Exploration of namespaces for std::simd}
\newcommand\wgName{Matthias Kretz <m.kretz@gsi.de>}
\newcommand\wgDocumentNumber{P3287R1}
\newcommand\wgGroup{LEWG}
\newcommand\wgTarget{\CC{}26}
\newcommand\wgAcknowledgements{Daniel Towner and Ruslan Arutyunyan contributed to this paper via discussions / reviews.}

\usepackage{mymacros}
\usepackage{wg21}
\setcounter{tocdepth}{2} % show sections and subsections in TOC
\hypersetup{bookmarksdepth=5}
\usepackage{changelog}
\usepackage{underscore}
\usepackage{multirow}

\addbibresource{extra.bib}

\newcommand\simd[1][]{\type{ba\-sic\_simd#1}\xspace}
\newcommand\simdT{\type{ba\-sic\_simd\MayBreak<\MayBreak{}T>}\xspace}
\newcommand\valuetype{\type{val\-ue\_type}\xspace}
\newcommand\referencetype{\type{ref\-er\-ence}\xspace}
\newcommand\mask[1][]{\type{ba\-sic\_simd\_mask#1}\xspace}
\newcommand\maskT{\type{ba\-sic\_simd\_mask\MayBreak<\MayBreak{}T>}\xspace}
\newcommand\wglink[1]{\href{https://wg21.link/#1}{#1}}
\newcommand\simdgeneric{SIMD-gener\-ic\xspace}

\newcommand\nativeabi{\UNSP{native-abi}}
\newcommand\deducet{\UNSP{deduce-t}}
\newcommand\simdsizev{\UNSP{simd-size-v}}
\newcommand\simdsizetype{\UNSP{simd-size-type}}
\newcommand\simdselect{\UNSP{simd-select-impl}}
\newcommand\maskelementsize{\UNSP{mask-element-size}}
\newcommand\integerfrom{\UNSP{integer-from}}
\newcommand\constexprwrapperlike{\UNSP{constexpr-wrapper-like}}
\newcommand\vectorizable{\UNSP{vectorizable}}

\renewcommand{\lst}[1]{Listing~\ref{#1}}
\renewcommand{\sect}[1]{Section~\ref{#1}}
\renewcommand{\ttref}[1]{Tony~Table~\ref{#1}}
\renewcommand{\tabref}[1]{Table~\ref{#1}}

\newcommand\myrating[1]{\par\noindent{\color{Headings}\scshape{}my rating:} #1}

\begin{document}
\selectlanguage{american}
\begin{wgTitlepage}
  In recent discussions about \code{simd} in LEWG, notably on 2023-06-16 while
  discussing \code{permute}, \code{expand}, and \code{compress}, there was a
  request for a paper exploring placing all \code{simd} non-member functions
  into a sub-namespace.
  \ldots or potentially any other means of using namespaces to improve the
  \code{simd} API.

  This paper explores a few ideas.
\end{wgTitlepage}

\pagestyle{scrheadings}

\section{Changelog}
(placeholder)
\begin{revision}
\item Add a simple example to the motivation section.
\item Expand the “Generalization” section to clearly define the feature rather
  than just sketching it.
  Also add a discussion of initial value and step.
\item Discuss why reusing the existing \code{iota} algorithm/view does not
  work/suffice for the \code{simd} use case.
\item Discuss why \code{iota_v} is the right name.
%  \todo
\end{revision}

\section{Straw Polls}
\subsection{SG1 at Kona 2022}
\wgPoll{After significant experience with the TS, we recommend that the next
version (the TS version with improvements) of \code{std::simd} target the IS (\CC{}26)}
{10&8&0&0&0}

\wgUnanimous{We like all of the recommended changes to \code{std::simd} proposed in p1928r1
(Includes making all of \code{std::simd} \code{constexpr}, and dropping an ABI stable type)}

\wgPoll{Future papers and future revisions of existing papers that target
\code{std::simd} should go directly to LEWG.
(We do not believe there are SG1 issues with \code{std::simd} today.)}
{9&8&0&0&0}


\section{Introduction \& Motivation}

Using the example of \std\code{permute(\simd, idx_perm)}, one of the unavoidable
LEWG discussions/decisions is about whether \code{simd} can grab the name
“permute”, potentially blocking its use for other facilities in the standard
library.\footnote{Just to clarify, I agree with the concern and I feel uneasy
with the need for \code{simd} to grab as many names as it would need to.}
With P3067R0 (“Provide named permutation functions for \std\code{simd}”), the
list of non-member functions to add to \std\ becomes:
\code{permute}, \code{expand}, \code{compress}, \code{grow}, \code{stride},
\code{chunk}, \code{reverse}, \code{re\-peat_all}, \code{re\-peat_each},
\code{transpose}, \code{zip}, \code{unzip}, \code{cat}, \code{ex\-tract},
\code{rotate}, \code{shift_left}, \code{shift_right}, and \code{align}.
All of these names would likely need a \code{simd} prefix if they want to go
into \std.

And then we're adding \simd overloads for all of \code{<cmath>} and
\code{<bit>}, \ldots.

So we need to understand whether there are viable alternatives to \code{simd}
naming.
This paper tries to explore the field as far as I believe is still sensible.
The goal is to come up with a consistent naming strategy for everything related
to \code{simd}.

\subsection{Motivation for a \code{std::simd} namespace}
\label{sec:namespacemotivation}

A \std\code{simd} namespace encapsulates everything related to data-parallel
types, creating an easy-to-explore, isolated space for these functionalities.
It avoids the need for (inconsistent) \code{simd_} prefixes by grouping related
functions into their own namespace.
It increases flexibility for future extensions to be organized within the
\std\code{simd} namespace.
Users can easily alias the namespace (e.g. \code{namespace simd = std::simd}),
reducing verbosity in the code but maintaining a clean and logical structure.

The \std\code{simd} namespace approach might be cleaner in the long term.
It avoids overloads like \code{std::reduce} and \code{std::all_of}, reducing
the likelihood of confusion.
While a \code{simd::simd} duplication in naming would be a bit awkward, it is a
one-time issue that users can learn to live with.
And we should consider renaming the alias template to
\std\code{simd::\MayBreak{}vec} (\std\code{simd::\MayBreak{}basic_vec} for the
class template).

This approach also aligns with other \CC{} standard library features,
where sub-namespaces are used to logically group related functionality:\\
\std\code{chrono},
\std\code{execution},
\std\code{filesystem},
\std\code{linalg},
\std\code{numbers},
\std\code{pmr},\\
\std\code{ranges}, and
\std\code{ranges::views}/\std\code{views}.

\subsection{\simdgeneric programming}

In this paper I want to use the term \emph{\simdgeneric} programming.
Note that in the space of types, \simdT is a generalization of \code{T} or ---
vice-versa --- \code{T} is the degenerate case of \simdT.
(The same is true for \mask and \code{bool}.)
We've touched upon this when we talked about regularity and how \simdT is
designed to retain regularity of each individual element inside the \simd,
leading to something I called “data-parallel regularity” of \simdT, for lack of
an existing term.

The \code{simd} design aims to allow users to replace \code{T} with \simdT in
their code without requiring any further code changes.
If this works (and because of branching on individual values of \code{T} it
cannot work for all code) I call such code \simdgeneric.

The following text uses this term because the use of namespaces opens an
interesting facility to opt in and out of some aspects of \simdgeneric
programming.

\section{Exploration}

When exploring naming and namespacing, I use the following functions to showcase the effect.
I then try to come up with all ways to use and abuse the facilities.
In addition I mention the effect of the choice on \simdgeneric programming.
To complete the picture, I added a concept that seems like something we might
want to add in the future, but for which there is currently no proposal coming
forward.

Note: we have to discuss the range vs. iterator argument to load/gather seperately.
This paper does not explore the issue.
I also removed \code{constexpr} and \code{noexcept} since they are irrelevant
to the exploration at hand.

\begin{enumerate}
  \item \label{fun:generator} \simd generator

    Status quo (P1928R9):
    \medskip\begin{lstlisting}[style=Vc]
std::simd<int> iota([](int i) { return i; });
    \end{lstlisting}

  \item \label{fun:load} \simd load from contiguous range

    Status quo (P1928R9):
    \medskip\begin{lstlisting}[style=Vc]
std::vector<int> data = {...};
std::simd<int> chunk(data.begin());
    \end{lstlisting}

  \item \label{fun:gather} \simd gather from contiguous range

    Status quo (P2664R6):
    \medskip\begin{lstlisting}[style=Vc]
std::vector<int> data = {/*...*/};
std::simd<int> idxs = /*...*/;
std::simd<int> std::gather_from(data, idxs);
    \end{lstlisting}

  \item \label{fun:permute} \simd permutations

    Status quo (P2664R6):
    \medskip\begin{lstlisting}[style=Vc]
std::simd<int> v = /*...*/;
std::simd<int> v2 = std::permute(v, [](int i) { return i ^ 1; });
    \end{lstlisting}

  \item \label{fun:select} \simd ternary operator replacement

    Status quo (P1928R9):
    \medskip\begin{lstlisting}[style=Vc]
std::simd<int> v = /*...*/;
std::simd<int> abs = std::simd_select(v >= 0, v, -v);
    \end{lstlisting}

  \item \label{fun:math} Math functions and algorithms

    Status quo (P1928R9):
    \medskip\begin{lstlisting}[style=Vc]
std::simd<float> x = /*...*/;
std::simd<float> y = std::exp(x);
std::simd<float> z = std::min(x, y);
    \end{lstlisting}

  \item \label{fun:maskred} Mask reductions

    Status quo (P1928R9):
    \medskip\begin{lstlisting}[style=Vc]
std::simd<float> x = /*...*/;
if (std::all_of(x > 0)) /*...*/
    \end{lstlisting}

  \item \label{concept} Simd concepts
    \begin{itemize}
      \item Constrain whether a type is a \simdT with \std\code{integral<T>}.
      \item Constrain whether a type is either \std\code{integral} or a \simdT
        with \std\code{integral<T>}.
    \end{itemize}

\end{enumerate}

\subsection{Status quo (latest revision of \code{simd} papers)}
\begin{description}
  \item[pros]
    \begin{itemize}
      \item \std\code{simd} is as concise as it could possibly be.

      \item Fairly good support for \simdgeneric programming.
    \end{itemize}

  \item[cons]
    \begin{itemize}
      \item We have a mix of non-member functions with and without
        \code{simd_} prefix.

      \item Most non-member functions would be nicer to read in code without
        the \code{simd} prefix.
        We introduce the prefix only because we are wary of the “name grab”
        in \code{std}.
        \Ie the motivation for the current naming scheme isn't the design of
        the \code{simd} API, but the freedom to evolve the standard library
        in the future.

      \item Load and gather (which are very similar in loading a SIMD
        “register” from a contiguous range of values) are inconsistent: One
        uses a constructor and member function, the other only a non-member
        function.

      \item Loads, stores, and the \code{simd} generator constructor cannot
        be used in \simdgeneric code.
    \end{itemize}
\end{description}

\subsection{Explorations in previous revision(s) of this paper}
\begin{itemize}
  \item Alternative 1: every function is a non-member with \code{simd} prefix
  \item Alternative 2: every function is a non-member without \code{simd} prefix
  \item Alternative 3: place everything but types into a namespace
  \item Alternative 4: make all non-member functions hidden friends
\end{itemize}

\subsection{Alternative 5: Place everything into a single namespace}
\label{sec:singlenamespace}

\medskip\begin{lstlisting}[style=Vc]
namespace std::simd {

template<class T, class Abi = /*...*/>
  class basic_simd;

template<class T, @\simdsizetype@ N = /*...*/>
  using simd = basic_simd<T, @\deducet@<T, N>>;

template<class V, class G>
  V
  generate(G&& gen);

template<class V = void, class It, class... Flags>
  conditional_t<is_same_v<V, void>, simd<iter_value_t<It>>, V>
  load_from(It first, simd_flags<Flags...> f = {});

template<class Rg, std::integral Idx, class AbiIdx, class... Flags>
  simd<ranges::range_value_t<Rg>, basic_simd<Idx, AbiIdx>::size()>
  gather_from(const Rg&& in, const basic_simd<Idx, AbiIdx>& indexes,
                   simd_flags<Flags...> f = {});

template<size_t SizeSelector = 0, class T, class Abi, class PermuteGenerator>
  simd<T, output-size>
  permute(const basic_simd<T, Abi>& v, PermuteGenerator&& fn);

template<size_t Bytes, class Abi, class T, class U>
  auto
  select(const basic_simd_mask<Bytes, Abi>& c, const T& a, const U& b)
    -> decltype(simd-select-impl(c, a, b));

template<class T, class Abi>
  basic_simd<T, Abi>
  exp(const basic_simd<T, Abi>& x);

template<class T, class Abi>
  basic_simd<T, Abi>
  min(const basic_simd<T, Abi>& x, const basic_simd<T, Abi>& y);

template<size_t Bs, class Abi>
  bool
  all_of(const basic_simd_mask<Bs, Abi>&);

template<class T>
  concept integral = /*...*/;

template<class T>
  concept generic_integral = std::integral<T> or std::simd::integral<T>;

}
\end{lstlisting}

(R0 discussed naming here, it was moved to \sect{sec:naming}.)

Usage example:
\medskip\begin{lstlisting}[style=Vc]
void f(std::simd::simd<float> vf, const std::vector<int>& data) {
  auto iota = std::simd::generate<std::simd::simd<int>>([](int i) { return i; });
  auto chunk = std::simd::load_from(data.begin());
  auto chunk_swapped = std::simd::gather_from(data, iota ^ 1);
  auto chunk_swapped2 = std::simd::permute(chunk, [](int i) { return i ^ 1; });
  assert(std::simd::all_of(chunk_swapped == chunk_swapped2));

  vf = std::simd::select(vf > 1.f, 1.f, vf);
  vf = std::simd::exp(vf);
  auto lo = std::simd::min(iota, chunk);
}
\end{lstlisting}

This is fairly verbose, so a user might decide to rather rely on ADL:
\medskip\begin{lstlisting}[style=Vc]
void f(std::simd::simd<float> vf, const std::vector<int>& data) {
  auto iota = std::simd::generate<std::simd::simd<int>>([](int i) { return i; });
  auto chunk = std::simd::load_from(data.begin());
  auto chunk_swapped = gather_from(data, iota ^ 1);
  auto chunk_swapped2 = permute(chunk, [](int i) { return i ^ 1; });
  assert(all_of(chunk_swapped == chunk_swapped2));

  vf = select(vf > 1.f, 1.f, vf);
  vf = exp(vf);
  auto lo = min(iota, chunk);
}
\end{lstlisting}

But as we can see, ADL only works for some of the functions.
If the function requires a template argument or none of the arguments are a
\simd / \mask, then the call still must be qualified.
Consequently, if a user wants to reduce the character overhead, a namespace
alias might be better suited:
\medskip\begin{lstlisting}[style=Vc]
namespace smd = std::simd;

void f(smd::simd<float> vf, const std::vector<int>& data) {
  auto iota = smd::generate<smd::simd<int>>([](int i) { return i; });
  auto chunk = smd::load_from(data.begin());
  auto chunk_swapped = smd::gather_from(data, iota ^ 1);
  auto chunk_swapped2 = smd::permute(chunk, [](int i) { return i ^ 1; });
  assert(smd::all_of(chunk_swapped == chunk_swapped2));

  vf = smd::select(vf > 1.f, 1.f, vf);
  vf = smd::exp(vf);
  auto lo = smd::min(iota, chunk);
}
\end{lstlisting}

The \simdgeneric programming example from previous sections now looks like
this:
\medskip\begin{lstlisting}[style=Vc]
template<std::integral T>
T scalar_only(T a, T b) {
  return 2 * std::min(a, b);
}

template<std::simd::integral T>
T simd_only(T a, T b) {
  return 2 * std::simd::min(a, b);
}

template<std::simd::generic_integral T>
T generic(T a, T b) {
  if constexpr (std::simd::integral<T>)
    return 2 * std::simd::min(a, b);
  else
    return 2 * std::min(a, b);
}
\end{lstlisting}

Another user might be looking for a way to qualify e.g. \code{<cmath>}
functions such that they work both with \code{T} and \simdT.
To that end one needs to basically inline \std\code{simd} into \code{std} and
thus write:
\medskip\begin{lstlisting}[style=Vc]
namespace xstd {
  using namespace std;
  using namespace std::simd;
}

void f(xstd::simd<float> vf, const xstd::vector<int>& data) {
  auto iota = xstd::generate<xstd::simd<int>>([](int i) { return i; });
  auto chunk = xstd::load_from(data.begin());
  auto chunk_swapped = xstd::gather_from(data, iota ^ 1);
  auto chunk_swapped2 = xstd::permute(chunk, [](int i) { return i ^ 1; });
  assert(xstd::all_of(chunk_swapped == chunk_swapped2));

  vf = xstd::select(vf > 1.f, 1.f, vf);
  vf = xstd::exp(vf);
  auto lo = xstd::min(iota, chunk);
}
\end{lstlisting}

I need to be convinced that the latter pattern isn't a liability, and
therefore I wouldn't allow this to go through code review without raising a
red flag.

\begin{description}
  \item[pros]
    \begin{itemize}
      \item We are free to grab names out of the new namespace.
      \item ADL still works.
      \item Consistent.
      \item[$\Rightarrow$] Users only need to learn: “If it's in the
        \std\code{simd} namespace then it works for \code{simd}s.
        When searching for a function for \code{simd}, look in the
        \std\code{simd} namespace.”
    \end{itemize}

  \item[cons]
    \begin{itemize}
      \item \simdgeneric programming just got harder.
      \item The class template name \std\code{simd::simd} is a bit awkward.
        (There are alternative names that we could adopt instead.)
    \end{itemize}
\end{description}

\myrating{unacceptable for lack of \simdgeneric programming;
interesting if we get rid of the out-of-the-box requirement for constexpr-if}

\pagebreak\subsection{Alternative 6: Place everything but obvious overloads into a
single namespace}

The preceding alternative probably went too far with moving \code{<cmath>}
overloads and algorithms like \code{min}, \code{clamp}, etc. into the
\std\code{simd} namespace.
So let's keep all functions that are a clear overload (\code{f(simd<T>)}) from
an existing function (\code{f(T)}) directly in the \code{std} namespace.
This is the “namespace equivalent” to the status-quo approach of whether a
\code{simd_} prefix is needed or not.

\medskip\begin{lstlisting}[style=Vc]
namespace std::simd {

template<class T, class Abi = /*...*/>
  class basic_simd;

template<class T, @\simdsizetype@ N = /*...*/>
  using simd = basic_simd<T, @\deducet@<T, N>>;

template<class V, class G>
  V
  generate(G&& gen);

template<class V = void, class It, class... Flags>
  conditional_t<is_same_v<V, void>, simd<iter_value_t<It>>, V>
  copy_from(It first, simd_flags<Flags...> f = {});

template<class Rg, std::integral Idx, class AbiIdx, class... Flags>
  simd<ranges::range_value_t<Rg>, basic_simd<Idx, AbiIdx>::size()>
  gather_from(const Rg&& in, const basic_simd<Idx, AbiIdx>& indexes,
                   simd_flags<Flags...> f = {});

template<size_t SizeSelector = 0, class T, class Abi, class PermuteGenerator>
  simd<T, output-size>
  permute(const basic_simd<T, Abi>& v, PermuteGenerator&& fn);

template<size_t Bytes, class Abi, class T, class U>
  auto
  select(const basic_simd_mask<Bytes, Abi>& c, const T& a, const U& b)
    -> decltype(simd-select-impl(c, a, b));

template<size_t Bs, class Abi>
  bool
  all_of(const basic_simd_mask<Bs, Abi>&);

template<class T>
  concept integral = /*...*/;

template<class T>
  concept generic_integral = std::integral<T> or std::simd::integral<T>;

}

namespace std {

template<class T, class Abi>
  simd::basic_simd<T, Abi>
  exp(const simd::basic_simd<T, Abi>& x);

template<class T, class Abi>
  simd::basic_simd<T, Abi>
  min(const simd::basic_simd<T, Abi>& x, const simd::basic_simd<T, Abi>& y);

}
\end{lstlisting}

Usage example:
\medskip\begin{lstlisting}[style=Vc]
void f(std::simd::simd<float> vf, const std::vector<int>& data) {
  auto iota = std::simd::generate<std::simd::simd<int>>([](int i) { return i; });
  auto chunk = std::simd::copy_from(data.begin());
  auto chunk_swapped = std::simd::gather_from(data, iota ^ 1);
  auto chunk_swapped2 = std::simd::permute(chunk, [](int i) { return i ^ 1; });
  assert(std::simd::all_of(chunk_swapped == chunk_swapped2));

  vf = std::simd::select(vf > 1.f, 1.f, vf);
  vf = std::exp(vf);
  auto lo = std::min(iota, chunk);
}
\end{lstlisting}

When relying on ADL, nothing changes compared to the example in the preceding
section.
However, if we now create a namespace alias and call everything fully
qualified, the necessary qualifications could be considered slightly
incoherent:

\medskip\begin{lstlisting}[style=Vc]
namespace smd = std::simd;

void f(smd::simd<float> vf, const std::vector<int>& data) {
  auto iota = smd::generate<smd::simd<int>>([](int i) { return i; });
  auto chunk = smd::copy_from(data.begin());
  auto chunk_swapped = smd::gather_from(data, iota ^ 1);
  auto chunk_swapped2 = smd::permute(chunk, [](int i) { return i ^ 1; });
  assert(smd::all_of(chunk_swapped == chunk_swapped2));

  vf = smd::select(vf > 1.f, 1.f, vf);
  vf = std::exp(vf);
  auto lo = std::min(iota, chunk);
}
\end{lstlisting}

At this point all functions already work for \simdgeneric code (or can be made
to work with suitable overloads in the \std\code{simd} namespace).
If LEWG were to adopt this naming style, then we need to decide on a per
function basis, whether the function is “SIMD-only” or whether an overload for
the value-type is useful on its own.
For the latter, the function goes into \code{std} otherwise it needs to go
into \std\code{simd}.

The \simdgeneric programming example from previous sections now looks like
this:
\medskip\begin{lstlisting}[style=Vc]
template<std::integral T>
T scalar_only(T a, T b) {
  return 2 * std::min(a, b);
}

template<std::simd::integral T>
T simd_only(T a, T b) {
  return 2 * std::min(a, b);
}

template<std::simd::generic_integral T>
T generic(T a, T b) {
  return 2 * std::min(a, b);
}
\end{lstlisting}

\begin{description}
  \item[pros]
    \begin{itemize}
      \item We are free to grab names out of the new namespace.
      \item ADL works.
      \item Fairly consistent.
      \item[$\Rightarrow$] Users need to learn: “If it's in the \std\code{simd}
        namespace then it works for \code{simd}s.
        When searching for a function for \code{simd}, if the same function
        exists / could exist for scalars look for it in \code{std}, otherwise
        look in the \std\code{simd} namespace.”
      \item \simdgeneric programming is straightforward to provide and use.
    \end{itemize}

  \item[cons]
    \begin{itemize}
      \item The class template name \std\code{simd::simd} is a bit awkward.

      \item We have a mix of non-member functions in \code{std} and
        \std\code{simd}.
    \end{itemize}
\end{description}

\myrating{acceptable; but not much different from the status quo --- not convinced this is actually \emph{better}}

\pagebreak\subsection{Alternative 7: Place \code{simd} into a single namespace with a different namespace for \simdgeneric interfaces}

\medskip\begin{lstlisting}[style=Vc]
namespace std::simd {

template<class T, class Abi = /*...*/>
  class basic_simd;

template<class T, @\simdsizetype@ N = /*...*/>
  using simd = basic_simd<T, @\deducet@<T, N>>;

template<class V, class G>
  V
  generate(G&& gen);

template<class V = void, class It, class... Flags>
  conditional_t<is_same_v<V, void>, simd<iter_value_t<It>>, V>
  load_from(It first, simd_flags<Flags...> f = {});

template<class Rg, std::integral Idx, class AbiIdx, class... Flags>
  simd<ranges::range_value_t<Rg>, basic_simd<Idx, AbiIdx>::size()>
  gather_from(const Rg&& in, const basic_simd<Idx, AbiIdx>& indexes,
                   simd_flags<Flags...> f = {});

template<size_t SizeSelector = 0, class T, class Abi, class PermuteGenerator>
  simd<T, output-size>
  permute(const basic_simd<T, Abi>& v, PermuteGenerator&& fn);

template<size_t Bytes, class Abi, class T, class U>
  auto
  select(const basic_simd_mask<Bytes, Abi>& c, const T& a, const U& b)
    -> decltype(simd-select-impl(c, a, b));

template<class T, class Abi>
  basic_simd<T, Abi>
  exp(const basic_simd<T, Abi>& x);

template<class T, class Abi>
  basic_simd<T, Abi>
  min(const basic_simd<T, Abi>& x, const basic_simd<T, Abi>& y);

template<size_t Bs, class Abi>
  bool
  all_of(const basic_simd_mask<Bs, Abi>&);

template<class T>
  concept integral = /*...*/;

} // std::simd

namespace std::simd_generic {

namespace scalar {

template<@\vectorizable@ T, class G>
  T
  generate(G&& gen);

template<@\vectorizable@ T, class It, class... Flags>
  T
  load_from(It first, simd_flags<Flags...> f = {});

template<class Rg, std::integral Idx, class... Flags>
  ranges::range_value_t<Rg>
  gather_from(const Rg&& in, Idx index, simd_flags<Flags...> f = {});

template<class T, class U>
  auto
  select(bool c, const T& a, const U& b)
    -> decltype(simd-select-impl(c, a, b));

using std::exp;

using std::min;

bool
all_of(same_as<bool>);

} // (std::simd_generic::)scalar

using namespace std::simd;

using namespace std::simd_generic::scalar;

template<class T>
  concept integral = std::integral<T> or std::simd::integral<T>;

} // std::simd_generic
\end{lstlisting}

The usage example looks exactly like in \sect{sec:singlenamespace}.
There is also no difference with regard to ADL and using a namespace alias.

However, the situation for \simdgeneric programming is rather different.
At this point a user can be very clear about “scalar-only” (\code{std}),
“simd-only” (\std\code{simd}), and \simdgeneric (\std\code{simd_generic})
code.
Thus, our recurring example becomes:
\medskip\begin{lstlisting}[style=Vc]
template<std::simd::integral T>
template<std::integral T>
T scalar_only(T a, T b) {
  return 2 * std::min(a, b);
}

T simd_only(T a, T b) {
  return 2 * std::simd::min(a, b);
}

template<std::simd_generic::integral T>
T fun(T a, T b) {
  return 2 * std::simd_generic::min(a, b);
}
\end{lstlisting}

Now the namespace of the \code{integral} concept matches the namespace of the
functions that we need to use.
There's a clear mechanism from opting into \simdgeneric overloads --- or avoiding them when they are not required.
All the prevision definitions of SIMD-integral and \simdgeneric-integral
concepts didn't have this clear association with a set of function overloads.

The ability to choose between \code{std::simd} and \code{std::simd_generic}
also provides another level of clarity in stating intent: Do you expect your
code to be called only with \simdT or also with \code{T}?

Note that, as declared above, also \code{<cmath>} overloads are in different
namespaces.
Thus, instead of writing \code{using std::exp}, I can now write \code{using
std::simd_generic::exp} and all unqualified \code{exp} calls are overloaded
for scalars and \code{simd}s.

I expect that many users might be interested in shortening \std\code{simd} and
even more \std\code{simd_generic}.
If that's the case, we're going to see many namespace aliases for the two
namespaces.

\begin{description}
  \item[pros]
    \begin{itemize}
      \item We are free to grab names out of the new namespace.
      \item ADL still works.
      \item Consistent.
      \item[$\Rightarrow$] Users only need to learn: “If it's in the
        \std\code{simd} namespace then it works for \code{simd}s.
        When searching for a function for \code{simd}, look in the
        \std\code{simd} namespace.
        When it needs to work generically for \code{simd} and scalars, just
        switch to \std\code{simd_generic}.”
      \item Opt-in \simdgeneric programming that is fairly “safe” with regard
        to accidentally calling the wrong overload.
    \end{itemize}

  \item[cons]
    \begin{itemize}
      \item The class template name \std\code{simd::simd} still is a bit
        awkward.
        \\(standard SIMD vector / \std\code{simd::vec}?)
      \item \std\code{simd_generic} is too long and will be abbreviated with
        different namespace aliases in different code bases\footnote{this is
        normal in other languages, \eg Python}.
    \end{itemize}
\end{description}

\myrating{
  sold; feels good after implementing it; happy about the clear separation of
  scalar / SIMD / \simdgeneric; happy about concise code through namespace
  aliases
}

\subsubsection{On renaming \code{std::simd::simd} to \code{std::simd::vec}}
Personally, I don't think \std\code{simd::simd} is a problem.
Especially, considering that users might introduce a namespace alias or even
--- heaven forbid --- import the whole \std\code{simd} (or
\std\code{simd_generic}) namespace into their local scope.
If \code{vec} needs to stand on its own without the \code{simd::} part of the
name, I fear we might lose clarity compared to \code{simd}.

I believe the situation is different for \std\code{simd::simd_mask}, which,
in my opinion, can live without the \code{simd_} part in its name.
Thus, even after a \code{using namespace std::simd;} the alias template name
\code{mask} is expressive enough.
(Because \code{mask} only appears in proximity to \code{simd} --- if it
appears in code at all.)

\pagebreak\subsection{Alternative 8: Everything in a single namespace with using-declarations in std}
\label{sec:singlenamespace2}

After LEWG feedback in St. Louis, I added this alternative, which is basically
a combination of Alternatives 5 and 6 (Sections \ref{sec:singlenamespace} and
\ref{sec:alt6}).

\medskip\begin{lstlisting}
namespace std::simd {

template<class T, class Abi = /*...*/>
  class basic_simd;

template<class T, @\simdsizetype@ N = /*...*/>
  using simd = basic_simd<T, @\deducet@<T, N>>;

template<class V, class G>
  V
  generate(G&& gen);

template<class V = void, class It, class... Flags>
  conditional_t<is_same_v<V, void>, simd<iter_value_t<It>>, V>
  load_from(It first, simd_flags<Flags...> f = {});

template<class Rg, std::integral Idx, class AbiIdx, class... Flags>
  simd<ranges::range_value_t<Rg>, basic_simd<Idx, AbiIdx>::size()>
  gather_from(const Rg&& in, const basic_simd<Idx, AbiIdx>& indexes,
                   simd_flags<Flags...> f = {});

template<size_t SizeSelector = 0, class T, class Abi, class PermuteGenerator>
  simd<T, output-size>
  permute(const basic_simd<T, Abi>& v, PermuteGenerator&& fn);

template<size_t Bytes, class Abi, class T, class U>
  auto
  select(const basic_simd_mask<Bytes, Abi>& c, const T& a, const U& b)
    -> decltype(simd-select-impl(c, a, b));

template<class T, class Abi>
  basic_simd<T, Abi>
  exp(const basic_simd<T, Abi>& x);

template<class T, class Abi>
  basic_simd<T, Abi>
  min(const basic_simd<T, Abi>& x, const basic_simd<T, Abi>& y);

template<size_t Bs, class Abi>
  bool
  all_of(const basic_simd_mask<Bs, Abi>&);

template<class T>
  concept integral = /*...*/;

template<class T>
  concept generic_integral = std::integral<T> or std::simd::integral<T>;

}

namespace std {
  using simd::exp;
  using simd::min;
}
\end{lstlisting}

Usage example:
\medskip\begin{lstlisting}[style=Vc]
void f(std::simd::simd<float> vf, const std::vector<int>& data) {
  auto iota = std::simd::generate<std::simd::simd<int>>([](int i) { return i; });
  auto chunk = std::simd::load_from(data.begin());
  auto chunk_swapped = std::simd::gather_from(data, iota ^ 1);
  auto chunk_swapped2 = std::simd::permute(chunk, [](int i) { return i ^ 1; });
  assert(std::simd::all_of(chunk_swapped == chunk_swapped2));

  vf = std::simd::select(vf > 1.f, 1.f, vf);
  vf = std::exp(vf);
  auto lo = std::min(iota, chunk);
}
\end{lstlisting}

Again, this is fairly verbose, so a user might decide to rather rely on ADL:
\medskip\begin{lstlisting}[style=Vc]
void f(std::simd::simd<float> vf, const std::vector<int>& data) {
  auto iota = std::simd::generate<std::simd::simd<int>>([](int i) { return i; });
  auto chunk = std::simd::load_from(data.begin());
  auto chunk_swapped = gather_from(data, iota ^ 1);
  auto chunk_swapped2 = permute(chunk, [](int i) { return i ^ 1; });
  assert(all_of(chunk_swapped == chunk_swapped2));

  vf = select(vf > 1.f, 1.f, vf);
  vf = exp(vf);
  auto lo = min(iota, chunk);
}
\end{lstlisting}

But as we can see, ADL only works for some of the functions.
If the function requires a template argument or none of the arguments are a
\simd / \mask, then the call still must be qualified.
Consequently, if a user wants to reduce the character overhead, a namespace
alias might be better suited:
\medskip\begin{lstlisting}[style=Vc]
namespace smd = std::simd;

void f(smd::simd<float> vf, const std::vector<int>& data) {
  auto iota = smd::generate<smd::simd<int>>([](int i) { return i; });
  auto chunk = smd::load_from(data.begin());
  auto chunk_swapped = smd::gather_from(data, iota ^ 1);
  auto chunk_swapped2 = smd::permute(chunk, [](int i) { return i ^ 1; });
  assert(smd::all_of(chunk_swapped == chunk_swapped2));

  vf = smd::select(vf > 1.f, 1.f, vf);
  vf = smd::exp(vf);
  auto lo = smd::min(iota, chunk);
}
\end{lstlisting}

The \simdgeneric programming example from previous sections now looks like
this:
\medskip\begin{lstlisting}[style=Vc]
template<std::integral T>
T scalar_only(T a, T b) {
  return 2 * std::min(a, b);
}

template<std::simd::integral T>
T simd_only(T a, T b) {
  return 2 * std::simd::min(a, b);
}

template<std::simd::generic_integral T>
T generic(T a, T b) {
  return 2 * std::min(a, b);
}
\end{lstlisting}

This is different to Alternative 7 (\sect{sec:alt7}), where simd-generic
programming requires an opt-in, i.e. a code-change from scalar code.
With this approach existing non-simd code (\code{scalar_only}) can be modified
to use data-parallel types and every function that exists in \code{std} and can
be used in simd-generic code needs no further work (\code{generic}).

\begin{description}
  \item[pros]
    \begin{itemize}
      \item We are free to grab names out of the new namespace.
      \item ADL still works.
      \item Consistent.
      \item[$\Rightarrow$] Users only need to learn: “If it's in the
        \std\code{simd} namespace then it works for \code{simd}s.
        When searching for a function for \code{simd}, look in the
        \std\code{simd} namespace.”
      \item \simdgeneric programming works.
    \end{itemize}

  \item[cons]
    \begin{itemize}
      \item The class template name \std\code{simd::simd} is a bit awkward.
        (There are alternative names that we could adopt instead.)
    \end{itemize}
\end{description}

\myrating{I like it. Clear separation of \code{simd} and non-\code{simd}
functions while still providing good support for \simdgeneric programming}


\section{Renaming simd and simd_mask}\label{sec:naming}
R0 of this paper discussed naming in “Alternative 5”.
But renaming isn't tied to one specific alternative but a general consideration
if the class templates are moved into a namespace.

\subsection{Namespace names}
Conceivable \emph{variations} for the \std\code{simd} namespace are
\begin{itemize}
  \item \std\code{datapar} (The \simd and \mask types are in the “Data-parallel types” section in the IS.)
  \item \std\code{dp} (data-parallel)
  \item \std\code{dpt} (data-parallel types)
  \item \std\code{unseq}
\end{itemize}
Personally, I don't believe any of these are an improvement.

\subsection{Class template names}
However, I suggest renaming \std\code{simd::}\MayBreak\mask to
\std\code{simd::}\MayBreak\code{basic_mask}, and accordingly
\std\code{simd::simd_mask} to \std\code{simd::mask}.

Consequently, if we're reading the namespace as part of the type name
(\code{simd::mask}) we should consider renaming \code{simd::simd} to:
\begin{list}{}{
  \setlength{\topsep}{0pt}%
  \setlength{\leftmargin}{7em}%
  \setlength{\rightmargin}{0pt}%
  \setlength{\labelwidth}{7em}%
}

  \item[\code{simd::vector}]
    We often speak about “SIMD vectors”; so in principle this a good name.
    However, I fear that using the heavily overloaded term “vector” has too
    much potential for confusion.
    Especially the use of \code{using namespace std; using namespace
    std::simd;}\footnote{huge foot-gun, which WG21 members will quickly
    recognize as such} or even just \code{using namespace std::simd} by
    itself would lead to a lot of confusion.

  \item[\code{simd::vec}]
    This name tries to avoid the confusion by spelling “vector” as an
    abbreviation (and thus avoid the “hold on, why does it say \code{vector}
    here?” moments when reviewing code)

  \item[\code{simd::value}]
    Note the naming precedent in \code{valarray}, which is called “value
    array”.

  \item[\code{simd::values}]

  \item[\code{simd::array}]
    The static extent matches \std\code{array}; it's a \std\code{array} with
    SIMD operations; also, I believe conversions between \code{simd} and
    \std\code{array} of equal extent should be implicit\ldots
\end{list}

From all of these, I'd prefer if we could use \code{simd::vector<T>} --- and
in the library where this work originates it was called \code{Vc::Vector<T>}
--- but I fear this will lead to confusion and just isn't worth the trouble.
It seems however that \code{simd::vec<T>} could resolve that issue and still
be fairly close to the term we use in speech.
Next best\ldots{} \code{simd::array} is starting to grow on me.
This term was never considered before, if I remember correctly.
It appeals to me be because I believe we should make CTAD and implicit
conversions work for \code{simd<T, N>} $\leftrightarrow$ \code{array<T, N>}\footnote{See P3299}.
In terms of bit-representation, they typically are the same thing.
They differ in alignment\footnote{Note that alignment can influence
\code{sizeof}.}, function argument passing\footnote{\Eg the Itanium ABI passes
  \code{array<float, 4>} as two \code{XMM} registers and \code{simd<float, 4>}
as one \code{XMM} register.}, and whether you can apply operators that the
value-type provides.

\subsection{On renaming \code{std::simd::simd} to \code{std::simd::vec}}
Personally, I don't think \std\code{simd::simd} is a problem.
Especially, considering that users might introduce a namespace alias or even
--- heaven forbid --- import the whole \std\code{simd} (or
\std\code{simd_generic}) namespace into their local scope.
If \code{vec} needs to stand on its own without the \code{simd::} part of the
name, I fear we might lose clarity compared to \code{simd}.

I believe the situation is different for \std\code{simd::simd_mask}, which,
in my opinion, can live without the \code{simd_} part in its name.
Thus, even after a \code{using namespace std::simd;} the alias template name
\code{mask} is expressive enough.
(Because \code{mask} only appears in proximity to \code{simd} --- if it
appears in code at all.)


\section{Recommendation: An example after renaming}

\medskip\begin{lstlisting}
namespace simd = std::simd;

// compute log for positive inputs
simd::vec<float> f(std::span<float> data)
{
  simd::vec<float>  x = simd::load_from(data);
  simd::mask<float> positive = x > 0.f;
  simd::vec<float>  l = std::log(simd::select(positive, x, 1.f));
  return simd::select(positive, l, x);
}
\end{lstlisting}

\section{Proposed polls}

Any vote would be against the status quo, which so far can be summarized as:
\begin{itemize}
  \item types and functions go directly into \code{std}

  \item when naming a function for \code{simd},
    \begin{itemize}
      \item if the same function exists / could exist for scalars or a range:
        no \code{simd_} prefix,
      \item otherwise the function name needs a \code{simd_} prefix
    \end{itemize}

  \item traits and types need a \code{simd} in their name
\end{itemize}

\wgPoll{Adopt wording instructions (Alternative 8) from \wgDocumentNumber\,
  renaming (\code{basic_})\code{simd_mask} to (\code{basic_})\code{mask} but
  \emph{not} (\code{basic_})\code{simd} to (\code{basic_})\code{vec}}
{&&&&}

\wgPoll{Adopt wording instructions (Alternative 8) from \wgDocumentNumber\,
  renaming (\code{basic_})\code{simd_mask} to (\code{basic_})\code{mask}
  \emph{and} (\code{basic_})\code{simd} to (\code{basic_})\code{vec}}
{&&&&}

\section{Wording}\label{sec:wording}

This is fairly mechanical work, adding namespaces and adjusting type and
function names.

In general the instructions go like this:
\begin{enumerate}
  \item Move every type and function in “Data-Parallel types [simd]” from
    namespace \code{std} to namespace \std\code{simd}.

  \item Replace all occurences of \code{basic_simd_mask} with
    \code{basic_mask}.

  \item Replace all occurences of \code{simd_mask} with \code{mask}.

  \item (TBD) Replace all occurences of \code{basic_simd} with \code{basic_vec}.

  \item (TBD) Replace all occurences of \code{simd} with \code{vec}.

  \item Rename all functions and remaining types / class templates: \code{s/\backslash<simd_//}.

  \item \code{s/rebind_simd/rebind/}

  \item \code{s/resize_simd/resize/}

  \item Add using-declarations in namespace \code{std} for the following functions:
    \begin{itemize}
      \item all function defined in [simd.math], and
      \item \code{min}, \code{max}, \code{minmax}, and \code{clamp} in [simd.alg].
    \end{itemize}
\end{enumerate}

\end{document}
% vim: sw=2 sts=2 ai et tw=0
