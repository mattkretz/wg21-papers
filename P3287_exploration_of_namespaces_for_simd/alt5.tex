\subsection{Alternative 5: Place everything into a single namespace}
\label{sec:singlenamespace}

\medskip\begin{lstlisting}[style=Vc]
namespace std::simd {

template<class T, class Abi = /*...*/>
  class basic_simd;

template<class T, @\simdsizetype@ N = /*...*/>
  using simd = basic_simd<T, @\deducet@<T, N>>;

template<class V, class G>
  V
  generate(G&& gen);

template<class V = void, class It, class... Flags>
  conditional_t<is_same_v<V, void>, simd<iter_value_t<It>>, V>
  copy_from(It first, simd_flags<Flags...> f = {});

template<class Rg, std::integral Idx, class AbiIdx, class... Flags>
  simd<ranges::range_value_t<Rg>, basic_simd<Idx, AbiIdx>::size()>
  gather_from(const Rg&& in, const basic_simd<Idx, AbiIdx>& indexes,
                   simd_flags<Flags...> f = {});

template<size_t SizeSelector = 0, class T, class Abi, class PermuteGenerator>
  simd<T, output-size>
  permute(const basic_simd<T, Abi>& v, PermuteGenerator&& fn);

template<size_t Bytes, class Abi, class T, class U>
  auto
  select(const basic_simd_mask<Bytes, Abi>& c, const T& a, const U& b)
    -> decltype(simd-select-impl(c, a, b));

template<class T, class Abi>
  basic_simd<T, Abi>
  exp(const basic_simd<T, Abi>& x);

template<class T, class Abi>
  basic_simd<T, Abi>
  min(const basic_simd<T, Abi>& x, const basic_simd<T, Abi>& y);

template<size_t Bs, class Abi>
  bool
  all_of(const basic_simd_mask<Bs, Abi>&);

template<class T>
  concept integral = /*...*/;

template<class T>
  concept generic_integral = std::integral<T> or std::simd::integral<T>;

}
\end{lstlisting}

Conceivable \emph{variations} for the \std\code{simd} namespace are
\begin{itemize}
  \item \std\code{datapar} (The \simd and \mask types are in the “Data-parallel types” section in the IS.)
  \item \std\code{dp} (data-parallel)
  \item \std\code{dpt} (data-parallel types)
  \item \std\code{unseq}
\end{itemize}
Personally, I don't believe any of these are an improvement.

However, I would suggest renaming \std\code{simd::}\MayBreak\mask to
\std\code{simd::}\MayBreak\code{basic_mask}, and accordingly \code{simd::simd_mask}
to \code{simd::mask}.

Consequently, if we're reading the namespace as part of the type name
(\code{simd::mask}) we should consider renaming \code{simd::simd} to:
\begin{list}{}{
  \setlength{\topsep}{0pt}%
  \setlength{\leftmargin}{7em}%
  \setlength{\rightmargin}{0pt}%
  \setlength{\labelwidth}{7em}%
}

  \item[\code{simd::vector}]
    We often speak about “SIMD vectors”; so in principle this a good name.
    However, I fear that using the heavily overloaded term “vector” has too
    much potential for confusion.
    Especially the use of \code{using namespace std; using namespace
    std::simd;}\footnote{huge foot-gun, which WG21 members will quickly
    recognize as such} or even just \code{using namespace std::simd} by
    itself would lead to a lot of confusion.

  \item[\code{simd::vec}]
    This name tries to avoid the confusion by spelling “vector” as an
    abbreviation (and thus avoid the “hold on, why does it say \code{vector}
    here?” moments when reviewing code)

  \item[\code{simd::value}]
    Note the naming precedent in \code{valarray}, which is called “value
    array”.

  \item[\code{simd::values}]

  \item[\code{simd::array}]
    The static extent matches \std\code{array}; it's a \std\code{array} with
    SIMD operations; also, I believe conversions between \code{simd} and
    \std\code{array} of equal extent should be implicit\ldots
\end{list}

From all of these, I'd prefer if we could use \code{simd::vector<T>} --- and
in the library where this work originates it was called \code{Vc::Vector<T>}
--- but I fear this will lead to confusion and just isn't worth the trouble.
It seems however that \code{simd::vec<T>} could resolve that issue and still
be fairly close to the term we use in speech.
Next best\ldots{} \code{simd::array} is starting to grow on me.
This term was never considered before (IIRC\footnote{if I remember
correctly}).
It appeals to me be because I believe we should make CTAD and implicit
conversions work for \code{simd<T, N>} $\leftrightarrow$ \code{array<T, N>}.
In terms of bit-representation, they typically are the same thing.
They differ in alignment\footnote{Note that alignment can influence
\code{sizeof}.}, function argument passing\footnote{\Eg the Itanium ABI passes
  \code{array<float, 4>} as two \code{XMM} registers and \code{simd<float, 4>}
as one \code{XMM} register.}, and whether you can apply operators that the
value-type provides.

For now, I don't want to propose a name change.
But please give me feedback if you think I should / should not (propose a name
change).

Usage example:
\medskip\begin{lstlisting}[style=Vc]
void f(std::simd::simd<float> vf, const std::vector<int>& data) {
  auto iota = std::simd::generate<std::simd::simd<int>>([](int i) { return i; });
  auto chunk = std::simd::copy_from(data.begin());
  auto chunk_swapped = std::simd::gather_from(data, iota ^ 1);
  auto chunk_swapped2 = std::simd::permute(chunk, [](int i) { return i ^ 1; });
  assert(std::simd::all_of(chunk_swapped == chunk_swapped2));

  vf = std::simd::select(vf > 1.f, 1.f, vf);
  vf = std::simd::exp(vf);
  auto lo = std::simd::min(iota, chunk);
}
\end{lstlisting}

This is fairly verbose, so a user might decide to rather rely on ADL:
\medskip\begin{lstlisting}[style=Vc]
void f(std::simd::simd<float> vf, const std::vector<int>& data) {
  auto iota = std::simd::generate<std::simd::simd<int>>([](int i) { return i; });
  auto chunk = std::simd::copy_from(data.begin());
  auto chunk_swapped = gather_from(data, iota ^ 1);
  auto chunk_swapped2 = permute(chunk, [](int i) { return i ^ 1; });
  assert(all_of(chunk_swapped == chunk_swapped2));

  vf = select(vf > 1.f, 1.f, vf);
  vf = exp(vf);
  auto lo = min(iota, chunk);
}
\end{lstlisting}

But as we can see, ADL only works for some of the functions.
If the function requires a template argument or none of the arguments are a
\simd / \mask, then the call still must be qualified.
Consequently, if a user wants to reduce the character overhead, a namespace
alias might be better suited:
\medskip\begin{lstlisting}[style=Vc]
namespace smd = std::simd;

void f(smd::simd<float> vf, const std::vector<int>& data) {
  auto iota = smd::generate<smd::simd<int>>([](int i) { return i; });
  auto chunk = smd::copy_from(data.begin());
  auto chunk_swapped = smd::gather_from(data, iota ^ 1);
  auto chunk_swapped2 = smd::permute(chunk, [](int i) { return i ^ 1; });
  assert(smd::all_of(chunk_swapped == chunk_swapped2));

  vf = smd::select(vf > 1.f, 1.f, vf);
  vf = smd::exp(vf);
  auto lo = smd::min(iota, chunk);
}
\end{lstlisting}

The \simdgeneric programming example from previous sections now looks like
this:
\medskip\begin{lstlisting}[style=Vc]
template<std::integral T>
T scalar_only(T a, T b) {
  return 2 * std::min(a, b);
}

template<std::simd::integral T>
T simd_only(T a, T b) {
  return 2 * std::simd::min(a, b);
}

template<std::simd::generic_integral T>
T generic(T a, T b) {
  if constexpr (std::simd::integral<T>)
    return 2 * std::simd::min(a, b);
  else
    return 2 * std::min(a, b);
}
\end{lstlisting}

Another user might be looking for a way to qualify e.g. \code{<cmath>}
functions such that they work both with \code{T} and \simdT.
To that end one needs to basically inline \std\code{simd} into \code{std} and
thus write:
\medskip\begin{lstlisting}[style=Vc]
namespace xstd {
  using namespace std;
  using namespace std::simd;
}

void f(xstd::simd<float> vf, const xstd::vector<int>& data) {
  auto iota = xstd::generate<xstd::simd<int>>([](int i) { return i; });
  auto chunk = xstd::copy_from(data.begin());
  auto chunk_swapped = xstd::gather_from(data, iota ^ 1);
  auto chunk_swapped2 = xstd::permute(chunk, [](int i) { return i ^ 1; });
  assert(xstd::all_of(chunk_swapped == chunk_swapped2));

  vf = xstd::select(vf > 1.f, 1.f, vf);
  vf = xstd::exp(vf);
  auto lo = xstd::min(iota, chunk);
}
\end{lstlisting}

I need to be convinced that the latter pattern isn't a liability, and
therefore I wouldn't allow this to go through code review without raising a
red flag.

\begin{description}
  \item[pros]
    \begin{itemize}
      \item We are free to grab names out of the new namespace.
      \item ADL still works.
      \item Consistent.
      \item[$\Rightarrow$] Users only need to learn: “If it's in the
        \std\code{simd} namespace then it works for \code{simd}s.
        When searching for a function for \code{simd}, look in the
        \std\code{simd} namespace.”
    \end{itemize}

  \item[cons]
    \begin{itemize}
      \item \simdgeneric programming just got harder.
      \item The class template name \std\code{simd::simd} is a bit awkward.
        (There are alternative names that we could adopt instead.)
    \end{itemize}
\end{description}

\myrating{unacceptable for lack of \simdgeneric programming;
interesting if we get rid of the out-of-the-box requirement for constexpr-if}
