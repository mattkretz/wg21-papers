\subsection{Alternative 7: Place \code{simd} into a single namespace with a different namespace for \simdgeneric interfaces}

\medskip\begin{lstlisting}[style=Vc]
namespace std::simd {

template<class T, class Abi = /*...*/>
  class basic_simd;

template<class T, @\simdsizetype@ N = /*...*/>
  using simd = basic_simd<T, @\deducet@<T, N>>;

template<class V, class G>
  V
  generate(G&& gen);

template<class V = void, class It, class... Flags>
  conditional_t<is_same_v<V, void>, simd<iter_value_t<It>>, V>
  copy_from(It first, simd_flags<Flags...> f = {});

template<class Rg, std::integral Idx, class AbiIdx, class... Flags>
  simd<ranges::range_value_t<Rg>, basic_simd<Idx, AbiIdx>::size()>
  gather_from(const Rg&& in, const basic_simd<Idx, AbiIdx>& indexes,
                   simd_flags<Flags...> f = {});

template<size_t SizeSelector = 0, class T, class Abi, class PermuteGenerator>
  simd<T, output-size>
  permute(const basic_simd<T, Abi>& v, PermuteGenerator&& fn);

template<size_t Bytes, class Abi, class T, class U>
  auto
  select(const basic_simd_mask<Bytes, Abi>& c, const T& a, const U& b)
    -> decltype(simd-select-impl(c, a, b));

template<class T, class Abi>
  basic_simd<T, Abi>
  exp(const basic_simd<T, Abi>& x);

template<class T, class Abi>
  basic_simd<T, Abi>
  min(const basic_simd<T, Abi>& x, const basic_simd<T, Abi>& y);

template<size_t Bs, class Abi>
  bool
  all_of(const basic_simd_mask<Bs, Abi>&);

template<class T>
  concept integral = /*...*/;

} // std::simd

namespace std::simd_generic {

namespace scalar {

template<@\vectorizable@ T, class G>
  T
  generate(G&& gen);

template<@\vectorizable@ T, class It, class... Flags>
  T
  copy_from(It first, simd_flags<Flags...> f = {});

template<class Rg, std::integral Idx, class... Flags>
  ranges::range_value_t<Rg>
  gather_from(const Rg&& in, Idx index, simd_flags<Flags...> f = {});

template<class T, class U>
  auto
  select(bool c, const T& a, const U& b)
    -> decltype(simd-select-impl(c, a, b));

using std::exp;

using std::min;

bool
all_of(same_as<bool>);

} // (std::simd_generic::)scalar

using namespace std::simd;

using namespace std::simd_generic::scalar;

template<class T>
  concept integral = std::integral<T> or std::simd::integral<T>;

} // std::simd_generic
\end{lstlisting}

The usage example looks exactly like in \sect{sec:singlenamespace}.
There is also no difference with regard to ADL and using a namespace alias.

However, the situation for \simdgeneric programming is rather different.
At this point a user can be very clear about “scalar-only” (\code{std}),
“simd-only” (\std\code{simd}), and \simdgeneric (\std\code{simd_generic})
code.
Thus, our recurring example becomes:
\medskip\begin{lstlisting}[style=Vc]
template<std::simd::integral T>
template<std::integral T>
T scalar_only(T a, T b) {
  return 2 * std::min(a, b);
}

T simd_only(T a, T b) {
  return 2 * std::simd::min(a, b);
}

template<std::simd_generic::integral T>
T fun(T a, T b) {
  return 2 * std::simd_generic::min(a, b);
}
\end{lstlisting}

Now the namespace of the \code{integral} concept matches the namespace of the
functions that we need to use.
There's a clear mechanism from opting into \simdgeneric overloads --- or avoiding them when they are not required.
All the prevision definitions of SIMD-integral and \simdgeneric-integral
concepts didn't have this clear association with a set of function overloads.

The ability to choose between \code{std::simd} and \code{std::simd_generic}
also provides another level of clarity in stating intent: Do you expect your
code to be called only with \simdT or also with \code{T}?

Note that, as declared above, also \code{<cmath>} overloads are in different
namespaces.
Thus, instead of writing \code{using std::exp}, I can now write \code{using
std::simd_generic::exp} and all unqualified \code{exp} calls are overloaded
for scalars and \code{simd}s.

I expect that many users might be interested in shortening \std\code{simd} and
even more \std\code{simd_generic}.
If that's the case, we're going to see many namespace aliases for the two
namespaces.

\begin{description}
  \item[pros]
    \begin{itemize}
      \item We are free to grab names out of the new namespace.
      \item ADL still works.
      \item Consistent.
      \item[$\Rightarrow$] Users only need to learn: “If it's in the
        \std\code{simd} namespace then it works for \code{simd}s.
        When searching for a function for \code{simd}, look in the
        \std\code{simd} namespace.
        When it needs to work generically for \code{simd} and scalars, just
        switch to \std\code{simd_generic}.”
      \item Opt-in \simdgeneric programming that is fairly “safe” with regard
        to accidentally calling the wrong overload.
    \end{itemize}

  \item[cons]
    \begin{itemize}
      \item The class template name \std\code{simd::simd} still is a bit
        awkward.
        \\(standard SIMD vector / \std\code{simd::vec}?)
      \item \std\code{simd_generic} is too long and will be abbreviated with
        different namespace aliases in different code bases\footnote{this is
        normal in other languages, \eg Python}.
    \end{itemize}
\end{description}

\myrating{
  sold; feels good after implementing it; happy about the clear separation of
  scalar / SIMD / \simdgeneric; happy about concise code through namespace
  aliases
}

\subsubsection{On renaming \code{std::simd::simd} to \code{std::simd::vec}}
Personally, I don't think \std\code{simd::simd} is a problem.
Especially, considering that users might introduce a namespace alias or even
--- heaven forbid --- import the whole \std\code{simd} (or
\std\code{simd_generic}) namespace into their local scope.
If \code{vec} needs to stand on its own without the \code{simd::} part of the
name, I fear we might lose clarity compared to \code{simd}.

I believe the situation is different for \std\code{simd::simd_mask}, which,
in my opinion, can live without the \code{simd_} part in its name.
Thus, even after a \code{using namespace std::simd;} the alias template name
\code{mask} is expressive enough.
(Because \code{mask} only appears in proximity to \code{simd} --- if it
appears in code at all.)
