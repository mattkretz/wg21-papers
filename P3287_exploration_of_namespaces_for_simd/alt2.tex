\subsection{Alternative 2: every function is a non-member without \code{simd} prefix}

\medskip\begin{lstlisting}[style=Vc]
template<class V, class G>
  V
  generate(G&& gen);

template<class V = void, class It, class... Flags>
  conditional_t<is_same_v<V, void>, simd<iter_value_t<It>>, V>
  load_from(It first, simd_flags<Flags...> f = {});

template<class Rg, std::integral Idx, class AbiIdx, class... Flags>
  simd<ranges::range_value_t<Rg>, basic_simd<Idx, AbiIdx>::size()>
  gather_from(const Rg&& in, const basic_simd<Idx, AbiIdx>& indexes,
                   simd_flags<Flags...> f = {});

template<size_t SizeSelector = 0, class T, class Abi, class PermuteGenerator>
  simd<T, output-size>
  permute(const basic_simd<T, Abi>& v, PermuteGenerator&& fn);

template<size_t Bytes, class Abi, class T, class U>
  auto
  select(const basic_simd_mask<Bytes, Abi>& c, const T& a, const U& b)
    -> decltype(simd-select-impl(c, a, b));

template<class T, class Abi>
  basic_simd<T, Abi>
  exp(const basic_simd<T, Abi>& x);

template<class T, class Abi>
  basic_simd<T, Abi>
  min(const basic_simd<T, Abi>& x, const basic_simd<T, Abi>& y);

template<size_t Bs, class Abi>
  bool
  all_of(const basic_simd_mask<Bs, Abi>&);

// no way around a prefix:
template<class T>
  concept simd_integral = /*...*/;

template<class T>
  concept simd_generic_integral = integral<T> or simd_integral<T>;
\end{lstlisting}

Usage example:
\medskip\begin{lstlisting}[style=Vc]
void f(std::simd<float> vf, const std::vector<int>& data) {
  auto iota = std::generate<std::simd<int>>([](int i) { return i; });
  auto chunk = std::load_from(data.begin());
  auto chunk_swapped = std::gather_from(data, iota ^ 1);
  auto chunk_swapped2 = std::permute(chunk, [](int i) { return i ^ 1; });
  assert(std::all_of(chunk_swapped == chunk_swapped2));

  vf = std::select(vf > 1.f, 1.f, vf);
  vf = std::exp(vf);
  auto lo = std::min(iota, chunk);
}
\end{lstlisting}

There is little variation possible for the above code.
The most important variation is using unqualified calls, relying on ADL:
\medskip\begin{lstlisting}[style=Vc]
void f(std::simd<float> vf, const std::vector<int>& data) {
  auto iota = std::generate<std::simd<int>>([](int i) { return i; });
  auto chunk = std::load_from(data.begin());
  auto chunk_swapped = gather_from(data, iota ^ 1);
  auto chunk_swapped2 = permute(chunk, [](int i) { return i ^ 1; });
  assert(all_of(chunk_swapped == chunk_swapped2));

  vf = select(vf > 1.f, 1.f, vf);
  vf = exp(vf);
  auto lo = min(iota, chunk);
}
\end{lstlisting}

For \simdgeneric programming the example now looks like this:
\medskip\begin{lstlisting}[style=Vc]
template<std::integral T>
T scalar_only(T a, T b) {
  return 2 * std::min(a, b);
}

template<std::simd_integral T>
T simd_only(T a, T b) {
  return 2 * std::min(a, b);
}

template<std::simd_generic_integral T>
T generic(T a, T b) {
  return 2 * std::min(a, b);
}
\end{lstlisting}

\begin{description}
  \item[pros]
    \begin{itemize}
      \item Consistent.
      \item[$\Rightarrow$] Simple to remember.
      \item \simdgeneric interfaces can easily be provided.
    \end{itemize}

  \item[cons]
    \begin{itemize}
      \item Nothing in e.g. \code{auto x = std::load_from(data.begin())} hints
        at the creation of a \simd object.
      \item Non-\code{simd} overloads for the same names become questionable
        as soon as the functionality isn't equivalent. (huge “name grab”)

      \item If we ever need to disambiguate an inconsistently overloaded term,
        then it will need a \code{simd_} prefix.
        \Eg the \code{simd_integral} concept would be such a term.
        This could be considered less consistent than what we'd like to aim
        for.
    \end{itemize}
\end{description}

\myrating{unacceptable “name grab” and potentially confusing overloads}
