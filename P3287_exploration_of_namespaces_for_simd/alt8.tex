\subsection{Alternative 8: Everything in a single namespace with using-declarations in std}
\label{sec:singlenamespace2}

After LEWG feedback in St. Louis, I added this alternative, which is basically
a combination of Alternatives 5 and 6 (Sections \ref{sec:singlenamespace} and
\ref{sec:alt6}).

\medskip\begin{lstlisting}
namespace std::simd {

template<class T, class Abi = /*...*/>
  class basic_simd;

template<class T, @\simdsizetype@ N = /*...*/>
  using simd = basic_simd<T, @\deducet@<T, N>>;

template<class V, class G>
  V
  generate(G&& gen);

template<class V = void, class It, class... Flags>
  conditional_t<is_same_v<V, void>, simd<iter_value_t<It>>, V>
  load_from(It first, simd_flags<Flags...> f = {});

template<class Rg, std::integral Idx, class AbiIdx, class... Flags>
  simd<ranges::range_value_t<Rg>, basic_simd<Idx, AbiIdx>::size()>
  gather_from(const Rg&& in, const basic_simd<Idx, AbiIdx>& indexes,
                   simd_flags<Flags...> f = {});

template<size_t SizeSelector = 0, class T, class Abi, class PermuteGenerator>
  simd<T, output-size>
  permute(const basic_simd<T, Abi>& v, PermuteGenerator&& fn);

template<size_t Bytes, class Abi, class T, class U>
  auto
  select(const basic_simd_mask<Bytes, Abi>& c, const T& a, const U& b)
    -> decltype(simd-select-impl(c, a, b));

template<class T, class Abi>
  basic_simd<T, Abi>
  exp(const basic_simd<T, Abi>& x);

template<class T, class Abi>
  basic_simd<T, Abi>
  min(const basic_simd<T, Abi>& x, const basic_simd<T, Abi>& y);

template<size_t Bs, class Abi>
  bool
  all_of(const basic_simd_mask<Bs, Abi>&);

template<class T>
  concept integral = /*...*/;

template<class T>
  concept generic_integral = std::integral<T> or std::simd::integral<T>;

}

namespace std {
  using simd::exp;
  using simd::min;
}
\end{lstlisting}

Usage example:
\medskip\begin{lstlisting}[style=Vc]
void f(std::simd::simd<float> vf, const std::vector<int>& data) {
  auto iota = std::simd::generate<std::simd::simd<int>>([](int i) { return i; });
  auto chunk = std::simd::load_from(data.begin());
  auto chunk_swapped = std::simd::gather_from(data, iota ^ 1);
  auto chunk_swapped2 = std::simd::permute(chunk, [](int i) { return i ^ 1; });
  assert(std::simd::all_of(chunk_swapped == chunk_swapped2));

  vf = std::simd::select(vf > 1.f, 1.f, vf);
  vf = std::exp(vf);
  auto lo = std::min(iota, chunk);
}
\end{lstlisting}

Again, this is fairly verbose, so a user might decide to rather rely on ADL:
\medskip\begin{lstlisting}[style=Vc]
void f(std::simd::simd<float> vf, const std::vector<int>& data) {
  auto iota = std::simd::generate<std::simd::simd<int>>([](int i) { return i; });
  auto chunk = std::simd::load_from(data.begin());
  auto chunk_swapped = gather_from(data, iota ^ 1);
  auto chunk_swapped2 = permute(chunk, [](int i) { return i ^ 1; });
  assert(all_of(chunk_swapped == chunk_swapped2));

  vf = select(vf > 1.f, 1.f, vf);
  vf = exp(vf);
  auto lo = min(iota, chunk);
}
\end{lstlisting}

But as we can see, ADL only works for some of the functions.
If the function requires a template argument or none of the arguments are a
\simd / \mask, then the call still must be qualified.
Consequently, if a user wants to reduce the character overhead, a namespace
alias might be better suited:
\medskip\begin{lstlisting}[style=Vc]
namespace smd = std::simd;

void f(smd::simd<float> vf, const std::vector<int>& data) {
  auto iota = smd::generate<smd::simd<int>>([](int i) { return i; });
  auto chunk = smd::load_from(data.begin());
  auto chunk_swapped = smd::gather_from(data, iota ^ 1);
  auto chunk_swapped2 = smd::permute(chunk, [](int i) { return i ^ 1; });
  assert(smd::all_of(chunk_swapped == chunk_swapped2));

  vf = smd::select(vf > 1.f, 1.f, vf);
  vf = smd::exp(vf);
  auto lo = smd::min(iota, chunk);
}
\end{lstlisting}

The \simdgeneric programming example from previous sections now looks like
this:
\medskip\begin{lstlisting}[style=Vc]
template<std::integral T>
T scalar_only(T a, T b) {
  return 2 * std::min(a, b);
}

template<std::simd::integral T>
T simd_only(T a, T b) {
  return 2 * std::simd::min(a, b);
}

template<std::simd::generic_integral T>
T generic(T a, T b) {
  return 2 * std::min(a, b);
}
\end{lstlisting}

This is different to Alternative 7 (\sect{sec:alt7}), where simd-generic
programming requires an opt-in, i.e. a code-change from scalar code.
With this approach existing non-simd code (\code{scalar_only}) can be modified
to use data-parallel types and every function that exists in \code{std} and can
be used in simd-generic code needs no further work (\code{generic}).

\begin{description}
  \item[pros]
    \begin{itemize}
      \item We are free to grab names out of the new namespace.
      \item ADL still works.
      \item Consistent.
      \item[$\Rightarrow$] Users only need to learn: “If it's in the
        \std\code{simd} namespace then it works for \code{simd}s.
        When searching for a function for \code{simd}, look in the
        \std\code{simd} namespace.”
      \item \simdgeneric programming works.
    \end{itemize}

  \item[cons]
    \begin{itemize}
      \item The class template name \std\code{simd::simd} is a bit awkward.
        (There are alternative names that we could adopt instead.)
    \end{itemize}
\end{description}

\myrating{I like it. Clear separation of \code{simd} and non-\code{simd}
functions while still providing good support for \simdgeneric programming}
