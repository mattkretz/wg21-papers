\newcommand\wgTitle{std::simd --- merge data-parallel types from the Parallelism TS 2}
\newcommand\wgName{Matthias Kretz <m.kretz@gsi.de>}
\newcommand\wgDocumentNumber{D1928R4}
\newcommand\wgGroup{LEWG}
\newcommand\wgTarget{\CC{}26}
%\newcommand\wgAcknowledgements{ }

\usepackage{mymacros}
\usepackage{wg21}
\setcounter{tocdepth}{2} % show sections and subsections in TOC
\hypersetup{bookmarksdepth=5}
\usepackage{changelog}
\usepackage{underscore}
\usepackage{comment}

\addbibresource{extra.bib}

\newcommand\simd[1][]{\type{simd#1}\xspace}
\newcommand\simdT{\type{simd<T>}\xspace}
\newcommand\valuetype{\type{value\_type}\xspace}
\newcommand\referencetype{\type{reference}\xspace}
\newcommand\whereexpression{\type{where\_expression}\xspace}
\newcommand\simdcast{\code{simd\_cast}\xspace}
\newcommand\mask[1][]{\type{simd\_mask#1}\xspace}
\newcommand\maskT{\type{simd\_mask<T>}\xspace}
\newcommand\fixedsizeN{\type{simd\_abi::fixed\_size<N>}\xspace}
\newcommand\fixedsizescoped{\type{simd\_abi::fixed\_size}\xspace}
\newcommand\fixedsize{\type{fixed\_size}\xspace}
\newcommand\wglink[1]{\href{https://wg21.link/#1}{#1}}
\DeclareRobustCommand\simdabi{\code{simd\_abi\MayBreak::\MayBreak}}

\renewcommand{\lst}[1]{Listing~\ref{#1}}
\renewcommand{\sect}[1]{Section~\ref{#1}}
\renewcommand{\ttref}[1]{Tony~Table~\ref{#1}}

\begin{document}
\selectlanguage{american}
\begin{wgTitlepage}
  After the Parallelism TS 2 was published in 2018, data-parallel types
  (\simdT) have been implemented and used.
  Now there is sufficient feedback to improve and merge Section 9 of the
  Parallelism TS 2 into the IS working draft.
\end{wgTitlepage}

\pagestyle{scrheadings}

\section{Changelog}
\begin{revision}
\item Target \CC{}26, addressing SG1 and LEWG.
\item Call for a merge of the (improved \& adjusted) TS specification to the IS.
\item Discuss changes to the ABI tags as consequence of TS experience; calls for polls to change the status quo.
\item Add template parameter \code{T} to \code{simd_abi::fixed_size}.
\item Remove \code{simd_abi::compatible}.
\item Add (but ask for removal) \code{simd_abi::abi_stable}.
\item Mention TS implementation in GCC releases.
\item Add more references to related papers.
\item Adjust the clause number for [numbers] to latest draft.
\item Add open question: what is the correct clause for [simd]?
\item Add open question: integration with ranges.
\item Add \code{simd_mask} generator constructor.
\item Consistently add simd and simd_mask to headings.
\item Remove experimental and parallelism_v2 namespaces.
\item Present the wording twice: with and without diff against N4808 (Parallelism TS 2).
\item Default load/store flags to \code{element_aligned}.
\item Generalize casts: conditionally \code{explicit} converting constructors.
\item Remove named cast functions.
\end{revision}

\begin{revision}
\item Add floating-point conversion rank to condition of \code{explicit} for converting constructors.
\item Call out different or equal semantics of the new ABI tags.
\item Update introductory paragraph of \sect{sec:changes}; R1 incorrectly kept the text from R0.
\item Define simd::size as a \code{constexpr} static data-member of type \code{integral_constant<size_t, N>}. This simplifies passing the size via function arguments and still be useable as a constant expression in the function body.
\item Document addition of \code{constexpr} to the API.
\item Add \code{constexpr} to the wording.
\item Removed ABI tag for passing \code{simd} over ABI boundaries.
\item Apply cast interface changes to the wording.
\item Explain the plan: what this paper wants to merge vs. subsequent papers for additional features. With an aim of minimal removal/changes of wording after this paper.
\item Document rationale and design intent for \code{where} replacement.
\end{revision}

\begin{revision}
\item Propose alternative to \code{hmin} and \code{hmax}.
\item Discuss \code{simd_mask} reductions wrt. consistency with \code{<bit>}. Propose better names to avoid ambiguity.
\item Remove \code{some_of}.
\item Add unary \code{\~{}} to \code{simd_mask}.
\item Discuss and ask for confirmation of masked ``overloads'' names and argument order.
\item Resolve inconsistencies wrt. \code{int} and \code{size_t}: Change \code{fixed_size} and \code{resize_simd} NTTPs from \code{int} to \code{size_t} (for consistency).
\item Discuss conversions on loads and stores.
\item Point to \cite{P2509R0} as related paper.
\item Generalize load and store from pointer to \code{contiguous_iterator}. (\sect{sec:contiguousItLoadStore})
\item Moved ``\code{element_reference} is overspecified'' to ``Open questions''.
\end{revision}

\begin{revision}
\item Remove wording diff.
\item Add std::simd to the paper title.
\item Update ranges integration discussion and mention formatting support via
  ranges (\sect{sec:formatting}).
\item Fix: pass iterators by value not const-ref.
\item Add lvalue-ref qualifier to subscript operators (\sect{sec:lvalue-subscript}).
\item Constrain \code{simd} operators: require operator to be well-formed on objects of \code{value_type} (\ref{sec:simd.unary}, \ref{sec:simd.binary}).
\item Rename mask reductions as decided in Issaquah.
\item Remove R3 ABI discussion and add follow-up question.
\item Add open question on first template parameter of \code{simd_mask} (\sect{sec:basicsimdmask}).
\item Overload loads and stores with mask argument (\ref{sec:simd.ctor}, \ref{sec:simd.copy}, \ref{sec:simd.mask.ctor}, \ref{sec:simd.mask.copy}).
\item Respecify \simd reductions to use a \mask argument instead of \code{const_where_expression} (\ref{sec:simd.reductions}).
\item Add \mask operators returning a \simd (\ref{sec:simd.mask.unary}, \ref{sec:simd.mask.conv})
\item Add conditional operator overloads as hidden friends to \simd and \mask
  (\ref{sec:simd.cond}, \ref{sec:simd.mask.cond}).
\item Discuss \std\code{hash} for \simd (\sect{sec:hash}).
\item Constrain some functions (e.g., min, max, clamp) to be \code{totally_ordered} (\ref{sec:simd.reductions}, \ref{sec:simd.alg}).
\item Asking for reconsideration of conversion rules.
\item Rename load/store flags (\sect{sec:renameandextendflags}).
\item Extend load/store flags with a new flag for conversions on load/store. (\sect{sec:renameandextendflags}).
\item Update \code{hmin}/\code{hmax} discussion with more extensive naming discussion (\sect{sec:hminhmax}).
\item Discuss freestanding \simd (\sect{sec:freestanding}).
\item Discuss \code{split} and \code{concat} (\sect{sec:splitandconcat}).
\item Apply the new library specification style from P0788R3.
\end{revision}

\begin{revision}
\item Added \code{simd_select} discussion.
\end{revision}

\begin{revision}
\item Updated the wording for changes discussed in and requested by LEWG in Varna.
\item Rename to \code{simd_cat} and \code{simd_split}.
\item Drop \code{simd_cat(array)} overload.
\item Replace \code{simd_split} by \code{simd_split} as proposed in P1928R4.
\item Use \code{indirectly_writable} instead of \code{output_iterator}.
\item Replace most \code{size_t} and \code{int} uses by \code{\textit{simd-size-type}} signed integer type.
\item Remove everything in \code{simd_abi} and the namespace itself.
\item Reword section on ABI tags using exposition-only ABI tag aliases.
\item Guarantee generator ctor calls callable exactly once per index.
\item Remove \code{int}/\code{unsigned int} exception from conversion rules of broadcast ctor.
\item Rename \code{loadstore_flags} to \code{simd_flags}.
\item Make \code{simd_flags::operator|} \code{consteval}.
\item Remove \code{simd_flags::operator\&} and \code{simd_flags::operator\^}.
\item Increase minimum SIMD width to 64.
\item Rename \code{hmin}/\code{hmax} to \code{reduce_min} and \code{reduce_max}.
\item Refactor \code{simd_mask<T, Abi>} to \code{basic_simd_mask<Bytes, Abi>} and replace all occurrences accordingly.
\item Rename \code{simd<T, Abi>} to \code{basic_simd<Bytes, Abi>} and replace all occurrences accordingly.
\item Remove \code{long double} from the set of vectorizable types.
\item Remove \code{is_abi_tag}, \code{is_simd}, and \code{is_simd_mask} traits.
\item Make \code{simd_size} exposition-only.
\end{revision}

\begin{revision}
\item Remove mask reduction precondition but ask LEWG for reversal of that decision (\sect{sec:removemaskreductionprecondition}).
\item Fix return type of \mask unary operators.
\item Fix \code{bool} overload of \simdselect (\sect{sec:simdselectwording}).
\item Remove unnecessary implementation freedom in \code{simd_split} (\sect{sec:bettersimdsplitwording}).
\item Use \code{class} instead of \code{typename} in template heads.
\item Implement LEWG decision to SFINAE on \emph{values} of
  constexpr-wrapper-like arguments to the broadcast ctor (\ref{sec:simd.ctor}).
\item Add relational operators to \mask as directed by LEWG (\ref{sec:simd.mask.comparison}).
\item Update section on \code{size_t} vs. \code{int} usage (\sect{sec:simdsizetype}).
\item Remove all open design questions, leaving LWG / wording questions.
\item Add LWG question on implementation note (\sect{sec:implnote}).
\item Add constraint for \code{BinaryOperation} to \code{reduce} overloads (\ref{sec:simd.reductions}).
%  \todo Add \code{numeric_limits} / numeric traits specializations since behavior of e.g. \code{simd<float>} and \code{float} may differ for reasonable implementations.
\end{revision}

\begin{revision}
\item Include \code{std::optional} return value from \code{reduce_min_index} and \code{reduce_max_index} in the exploration.
\item Fix \LaTeX{} markup errors.
\item Remove repetitive mention of “exposition-only” before \deducet.
\item Replace “TU” with “translation unit”.
\item Reorder first paragraphs in the wording, especially reducing the note on compiling down to SIMD instructions.
\item Replace cv-unqualified arithmetic types with a more precise list of types.
\item Move the place where “supported” is defined.
\end{revision}

\begin{revision}
\item Improve wording that includes the \CC{}23 extended floating-point types in the set of vectorizable types (\ref{wording.vectorizable.types}).
\item Improve wording that defines “selected indices” and “selected elements” (\ref{wording.selected.indices}).
\item Remove superfluous introduction paragraph.
\item Improve wording introducing the intent of ABI tags (\ref{wording.ABI.tag})
\item Consistently use \code{size} as a callable in the wording.
\item Add missing \code{type_identity_t} for \code{reduce} (\ref{sec:simd.syn}, \ref{sec:simd.reductions}).
\item Spell out “iff” (\ref{wording.deducet}).
\item Fixed template argument to \nativeabi\ in the default template argument of \code{basic_simd_mask} (\ref{sec:simd.syn}).
\item Fixed default template argument to \code{simd_mask} to be consistent with \code{simd} (\ref{sec:simd.syn}).
\item Add instructions to add \code{<simd>} to the table of headers in [headers].
\item Add instructions to add a new subclause to the table in [numerics.general].
\item Add instructions to add \code{<simd>} [diff.23.library].
\item Add \simdsizev to the wording and replace \code{simd_size_v} to actually implement “Make \code{simd_size} exposition-only.”
\item Restored precondition (and removed \code{noexcept}) on
  \code{reduce_min_index} and \code{reduce_max_index} as directed by LEWG.
\end{revision}

\begin{revision}
\item Strike through wording removed by P3275 (non-const \code{operator[]}).
\item Remove “exposition only” from detailed prose, it's already marked as such in the synopsis.
\item Reorder defintion of \emph{vectorizable type} above its first use.
\item Commas, de-duplication, word order, \code{s/may/can/} in a note.
\item Use text font for “[)” when defining a range of integers.
\item Several small changes from LWG review on 2024-06-26.
\item Reword \code{rebind_simd} and \code{resize_simd}.
\item Remove mention of implementation-defined load/store flags.
\item Remove paragraph about default initialization of \simd.
\item Reword all constructor \emph{Effects} from “Constructs an object \ldots”
  to “Initializes \ldots”.
\item Instead of writing “satisfies X” in \emph{Constraints} and “models X” in
  \emph{Preconditions}, say only “models X” in \emph{Constraints}.
\item Replace \code{is_trivial_v} with “is trivially copyable”.
\item First shot at improving generator function constraints.
\item Reword constraints on unary and binary operators.
\item Add missing/inconsistent \code{explicit} on load constructors.
\item Fix preconditions of subscript operators.
\item Reword effects of compound assignment operators.
\item Add that \code{BinaryOperation} may not modify input \simd.
\item Fix definition of GENERALIZED_SUMs.
\end{revision}

\begin{revision}
\item Say “\textit{op}” instead of “the indicated operator”
\item Fix constraints on shift operators with \simdsizetype{} on the right operand.
\item Remove wording removed by P3275 (non-const \code{operator[]}).
\item Make intrinsics conversion recommended practice.
\item Make \code{simd_flags} template arguments exposition-only.
\item Make \code{simd_alignment} \emph{not} implementation-defined.
\item Reword “supported” to “enabled or disabled”.
\item Apply improved wording from \ref{sec:simd.overview} to \ref{sec:simd.mask.overview}.
\item Add comments for LWG to address to broadcast ctor (\ref{sec:simd.ctor}).
\item Respecify generator ctor to not reuse broadcast constraint (\ref{sec:simd.ctor}).
\item Use \code{to_address} on contiguous iterators (\ref{sec:simd.ctor} and \ref{sec:simd.copy}).
  This is more explicit about allowing memcpy on the complete range rather than
  having to iterate the range per element.
\end{revision}

\begin{revision}
\item Fix default size of \code{simd} and \code{simd_mask} aliases
  (\ref{sec:simd.syn}, necessary for
  \std\code{destructible<\MayBreak{}\std{}simd<\MayBreak\std{}string>>} to be well-formed).
\item Extend value-preserving to encompass conversions from all arithmetic
  types. Use this new freedom in \ref{sec:simd.ctor} to fully constrain the
  generator constructor and to plug a specification hole in the broadcast
  constructor.
\item Fix broadcast constructor wording by constraining \constexprwrapperlike
  arguments to arithmetic types.
  %\todo Reorder \code{simd} and \code{simd_mask} specification in the wording (mask first).
\end{revision}

\section{Straw Polls}


\section{Introduction}
\cite{P0214R9} introduced \simdT and related types and functions into the Parallelism TS 2 Section 9.
The TS was published in 2018.
An incomplete and non-conforming (because P0214 evolved) implementation existed for the whole time P0214 progressed through the committee.
Shortly after the GCC 9 release, a complete implementation of Section 9 of the TS was made available.
Since GCC 11 a complete \code{simd} implementation of the TS is part of its standard library.

In the meantime the TS feedback progressed to a point where a merge should happen ASAP.
This paper proposes to merge only the feature-set that is present in the Parallelism TS 2.
(Note: The first revision of this paper did not propose a merge.)
If, due to feedback, any of these features require a change, then this paper (P1928) is the intended vehicle.
If a new feature is basically an addition to the wording proposed here, then it will progress in its own paper.

\subsection{Related papers}
\begin{description}
  \item[\wglink{P0350}] Before publication of the TS, SG1 approved \cite{P0350R0} which did not progress in time in LEWG to make it into the TS.
    \wglink{P0350} is moving forward independently.

  \item[\wglink{P0918}] After publication of the TS, SG1 approved \cite{P0918R2} which adds \code{shuffle}, \code{interleave}, \code{sum_to}, \code{multiply_sum_to}, and \code{saturated_simd_cast}.
    \wglink{P0918} will move forward independently.

  \item[\wglink{P1068}] R3 of the paper removed discussion/proposal of a \code{simd} based API because it was targeting \CC{}23 with the understanding of \code{simd} not being ready for \CC{}23.
    This is unfortunate as the presence of \code{simd} in the IS might lead to a considerably different assessment of the iterator/range-based API proposed in P1068.

  \item[\wglink{P0917}] The ability to write code that is generic wrt. arithmetic types and \code{simd} types is considered to be of high value (TS feedback).
    Conditional expressions via the \code{where} function were not all too well received.
    Conditional expressions via the conditional operator would provide a solution deemed perfect by those giving feedback (myself included).

  \item[draft on non-member {operator[]}] TODO

  \item[\wglink{P2600}] The fix for ADL is important to ensure the above two papers do not break existing code.

  \item[\wglink{P0543}] The paper proposing functions for saturation arithmetic expects \code{simd} overloads as soon as \code{simd} is merged to the IS.

  \item[\wglink{P0553}] The bit operations that are part of \CC{}20 expects \code{simd} overloads as soon as \code{simd} is merged to the IS.

  \item[\wglink{P2638}] Intel’s response to \wglink{P1915R0} for \code{std::simd}

  \item[\wglink{P2663}] \code{std::simd<std::complex<T>>}.

  \item[\wglink{P2664}] Permutations for \code{simd}.

  \item[\wglink{P2509}] \textcite{P2509R0} proposes a ``type trait to detect
    conversions between arithmetic-like types that always preserve the numeric
    value of the source object''. This matches the \textit{value-preserving}
    conversions the \code{simd} specification uses.
\end{description}
The papers \wglink{P0350}, \wglink{P0918}, \wglink{P2663}, \wglink{P2664}, and
the \code{simd}-based \wglink{P1068} fork currently have no shipping vehicle
and are basically blocked on this paper.

\section{Changes after TS feedback}\label{sec:changes}
\cite{P1915R0} (Expected Feedback from \code{simd} in the Parallelism TS 2) was published in 2019, asking for feedback to the TS.
I received feedback on the TS via the GitHub issue tracker, e-mails, and personal conversations.
There is also a lot of valuable feedback published in \wglink{P2638} ``Intel’s response to \wglink{P1915R0} for \code{std::simd}''.
This paper captures the major change requests but should still be considered a work-in-progress.

\subsection{improve ABI tags}
Summary:
\begin{itemize}
  \item Change the default SIMD ABI tag to \simdabi\code{native<T>} instead of \simdabi\code{compatible<T>}.
  \item Change \simdabi\code{fixed_size} to not recommend implementations make it ABI compatible.
\end{itemize}

For a discussion, see \wglink{P1928R3} Section 4.1.

Follow-up open questions are discussed in \sect{sec:simplifyfixedsize}.

\subsection{Simplify/generalize casts}\label{sec:casts}

The change to the ABI tags requires a reconsideration of cast functions and
implicit and explicit casts between data-parallel types of different ABI tags.
This is in addition to TS feedback on casts being too strict or cumbersome to use.

\subsubsection{More (explicit) converting constructors}

The TS allows implicit casts between \code{fixed_size<N>} types that only
differ in element type and where the values are preserved (“every possible
value of \code{U} can be represented with type \code{value_type}”).

However, from experience with the TS, it is better to also enable implicit
conversions between any \code{simd} specializations with equal element count,
even if such a conversion might be non-portable between targets with different
native SIMD widths.
The expectation is, that users set up their types according to a pattern
similar to \lst{lst:simdtypespattern}.
\begin{lstlisting}[numbers=left,float={hbtp},label=lst:simdtypespattern,caption={
  Recommended setup of \code{simd} types
}]
using  floatv = std::simd<float>;
using doublev = std::rebind_simd_t<double, floatv>;
using  int32v = std::rebind_simd_t<std:: int32_t, floatv>;
using uint32v = std::rebind_simd_t<std::uint32_t, floatv>;
using  int16v = std::rebind_simd_t<std:: int16_t, floatv>;
using uint16v = std::rebind_simd_t<std::uint16_t, floatv>;
// ...
\end{lstlisting}
Thus, users will work with a set of types that have equal number of elements by
construction.
Some of the types may use the \code{fixed_size} ABI tag and some may use an
extended ABI tag.
This detail should not stop the user from being able to cast between a
compiler-flag dependent subset of these types.

Besides a constraint on the number of elements being equal, the converting
constructor should be conditionally \code{explicit}:
Implicit casts are only allowed if the element type conversion is
value-preserving (same wording as in the TS).

This resolves major inconveniences when working with mixed-precision
operations (cf. \ttref{tt:better with conv ctors}).
\begin{tonytable}[Parallelism TS 2]{Improved generic code after adding converting constructors}\label{tt:better with conv ctors}
  \begin{lstlisting}
namespace stdx = std::experimental;

template <class T> void f(T a, int b)
{
  using I = std::conditional_t<
              stdx::is_simd_v<T>,
              stdx::rebind_simd_t<int, T>, int>;
  I c;
  if constexpr (stdx::is_simd_v<T>) {
    c = stdx::static_simd_cast<I>(a) + b;
  } else {
    c = static_cast<int>(a) + b;
  }
  g(c);
}
  \end{lstlisting}
  &
  \begin{lstlisting}
// assuming simd in namespace std

template <class T> void f(T a, int b)
{
  using I = std::conditional_t<
              std::is_simd_v<T>,
              std::rebind_simd_t<int, T>, int>;
  I c = static_cast<I>(a) + b;





  g(c);
}
  \end{lstlisting}
\end{tonytable}%
Type conversions for \code{simd} are still less error-prone than builtin types,
because conversions that might lose information require an explicit cast.
Also, unintended widening of the SIMD register size can happen, but typically
leads to the need for an explicit cast in the complete statement (cf.
\lst{lst:mixedprecision}).

\begin{lstlisting}[numbers=left,float={hbtp},label=lst:mixedprecision,caption={
  Mixed precision code using the types from \lst{lst:simdtypespattern}, ensuring equal element count
}]
void f(int32v a, doublev b, floatv c)
{
  doublev x = a * b + c; // OK: implicit (value-preserving) conversion from int and float
    // to double. Requires twice the register space, but there's no way around it and the
    // result type requires it anyway.
  int32v y = a * b; // ERROR: implicit conversion from double to int not value-preserving
  int32v z1 = static_cast<int32v>(a * b); // OK: cast hints at implicit register widening
  int32v z2 = a * static_cast<int32v>(b); // OK
}
\end{lstlisting}

\subsubsection{Remove named cast functions}

From the cast functions \stdx\code{to_fixed_size}, \stdx\code{to_native}, and
\stdx\code{to_compatible} only the conversions from \simdabi\code{fixed_size<T,
N>} to \simdabi\code{native<T>} and back may still benefit from a named cast
function.
Most importantly, the conversion from \code{native} to its \code{fixed_size}
counterpart benefits from a cast expression that does not require spelling out
the destination type.
However, since converting constructors are provided by the standard library, it
is simple for users to define their own \code{to_fixed_size} function if they
want one (e.g.
\begin{lstlisting}[numbers=left,float={hbtp},label=lst:userdefined-to-fixed-size,caption={
  Example of a user-defined \code{to_fixed_size} implementation if explicit casts are provided
}]
template <typename T>
constexpr std::fixed_size_simd<T, std::simd_size_v<T>> to_fixed_size(std::simd<T> x)
{
  return x;
}
\end{lstlisting}
\lst{lst:userdefined-to-fixed-size}).
The reverse cast can trivially be spelled out as \code{static_cast<simd<T>>(y)}
in program code.
The only motivation for adding a \code{to_native} function would be the
provision of a counterpart for the \code{to_fixed_size} cast function.

Besides the functions only implementing trivial implicit casts, there is little
to no need for these functions.
The named cast functions are therefore removed altogether.

\subsubsection{Remove \code{simd_cast} and \code{static_simd_cast}}
There are two cast function templates in the TS: \code{simd_cast} and
\code{static_simd_cast}.
The former is equivalent to the latter except that only value-preserving
conversions are allowed.
The template parameter can either be a \code{simd} specialization or a
vectorizable type \code{T}.
In the latter case, the cast function determines the return type as
\code{fixed_size_simd<T, input.size()>}.

Since we allow all conversions covered by \stdx\code{simd_cast} and
\stdx\code{static_simd_cast} via \std\code{simd} constructors, the cast
functions can be removed altogether.
The lost feature (cast via element type) can be replaced using
\code{rebind_simd} as shown in \ttref{tt:tsvsp1928casts}.
\begin{tonytable}[Parallelism TS 2]{Casting without specifying the target ABI tag}\label{tt:tsvsp1928casts}
  \begin{lstlisting}
template <typename V>
void f(V x)
{
  const auto y = stdx::static_simd_cast<double>(x);
  // ...
}
  \end{lstlisting}
  &
  \begin{lstlisting}
template <typename V>
void f(V x)
{
  const auto y = std::rebind_simd_t<double, V>(x);
  // ...
}
  \end{lstlisting}
\end{tonytable}%

\subsubsection{mask casts}
\code{simd_mask} casts should work when \code{simd} casts work.
I.e. if \code{simd<T0, A0>} is implicitly convertible to \code{simd<T1, A1>}
then \code{simd_mask<T0, A0>} is implicitly convertible to \code{simd_mask<T1,
A1>}.
The reverse (if \code{simd_mask} is convertible then \code{simd} is
convertible) does not have to be true.
Specifically, the TS allows all \code{fixed_size<N>} mask to be
interconvertible, irrespective of the element type.
For the IS merge, the proposal is to make this more consistent with \code{simd}
while also preserving most of the convenience:
Allow implicit conversions if the \code{sizeof} the element types are equal,
otherwise the conversion must be explicit.

Conversions with different element count are not possible via a constructor
(consistent with \code{simd}).
This would require a different function, such as the \code{resize<N>(simd)}
function proposed by \textcite{P2638R0}.

\subsubsection{Complete casts for \code{simd_mask}}
The \code{simd_cast} and \code{static_simd_cast} overloads for \code{simd_mask} were forgotten for the TS.
Without those casts (and no casts via constructors) mixing different arithmetic types is painful.
There is no motivation for forbidding casts on \code{simd_mask}.

The proposed changes for casts solve this issue.

\subsubsection{Summary of casts}

\begin{enumerate}
  \item \code{simd<T0, A0>} is convertible to \code{simd<T1, A1>} if
    \code{simd_size_v<T0, A0> == simd_size_v<T1, A1>}.

  \item \code{simd<T0, A0>} is implicitly convertible to \code{simd<T1, A1>}
    if, additionally, the conversion \code{T0} to \code{T1} is
    value-preserving.

  \item \code{simd_mask<T0, A0>} is convertible to \code{simd_mask<T1, A1>} if
    \code{simd_size_v<T0, A0> == simd_size_v<T1, A1>}.

  \item \code{simd_mask<T0, A0>} is implicitly convertible to
    \code{simd_mask<T1, A1>} if, additionally, \code{sizeof(T0) ==
    sizeof(T1)}.

  \item \code{simd<T0, A0>} can be \code{bit_cast}ed to \code{simd<T1, A1>} if
    \code{sizeof(simd<T0, A0>) == sizeof(simd<T1, A1>)}.

  \item \code{simd_mask<T0, A0>} can be \code{bit_cast}ed to \code{simd_mask<T1, A1>} if
    \code{sizeof(simd_mask<T0, A0>) == sizeof(simd_mask<T1, A1>)}.
\end{enumerate}

\subsection{Add \code{simd_mask} generator constructor}
The \code{simd} generator constructor is very useful for initializing objects
from scalars in a portable (i.e. different \code{simd::size()}) fashion.
The need for a similar constructor for \code{simd_mask} is less frequent, but,
even if only for consistency, there should be one.
Besides consistency, it is also useful, of course.
Consider a predicate function that is given without \code{simd} interface (e.g. from a library).
How do you construct a \code{simd_mask} from it?
With a generator constructor it is easy:
\medskip\begin{lstlisting}[style=Vc]
simd<T> f(simd<T> x, Predicate p) {
  const simd_mask<T> k([&](auto i) { return p(x[i]); });
  where(k, x) = 0;
  return x;
}
\end{lstlisting}
Without the generator constructor one has to write e.g.:
\medskip\begin{lstlisting}[style=Vc]
simd<T> f(simd<T> x, Predicate p) {
  simd_mask<T> k;
  for (size_t i = 0; i < simd<T>::size(); ++i) {
    k[i] = p(x[i]);
  }
  where(k, x) = 0;
  return x;
}
\end{lstlisting}
The latter solution makes it hard to initialize the \code{simd_mask} as \code{const}, is more verbose, is harder to optimize, and cannot use the sequencing properties the generator constructor allows.

Therefore add:
\begin{wgText}
\begin{itemdecl}
template<class G> simd_mask(G&& gen) noexcept;
\end{itemdecl}
\end{wgText}

\subsection{Default load/store flags to \code{element_aligned}}

Consider:
\medskip\begin{lstlisting}[style=Vc,numbers=left]
std::simd<float> v(addr, std::vector_aligned); @\label{lstline:vector_aligned}@
v.copy_from(addr + 1, std::element_aligned); @\label{lstline:load element_aligned}@
v.copy_to(dest, std::element_aligned); @\label{lstline:store element_aligned}@
\end{lstlisting}
Line~\ref{lstline:vector_aligned} supplies an optimization hint to the load operation.
Line~\ref{lstline:load element_aligned} says what really?
“Please don't crash.
I know this is not a vector aligned access\footnote{Of course, vector aligned is equivalent to element aligned if \code{simd<float>::size() == 1}}.”
Line~\ref{lstline:store element_aligned} says:
“I don't know whether it's vector aligned or not.
Compiler, if you know more, please optimize, otherwise just don't make it crash.”
(To clarify, the difference between lines~\ref{lstline:load element_aligned} and~\ref{lstline:store element_aligned} is what line~\ref{lstline:vector_aligned} says about the alignment of \code{addr}.)
In both cases of \code{element_aligned} access, the developer requested a behavior we take as given in all other situations.
Why does the TS force to spell it out in this case?

Since \CC{}20, we also have another option:
\medskip\begin{lstlisting}[numbers=left]
std::simd<float> v(std::assume_aligned<std::memory_alignment_v<std::simd<float>>>(addr)); @\label{lstline:assume_aligned}@
v.copy_from(addr + 1);
v.copy_to(dest);
\end{lstlisting}
This seems to compose well, except that line \ref{lstline:assume_aligned} is rather long for a common pattern in this interface.
Also, this removes implementation freedom because the library cannot statically determine the alignment properties of the pointer.

Consequently, as a minimal improvement to the TS keep the load/store flags as
is, but default them to \code{element_aligned}.
I.e.:
\medskip\begin{lstlisting}[numbers=left]
std::simd<float> v(addr, std::vector_aligned);
v.copy_from(addr + 1);
v.copy_to(dest);
\end{lstlisting}

\sect{sec:convertingLoadsAndStores} discusses an option for additional flags.

\subsection{Contiguous iterators for loads and stores}\label{sec:contiguousItLoadStore}

After Ranges and Concepts introduced \code{std::contiguous_iterator}, the
load/store interface for \code{simd} can easily be generalized from \code{U*}
to \code{std::contiguous_iterator} with additional constraints for
\code{input_iterator}/\code{output_iterator} and \code{iter_value_t}.
This was not a possible design choice for the TS but does make a lot of sense
to modernize with the merge.
Therefore, the merge generalizes the load/store interfaces to look like
\lst{lst:copyImpl}.
\begin{lstlisting}[numbers=left,float={hbtp},label=lst:copyImpl,caption={
    \code{copy_from} and \code{copy_to} declarations using \code{contiguous_iterator}
}]
template <contiguous_iterator It, typename Flags = element_aligned_tag>
requires detail::vectorizable<iter_value_t<It>>
constexpr void copy_from(It first, Flags f = {});

template <contiguous_iterator It, typename Flags = element_aligned_tag>
requires output_iterator<It, Tp> && detail::vectorizable<iter_value_t<It>>
constexpr void copy_to(It first, Flags f = {}) const;
\end{lstlisting}

\subsection{\code{constexpr} everything}
The libstdc++ implementation implements the complete TS API as \code{constexpr} as an optional extension.
This is useful (e.g. for computing constants) and not a significant implementation burden.
Users (as well as \textcite{P2638R0}) have called for \code{constexpr}.
The merge consequently adds \code{constexpr} to all functions.

\subsection{Specify \code{simd::size} as \code{integral_constant}}
The TS specifies \code{simd::size} as a \code{static constexpr} function returning the number of elements of the \code{simd} specialization.
Instead of a function, this paper uses a static data member of type \code{std::integral_constant<std::size_t, N>}, which is both convertible to \code{std::size_t} and callable.
The upside of using a static data member is that it can be used as function parameter without conversion to integer and thus easily pass the size into a function as constant expression.
See \lst{lst:sizeparam} for an example.
\begin{lstlisting}[numbers=left,float={hbtp},label=lst:sizeparam,caption={
    Example: Pass \code{simd::size} as ``constant expression function argument''
}]
template <std::ranges::contiguous_range R, std::size_t Size>
std::span<const std::ranges::range_value_t<R>, Size>
auto subscript(const R& r, std::size_t first, std::integral_constant<std::size_t, Size>) {
  return std::span<const std::ranges::range_value_t<R>, Size>(
    std::ranges::data(r) + first, Size());
}

void g(std::vector<float> data) {
  std::simd<float> v;
  for (std::size_t i = 0; i + v.size < data.size(); i += v.size) {
    v = subscript(data, i, v.size);  // simd::simd(span) to be proposed
    // ...
  }
}
\end{lstlisting}


\subsection{Replace \code{where} facilities}

The \code{where} functions and corresponding \code{where_expression} have been
the most controversial part going into the TS.
My interpretation of the feedback I received is that users can work with it but
do not find it intuitive.
Instead, many have asked for a blend / select / conditional operator instead.
Whenever I asked users whether they would like to use the \code{?:} operator I
got positive and often enthusiastic responses.
An overloaded \code{operator?:} would open the door to generic and intuitive
SIMD code.

A major motivation for the \code{where} function in the TS was its ability to
express masked \emph{operations} in addition to masked assignments.
This enables library implementations to explicitly use masked operation
intrinsics instead of resorting to an unmasked operation with subsequent masked
assignment.
The latter can be contracted to a masked operation by compilers, but obviously
there's no guarantee.
In any case, the topic is a QoI issue that doesn't have to dictate the API.

If \code{operator?:} had been overloadable when I designed \stdx\simd{} then I
would have propopsed \code{?:} overloads for \code{simd_mask} and \code{simd}.
Consequently, \code{where} would likely not have existed.
Sadly we still cannot overload \code{operator?:} even though there has been
positive feedback in EWG-I.
That work is currently blocked on \cite{P2600R0}.

\subsubsection{Proposed replacements for \code{where}}

This paper proposes the following replacements for \stdx\code{where}:

\begin{itemize}
  \item Overloads for \code{simd\MayBreak{}::\MayBreak{}copy_from},
    \code{simd\MayBreak{}::\MayBreak{}copy_to},
    \code{simd_mask\MayBreak{}::\MayBreak{}copy_from},
    \code{simd_mask\MayBreak{}::\MayBreak{}copy_to}, \code{reduce},
    \code{hmin}, and \code{hmax} with additional \code{simd_mask} parameter.

    There are still open questions on these functions, discussed in \sect{sec:maskedOverloads}.

  \item hidden friend \code{operator?:}\footnote{as soon as EWG lifts the
    restriction\ldots} / \code{conditional_operator} functions in \code{simd}
    and \code{simd_mask}:
    \begin{itemize}
      \item \lstinline@simd simd::operator?:(mask_type, simd, simd)@
      \item \lstinline@template <class U1, class U2>@\\
        \lstinline@requires convertible_to<simd_mask, rebind_simd_t<common_type_t<U1, U2>, simd_mask>@\\
        \lstinline@friend constexpr rebind_simd_t<common_type_t<U1, U2>, simd_type>@\\
        \lstinline@simd_mask::operator?:(simd_mask, U1, U2)@
      \item \lstinline@simd_mask simd_mask::operator?:(simd_mask, simd_mask, simd_mask)@
      \item \lstinline@simd_mask simd_mask::operator?:(simd_mask<K, KAbi>, simd_mask, simd_mask<U, UAbi>)@\\
        (for disambiguation of the above because \code{simd_mask}s can be interconvertible)
      \item \lstinline@simd_mask simd_mask::operator?:(simd_mask, bool, bool)@\\
        (for consistency; it's not very useful)
    \end{itemize}

  \item facilities for converting \code{simd_mask<T>} to \code{simd<T>} with
    values \code{0} or \code{1}:
    \begin{itemize}
      \item \code{simd_mask::operator simd_type} (not \code{explicit},
        preferably with 4.1 of \cite{P2600R0} adopted)
      \item unary \code{simd_mask::operator+}; equivalent to \code{+(operator simd_type())}
      \item unary \code{simd_mask::operator-}; equivalent to \code{-(operator simd_type())}
      \item unary \code{simd_mask::operator\~{}}; equivalent to \code{\~{}(operator simd_type())}
    \end{itemize}
\end{itemize}

(Wording for the above is still TBD.)

\subsubsection{Examples}

\lst{lst:simdconditionals} presents a few simple examples of working with a
\code{simd_mask} result in the absence of \code{where}.
Note that the compiler I used implements the ADL fix proposed in \cite{P2600R0}
and implements \code{opertaor?:} overloading as explored in \cite{D0917}.

\begin{lstlisting}[numbers=left,float={hbtp},label=lst:simdconditionals,caption={
    \code{simd} conditionals without \code{where} and with \cite{P2600R0} and \cite{D0917}, showing the corresponding assembly output (\code{gcc -O2 -std=c++23 -march=skylake-avx512}; personal GCC 12.1 branch with patches implementing \cite{P2600R0} and \cite{D0917})
}]
auto f0(std::simd<int> x) { return x > 0 ? 2 * x : x; }
/*	vpxor	xmm1, xmm1, xmm1
	vpcmpd	k1, zmm1, zmm0, 1
	vpslld	zmm0{k1}, zmm0, 1
	ret
*/
auto f1(std::simd<int> x) { return x > 0 ? 1 : 0; }
/*	vmovdqa32	zmm1, zmm0
	vpxor		xmm0, xmm0, xmm0
	vpcmpd		k1, zmm0, zmm1, 1
	mov		eax, 1
	vpbroadcastd	zmm0{k1}{z}, eax
	ret
*/
auto f2(std::simd<int> x) { return std::simd(x > 0); }
/*	vmovdqa32	zmm1, zmm0
	vpxor		xmm0, xmm0, xmm0
	vpcmpd		k1, zmm0, zmm1, 1
	mov		eax, 1
	vpbroadcastd	zmm0{k1}{z}, eax
	ret
*/
auto f3(std::simd<int> x) { return -(x > 0); }
/*	vmovdqa32	zmm1, zmm0
	vpxor		xmm0, xmm0, xmm0
	vpcmpd		k0, zmm0, zmm1, 1
	vpmovm2d	zmm0, k0
	ret
*/
auto f4(std::simd<int> x) { return x > 0 ? -1 : 0; }
/*	vmovdqa32	zmm1, zmm0
	vpxor		xmm0, xmm0, xmm0
	vpcmpd		k0, zmm0, zmm1, 1
	vpmovm2d	zmm0, k0
	ret
*/
auto f5(std::simd<int> x) { return x > 0 ? true : false; }
/*	vmovdqa32	zmm1, zmm0
	vpxor		xmm0, xmm0, xmm0
	vpcmpd		k0, zmm0, zmm1, 1
	kmovw		eax, k0
	ret
*/
\end{lstlisting}

\begin{itemize}
  \item The function \code{f0} scales all positive values in \code{x} by 2.
    The compiler contracts the blending of \code{2 * x} and \code{x} with the
    multiply operation (a left shift by 1) to a masked left shift instruction.

  \item The functions \code{f1} and \code{f2} both return a \code{simd<int>}
    where all positive entries of \code{x} are replaced by 1 and the remaining
    entries are 0.
    I.e. converting the comparison result to \code{simd} works analogue to
    promotion of \bool to \intt.

  \item The functions \code{f3} and \code{f4} both return a \code{simd<int>}
    where all positive entries of \code{x} are replaced by -1 and the remaining
    entries are 0.
    The ISA allows a more efficient translation and the compiler recognizes the
    pattern in both variants.

  \item Finally, to complete the set, \code{f5} shows how one could even blend
    \code{bool} arguments into a \code{simd_mask}.
    The compiler recognizes that the conditional operator is a no-op and simply
    returns the result of the comparison itself.
\end{itemize}

\begin{tonytable}[Parallelism TS 2]{Counting positive values in a \std\code{vector}}\label{tt:countingexample}
  \begin{lstlisting}
namespace stdx = std::experimental;
int count_positive(
  const std::vector<stdx::native_simd<float>>& x)
{
  // simplify generated assembly:
  if (x.size() == 0) std::unreachable();
  using floatv = stdx::native_simd<float>;
  using intv = stdx::rebind_simd_t<int, floatv>;
  intv counter = {};
  for (stdx::simd v : x) {
    auto k = stdx::static_simd_cast<
               intv::mask_type>(v > 0);
    ++where(k, counter);
  }
  return reduce(counter);
}
/*	mov	edx, 1
	mov	rcx, QWORD PTR [rdi+8]
	mov	rax, QWORD PTR [rdi]
	vpxor	xmm0, xmm0, xmm0
	vxorps	xmm2, xmm2, xmm2
	vpbroadcastd	zmm1, edx
.L12:
	vcmpps	k1, zmm2, ZMMWORD PTR [rax], 1
	add	rax, 64

	vpaddd	zmm0{k1}, zmm0, zmm1
	cmp	rcx, rax
	jne	.L12
	vmovdqa	ymm1, ymm0
	vextracti64x4	ymm0, zmm0, 0x1
	vpaddd	ymm0, ymm1, ymm0
	vmovdqa	xmm1, xmm0
	vextracti64x2	xmm0, ymm0, 0x1
	vpaddd	xmm1, xmm1, xmm0
	vpshufd	xmm0, xmm1, 27
	vpaddd	xmm0, xmm0, xmm1
	vpunpckhqdq	xmm1, xmm0, xmm0
	vpaddd	xmm0, xmm0, xmm1
	vmovd	eax, xmm0
	vzeroupper
	ret
*/
  \end{lstlisting}
  &
  \begin{lstlisting}

int count_positive(
  const std::vector<std::simd<float>>& x)
{
  // simplify generated assembly:
  if (x.size() == 0) std::unreachable();
  using floatv = std::simd<float>;
  using intv = std::rebind_simd_t<int, floatv>;
  intv counter = {};
  for (std::simd v : x) {
    counter += v > 0;


  }
  return reduce(counter);
}
/*	mov	edx, 1
	mov	rcx, QWORD PTR [rdi+8]
	mov	rax, QWORD PTR [rdi]
	vpxor	xmm0, xmm0, xmm0
	vxorps	xmm3, xmm3, xmm3
	vpbroadcastd	zmm2, edx
.L9:
	vcmpps	k1, zmm3, ZMMWORD PTR [rax], 1
	add	rax, 64
	vmovdqa32	zmm1{k1}{z}, zmm2
	vpaddd	zmm0, zmm0, zmm1
	cmp	rcx, rax
	jne	.L9
	vmovdqa	ymm1, ymm0
	vextracti64x4	ymm0, zmm0, 0x1
	vpaddd	ymm0, ymm1, ymm0
	vmovdqa	xmm1, xmm0
	vextracti64x2	xmm0, ymm0, 0x1
	vpaddd	xmm1, xmm1, xmm0
	vpshufd	xmm0, xmm1, 27
	vpaddd	xmm0, xmm0, xmm1
	vpunpckhqdq	xmm1, xmm0, xmm0
	vpaddd	xmm0, xmm0, xmm1
	vmovd	eax, xmm0
	vzeroupper
	ret
*/
  \end{lstlisting}
\end{tonytable}%

\ttref{tt:countingexample} presents an algorithm for counting all positive
\code{float} values in a \code{std::vector}.
For simplicity, the code uses \code{vector<simd<float>>} and assumes the \code{vector}
is not empty.
If a \code{simd_mask} implicitly converts to a \code{simd} (as proposed and
analogue to \code{bool}), the code is simplified significantly.
However, at this point, the TS implementation compiles to a masked add
instruction while the implementation for this paper does not.
The difference is that the former executes an unmasked addition followed up by
a masked assignment while the latter converts the mask into a \code{simd} of 1s
and 0s followed up by an unmasked addition.
The compiler needs to recognize this pattern in order to reach the same
performance (QoI).

\subsection{Make use of \code{int} and \code{size_t} consistent}

The TS uses \code{int} as NTTP for
\begin{itemize}
  \item \stdx\simdabi\code{fixed_size},
  \item \stdx\code{fixed_size_simd},
  \item \stdx\code{fixed_size_simd_mask}, and
  \item \stdx\code{resize_simd}.
\end{itemize}
The constant \stdx\simdabi\code{max_fixed_size} is of type \code{int}.
The TS uses \code{size_t} as NTTP for
\begin{itemize}
  \item \code{split}, and
  \item \code{split_by}.
\end{itemize}
This paper uses \code{integral_constant<size_t, }\VSize{}\code{>} for
\begin{itemize}
  \item \code{simd_size<T, Abi>},
  \item \code{simd<T, Abi>::size}, and
  \item \code{simd_mask<T, Abi>::size}.
\end{itemize}
Finally, \code{simd_size_v<T, Abi>} is of type \code{size_t}.

All of these integers denote a SIMD width.
They should be consistent.
Since the \code{size} member will never get concensus to use type \code{int},
the decision falls on \code{size_t} for all.

The merge proposal therefore uses \code{size_t} for
\stdx\simdabi\code{fixed_size}, \stdx\code{fixed_size_simd},
\stdx\code{fixed_size_simd_mask}, \stdx\code{resize_simd}, and
\stdx\simdabi\code{max_fixed_size}.

\subsection{Clean up math function overloads}
The wording that produces \code{simd} overloads misses a few cases and leaves room for ambiguity.
There is also no explicit mention of integral overloads that are supported in \code{<cmath>} (e.g. \code{std::cos(1)} calling \code{std::cos(double)}).
At the very least, \code{std::abs(simd<\textit{signed-integral}>)} should be specified.

Also, from implementation experience, ``undefined behavior'' for domain, pole,
or range error is unnecessary.
It could either be an unspecified result or even match the expected result of
the function according to Annex F in the C standard.
The latter could possibly be a recommendation, i.e. QoI.
The intent is to avoid \code{errno} altogether, while still supporting
floating-point exceptions (possibly depending on compiler flags).

This needs more work and is not reflected in the wording at this point.

%Wording idea:
%\begin{wgText}
%  \setcounter{Paras}{2}
%  \pnum
%  For each function with at least one parameter of type
%  \term{floating-point-type} other than \tcode{abs}, the implementation also
%  provides additional overloads sufficient to ensure that:
%  \begin{enumerate}
%    \item If at least one argument has a type that is a specialization of \tcode{simd}, then,
%      \begin{itemize}
%        \item if any two arguments are specializations of \tcode{simd} with
%          different width, then overload resolution does not result in a usable
%          candidate ([over.match.general]) from the overloads provided by the
%          implementation; otherwise
%        % now all simd arguments have equal width, or arguments are non-simd
%        \item every argument that is a specialization of \tcode{simd}
%      \end{itemize}
%      if every argument corresponding to a \term{floating-point-type} parameter
%      has arithmetic type or is a specialization of \tcode{simd}, then every
%      such
%      argument is effectively cast to the floating-point type with the greatest
%      floating-point conversion rank and greatest floating-point conversion
%      subrank among the types of all such arguments, where arguments of integer
%      type are considered to have the same floating-point conversion rank as
%      \tcode{double}.
%
%    \item Otherwise, if every argument corresponding to a
%      \term{floating-point-type} parameter has arithmetic type, then every such
%      argument is effectively cast to the floating-point type with the greatest
%      floating-point conversion rank and greatest floating-point conversion
%      subrank among the types of all such arguments, where arguments of integer
%      type are considered to have the same floating-point conversion rank as
%      \tcode{double}.
%      If no such floating-point type with the greatest rank and subrank exists,
%      then overload resolution does not result in a usable candidate
%      ([over.match.general]) from the overloads provided by the implementation.
%  \end{enumerate}
%\end{wgText}

\subsection{Add lvalue-qualifier to non-const subscript}\label{sec:lvalue-subscript}
The \code{operator[]} overloads of \code{simd} and \code{simd_mask} returned a
proxy reference object for non-\code{const} objects and the \code{value_type}
for \code{const} objects.
This made expressions such as \code{(x * 2)[0] = 1} well-formed.
However, assignment to temporaries can only be an error in the code (or code obfuscation).
Both \code{operator[]} overloads should be lvalue-ref qualified to make
\code{(x * 2)[0]} pick the const overload, which returns a prvalue that is not
assignable.

\subsection{Rename \code{simd_mask} reductions}
Summary:
\begin{itemize}
  \item The function \stdx\code{some_of} was removed.
  \item The function \stdx\code{popcount} was renamed to \std\code{reduce_count}.
  \item The function \stdx\code{find_first_set} was renamed to \std\code{reduce_min_index}.
  \item The function \stdx\code{find_last_set} was renamed to \std\code{reduce_max_index}.
\end{itemize}

For a discussion of this topic see \wglink{P1928R3} Section 5.2.

\section{Open questions}

\subsection{Alternatives to \code{hmin} and \code{hmax}}
The functions \code{hmin(simd)} and \code{hmax(simd)} are basically
specializations of \code{reduce(simd)}.
I received feedback asking for better names.

With \CC{}17, there was nothing equivalent to \code{std::plus<>} for minimum
and maximum.
Since the merge of Ranges (\CC{}20), we have \stdranges\code{min} and
\stdranges\code{max}.
The \code{reduce(simd)} specification requires the \code{binary_op} to be
callable with two \code{simd} arguments, though (split initial argument in
half, call \code{binary_op}, split again, call \code{binary_op}, \ldots until
only a scalar is left).
This doesn't work with \stdranges\code{min} (and \code{max}) because it
requires an lvalue reference as return type.
If we added another \code{operator()} to \stdranges\code{min}, then their
use with \code{reduce(simd)} would be slightly inconsistent:
\medskip\begin{lstlisting}
simd<unsigned> v = ...;
auto a = reduce(v, std::bit_and<>); // must type <>
auto b = reduce(v, std::ranges::min); // must *not* type <>
\end{lstlisting}

However, if \code{simd} will be an \code{input_range} (see \sect{sec:ranges})
then the
\stdranges\code{min(\MayBreak{}ranges\MayBreak{}::\MayBreak{}input_range
auto\&\&, ...)} overload matches and \stdranges\code{min(simd)} works out of
the box.
We could then leave it up to QoI to recognize the opportunity for a SIMD
implementation of the reduction.

Alternatively (or in addition) we could rename the TS functions to
\code{reduce_min(simd)} and \code{reduce_max(simd)}.

\subsubsection{Suggested Polls}

\wgPoll{We want to do something about \code{hmin} and \code{hmax}; i.e. the TS
status quo is not acceptable for the IS.}
{&&&&}

\wgPoll{Rename to \code{reduce_min} and \code{reduce_max}.}
{&&&&}

\wgPoll{Extend \stdranges\code{min} and \code{max} to allow prvalue return types.}
{&&&&}

\wgPoll{Remove \code{hmin} and \code{hmax} expecting \code{simd} to become a range.}
{&&&&}

\subsection{Argument order and naming of masked overloads}\label{sec:maskedOverloads}

In the TS, where-expressions made it possible to reuse existing function names
and argument orders for masked operations.
With the removal of where-expressions the mask must become a function argument.
See \tabref{tab:callsWithoutWhere} for a possible pattern to replace where-expressions.
\begin{beforeaftertable}[Parallelism TS 2 & possible replacement]{Possible replacement for where-expressions}
  \label{tab:callsWithoutWhere}
  \begin{lstlisting}
stdx::native_simd<float> v = ...;
where(v > 0, v).copy_from(ptr, stdx::element_aligned);
where(v < 0, v).copy_to(ptr, stdx::element_aligned);
float pos_sum = reduce(where(v > 0, v));
  \end{lstlisting}
  &
  \begin{lstlisting}
std::simd<float> v = ...;
v.copy_from_if(ptr, v > 0);
v.copy_to_if(ptr, v < 0);
float pos_sum = reduce_if(v, v > 0);
  \end{lstlisting}
\end{beforeaftertable}%

There are more options, of course.
E.g. possible replacements for \stdx\code{where_expression::copy_from}:
\begin{itemize}
  \item \lstinline@v.copy_from_if(v > 0, ptr)@\\
    Here the condition directly follows the word ``if'', which seems helpful.
    However:
  \item \lstinline@v.copy_from_if(ptr, v > 0)@\\
    This argument order follows precedent from algorithms, which always append
    the predicate to the list of arguments (e.g. \code{copy_if(first, last,
    d_first, pred)}).
    In addition, the argument order matches the order of words in the function
    name: ``from'' --- \code{ptr}, ``if'' --- \code{mask}.
  \item \lstinline@v.copy_from(v > 0, ptr)@
  \item \lstinline@v.copy_from(ptr, v > 0)@
  \item \lstinline@v.copy_from_where(v > 0, ptr)@
  \item \lstinline@v.copy_from_where(ptr, v > 0)@
\end{itemize}

The TS has no facility for masked load constructors.
I did not receive feedback that such a constructor is needed/wanted, so this paper will not propose one.

I propose to append \code{_if} to the masked functions and append the mask
argument (but before default arguments or additional arguments required for
masked operations, such as the \code{identity_element} in \code{reduce}).

\subsubsection{Suggested Polls}

\wgPoll{The names of masked ``overloads'' should include an \code{if} and follow the argument order proposed in \wgDocumentNumber{} \sect{sec:maskedOverloads}}
{&&&&}

%\subsection{Tuning masked loads and stores}
%
%The TS specifies masked loads and stores to prefer memory safety over performance:
%\begin{quote}{}
%  [§9.5 p9] Requires: [\ldots] the largest selected index is less than the number of values pointed to by mem.]
%
%  [§9.5 p19] Requires: [\ldots] for all selected indices i, i shall be less than the number of values pointed to by mem.
%\end{quote}
%I.e. the implementation is not allowed to read or write the memory locations that are masked off.
%Consequently,\\
%\lstinline@where(stdx::simd_mask<int>(false), v).copy_from(&*data.end(), stdx::element_aligned)@\\
%would not access any memory and the code would not invoke undefined behavior.
%However, this precludes efficient implementations on ISAs that have no native
%support for masked load and store operations.
%If a user ensures his memory allocations are always padded as necessary and
%thus expects highest performance, there is no good knob to turn to ``fast and
%faults are on you''.
%
%Such a knob could easily be provided via an additional load/store flag:\\
%\lstinline@v.copy_from_if(ptr, v > 0, std::vector_aligned | std::may_dereference_all)@.
\subsection{Converting loads \& stores consistency}\label{sec:convertingLoadsAndStores}

For the TS, we allowed pointers to any \emph{vectorizable} type as valid
arguments to \code{copy_from} and \code{copy_to}.
I.e. loads and stores can be converting operations without a clue in the code
other than the type of the pointer.
It can therefore happen that a conversion that is not value-preserving goes
unnoticed.
The broadcast and \code{simd} conversion constructurs guard against accidental
use of such conversions.

I have not received feedback that users wrote buggy because of this liberal
interface.
However, in the TS process this question was never really considered.
Therefore, I just wanted to show a suggestion for a stricter but just as
powerful interface.
\lst{lst:saferConvertingLoads} presents converting broadcast and cast
\begin{lstlisting}[numbers=left,float={hbtp},label=lst:saferConvertingLoads,caption={
    Load-store flags as opt-in to converting loads and stores
}]
float  fmem[std::simd_size_v<float>] = {};
double dmem[std::simd_size_v<float>] = {};
short  smem[std::simd_size_v<float>] = {};

std::simd<float> a = 1.; // ERROR: double -> float conversion is not value-preserving
std::simd<float> b = std::rebind_simd_t<int, std::simd<float>>(1); // ERROR:
                                                  // int -> float is not value-preserving

// TS:
stdx::simd<float> ts;
ts.copy_from(fmem, stdx::element_aligned); // OK
ts.copy_from(dmem, stdx::element_aligned); // OK
ts.copy_from(smem, stdx::element_aligned); // OK

// idea, not status quo of this paper:
std::simd<float> v;
v.copy_from(fmem); // OK
v.copy_from(dmem); // ERROR: converting load
v.copy_from(smem); // ERROR: converting load

// Option (a) - one flag only:
v.copy_from(dmem, std::simd_converting); // OK
v.copy_from(smem, std::simd_converting); // OK

// Option (b) - two flags:
v.copy_from(dmem, std::simd_safe_cvt); // ERROR: double -> float is not value-preserving
v.copy_from(dmem, std::simd_any_cvt);  // OK
v.copy_from(smem, std::simd_safe_cvt); // OK
v.copy_from(smem, std::simd_any_cvt);  // OK
\end{lstlisting}
expressions, which are ill-formed because the type conversion is not
value-preserving.
The equivalent conversions on \code{copy_from} are well-formed, though.
I believe it would be better for users to opt-in to conversions on load and
store.
There are value-preserving and non-value-preserving conversions, which could be
combined into the same opt-in or we provide a separate spelling for
value-preserving conversions (the safe kind of conversion).

If LEWG is interested, I would be thankful for naming suggestions.
I do not believe that using ``safe'' is a good term here.

%
\subsection{Integration with ranges}\label{sec:ranges}
\code{simd} itself is not a container \cite{P0851R0}.
The value of a data-parallel object is not an array of elements but rather needs to be understood as a single opaque value that happens to have means for reading and writing element values.
I.e. \code{simd<int> x = \{\};} does not start the lifetime of \type{int} objects.
This implies that \code{simd} cannot model a contiguous range.
But \code{simd} can trivially model \code{random_access_range}.
However, in order to model \code{output_range}, the iterator of every non-const
\code{simd} would have to return an \code{element_reference} on dereference.
Without the ability of \code{element_reference} to decay to the element type
(similar to how arrays decay to pointers on deduction), I would prefer to
simply make \code{simd} model only \code{random_access_range}.

If \code{simd} is a range, then \code{std::vector<std::simd<float>> data} can
be flattened trivially via \code{data | std::views::join}.
This makes the use of ``arrays of \code{simd<T>}'' easier to integrate into
existing interfaces the expect ``array of \code{T}''.

I plan to pursue adding iterators and conversions to array and from
random-access ranges, specifically \code{span} with static extent, in a
follow-up paper.
I believe it is not necessary to resolve this question before merging
\code{simd} from the TS.

\subsection{Formatting support}\label{sec:formatting}
If \code{simd} \emph{is a} range, as suggested above and to be proposed in a
follow-up paper, then \code{simd} will automatically be formatted as a range.
This seems to be a good solution unless there is a demand to format \code{simd}
objects differently from \code{random_access_range}.

\subsection{Correct place for \code{simd} in the IS?}

While \code{simd} is certainly very important for numerics and therefore fits into the “Numerics library” clause, it is also more than that.
E.g. \code{simd} can be used for vectorization of text processing.
In principle \code{simd} should be understood similar to fundamental types.
Is the “General utilities library” clause a better place?
Or rename “Concurrency support library” to “Parallelism and concurrency support library” and put it there?
Alternatively, add a new library clause?

I am seeking feedback before making a recommendation.

\subsection{\code{element_reference} is overspecified}
\code{element_reference} is spelled out in a lot of detail.
It may be better to define its requirements in a list of requirements or a table instead.

This change is not reflected in the wording, pending encouragement from WG21 (mostly LWG).

\subsection{Simplify fixed_size}\label{sec:simplifyfixedsize}
To understand the following discussion it is worth remembering that given an
element type \code{T} and SIMD width \code{N}, the ABI tag is not necessarily
distinct.
For example, given AVX-512, \code{fixed_size_simd<float, 16>} could either be
one \code{zmm} register or two \code{ymm} registers.
Orthogonally, the \code{simd_mask} type could either be stored as a bit-mask or
as a vector-mask.
Thus, we already have four reasonable choices for the ABI tag.

There are only three possible reasons why \simdabi\code{fixed_size<T, N>} should
not be an alias for \simdabi\code{deduce_t<T, N>} (or rather a rename of
\code{deduce_t}):
\begin{enumerate}
  \item We want \code{N} in \code{fixed_size_simd<T, N>} to be deducible.

  \item We want to support overloading via differently named ABI tags (i.e. if
    the user types a different name, then the actual type should be different).

  \item We want the actual type name of \simdabi\code{fixed_size<T, N>} to be
    recognizable as ``fixed size'' and not just hide behind some
    implementation-defined ABI tag.
\end{enumerate}

In the TS, \code{fixed_size_simd<T, N>} is required to be deducible.
However, after the changes we did to \code{fixed_size} and conversions, there is
no good motivation over deducing \code{fixed_size_simd<T, N>} instead of
\code{simd<T, Abi>}.
Let's just communicate that \code{simd<T, Abi>} is the one and only way to
deduce arbitrary length \code{simd} types.

In the TS, a user could write the following overload set:
\medskip\begin{lstlisting}
void f(stdx::native_simd<float>);

template <int N>
void f(stdx::fixed_size_simd<float, N>);
\end{lstlisting}
Besides \code{N} not being deducible anymore, if \code{N ==
stdx::native_simd<float>::size()}, then the overloads would clash, using the
same argument type.
However, again, after the changes we did to \code{fixed_size} and conversions,
there is probably no good reason left for declaring such an overload set.
Instead a user should write a single function template passing
\code{simd<float, Abi>}, deducing \code{Abi} and potentially constraining the
function using Concepts.

That leaves the question of diagnostics / debugging with regard to ABI tag names.
This argument can also go the other way:
By hiding the actual ABI tag used for implementing \code{fixed_size<T, N>} the
user has a harder time understanding what is going on.
So I don't think this argument has much weight.

\subsubsection{Suggested Poll}

Consequently, I propose the following poll:\\
\wgPoll{Remove \simdabi\code{fixed_size}, rename \simdabi\code{deduce_t} to
  \simdabi\code{fixed_size}, and remove (no public API) \simdabi\code{deduce}.
  Require \code{fixed_size_simd<T, simd_size_v<T>>} to be the same type as
  \code{simd<T>}. Require \code{fixed_size_simd<T, 1>} to be the same type as
  \code{simd<T, \simdabi{}scalar>}.}
{&&&&}

\section{Wording: Add Section 9 of N4808 with modifications}\label{sec:wording}

The following section presents the wording to be applied against the \CC{}
working draft.

The wording still needs work:
\begin{itemize}
  \item Replace \code{where} \& \code{where_expression} wording with \code{conditional_operator} and masked overloads.
  \item Apply the new library specification style from P0788R3.
\end{itemize}

\begin{wgText}[In {[version.syn]}, add]
  \begin{codeblock}
    #define __cpp_lib_simd YYYYMML // also in <simd>
  \end{codeblock}
\end{wgText}
Adjust the placeholder value as needed so as to denote this proposal's date of adoption.

\begin{wgText}[Add a new subclause after §28.8 {[numerics.numbers]}]
  \setcounter{WGClause}{28}
  \setcounter{WGSubSection}{8}
  \lstset{%
    columns=fullflexible,
    deletedelim=**[is]{|-}{-|},
    moredelim=[is][\color{white}\fontsize{0.1pt}{0.1pt}\selectfont{}]{|-}{-|}
  }
  \def\rSec#1[#2]#3{%
  \ifcase#1\wgSubsection[subsection]{#3}{#2}
  \or\wgSubsubsection[subsubsection]{#3}{#2}
  \or\wgSubsubsubsection[paragraph]{#3}{#2}
  \or\error
\fi}

\renewcommand\foralli[1][]{for all $i$ in the range of \range{0}{#1size()}}
\renewcommand\forallmaskedi{for all selected indices $i$ of \tcode{mask}}

\newcommand\validMaskedRange[1][first]{For all selected indices $i$,
\range{#1}{#1 + $i$ + 1} is a valid range.}

\newcommand\flagsRequires[2]{
\item If the template parameter pack \tcode{Flags} contains
  \tcode{\alignedflag}, \tcode{to_address(first)} points to storage
  aligned by \tcode{simd_alignment_v<#1>}.
\item If the template parameter pack \tcode{Flags} contains
  \tcode{\overalignedflag<N>}, \tcode{to_address(first)}
  points to storage aligned by \tcode{N}.
}

\newcommand\conversionFlagsMandate[2]{
  If the template parameter pack \tcode{Flags} does not contain
  \tcode{\convertflag}, then the conversion from \tcode{#1} to
  \tcode{#2} is value-preserving.
}

\newcommand\op{\textrm{\textit{op}}}

\newcommand\ConstraintUnaryOperatorWellFormed[2][const ]{%
  \constraints \tcode{requires (#1value_type a) \{ #2; \}} is \tcode{true}.
}

\newcommand\ConstraintOperatorTWellFormed{%
  \constraints \tcode{requires (value_type a, value_type b) \{ a \op{} b; \}} is \tcode{true}.
}

\rSec0[simd]{Data-parallel types}
\rSec1[simd.general]{General}

\pnum
[simd] defines data-parallel types and operations on these types.
\begin{note}
The intent is to support acceleration through data-parallel execution resources
where available, such as SIMD registers and instructions or execution units
driven by a common instruction decoder.
%If such execution resources are unavailable, the interfaces support a
%transparent fallback to sequential execution.
\end{note}

\pnum\label{wording.vectorizable.types}%
The set of \defn{vectorizable types} comprises all standard integer types,
character types, and the types \tcode{float} and \tcode{double}
([basic.fundamental]).
In addition, \tcode{std::float16_t}, \tcode{std::float32_t}, and
\tcode{std::float64_t} are vectorizable types if defined ([basic.extended.fp]).

\pnum
The term \defn{data-parallel type} refers to all enabled specializations of
the \tcode{basic_simd} and \tcode{basic_simd_mask} class templates. A \defn{data-parallel object} is
an object of \term{data-parallel type}.

\pnum
Each specialization of \tcode{basic_simd} or \tcode{basic_simd_mask} is either enabled or disabled,
as described in \ref{simd.overview} and \ref{simd.mask.overview}.

\pnum
A data-parallel type consists of one or more elements of an underlying vectorizable type,
called the \defn{element type}.
The number of elements is a constant for each data-parallel type and called the
\defn{width} of that type.
The elements in a data-parallel type are indexed from 0 to $\textrm{width} - 1$.

\pnum
An \defn{element-wise operation} applies a specified operation to the elements of one or more
data-parallel objects. Each such application is unsequenced with respect to the others. A
\defn{unary element-wise operation} is an element-wise operation that applies a unary operation to
each element of a data-parallel object. A \defn{binary element-wise operation} is an element-wise
operation that applies a binary operation to corresponding elements of two data-parallel objects.

\pnum\label{wording.selected.indices}%
Given a \tcode{basic_simd_mask<Bytes, Abi>} object \tcode{mask}, the
\defn{selected indices} signify the integers $i$ in the range
\range{0}{mask.size()} for which \tcode{mask[$i$]} is \tcode{true}.
Given an object \tcode{data} of type \tcode{basic_simd<T, Abi>} or \tcode{basic_simd_mask<Bytes, Abi>},
the \defn{selected elements} signify the elements \tcode{data[$i$]} for all selected indices $i$.

\pnum
The conversion from vectorizable type \tcode{U} to vectorizable type \tcode{T} is
\defn{value-preserving} if
all possible values of \tcode{U} can be represented with type \tcode{T}.

\rSec1[simd.syn]{Header \texorpdfstring{\tcode{<simd>}}{<simd>} synopsis}

%\indexhdr{simd}
\begin{codeblock}
namespace std {
  using @\simdsizetype@ = @\seebelow@;                                  // \expos
  template <class T, class Abi> constexpr @\simdsizetype\ \simdsizev@ = @\seebelow@; // \expos

  template <class T> constexpr size_t @\maskelementsize@ = @\seebelow@; // \expos
  template <size_t Bytes> using @\integerfrom@ = @\seebelow@;            // \expos

  template <class T>
    concept @\constexprwrapperlike@ =                                 // \expos
      convertible_to<T, decltype(T::value)> &&
      equality_comparable_with<T, decltype(T::value)> &&
      bool_constant<T() == T::value>::value &&
      bool_constant<static_cast<decltype(T::value)>(T()) == T::value>::value;

  // \ref{simd.abi}, \tcode{simd} ABI tags
  template<class T> using @\nativeabi@ = @\seebelow@;                    // \expos
  template<class T, @\simdsizetype@ N> using @\deducet@ = @\seebelow@;    // \expos

  // \ref{simd.traits}, \tcode{simd} type traits
  template<class T, class U = typename T::value_type> struct simd_alignment;
  template<class T, class U = typename T::value_type>
    inline constexpr size_t simd_alignment_v = simd_alignment<T, U>::value;

  template<class T, class V> struct rebind_simd { using type = @\seebelow@; };
  template<class T, class V> using rebind_simd_t = typename rebind_simd<T, V>::type;
  template<@\simdsizetype@ N, class V> struct resize_simd { using type = @\seebelow@; };
  template<@\simdsizetype@ N, class V> using resize_simd_t = typename resize_simd<N, V>::type;

  // \ref{simd.flags}, Load and store flags
  struct @\convertflag@; // \expos
  struct @\alignedflag@; // \expos
  template<size_t N> struct @\overalignedflag@; // \expos

  template <class... Flags> struct simd_flags;
  inline constexpr simd_flags<> simd_flag_default{};
  inline constexpr simd_flags<@\convertflag@> simd_flag_convert{};
  inline constexpr simd_flags<@\alignedflag@> simd_flag_aligned{};
  template<size_t N> requires (has_single_bit(N))
    inline constexpr simd_flags<@\overalignedflag<N>@> simd_flag_overaligned{};

  // \ref{simd.class}, Class template \tcode{basic_simd}
  template<class T, class Abi = @\nativeabi@<T>> class basic_simd;
  template<class T, @\simdsizetype@ N = @\simdsizev@<T, @\nativeabi@<T>>>
    using simd = basic_simd<T, @\deducet@<T, N>>;

  // \ref{simd.mask.class}, Class template \tcode{basic_simd_mask}
  template<size_t Bytes, class Abi = @\nativeabi@<@\integerfrom@<Bytes>>> class basic_simd_mask;
  template<class T, @\simdsizetype@ N = @\simdsizev@<T, @\nativeabi@<T>>>
    using simd_mask = basic_simd_mask<sizeof(T), @\deducet@<T, N>>;

  // \ref{simd.creation}, \tcode{basic_simd} and \tcode{basic_simd_mask} creation
  template<class V, class Abi>
    constexpr auto
      simd_split(const basic_simd<typename V::value_type, Abi>& x) noexcept;
  template<class M, class Abi>
    constexpr auto
      simd_split(const basic_simd_mask<@\maskelementsize@<M>, Abi>& x) noexcept;

  template<class T, class... Abis>
    constexpr basic_simd<T, @\deducet@<T, (basic_simd<T, Abis>::size() + ...)>>
      simd_cat(const basic_simd<T, Abis>&...) noexcept;
  template<size_t Bs, class... Abis>
    constexpr basic_simd_mask<Bs, @\deducet@<@\integerfrom@<Bs>,
                              (basic_simd_mask<Bs, Abis>::size() + ...)>>
      simd_cat(const basic_simd_mask<Bs, Abis>&...) noexcept;

  // \ref{simd.mask.reductions}, \tcode{basic_simd_mask} reductions
  template<size_t Bs, class Abi>
    constexpr bool all_of(const basic_simd_mask<Bs, Abi>&) noexcept;
  template<size_t Bs, class Abi>
    constexpr bool any_of(const basic_simd_mask<Bs, Abi>&) noexcept;
  template<size_t Bs, class Abi>
    constexpr bool none_of(const basic_simd_mask<Bs, Abi>&) noexcept;
  template<size_t Bs, class Abi>
    constexpr @\simdsizetype@ reduce_count(const basic_simd_mask<Bs, Abi>&) noexcept;
  template<size_t Bs, class Abi>
    constexpr @\simdsizetype@ reduce_min_index(const basic_simd_mask<Bs, Abi>&);
  template<size_t Bs, class Abi>
    constexpr @\simdsizetype@ reduce_max_index(const basic_simd_mask<Bs, Abi>&);

  constexpr bool all_of(same_as<bool> auto) noexcept;
  constexpr bool any_of(same_as<bool> auto) noexcept;
  constexpr bool none_of(same_as<bool> auto) noexcept;
  constexpr @\simdsizetype@ reduce_count(same_as<bool> auto) noexcept;
  constexpr @\simdsizetype@ reduce_min_index(same_as<bool> auto);
  constexpr @\simdsizetype@ reduce_max_index(same_as<bool> auto);

  // \ref{simd.reductions}, \tcode{basic_simd} reductions
  template<class T, class Abi, class BinaryOperation = plus<>>
    constexpr T reduce(const basic_simd<T, Abi>&, BinaryOperation = {});
  template<class T, class Abi, class BinaryOperation>
    constexpr T reduce(const basic_simd<T, Abi>& x,
      const typename basic_simd<T, Abi>::mask_type& mask, type_identity_t<T> identity_element,
      BinaryOperation binary_op);
  template<class T, class Abi>
    constexpr T reduce(const basic_simd<T, Abi>& x,
      const typename basic_simd<T, Abi>::mask_type& mask, plus<> binary_op = {}) noexcept;
  template<class T, class Abi>
    constexpr T reduce(const basic_simd<T, Abi>& x,
      const typename basic_simd<T, Abi>::mask_type& mask, multiplies<> binary_op) noexcept;
  template<class T, class Abi>
    constexpr T reduce(const basic_simd<T, Abi>& x,
      const typename basic_simd<T, Abi>::mask_type& mask, bit_and<> binary_op) noexcept;
  template<class T, class Abi>
    constexpr T reduce(const basic_simd<T, Abi>& x,
      const typename basic_simd<T, Abi>::mask_type& mask, bit_or<> binary_op) noexcept;
  template<class T, class Abi>
    constexpr T reduce(const basic_simd<T, Abi>& x,
      const typename basic_simd<T, Abi>::mask_type& mask, bit_xor<> binary_op) noexcept;

  template<class T, class Abi>
    constexpr T reduce_min(const basic_simd<T, Abi>&) noexcept;
  template<class T, class Abi>
    constexpr T reduce_min(const basic_simd<T, Abi>&,
                           const typename basic_simd<T, Abi>::mask_type&) noexcept;
  template<class T, class Abi>
    constexpr T reduce_max(const basic_simd<T, Abi>&) noexcept;
  template<class T, class Abi>
    constexpr T reduce_max(const basic_simd<T, Abi>&,
                           const typename basic_simd<T, Abi>::mask_type&) noexcept;

  // \ref{simd.alg}, Algorithms
  template<class T, class Abi>
    constexpr basic_simd<T, Abi>
      min(const basic_simd<T, Abi>& a, const basic_simd<T, Abi>& b) noexcept;
  template<class T, class Abi>
    constexpr basic_simd<T, Abi>
      max(const basic_simd<T, Abi>& a, const basic_simd<T, Abi>& b) noexcept;
  template<class T, class Abi>
    constexpr pair<basic_simd<T, Abi>, basic_simd<T, Abi>>
      minmax(const basic_simd<T, Abi>& a, const basic_simd<T, Abi>& b) noexcept;
  template<class T, class Abi>
    constexpr basic_simd<T, Abi>
      clamp(const basic_simd<T, Abi>& v, const basic_simd<T, Abi>& lo,
            const basic_simd<T, Abi>& hi);

  template<class T, class U>
    constexpr auto simd_select(bool c, const T& a, const U& b)
    -> remove_cvref_t<decltype(c ? a : b)>;
  template<size_t Bytes, class Abi, class T, class U>
    constexpr auto simd_select(const basic_simd_mask<Bytes, Abi>& c, const T& a, const U& b)
    noexcept -> decltype(@\simdselect@(c, a, b));
}
\end{codeblock}

\pnum
\simdsizetype{} is an alias for a signed integer type.

\pnum
\tcode{\simdsizev<T, Abi>} denotes the width of \tcode{basic_simd<T, Abi>} if
the specialization \tcode{basic_simd<T, Abi>} is enabled, or \tcode{0} otherwise.
\begin{note}
  \tcode{\simdsizev<T, Abi>} does not require instantiation of \tcode{basic_simd<T, Abi>}.
\end{note}
\FIXME{drop the note?}

\pnum
\tcode{\maskelementsize<basic_simd_mask<Bytes, Abi>>} has the value \tcode{Bytes}.

\pnum
\tcode{\integerfrom<Bytes>} is an alias for a signed integer type \tcode{T} so that \tcode{sizeof(T)
== Bytes}.

\rSec1[simd.abi]{\tcode{simd} ABI tags}

\begin{codeblock}
template<class T> using @\nativeabi@ = @\seebelow@; // \expos
template<class T, @\simdsizetype@ N> using @\deducet@ = @\seebelow@; // \expos
\end{codeblock}

\pnum\label{wording.ABI.tag}
An \defn{ABI tag} is a type that indicates a choice of size and binary
representation for objects of data-parallel type.
\begin{note}
  The intent is for the size and binary representation to depend on the target
  architecture and compiler flags.
\end{note}
The ABI tag, together with a given element type, implies the width.

\pnum
\begin{note}
The ABI tag is orthogonal to selecting the machine instruction set.
The selected machine instruction set limits the usable ABI tag types, though
(see \ref{simd.overview}).
The ABI tags enable users to safely pass objects of data-parallel type between
translation unit boundaries (e.g. function calls or I/O).
\end{note}

\pnum
An implementation defines ABI tag types as necessary for the following aliases.

\pnum\label{wording.deducet}
\tcode{\deducet<T, N>} is defined if
\begin{itemize}
  \item \tcode{T} is a vectorizable type,
  \item \tcode{N} is greater than zero, and
  \item \tcode{N} is not larger than an implementation-defined maximum.
\end{itemize}
The implementation-defined maximum for \tcode{N} is not smaller than 64
and can differ depending on \tcode{T}.

\pnum
Where present, \tcode{\deducet<T, N>} names an ABI tag type that satisfies
\begin{itemize}
  \item \tcode{\simdsizev<T, \deducet<T, N>> == N}, and
  \item \tcode{basic_simd<T, \deducet<T, N>>} is enabled (see \ref{simd.overview}).
\end{itemize}

% TODO: If we really want the Abis... pack back this needs a paper to LEWG.
%       The pack allows e.g. to stay with ymm registers even when zmm is available,
%         or to implement an MMX ABI that doesn't get used unless explicitly called for.
%       Also one can imagine a target with multiple different SIMD execution facilities
%         where moving between them has a high cost and shouldn't happen without a request.
%\begin{note}
  %If multiple ABI tags can satisfy the above conditions, differences in \tcode{Abis...} can lead to different results.
%\end{note}

\INFO{I removed the paragraph saying “The type of \tcode{\deducet<T, N>} in
  translation unit 1 differs from the type of \tcode{\deducet<T, N>} in
  translation unit 2 if and only if the type of \tcode{\nativeabi<T>} in
  translation unit 1 differs from the type of \tcode{\nativeabi<T>} in
  translation unit 2.” after consulting Jens.
  He said I can't reasonably say anything about working around ODR problems in an implementation.
Implementations thus have to figure this out on their own.}

\pnum
\tcode{\nativeabi<T>} is an implementation-defined alias for an ABI tag.
\tcode{basic_simd<T, \nativeabi<T>} is an enabled specialization.
\begin{note}
The intent is to use the ABI tag producing the most efficient data-parallel
execution for the element type \tcode{T} on the currently
targeted system.
For target architectures with ISA extensions, compiler flags can change the
type of the \tcode{\nativeabi<T>} alias.
\end{note}\\
\begin{example}
  Consider a target architecture supporting the ABI tags
  \tcode{__simd128} and \tcode{__simd256}, where hardware support for
  \tcode{__simd256} exists only for floating-point types.
  The implementation therefore defines \tcode{\nativeabi<T>} as an alias for
  \begin{itemize}
    \item \tcode{__simd256} if \tcode{T} is a floating-point type, and
    \item \tcode{__simd128} otherwise.
  \end{itemize}
\end{example}

\rSec1[simd.traits]{\tcode{simd} type traits}

\begin{itemdecl}
template<class T, class U = typename T::value_type> struct simd_alignment { @\seebelow@ };
\end{itemdecl}

\begin{itemdescr}
\pnum
\tcode{simd_alignment<T, U>} has a member \tcode{value} if and only if
\begin{itemize}
  \item \tcode{T} is a specialization of \tcode{basic_simd_mask} and \tcode{U} is \tcode{bool}, or
  \item \tcode{T} is a specialization of \tcode{basic_simd} and \tcode{U} is a vectorizable type.
\end{itemize}

\pnum
If \tcode{value} is present, the type \tcode{simd_alignment<T, U>} is a \tcode{BinaryTypeTrait} with
a base characteristic of \tcode{integral_constant<size_t, N>} for some unspecified
\tcode{N} (see \ref{simd.copy} and \ref{simd.mask.copy}). \begin{note}\tcode{value} identifies the
alignment restrictions on pointers used for (converting) loads and stores for the given type
\tcode{T} on arrays of type \tcode{U}.\end{note}

\pnum
The behavior of a program that adds specializations for \tcode{simd_alignment} is undefined.
\end{itemdescr}

\begin{itemdecl}
template<class T, class V> struct rebind_simd { using type = @\seebelow@; };
\end{itemdecl}

\begin{itemdescr}
  \pnum
  The member \tcode{type} is present if and only if
  \begin{itemize}
    \item \tcode{V} is a specialization of either \tcode{basic_simd} or
      \tcode{basic_simd_mask},
    \item \tcode{T} is a vectorizable type, and
    \item \tcode{\deducet<T, V::size()>} has a member type \tcode{type}.
  \end{itemize}

  \pnum
  Let \tcode{Abi1} denote an ABI tag such that \tcode{basic_simd<T,
  Abi1>::size() == V::size()}.
  Where present, the member typedef \tcode{type} names \tcode{basic_simd<T,
  Abi1>} if \tcode V is a specialization of \tcode{basic_simd} or
  \tcode{basic_simd_mask<sizeof(T), Abi1>} if \tcode V is a specialization of
  \tcode{basic_simd_mask}.
\end{itemdescr}

\begin{itemdecl}
template<@\simdsizetype@ N, class V> struct resize_simd { using type = @\seebelow@; };
\end{itemdecl}

\begin{itemdescr}
  \pnum Let \tcode{T} denote
  \begin{itemize}
    \item \tcode{typename V::value_type} if \tcode{V} is a specialization of
      \tcode{basic_simd} or
    \item \tcode{\integerfrom<\maskelementsize<V>>} if \tcode{V} is a
      specialization of \tcode{basic_simd_mask}.
  \end{itemize}

  \pnum
  The member \tcode{type} is present if and only if
  \begin{itemize}
    \item \tcode{V} is a specialization of either \tcode{basic_simd} or
      \tcode{basic_simd_mask}, and
    \item \tcode{\deducet<T, N>} has a member type \tcode{type}.
  \end{itemize}

  \pnum
  Let \tcode{Abi1} denote an ABI tag such that \tcode{basic_simd<T,
  Abi1>::size() == V::size()}.
  Where present, the member typedef \tcode{type} names \tcode{basic_simd<T,
  Abi1>} if \tcode V is a specialization of \tcode{basic_simd} or
  \tcode{basic_simd_mask<sizeof(T), Abi1>} if \tcode V is a specialization of
  \tcode{basic_simd_mask}.
\end{itemdescr}

\rSec1[simd.flags]{Load and store flags}

\rSec2[simd.flags.overview]{Class template \tcode{simd_flags} overview}

\begin{codeblock}
template <class... Flags> struct simd_flags {
  // \ref{simd.flags.oper}, \tcode{simd_flags} operators
  template <class... Other>
    friend consteval auto operator|(simd_flags, simd_flags<Other...>);
};
\end{codeblock}

\pnum
\begin{note}
The class template \tcode{simd_flags} acts like a integer bit-flag for types.
\end{note}

\pnum\constraints
Every type in \tcode{Flags} is one of \tcode{\convertflag},
\tcode{\alignedflag}, or \tcode{\overalignedflag<N>}.

\rSec2[simd.flags.oper]{\tcode{simd_flags} operators}

\begin{itemdecl}
template <class... Other>
  friend consteval auto operator|(simd_flags a, simd_flags<Other...> b);
\end{itemdecl}

\begin{itemdescr}
  \pnum\returns
  A default-initialized object of type \tcode{simd_flags<Flags2...>} where
  every type in pack \tcode{Flags2} is present either in pack \tcode{Flags} or
  pack \tcode{Other} and every type in packs \tcode{Flags} and \tcode{Other} is
  present in \tcode{Flags2}.
  If the packs \tcode{Flags} and \tcode{Other} contain two
  different specializations \tcode{\overalignedflag<N1>} and
  \tcode{\overalignedflag<N2>}, \tcode{Flags2} is not required to contain the
  specialization \tcode{\overalignedflag<std\colcol{}min(N1, N2)>}.
\end{itemdescr}

\rSec1[simd.class]{Class template \tcode{basic_simd}}

\rSec2[simd.overview]{Class template \tcode{basic_simd} overview}

\begin{codeblock}
template<class T, class Abi> class basic_simd {
public:
  using value_type = T;
  using mask_type = basic_simd_mask<sizeof(T), Abi>;
  using abi_type = Abi;

  static constexpr integral_constant<@\simdsizetype@, @\simdsizev@<T, Abi>> size {};

  constexpr basic_simd() noexcept = default;

  // \ref{simd.ctor}, \tcode{basic_simd} constructors
  template<class U> constexpr basic_simd(U&& value) noexcept;
  template<class U, class UAbi>
    constexpr explicit(@\seebelow@) basic_simd(const basic_simd<U, UAbi>&) noexcept;
  template<class G> constexpr explicit basic_simd(G&& gen) noexcept;
  template<class It, class... Flags>
    constexpr explicit basic_simd(It first, simd_flags<Flags...> = {});
  template<class It, class... Flags>
    constexpr explicit basic_simd(It first, const mask_type& mask, simd_flags<Flags...> = {});

  // \ref{simd.copy}, \tcode{basic_simd} copy functions
  template<class It, class... Flags>
    constexpr void copy_from(It first, simd_flags<Flags...> f = {});
  template<class It, class... Flags>
    constexpr void copy_from(It first, const mask_type& mask, simd_flags<Flags...> f = {});
  template<class Out, class... Flags>
    constexpr void copy_to(Out first, simd_flags<Flags...> f = {}) const;
  template<class Out, class... Flags>
    constexpr void copy_to(Out first, const mask_type& mask, simd_flags<Flags...> f = {}) const;

  // \ref{simd.subscr}, \tcode{basic_simd} subscript operators
  constexpr value_type operator[](@\simdsizetype@) const;

  // \ref{simd.unary}, \tcode{basic_simd} unary operators
  constexpr basic_simd& operator++() noexcept;
  constexpr basic_simd operator++(int) noexcept;
  constexpr basic_simd& operator--() noexcept;
  constexpr basic_simd operator--(int) noexcept;
  constexpr mask_type operator!() const noexcept;
  constexpr basic_simd operator~() const noexcept;
  constexpr basic_simd operator+() const noexcept;
  constexpr basic_simd operator-() const noexcept;

  // \ref{simd.binary}, \tcode{basic_simd} binary operators
  friend constexpr basic_simd operator+(const basic_simd&, const basic_simd&) noexcept;
  friend constexpr basic_simd operator-(const basic_simd&, const basic_simd&) noexcept;
  friend constexpr basic_simd operator*(const basic_simd&, const basic_simd&) noexcept;
  friend constexpr basic_simd operator/(const basic_simd&, const basic_simd&) noexcept;
  friend constexpr basic_simd operator%(const basic_simd&, const basic_simd&) noexcept;
  friend constexpr basic_simd operator&(const basic_simd&, const basic_simd&) noexcept;
  friend constexpr basic_simd operator|(const basic_simd&, const basic_simd&) noexcept;
  friend constexpr basic_simd operator^(const basic_simd&, const basic_simd&) noexcept;
  friend constexpr basic_simd operator<<(const basic_simd&, const basic_simd&) noexcept;
  friend constexpr basic_simd operator>>(const basic_simd&, const basic_simd&) noexcept;
  friend constexpr basic_simd operator<<(const basic_simd&, @\simdsizetype@) noexcept;
  friend constexpr basic_simd operator>>(const basic_simd&, @\simdsizetype@) noexcept;

  // \ref{simd.cassign}, \tcode{basic_simd} compound assignment
  friend constexpr basic_simd& operator+=(basic_simd&, const basic_simd&) noexcept;
  friend constexpr basic_simd& operator-=(basic_simd&, const basic_simd&) noexcept;
  friend constexpr basic_simd& operator*=(basic_simd&, const basic_simd&) noexcept;
  friend constexpr basic_simd& operator/=(basic_simd&, const basic_simd&) noexcept;
  friend constexpr basic_simd& operator%=(basic_simd&, const basic_simd&) noexcept;
  friend constexpr basic_simd& operator&=(basic_simd&, const basic_simd&) noexcept;
  friend constexpr basic_simd& operator|=(basic_simd&, const basic_simd&) noexcept;
  friend constexpr basic_simd& operator^=(basic_simd&, const basic_simd&) noexcept;
  friend constexpr basic_simd& operator<<=(basic_simd&, const basic_simd&) noexcept;
  friend constexpr basic_simd& operator>>=(basic_simd&, const basic_simd&) noexcept;
  friend constexpr basic_simd& operator<<=(basic_simd&, @\simdsizetype@) noexcept;
  friend constexpr basic_simd& operator>>=(basic_simd&, @\simdsizetype@) noexcept;

  // \ref{simd.comparison}, \tcode{basic_simd} compare operators
  friend constexpr mask_type operator==(const basic_simd&, const basic_simd&) noexcept;
  friend constexpr mask_type operator!=(const basic_simd&, const basic_simd&) noexcept;
  friend constexpr mask_type operator>=(const basic_simd&, const basic_simd&) noexcept;
  friend constexpr mask_type operator<=(const basic_simd&, const basic_simd&) noexcept;
  friend constexpr mask_type operator>(const basic_simd&, const basic_simd&) noexcept;
  friend constexpr mask_type operator<(const basic_simd&, const basic_simd&) noexcept;

  // \ref{simd.cond}, \tcode{basic_simd} exposition-only conditional operators
  friend constexpr basic_simd @\simdselect@(
    const mask_type&, const basic_simd&, const basic_simd&) noexcept;
};
\end{codeblock}

\pnum
The specializations of class template \tcode{basic_simd} are data-parallel types.

\pnum
Every specialization of \tcode{basic_simd} is a complete type.
The types \tcode{basic_simd<T, \deducet<T, N>>} for all vectorizable
\tcode{T} and with \tcode{N} in the range of \crange{1}{64} are enabled.
It is implementation-defined whether any other \tcode{basic_simd<T, Abi>} specialization
with vectorizable \tcode{T} is enabled.
Any other specialization of \tcode{basic_simd} is disabled.

\begin{note}
  The intent is for implementations to determine on the basis of the currently
  targeted system, whether \tcode{basic_simd<T, Abi>} is enabled.
\end{note}
\FIXME{drop the note?}

If \tcode{basic_simd<T, Abi>} is disabled, the specialization has a
deleted default constructor, deleted destructor, deleted copy constructor, and
deleted copy assignment.
In addition only the \tcode{value_type}, \tcode{abi_type}, and
\tcode{mask_type} members are present.

If \tcode{basic_simd<T, Abi>} is enabled, \tcode{basic_simd<T, Abi>} is
trivially copyable.

\pnum\recommended:
Implementations should enable explicit conversion from and to
implementation-defined types. This adds one or more of the following
declarations to class \tcode{basic_simd}:

\begin{codeblock}
constexpr explicit operator @\impdef@() const;
constexpr explicit basic_simd(const @\impdef@& init);
\end{codeblock}

\begin{example}
  Consider an implementation that supports the type \tcode{__vec4f} and the function \tcode{__vec4f
  _vec4f_addsub(__vec4f, __vec4f)} for the currently targeted system.
  A user may require the use of \tcode{_vec4f_addsub} for maximum performance and thus writes:
  \begin{codeblock}
    using V = basic_simd<float, simd_abi::__simd128>;
    V addsub(V a, V b) {
      return static_cast<V>(_vec4f_addsub(static_cast<__vec4f>(a), static_cast<__vec4f>(b)));
    }
  \end{codeblock}
\end{example}


\rSec2[simd.ctor]{\tcode{basic_simd} constructors}

\begin{itemdecl}
template<class U> constexpr basic_simd(U&&) noexcept;
\end{itemdecl}

\begin{itemdescr}
  \pnum Let \tcode{From} denote the type \tcode{remove_cvref_t<U>}.

  \pnum\constraints
  \tcode{From} satisfies \tcode{convertible_to<value_type>}, and either
  \begin{itemize}
    \item \tcode{From} is a vectorizable type and the conversion from
      \tcode{From} to \tcode{value_type} is value-preserving
      (\ref{simd.general}), or

    \item \tcode{From} is not an arithmetic type and does not satisfy
      \tcode{\constexprwrapperlike}, or

    \item \tcode{From} satisfies \tcode{\constexprwrapperlike} (\ref{simd.syn})
      and the actual value of \tcode{From::value} after conversion to
      \tcode{value_type} will fit into \tcode{value_type} and will produce the
      original value when converted back to \tcode{decltype(From::value)}.
    \FIXME{
      A value “after conversion to \tcode{To}” is always representable by \tcode{To}.
      What I actually implemented is
      \tcode{!(unsigned_integral<To> \&\& From::value < decltype(From::value)()
          \&\& From::value <= numeric_limits<To>::max()
        \&\& From::value >= numeric_limits<To>::lowest()}
    }
    \INFO{
      Design intent:
      I'm trying to allow \tcode{1.f $\rightarrow$ int} while disallowing \tcode{1.1f $\rightarrow$
      int}.
      Also, if \tcode{From::value} is a UDT, e.g. fixed-point, I believe we cannot use wording
      such as “value can be represented” because how can we speak about the numerical value of a
      UDT? Or more importantly, how would you implement such a constraint? That'd be hand waving
      at best. We can speak about the value after conversion. But then we don't know what was lost
      until we convert it back.\\
      Ultimately, I think we need to aim for a reasonable heuristic, no more.
    }
  \end{itemize}

  \pnum\effects
  Initializes each element to the value of the argument after conversion to \tcode{value_type}.
\end{itemdescr}

\begin{itemdecl}
template<class U, class UAbi>
  constexpr explicit(@\seebelow@) basic_simd(const basic_simd<U, UAbi>& x) noexcept;
\end{itemdecl}

\begin{itemdescr}
  \pnum\constraints
  \tcode{\simdsizev<U, UAbi> == size()} is \tcode{true}.

  \pnum\effects
  Initializes the $i^\text{th}$ element with \tcode{static_cast<T>(x[$i$])} \foralli.

  \pnum\remarks
  %The constructor is \tcode{explicit} if
  The expression inside \tcode{explicit} evaluates to \tcode{true} if either
  \begin{itemize}
    \item the conversion from \tcode{U} to \tcode{value_type} is not
      value-preserving, or

    \item both \tcode{U} and \tcode{value_type} are integral types and the
      integer conversion rank (\iref{conv.rank}) of \tcode{U} is greater than
      the integer conversion rank of \tcode{value_type}, or

    \item both \tcode{U} and \tcode{value_type} are floating-point types and
      the floating-point conversion rank (\iref{conv.rank}) of \tcode{U} is
      greater than the floating-point conversion rank of \tcode{value_type}.
  \end{itemize}
\end{itemdescr}

\begin{itemdecl}
template<class G> constexpr explicit basic_simd(G&& gen) noexcept;
\end{itemdecl}

\begin{itemdescr}
  \pnum Let \tcode{From}$_i$ denote the type
  \tcode{decltype(gen(integral_constant<\simdsizetype, $i$>()))}.

  \pnum\constraints
  \tcode{From}$_i$ satisfies \tcode{convertible_to<value_type>} \foralli.
  In addition, \foralli, if \tcode{From}$_i$ is a vectorizable type, conversion from
  \tcode{From}$_i$ to \tcode{value_type} is value-preserving.
  \FIXME{
    This allows \tcode{long double} and \tcode{std::float128_t} as return types.
    So I think the last sentence needs to constrain on “arithmetic type” rather than “vectorizable
    type”.
    At the same time, we probably want to extend the definition of value-preserving from
    vectorizable types to arithmetic types.
  }

  \pnum\effects
  Initializes the $i^\text{th}$ element with
  \tcode{static_cast<value_type>(gen(integral_constant<\simdsizetype, i>()))} \foralli.

  \pnum
    The calls to \tcode{gen} are unsequenced with respect to each other.
    Vectorization-unsafe (\iref{algorithms.parallel.defns}) standard library
    functions may not be invoked by \tcode{gen}.
    \tcode{gen} is invoked exactly once for each $i$.
\end{itemdescr}

\newcommand\SimdLoadDescr[2]{
  \pnum\constraints
  \begin{itemize}
    \item \tcode{iter_value_t<It>} is a vectorizable type, and
    \item \tcode{It} models \tcode{contiguous_iterator}.
  \end{itemize}

  \pnum\mandates
  \conversionFlagsMandate{iter_value_t<It>}{value_type}

  \pnum\expects
  \begin{itemize}
    \item #1
    \flagsRequires{basic_simd, iter_value_t<It>}{iter_value_t<It>}
  \end{itemize}

  \pnum\effects #2
}

\begin{itemdecl}
template<class It, class... Flags>
  constexpr explicit basic_simd(It first, simd_flags<Flags...> = {});
\end{itemdecl}

\begin{itemdescr}
  \SimdLoadDescr
    {\range{first}{first + size()} is a valid range.}
    {Initializes the $i^\text{th}$ element with \tcode{static_cast<T>(to_address(first)[$i$])}
    \foralli.}
\end{itemdescr}

\begin{itemdecl}
template<class It, class... Flags>
  constexpr explicit basic_simd(It first, const mask_type& mask, simd_flags<Flags...> = {});
\end{itemdecl}

\begin{itemdescr}
  \SimdLoadDescr
    {\validMaskedRange}
    {Initializes the $i^\text{th}$ element with \tcode{mask[$i$] ?
      static_cast<T>(to_address(first)[$i$]) : T()}
    \foralli.}
\end{itemdescr}

\rSec2[simd.copy]{\tcode{basic_simd} copy functions}

\begin{itemdecl}
template<class It, class... Flags>
  constexpr void copy_from(It first, simd_flags<Flags...> f = {});
\end{itemdecl}

\begin{itemdescr}
  \SimdLoadDescr
    {\range{first}{first + size()} is a valid range.}
    {Replaces the elements of the \tcode{basic_simd} object such that the $i^\text{th}$ element is
    assigned with \tcode{static_cast<T>(to_address(first)[$i$])} \foralli.}
\end{itemdescr}

\begin{itemdecl}
template<class It, class... Flags>
  constexpr void copy_from(It first, const mask_type& mask, simd_flags<Flags...> f = {});
\end{itemdecl}

\begin{itemdescr}
  \SimdLoadDescr
    {\validMaskedRange}
    {Replaces the selected elements of the \tcode{basic_simd} object such that the $i^\text{th}$
    element is replaced with \tcode{static_cast<T>(to_address(first)[$i$])} \forallmaskedi.}
\end{itemdescr}

\newcommand\SimdStoreDescr[2]{
  \pnum\constraints
  \begin{itemize}
    \item \tcode{iter_value_t<Out>} is a vectorizable type, and
    \item \tcode{Out} models \tcode{contiguous_iterator}, and
    \item \tcode{Out} models \tcode{indirectly_writable<value_type>}.
  \end{itemize}

  \pnum\mandates
  \conversionFlagsMandate{value_type}{iter_value_t<Out>}

  \pnum\expects
  \begin{itemize}
    \item #1
    \flagsRequires{basic_simd, iter_value_t<Out>}{iter_value_t<Out>}
  \end{itemize}

  \pnum\effects #2
}

\begin{itemdecl}
template<class Out, class... Flags>
  constexpr void copy_to(Out first, simd_flags<Flags...> f = {}) const;
\end{itemdecl}

\begin{itemdescr}
  \SimdStoreDescr
    {\range{first}{first + size()} is a valid range.}
    {Copies all \tcode{basic_simd} elements as if \tcode{to_address(first)[$i$] =
    static_cast<iter_value_t<Out>>(operator[]($i$))} \foralli.}
\end{itemdescr}

\begin{itemdecl}
template<class Out, class... Flags>
  constexpr void copy_to(Out first, const mask_type& mask, simd_flags<Flags...> f = {}) const;
\end{itemdecl}

\begin{itemdescr}
  \SimdStoreDescr
    {\validMaskedRange}
    {Copies the selected elements as if \tcode{to_address(first)[$i$] =
    static_cast<iter_value_t<Out>>(operator[]($i$))} \forallmaskedi.}
\end{itemdescr}

\rSec2[simd.subscr]{\tcode{basic_simd} subscript operator}

\begin{itemdecl}
constexpr value_type operator[](@\simdsizetype@ i) const;
\end{itemdecl}

\begin{itemdescr}
  \pnum\expects
  \tcode{i >= 0 \&\& i < size()} is \tcode{true}.

  \pnum\returns
  The value of the $i^\text{th}$ element.

  \pnum\throws Nothing.
\end{itemdescr}

\rSec2[simd.unary]{\tcode{basic_simd} unary operators}

\pnum
Effects in [simd.unary] are applied as unary element-wise operations.

\begin{itemdecl}
constexpr basic_simd& operator++() noexcept;
\end{itemdecl}

\begin{itemdescr}
  \pnum\ConstraintUnaryOperatorWellFormed[]{++a}

  \pnum\effects
  Increments every element by one.

  \pnum\returns
  \tcode{*this}.
\end{itemdescr}

\begin{itemdecl}
constexpr basic_simd operator++(int) noexcept;
\end{itemdecl}

\begin{itemdescr}
  \pnum\ConstraintUnaryOperatorWellFormed[]{a++}

  \pnum\effects
  Increments every element by one.

  \pnum\returns
  A copy of \tcode{*this} before incrementing.
\end{itemdescr}

\begin{itemdecl}
constexpr basic_simd& operator--() noexcept;
\end{itemdecl}

\begin{itemdescr}
  \pnum\ConstraintUnaryOperatorWellFormed[]{--a}

  \pnum\effects
  Decrements every element by one.

  \pnum\returns
  \tcode{*this}.
\end{itemdescr}

\begin{itemdecl}
constexpr basic_simd operator--(int) noexcept;
\end{itemdecl}

\begin{itemdescr}
  \pnum\ConstraintUnaryOperatorWellFormed[]{a--}

  \pnum\effects
  Decrements every element by one.

  \pnum\returns
  A copy of \tcode{*this} before decrementing.
\end{itemdescr}

\begin{itemdecl}
constexpr mask_type operator!() const noexcept;
\end{itemdecl}

\begin{itemdescr}
  \pnum\ConstraintUnaryOperatorWellFormed{!a}

  \pnum\returns
  A \tcode{basic_simd_mask} object with the $i^\text{th}$ element set to \tcode{!operator[]($i$)}
  \foralli.
\end{itemdescr}

\begin{itemdecl}
constexpr basic_simd operator~() const noexcept;
\end{itemdecl}

\begin{itemdescr}
  \pnum\ConstraintUnaryOperatorWellFormed{\~{}a}

  \pnum\returns
  A \tcode{basic_simd} object with the $i^\text{th}$ element set to \tcode{\~{}operator[]($i$)}
  \foralli.
\end{itemdescr}

\begin{itemdecl}
constexpr basic_simd operator+() const noexcept;
\end{itemdecl}

\begin{itemdescr}
  \pnum\ConstraintUnaryOperatorWellFormed{+a}

  \pnum\returns
  \tcode{*this}.
\end{itemdescr}

\begin{itemdecl}
constexpr basic_simd operator-() const noexcept;
\end{itemdecl}

\begin{itemdescr}
  \pnum\ConstraintUnaryOperatorWellFormed{-a}

  \pnum\returns
  A \tcode{basic_simd} object where the $i^\text{th}$ element is initialized to
  \tcode{-operator[]($i$)} \foralli.
\end{itemdescr}

\rSec1[simd.nonmembers]{\tcode{basic_simd} non-member operations}

\rSec2[simd.binary]{\tcode{basic_simd} binary operators}

\begin{itemdecl}
friend constexpr basic_simd operator+(const basic_simd& lhs, const basic_simd& rhs) noexcept;
friend constexpr basic_simd operator-(const basic_simd& lhs, const basic_simd& rhs) noexcept;
friend constexpr basic_simd operator*(const basic_simd& lhs, const basic_simd& rhs) noexcept;
friend constexpr basic_simd operator/(const basic_simd& lhs, const basic_simd& rhs) noexcept;
friend constexpr basic_simd operator%(const basic_simd& lhs, const basic_simd& rhs) noexcept;
friend constexpr basic_simd operator&(const basic_simd& lhs, const basic_simd& rhs) noexcept;
friend constexpr basic_simd operator|(const basic_simd& lhs, const basic_simd& rhs) noexcept;
friend constexpr basic_simd operator^(const basic_simd& lhs, const basic_simd& rhs) noexcept;
friend constexpr basic_simd operator<<(const basic_simd& lhs, const basic_simd& rhs) noexcept;
friend constexpr basic_simd operator>>(const basic_simd& lhs, const basic_simd& rhs) noexcept;
\end{itemdecl}

\begin{itemdescr}
  \pnum Let \op{} be the operator.

  \pnum\ConstraintOperatorTWellFormed

  \pnum\returns
  A \tcode{basic_simd} object initialized with the results of applying \op{} to \tcode{lhs} and
  \tcode{rhs} as a binary element-wise operation.
\end{itemdescr}

\begin{itemdecl}
friend constexpr basic_simd operator<<(const basic_simd& v, @\simdsizetype@ n) noexcept;
friend constexpr basic_simd operator>>(const basic_simd& v, @\simdsizetype@ n) noexcept;
\end{itemdecl}

\begin{itemdescr}
  \pnum Let \textit{op} be the operator.

  \pnum\constraints
  \tcode{requires (value_type a, \simdsizetype{} b) \{ a \op{} b; \}} is \tcode{true}.

  \pnum\returns
  A \tcode{basic_simd} object where the $i^\text{th}$ element is initialized to the result of
  applying \op{} to \tcode{v[$i$]} and \tcode{n} \foralli.
\end{itemdescr}

\rSec2[simd.cassign]{\tcode{basic_simd} compound assignment}

\begin{itemdecl}
friend constexpr basic_simd& operator+=(basic_simd& lhs, const basic_simd& rhs) noexcept;
friend constexpr basic_simd& operator-=(basic_simd& lhs, const basic_simd& rhs) noexcept;
friend constexpr basic_simd& operator*=(basic_simd& lhs, const basic_simd& rhs) noexcept;
friend constexpr basic_simd& operator/=(basic_simd& lhs, const basic_simd& rhs) noexcept;
friend constexpr basic_simd& operator%=(basic_simd& lhs, const basic_simd& rhs) noexcept;
friend constexpr basic_simd& operator&=(basic_simd& lhs, const basic_simd& rhs) noexcept;
friend constexpr basic_simd& operator|=(basic_simd& lhs, const basic_simd& rhs) noexcept;
friend constexpr basic_simd& operator^=(basic_simd& lhs, const basic_simd& rhs) noexcept;
friend constexpr basic_simd& operator<<=(basic_simd& lhs, const basic_simd& rhs) noexcept;
friend constexpr basic_simd& operator>>=(basic_simd& lhs, const basic_simd& rhs) noexcept;
\end{itemdecl}

\begin{itemdescr}
  \pnum Let \textit{op} be the operator.

  \pnum\ConstraintOperatorTWellFormed

  \pnum\effects
  These operators apply the indicated operator to \tcode{lhs} and \tcode{rhs} as an element-wise
  operation.

  \pnum\returns
  \tcode{lhs}.
\end{itemdescr}

\begin{itemdecl}
friend constexpr basic_simd& operator<<=(basic_simd& lhs, @\simdsizetype@ n) noexcept;
friend constexpr basic_simd& operator>>=(basic_simd& lhs, @\simdsizetype@ n) noexcept;
\end{itemdecl}

\begin{itemdescr}
  \pnum Let \textit{op} be the operator.

  \pnum\constraints
  \tcode{requires (value_type a, \simdsizetype{} b) \{ a \op{} b; \}} is \tcode{true}.

  \pnum\effects
  Equivalent to: \tcode{return operator \textit{op} (lhs, basic_simd(n));}
\end{itemdescr}

\rSec2[simd.comparison]{\tcode{basic_simd} compare operators}

\begin{itemdecl}
friend constexpr mask_type operator==(const basic_simd& lhs, const basic_simd& rhs) noexcept;
friend constexpr mask_type operator!=(const basic_simd& lhs, const basic_simd& rhs) noexcept;
friend constexpr mask_type operator>=(const basic_simd& lhs, const basic_simd& rhs) noexcept;
friend constexpr mask_type operator<=(const basic_simd& lhs, const basic_simd& rhs) noexcept;
friend constexpr mask_type operator>(const basic_simd& lhs, const basic_simd& rhs) noexcept;
friend constexpr mask_type operator<(const basic_simd& lhs, const basic_simd& rhs) noexcept;
\end{itemdecl}

\begin{itemdescr}
  \pnum Let \textit{op} be the operator.

  \pnum\ConstraintOperatorTWellFormed

  \pnum\returns
  A \tcode{basic_simd_mask} object initialized with the results of applying \op{} to \tcode{lhs} and
  \tcode{rhs} as a binary element-wise operation.
\end{itemdescr}

\rSec2[simd.cond]{\tcode{basic_simd} exposition-only conditional operators}

\begin{itemdecl}
friend constexpr basic_simd
@\simdselect@(const mask_type& mask, const basic_simd& a, const basic_simd& b) noexcept;
\end{itemdecl}

\begin{itemdescr}
  \pnum\returns
  A \tcode{basic_simd} object where the $i^\text{th}$ element equals \tcode{mask[$i$] ? a[$i$] :
  b[$i$]} \foralli.
\end{itemdescr}

\rSec2[simd.reductions]{\tcode{basic_simd} reductions}

\pnum
In [simd.reductions], \tcode{BinaryOperation} shall be a binary element-wise operation.

\begin{itemdecl}
template<class T, class Abi, class BinaryOperation = plus<>>
  constexpr T reduce(const basic_simd<T, Abi>& x, BinaryOperation binary_op = {});
\end{itemdecl}

\begin{itemdescr}
  \pnum\constraints
  \tcode{BinaryOperation} satisfies \tcode{invocable<simd<T, 1>, simd<T, 1>>}.

  \pnum\mandates
  \tcode{binary_op} can be invoked with two arguments of type
  \tcode{basic_simd<T, A1>} returning \tcode{basic_simd<T, A1>} for every
  \tcode{A1} that is an ABI tag type.
  \INFO{Better alternative? “[\ldots] for zero or more unspecified ABI tag types \tcode{A1}.”}
  \FIXME{This is not supposed to require exhaustive testing of all ABI tags.
    What we need to express is that the user-supplied \tcode{binary_op} \emph{can} be called
    with every possible ABI tag since different implementations / compiler flags / targets will lead
    to a different subset getting called.
    Basically, (start waving hands) “\tcode{binary_op} can be invoked with the specializations of
  \tcode{basic_simd} that the implementation needs” (stop waving hands).}

  \pnum\expects
  \tcode{BinaryOperation} does not modify \tcode{x}.

  \pnum\returns
  \tcode{\placeholdernc{GENERALIZED_SUM}(binary_op, simd<T, 1>(x[$i$]), ...)[0]} \foralli{}
  (\iref{numerics.defns}).

  \pnum\throws
  Any exception thrown from \tcode{binary_op}.
\end{itemdescr}

\begin{itemdecl}
template<class T, class Abi, class BinaryOperation>
  constexpr T reduce(const basic_simd<T, Abi>& x, const typename basic_simd<T, Abi>::mask_type& mask,
                     type_identity_t<T> identity_element, BinaryOperation binary_op);
\end{itemdecl}

\begin{itemdescr}
  \pnum\constraints
  \tcode{BinaryOperation} satisfies \tcode{invocable<simd<T, 1>, simd<T, 1>>}.

  \pnum\mandates
  \tcode{binary_op} can be invoked with two arguments of type \tcode{basic_simd<T, A1>} returning
  \tcode{basic_simd<T, A1>} for every \tcode{A1} that is an ABI tag type.

  \pnum\expects
  \begin{itemize}
    \item \tcode{BinaryOperation} does not modify \tcode{x}.

    \item For every \tcode{A1} that is an ABI tag type and for all finite
      values \tcode{y} representable by \tcode{T}, the results of
      \tcode{all_of(y == binary_op(basic_simd<T, A1>(identity_element),
      basic_simd<T, A1>(y)))} and \tcode{all_of(basic_simd<T, A1>(y) ==
      binary_op(y, basic_simd<T, A1>(identity_element)))} are \tcode{true}.
  \end{itemize}

  \pnum\returns
  If \tcode{none_of(mask)} is \tcode{true}, returns \tcode{identity_element}.
  Otherwise, returns \tcode{\placeholdernc{GENERALIZED_SUM}(binary_op, simd<T, 1>(x[$i$]), ...)[0]}
  \forallmaskedi.

  \pnum\throws
  Any exception thrown from \tcode{binary_op}.
\end{itemdescr}

\begin{itemdecl}
template<class T, class Abi>
  constexpr T reduce(const basic_simd<T, Abi>& x, const typename basic_simd<T, Abi>::mask_type& mask,
                     plus<> binary_op = {}) noexcept;
\end{itemdecl}

\begin{itemdescr}
  \pnum\returns
  If \tcode{none_of(mask)} is \tcode{true}, returns \tcode{T()}. Otherwise, returns
  \tcode{\placeholdernc{GENERALIZED_SUM}(binary_op, x[$i$], ...)} \forallmaskedi.
\end{itemdescr}

\begin{itemdecl}
template<class T, class Abi>
  constexpr T reduce(const basic_simd<T, Abi>& x, const typename basic_simd<T, Abi>::mask_type& mask,
                     multiplies<> binary_op) noexcept;
\end{itemdecl}

\begin{itemdescr}
  \pnum\returns
  If \tcode{none_of(x)} is \tcode{true}, returns \tcode{T(1)}. Otherwise, returns
  \tcode{\placeholdernc{GENERALIZED_SUM}(binary_op, x[$i$], ...)} \forallmaskedi.
\end{itemdescr}

\begin{itemdecl}
template<class T, class Abi>
  constexpr T reduce(const basic_simd<T, Abi>& x, const typename basic_simd<T, Abi>::mask_type& mask,
                     bit_and<> binary_op) noexcept;
\end{itemdecl}

\begin{itemdescr}
  \pnum\constraints
  \tcode{is_integral_v<T>} is \tcode{true}.

  \pnum\returns
  If \tcode{none_of(mask)} is \tcode{true}, returns \tcode{T(\~{}T())}. Otherwise, returns
  \tcode{\placeholdernc{GENERALIZED_SUM}(binary_op, x[$i$], ...)} \forallmaskedi.
\end{itemdescr}

\begin{itemdecl}
template<class T, class Abi>
  constexpr T reduce(const basic_simd<T, Abi>& x, const typename basic_simd<T, Abi>::mask_type& mask,
                     bit_or<> binary_op) noexcept;
template<class T, class Abi>
  constexpr T reduce(const basic_simd<T, Abi>& x, const typename basic_simd<T, Abi>::mask_type& mask,
                     bit_xor<> binary_op) noexcept;
\end{itemdecl}

\begin{itemdescr}
  \pnum\constraints
  \tcode{is_integral_v<T>} is \tcode{true}.

  \pnum\returns
  If \tcode{none_of(mask)} is \tcode{true}, returns \tcode{T()}. Otherwise, returns
  \tcode{\placeholdernc{GENERALIZED_SUM}(binary_op, x[$i$], ...)} \forallmaskedi.
\end{itemdescr}

\begin{itemdecl}
template<class T, class Abi> constexpr T reduce_min(const basic_simd<T, Abi>& x) noexcept;
\end{itemdecl}

\begin{itemdescr}
  \pnum\constraints
  \tcode{T} models \tcode{totally_ordered}.

  \pnum\returns
  The value of an element \tcode{x[$j$]} for which \tcode{x[$i$] < x[$j$]} is \tcode{false}
  \foralli.
\end{itemdescr}

\begin{itemdecl}
template<class T, class Abi>
  constexpr T reduce_min(
    const basic_simd<T, Abi>&, const typename basic_simd<T, Abi>::mask_type&) noexcept;
\end{itemdecl}

\begin{itemdescr}
  \pnum\constraints
  \tcode{T} models \tcode{totally_ordered}.

  \pnum\returns
  If \tcode{none_of(mask)} is \tcode{true}, returns \tcode{numeric_limits<T>::max()}.
  Otherwise, returns the value of a selected element \tcode{x[$j$]} for which \tcode{x[$i$] <
  x[$j$]} is \tcode{false} \forallmaskedi.
\end{itemdescr}

\begin{itemdecl}
template<class T, class Abi> constexpr T reduce_max(const basic_simd<T, Abi>& x) noexcept;
\end{itemdecl}

\begin{itemdescr}
  \pnum\constraints
  \tcode{T} models \tcode{totally_ordered}.

  \pnum\returns
  The value of an element \tcode{x[$j$]} for which \tcode{x[$j$] < x[$i$]} is \tcode{false}
  \foralli.
\end{itemdescr}

\begin{itemdecl}
template<class T, class Abi>
  constexpr T reduce_max(
    const basic_simd<T, Abi>&, const typename basic_simd<T, Abi>::mask_type&) noexcept;
\end{itemdecl}

\begin{itemdescr}
  \pnum\constraints
  \tcode{T} models \tcode{totally_ordered}.

  \pnum\returns
  If \tcode{none_of(mask)} is \tcode{true}, returns \tcode{numeric_limits<V::value_type>::lowest()}.
  Otherwise, returns the value of a selected element \tcode{x[$j$]} for which \tcode{x[$j$] <
  x[$i$]} is \tcode{false} \forallmaskedi.
\end{itemdescr}

\rSec2[simd.creation]{\tcode{basic_simd} and \tcode{basic_simd_mask} creation}

\begin{itemdecl}
template<class T, class Abi>
  constexpr auto simd_split(const basic_simd<typename T::value_type, Abi>& x) noexcept;
template<class T, class Abi>
  constexpr auto simd_split(const basic_simd_mask<@\maskelementsize@<T>, Abi>& x) noexcept;
\end{itemdecl}

\begin{itemdescr}
  % probably not necessary/helpful:
  %\pnum\mandates \tcode{T::size() <= \simdsizev<V::value_type, Abi>}.

  \pnum\constraints
  \begin{itemize}
    \item For the first overload \tcode{T} is a specialization of \tcode{basic_simd}.
    \item For the second overload \tcode{T} is a specialization of \tcode{basic_simd_mask}.
  \end{itemize}

  \pnum Let $N$ be \tcode{x.size() / T::size()}.

    \pnum\returns
    \begin{itemize}
      \item If \tcode{x.size() \% T::size() == 0}, an \tcode{array<T, $N$>} with
        the $i^\text{th}$ \simd or \mask element of the $j^\text{th}$ \tcode{array}
        element initialized to the value of the element in \tcode{x} with index
        \tcode{$i$ + $j$ * T::size()}.

      \item Otherwise, a \tcode{tuple} of $N$ objects of type \tcode{T} and one
        object of type \tcode{resize_simd_t<x.size() \% T::size(), T>}.
        The $i^\text{th}$ \simd or \mask element of the $j^\text{th}$
        \tcode{tuple} element of type \tcode{T} is initialized to the value of
        the element in \tcode{x} with index \tcode{$i$ + $j$ * T::size()}.
        The $i^\text{th}$ \simd or \mask element of the \tcode{N}$^\text{th}$
        \tcode{tuple} element is initialized to the value of the element in
        \tcode{x} with index \tcode{$i$ + $N$ * T::size()}.
    \end{itemize}
  \end{itemdescr}

\begin{itemdecl}
template<class T, class... Abis>
  constexpr simd<T, (basic_simd<T, Abis>::size() + ...)>
    simd_cat(const basic_simd<T, Abis>&... xs) noexcept;
template<size_t Bytes, class... Abis>
  constexpr simd_mask<@\deducet@<@\integerfrom@<Bytes>, (basic_simd_mask<Bytes, Abis>::size() + ...)>
    simd_cat(const basic_simd_mask<Bytes, Abis>&... xs) noexcept;

\end{itemdecl}

\begin{itemdescr}
  \pnum\returns
  A data-parallel object initialized with the concatenated values in the \tcode{xs} pack of
  data-parallel objects: The $i^\text{th}$ \tcode{basic_simd}/\tcode{basic_simd_mask} element of the
  $j^\text{th}$ parameter in the \tcode{xs} pack is copied to the return value's element with index
  $i$ + the sum of the width of the first $j$ parameters in the \tcode{xs} pack.
\end{itemdescr}

\rSec2[simd.alg]{Algorithms}

\begin{itemdecl}
template<class T, class Abi>
  constexpr basic_simd<T, Abi> min(const basic_simd<T, Abi>& a, const basic_simd<T, Abi>& b) noexcept;
\end{itemdecl}

\begin{itemdescr}
  \pnum\constraints
  \tcode{T} models \tcode{totally_ordered}.

  \pnum\returns
  The result of the element-wise application of \tcode{std::min(a[$i$], b[$i$])} \foralli.
\end{itemdescr}

\begin{itemdecl}
template<class T, class Abi>
  constexpr basic_simd<T, Abi> max(const basic_simd<T, Abi>& a, const basic_simd<T, Abi>& b) noexcept;
\end{itemdecl}

\begin{itemdescr}
  \pnum\constraints
  \tcode{T} models \tcode{totally_ordered}.

  \pnum\returns
  The result of the element-wise application of \tcode{std::max(a[$i$], b[$i$])} \foralli.
\end{itemdescr}

\begin{itemdecl}
template<class T, class Abi>
  constexpr pair<basic_simd<T, Abi>, basic_simd<T, Abi>>
  minmax(const basic_simd<T, Abi>& a, const basic_simd<T, Abi>& b) noexcept;
\end{itemdecl}

\begin{itemdescr}
  \pnum\constraints
  \tcode{T} models \tcode{totally_ordered}.

  \pnum\returns
  A \tcode{pair} initialized with
  \begin{itemize}
    \item the result of element-wise application of \tcode{std::min(a[$i$], b[$i$])} \foralli{} in
      the \tcode{first} member, and
    \item the result of element-wise application of \tcode{std::max(a[$i$], b[$i$])} \foralli{} in
      the \tcode{second} member.
  \end{itemize}
\end{itemdescr}

\begin{itemdecl}
template<class T, class Abi>
  constexpr basic_simd<T, Abi> clamp(
    const basic_simd<T, Abi>& v, const basic_simd<T, Abi>& lo, const basic_simd<T, Abi>& hi);
\end{itemdecl}

\begin{itemdescr}
  \pnum\constraints
  \tcode{T} models \tcode{totally_ordered}.

  \pnum\expects
  No element in \tcode{lo} shall be greater than the corresponding element in \tcode{hi}.

  \pnum\returns
  The result of element-wise application of \tcode{std::clamp(v[$i$], lo[$i$], hi[$i$])} \foralli.
\end{itemdescr}

\begin{itemdecl}
  template<class T, class U>
    constexpr auto simd_select(bool c, const T& a, const U& b)
    -> remove_cvref_t<decltype(c ? a : b)>;
\end{itemdecl}

\begin{itemdescr}
    \pnum\returns As-if \tcode{c ? a : b}.
\end{itemdescr}

\begin{itemdecl}
  template<size_t Bytes, class Abi, class T, class U>
    constexpr auto simd_select(const basic_simd_mask<Bytes, Abi>& c, const T& a, const U& b)
    noexcept -> decltype(@\simdselect@(c, a, b));
\end{itemdecl}

\begin{itemdescr}
    \pnum\returns As-if \tcode{\simdselect(c, a, b)}.
\end{itemdescr}

\rSec2[simd.math]{\tcode{basic_simd} math library}

\pnum
For each set of overloaded functions within \tcode{<cmath>}, there shall be additional overloads
sufficient to ensure that if any argument corresponding to a \tcode{double} parameter has type
\tcode{basic_simd<T, Abi>}, where \tcode{is_floating_point_v<T>} is \tcode{true}, then:
\begin{itemize}
  \item All arguments corresponding to \tcode{double} parameters shall be convertible to
    \tcode{basic_simd<T, Abi>}.
  \item All arguments corresponding to \tcode{double*} parameters shall be of type
    \tcode{basic_simd<T, Abi>*}.
  \item All arguments corresponding to parameters of integral type \tcode{U} shall be convertible to
    \tcode{rebind_simd_t<U, basic_simd<T, Abi>>}.
  \item All arguments corresponding to \tcode{U*}, where \tcode{U} is integral, shall be of type
    \tcode{rebind_simd_t<U, basic_simd<T, Abi>>*}.
  \item If the corresponding return type is \tcode{double}, the return type of the additional
    overloads is \tcode{basic_simd<T, Abi>}. Otherwise, if the corresponding return type is
    \tcode{bool}, the return type of the additional overload is \tcode{basic_simd<T,
    Abi>::mask_type}. Otherwise, the return type is \tcode{rebind_simd_t<R, basic_simd<T, Abi>>},
    with \tcode{R} denoting the corresponding return type.
\end{itemize}
It is unspecified whether a call to these overloads with arguments that are all convertible to
\tcode{basic_simd<T, Abi>} but are not of type \tcode{basic_simd<T, Abi>} is well-formed.

\pnum
Each function overload produced by the above rules applies the indicated \tcode{<cmath>} function
element-wise. For the mathematical functions, the results per element only need to be approximately
equal to the application of the function which is overloaded for the element type.

\pnum
The result is unspecified if a domain, pole, or range error occurs when the input argument(s) are
applied to the indicated \tcode{<cmath>} function.
\begin{note}Implementations are encouraged to follow the C specification (especially Annex
F).\end{note}

\pnum
\FIXME{Allow \tcode{abs(basic_simd<\textrm{signed-integral}>)}.}

\pnum
If \tcode{abs} is called with an argument of type \tcode{basic_simd<X, Abi>} for which
\tcode{is_unsigned_v<X>} is \tcode{true}, the program is ill-formed.

\rSec1[simd.mask.class]{Class template \tcode{basic_simd_mask}}

\rSec2[simd.mask.overview]{Class template \tcode{basic_simd_mask} overview}

\begin{codeblock}
template<size_t Bytes, class Abi> class basic_simd_mask {
public:
  using value_type = bool;
  using abi_type = Abi;

  static constexpr integral_constant<@\simdsizetype@, @\simdsizev@<@\integerfrom@<Bytes>, Abi>> size {};

  constexpr basic_simd_mask() noexcept = default;

  // \ref{simd.mask.ctor}, \tcode{basic_simd_mask} constructors
  constexpr explicit basic_simd_mask(value_type) noexcept;
  template<size_t UBytes, class UAbi>
    constexpr explicit basic_simd_mask(const basic_simd_mask<UBytes, UAbi>&) noexcept;
  template<class G> constexpr explicit basic_simd_mask(G&& gen) noexcept;
  template<class It, class... Flags>
    constexpr basic_simd_mask(It first, Flags = {});
  template<class It, class... Flags>
    constexpr basic_simd_mask(It first, const basic_simd_mask& mask, simd_flags<Flags...> = {});

  // \ref{simd.mask.copy}, \tcode{basic_simd_mask} copy functions
  template<class It, class... Flags>
    constexpr void copy_from(It first, simd_flags<Flags...> = {});
  template<class It, class... Flags>
    constexpr void copy_from(It first, const basic_simd_mask& mask, simd_flags<Flags...> = {});
  template<class Out, class... Flags>
    constexpr void copy_to(Out first, simd_flags<Flags...> = {}) const;
  template<class Out, class... Flags>
    constexpr void copy_to(Out first, const basic_simd_mask& mask, simd_flags<Flags...> = {}) const;

  // \ref{simd.mask.subscr}, \tcode{basic_simd_mask} subscript operators
  constexpr value_type operator[](@\simdsizetype@) const;

  // \ref{simd.mask.unary}, \tcode{basic_simd_mask} unary operators
  constexpr basic_simd_mask operator!() const noexcept;
  constexpr basic_simd<@\integerfrom@<Bytes>, Abi> operator+() const noexcept;
  constexpr basic_simd<@\integerfrom@<Bytes>, Abi> operator-() const noexcept;
  constexpr basic_simd<@\integerfrom@<Bytes>, Abi> operator~() const noexcept;

  // \ref{simd.mask.conv}, \tcode{basic_simd_mask} conversion operators
  template <class U, class A>
    constexpr explicit(sizeof(U) != Bytes) operator basic_simd<U, A>() const noexcept;

  // \ref{simd.mask.binary}, \tcode{basic_simd_mask} binary operators
  friend constexpr basic_simd_mask
    operator&&(const basic_simd_mask&, const basic_simd_mask&) noexcept;
  friend constexpr basic_simd_mask
    operator||(const basic_simd_mask&, const basic_simd_mask&) noexcept;
  friend constexpr basic_simd_mask
    operator&(const basic_simd_mask&, const basic_simd_mask&) noexcept;
  friend constexpr basic_simd_mask
    operator|(const basic_simd_mask&, const basic_simd_mask&) noexcept;
  friend constexpr basic_simd_mask
    operator^(const basic_simd_mask&, const basic_simd_mask&) noexcept;

  // \ref{simd.mask.cassign}, \tcode{basic_simd_mask} compound assignment
  friend constexpr basic_simd_mask&
    operator&=(basic_simd_mask&, const basic_simd_mask&) noexcept;
  friend constexpr basic_simd_mask&
    operator|=(basic_simd_mask&, const basic_simd_mask&) noexcept;
  friend constexpr basic_simd_mask&
    operator^=(basic_simd_mask&, const basic_simd_mask&) noexcept;

  // \ref{simd.mask.comparison}, \tcode{basic_simd_mask} comparisons
  friend constexpr basic_simd_mask
    operator==(const basic_simd_mask&, const basic_simd_mask&) noexcept;
  friend constexpr basic_simd_mask
    operator!=(const basic_simd_mask&, const basic_simd_mask&) noexcept;
  friend constexpr basic_simd_mask
    operator>=(const basic_simd_mask&, const basic_simd_mask&) noexcept;
  friend constexpr basic_simd_mask
    operator<=(const basic_simd_mask&, const basic_simd_mask&) noexcept;
  friend constexpr basic_simd_mask
    operator>(const basic_simd_mask&, const basic_simd_mask&) noexcept;
  friend constexpr basic_simd_mask
    operator<(const basic_simd_mask&, const basic_simd_mask&) noexcept;

  // \ref{simd.mask.cond}, \tcode{basic_simd_mask} exposition-only conditional operators
  friend constexpr basic_simd_mask @\simdselect@(
    const basic_simd_mask&, const basic_simd_mask&, const basic_simd_mask&) noexcept;
  friend constexpr basic_simd_mask @\simdselect@(
    const basic_simd_mask&, same_as<bool> auto, same_as<bool> auto) noexcept;
  template <class T0, class T1>
    friend constexpr simd<@\seebelow@, size()>
      @\simdselect@(const basic_simd_mask&, const T0&, const T1&) noexcept;
};
\end{codeblock}

\pnum
The specializations of class template \tcode{basic_simd_mask} are data-parallel types with element
type \tcode{bool}.

\pnum
Every specialization of \tcode{basic_simd_mask} is a complete type.
The types \tcode{basic_simd_mask<sizeof(T), \deducet<T, N>>} for all vectorizable
\tcode{T} and with \tcode{N} in the range of \crange{1}{64} are enabled.
It is implementation-defined whether any other \tcode{basic_simd_mask<sizeof(T), Abi>}
specialization with vectorizable \tcode{T} is enabled.
Any other specialization of \tcode{basic_simd_mask} is disabled.

\begin{note}
  The intent is for implementations to determine on the basis of the currently
  targeted system, whether \tcode{basic_simd_mask<Bytes, Abi>} is enabled.
\end{note}
\FIXME{drop the note?}

If \tcode{basic_simd_mask<Bytes, Abi>} is disabled, the specialization has a deleted
default constructor, deleted destructor, deleted copy constructor, and deleted copy assignment.
In addition only the \tcode{value_type} and \tcode{abi_type} members are present.

If \tcode{basic_simd_mask<Bytes, Abi>} is enabled, \tcode{basic_simd_mask<Bytes, Abi>} is
trivially copyable.

\pnum
Implementations should enable explicit conversion from and to implementation-defined types. This
adds one or more of the following declarations to class \tcode{basic_simd_mask}:

\begin{codeblock}
constexpr explicit operator @\impdef@() const;
constexpr explicit basic_simd_mask(const @\impdef@& init);
\end{codeblock}

\rSec2[simd.mask.ctor]{\tcode{basic_simd_mask} constructors}

\begin{itemdecl}
constexpr explicit basic_simd_mask(value_type x) noexcept;
\end{itemdecl}

\begin{itemdescr}
  \pnum\effects
  Initializes each element with \tcode{x}.
\end{itemdescr}

\begin{itemdecl}
template<size_t UBytes, class UAbi>
  constexpr explicit basic_simd_mask(const basic_simd_mask<UBytes, UAbi>& x) noexcept;
\end{itemdecl}

\begin{itemdescr}
  \pnum\constraints
  \tcode{\simdsizev<U, UAbi> == size()}.

  \pnum\effects
  Initializes the $i^\text{th}$ element with \tcode{x[$i$]} \foralli.
\end{itemdescr}

\begin{itemdecl}
template<class G> constexpr explicit basic_simd_mask(G&& gen) noexcept;
\end{itemdecl}

\begin{itemdescr}
  \pnum\constraints
  \tcode{static_cast<bool>(gen(integral_constant<\simdsizetype, i>()))} is
  well-formed \foralli.

  \pnum\effects
  Initializes the $i^\text{th}$ element with
  \tcode{gen(integral_constant<\simdsizetype, i>())} \foralli.

  \pnum
  The calls to \tcode{gen} are unsequenced with respect to each other.
  Vectorization-unsafe standard library functions may not be invoked by \tcode{gen}
  (\iref{algorithms.parallel.exec}).
\end{itemdescr}

\newcommand\MaskLoadDescr[2]{
  \pnum\constraints
  \begin{itemize}
    \item \tcode{iter_value_t<It>} is of type \tcode{bool}, and
    \item \tcode{It} models \tcode{contiguous_iterator}.
  \end{itemize}

  \pnum\expects
  \begin{itemize}
    \item #1
    \flagsRequires{basic_simd_mask}{value_type}
  \end{itemize}

  \pnum\effects #2

  \pnum\throws Nothing.
}

\begin{itemdecl}
template<class It, class... Flags>
  constexpr basic_simd_mask(It first, simd_flags<Flags...> = {});
\end{itemdecl}

\begin{itemdescr}
  \MaskLoadDescr
    {\range{first}{first + size()} is a valid range.}
    {Initializes the $i^\text{th}$ element with \tcode{first[$i$]} \foralli.}
\end{itemdescr}

\begin{itemdecl}
template<class It, class... Flags>
  constexpr basic_simd_mask(It first, const basic_simd_mask& mask, simd_flags<Flags...> = {});
\end{itemdecl}

\begin{itemdescr}
  \MaskLoadDescr
    {\validMaskedRange}
    {Initializes the $i^\text{th}$ element with \tcode{mask[$i$] ? first[$i$] : false} \foralli.}
\end{itemdescr}

\rSec2[simd.mask.copy]{\tcode{basic_simd_mask} copy functions}

\begin{itemdecl}
template<class It, class... Flags>
  constexpr void copy_from(It first, simd_flags<Flags...> = {});
\end{itemdecl}

\begin{itemdescr}
  \MaskLoadDescr
  {\range{first}{first + size()} is a valid range.}
  {Replaces the elements of the \tcode{basic_simd_mask} object such that the $i^\text{th}$ element
  is replaced with \tcode{first[$i$]} \foralli.}
\end{itemdescr}

\begin{itemdecl}
template<class It, class... Flags>
  constexpr void copy_from(It first, const basic_simd_mask& mask, simd_flags<Flags...> = {});
\end{itemdecl}

\begin{itemdescr}
  \MaskLoadDescr
    {\validMaskedRange}
    {Replaces the selected elements of the \tcode{basic_simd_mask} object such that the
    $i^\text{th}$ element is replaced with \tcode{first[$i$]} \forallmaskedi.}
\end{itemdescr}

\newcommand\MaskStoreDescr[2]{
  \pnum\constraints
  \begin{itemize}
    \item \tcode{iter_value_t<Out>} is of type \tcode{bool}, and
    \item \tcode{Out} models \tcode{contiguous_iterator}, and
    \item \tcode{Out} models \tcode{indirectly_writable<value_type>}.
  \end{itemize}

  \pnum\expects
  \begin{itemize}
    \item #1
    \flagsRequires{basic_simd_mask}{value_type}
  \end{itemize}

  \pnum\effects #2

  \pnum\throws Nothing.
}

\begin{itemdecl}
template<class Out, class... Flags>
  constexpr void copy_to(Out first, simd_flags<Flags...> = {}) const;
\end{itemdecl}

\begin{itemdescr}
  \MaskStoreDescr
    {\range{first}{first + size()} is a valid range.}
    {Copies all \tcode{basic_simd_mask} elements as if \tcode{first[$i$] = operator[]($i$)}
    \foralli.}
\end{itemdescr}

\begin{itemdecl}
template<class Out, class... Flags>
  constexpr void copy_to(Out first, const basic_simd_mask& mask, simd_flags<Flags...> = {}) const;
\end{itemdecl}

\begin{itemdescr}
  \MaskStoreDescr
  {\validMaskedRange}
  {Copies the selected elements as if \tcode{first[$i$] = operator[]($i$)} \forallmaskedi.}
\end{itemdescr}

\rSec2[simd.mask.subscr]{\tcode{basic_simd_mask} subscript operator}

\begin{itemdecl}
constexpr value_type operator[](@\simdsizetype@ i) const;
\end{itemdecl}

\begin{itemdescr}
  \pnum\expects
  \tcode{i >= 0 \&\& i < size()} is \tcode{true}.

  \pnum\returns
  The value of the $i^\text{th}$ element.

  \pnum\throws Nothing.
\end{itemdescr}

\rSec2[simd.mask.unary]{\tcode{basic_simd_mask} unary operators}

\begin{itemdecl}
constexpr basic_simd_mask operator!() const noexcept;
\end{itemdecl}

\begin{itemdescr}
  \pnum\returns
  The result of the element-wise application of \tcode{operator!}.
\end{itemdescr}

\begin{itemdecl}
constexpr basic_simd<@\integerfrom@<Bytes>, Abi> operator+() const noexcept;
constexpr basic_simd<@\integerfrom@<Bytes>, Abi> operator-() const noexcept;
constexpr basic_simd<@\integerfrom@<Bytes>, Abi> operator~() const noexcept;
\end{itemdecl}

\begin{itemdescr}
  \pnum\constraints
  Application of the indicated unary operator to objects of type \tcode{T} is well-formed.

  \pnum\returns
  The result of applying the indicated operator to \tcode{static_cast<simd_type>(*this)}.
\end{itemdescr}

\rSec2[simd.mask.conv]{\tcode{basic_simd_mask} conversion operators}

\begin{itemdecl}
template <class U, class A>
  constexpr explicit(sizeof(U) != Bytes) operator basic_simd<U, A>() const noexcept;
\end{itemdecl}

\begin{itemdescr}
  \pnum\constraints
  \tcode{\simdsizev<U, A> == \simdsizev<T, Abi>}.

  \pnum\returns
  An object where the $i^\text{th}$ element is initialized to
  \tcode{static_cast<U>(operator[]($i$))}.
\end{itemdescr}

\rSec1[simd.mask.nonmembers]{Non-member operations}

\rSec2[simd.mask.binary]{\tcode{basic_simd_mask} binary operators}

\begin{itemdecl}
friend constexpr basic_simd_mask
  operator&&(const basic_simd_mask& lhs, const basic_simd_mask& rhs) noexcept;
friend constexpr basic_simd_mask
  operator||(const basic_simd_mask& lhs, const basic_simd_mask& rhs) noexcept;
friend constexpr basic_simd_mask
  operator& (const basic_simd_mask& lhs, const basic_simd_mask& rhs) noexcept;
friend constexpr basic_simd_mask
  operator| (const basic_simd_mask& lhs, const basic_simd_mask& rhs) noexcept;
friend constexpr basic_simd_mask
  operator^ (const basic_simd_mask& lhs, const basic_simd_mask& rhs) noexcept;
\end{itemdecl}

\begin{itemdescr}
  \pnum\returns
  A \tcode{basic_simd_mask} object initialized with the results of applying the indicated operator
  to \tcode{lhs} and \tcode{rhs} as a binary element-wise operation.
\end{itemdescr}

\rSec2[simd.mask.cassign]{\tcode{basic_simd_mask} compound assignment}

\begin{itemdecl}
friend constexpr basic_simd_mask&
  operator&=(basic_simd_mask& lhs, const basic_simd_mask& rhs) noexcept;
friend constexpr basic_simd_mask&
  operator|=(basic_simd_mask& lhs, const basic_simd_mask& rhs) noexcept;
friend constexpr basic_simd_mask&
  operator^=(basic_simd_mask& lhs, const basic_simd_mask& rhs) noexcept;
\end{itemdecl}

\begin{itemdescr}
  \pnum\effects
  These operators apply the indicated operator to \tcode{lhs} and \tcode{rhs} as a binary
  element-wise operation.

  \pnum\returns
  \tcode{lhs}.
\end{itemdescr}

\rSec2[simd.mask.comparison]{\tcode{basic_simd_mask} comparisons}

\begin{itemdecl}
friend constexpr basic_simd_mask
  operator==(const basic_simd_mask&, const basic_simd_mask&) noexcept;
friend constexpr basic_simd_mask
  operator!=(const basic_simd_mask&, const basic_simd_mask&) noexcept;
friend constexpr basic_simd_mask
  operator>=(const basic_simd_mask&, const basic_simd_mask&) noexcept;
friend constexpr basic_simd_mask
  operator<=(const basic_simd_mask&, const basic_simd_mask&) noexcept;
friend constexpr basic_simd_mask
  operator>(const basic_simd_mask&, const basic_simd_mask&) noexcept;
friend constexpr basic_simd_mask
  operator<(const basic_simd_mask&, const basic_simd_mask&) noexcept;
\end{itemdecl}

\begin{itemdescr}
  \pnum\returns
  A \tcode{basic_simd_mask} object initialized with the results of applying the indicated operator
  to \tcode{lhs} and \tcode{rhs} as a binary element-wise operation.
\end{itemdescr}

\rSec2[simd.mask.cond]{\tcode{basic_simd_mask} exposition-only conditional operators}

\begin{itemdecl}
friend constexpr basic_simd_mask @\simdselect@(
  const basic_simd_mask& mask, const basic_simd_mask& a, const basic_simd_mask& b) noexcept;
\end{itemdecl}

\begin{itemdescr}
  \pnum\returns
  A \tcode{basic_simd_mask} object where the $i^\text{th}$ element equals \tcode{mask[$i$] ? a[$i$]
  : b[$i$]} \foralli.
\end{itemdescr}

\begin{itemdecl}
friend constexpr basic_simd_mask
@\simdselect@(const basic_simd_mask& mask, same_as<bool> auto a, same_as<bool> auto b) noexcept;
\end{itemdecl}

\begin{itemdescr}
  \pnum\returns
  A \tcode{basic_simd_mask} object where the $i^\text{th}$ element equals \tcode{mask[$i$] ? a : b}
  \foralli.
\end{itemdescr}

\begin{itemdecl}
template <class T0, class T1>
  friend constexpr simd<@\seebelow@, size()>
    @\simdselect@(const basic_simd_mask& mask, const T0& a, const T1& b) noexcept;
\end{itemdecl}

\begin{itemdescr}
  \pnum Let \tcode{U} be the common type of \tcode{T0} and \tcode{T1} without
  applying integral promotions on integral types with integer conversion rank
  less than the rank of \tcode{int}.

  \pnum\constraints
  \begin{itemize}
    \item \tcode{U} is a vectorizable type, and
    \item \tcode{sizeof(U) == Bytes}, and
    \item \tcode{T0} satisfies \tcode{convertible_to<simd<U, size()>>}, and
    \item \tcode{T1} satisfies \tcode{convertible_to<simd<U, size()>>}.
  \end{itemize}

  \pnum\returns
  A \tcode{basic_simd<U, Abi>} object where the $i^\text{th}$ element equals \tcode{mask[$i$] ? a :
  b} \foralli.
\end{itemdescr}

\rSec2[simd.mask.reductions]{\tcode{basic_simd_mask} reductions}

\begin{itemdecl}
template<size_t Bytes, class Abi>
  constexpr bool all_of(const basic_simd_mask<Bytes, Abi>& k) noexcept;
\end{itemdecl}

\begin{itemdescr}
  \pnum\returns
  \tcode{true} if all boolean elements in \tcode{k} are \tcode{true}, \tcode{false} otherwise.
\end{itemdescr}

\begin{itemdecl}
template<size_t Bytes, class Abi>
  constexpr bool any_of(const basic_simd_mask<Bytes, Abi>& k) noexcept;
\end{itemdecl}

\begin{itemdescr}
  \pnum\returns
  \tcode{true} if at least one boolean element in \tcode{k} is \tcode{true}, \tcode{false}
  otherwise.
\end{itemdescr}

\begin{itemdecl}
template<size_t Bytes, class Abi>
  constexpr bool none_of(const basic_simd_mask<Bytes, Abi>& k) noexcept;
\end{itemdecl}

\begin{itemdescr}
  \pnum\returns
  \tcode{true} if none of the one boolean elements in \tcode{k} is \tcode{true}, \tcode{false}
  otherwise.
\end{itemdescr}

\begin{itemdecl}
template<size_t Bytes, class Abi>
  constexpr @\simdsizetype@ reduce_count(const basic_simd_mask<Bytes, Abi>& k) noexcept;
\end{itemdecl}

\begin{itemdescr}
  \pnum\returns
  The number of boolean elements in \tcode{k} that are \tcode{true}.
\end{itemdescr}

\begin{itemdecl}
template<size_t Bytes, class Abi>
  constexpr @\simdsizetype@ reduce_min_index(const basic_simd_mask<Bytes, Abi>& k);
\end{itemdecl}

\begin{itemdescr}
  \pnum\expects
  \tcode{any_of(k)} is \tcode{true}.

  \pnum\returns
  The lowest element index $i$ where \tcode{k[$i$]} is \tcode{true}.
\end{itemdescr}

\begin{itemdecl}
template<size_t Bytes, class Abi>
  constexpr @\simdsizetype@ reduce_max_index(const basic_simd_mask<Bytes, Abi>& k);
\end{itemdecl}

\begin{itemdescr}
  \pnum\expects
  \tcode{any_of(k)} is \tcode{true}.

  \pnum\returns
  The greatest element index $i$ where \tcode{k[$i$]} is \tcode{true}.
\end{itemdescr}

\begin{itemdecl}
constexpr bool all_of(same_as<bool> auto) noexcept;
constexpr bool any_of(same_as<bool> auto) noexcept;
constexpr bool none_of(same_as<bool> auto) noexcept;
constexpr @\simdsizetype@ reduce_count(same_as<bool> auto x) noexcept;
\end{itemdecl}

\begin{itemdescr}
  \pnum\returns
  \tcode{all_of} and \tcode{any_of} return their arguments; \tcode{none_of}
  returns the negation of its argument; \tcode{reduce_count} returns the
  integral representation of \tcode{x}.
\end{itemdescr}

\begin{itemdecl}
constexpr @\simdsizetype@ reduce_min_index(same_as<bool> auto y);
constexpr @\simdsizetype@ reduce_max_index(same_as<bool> auto z);
\end{itemdecl}

\begin{itemdescr}
  \pnum\expects
  The value of the argument is \tcode{true}.

  \pnum\returns \tcode{0}.
\end{itemdescr}

% vim: tw=100

\end{wgText}

\end{document}
% vim: sw=2 sts=2 ai et tw=0
