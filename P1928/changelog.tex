\section{Changelog}
\begin{revision}
\item Target \CC{}26, addressing SG1 and LEWG.
\item Call for a merge of the (improved \& adjusted) TS specification to the IS.
\item Discuss changes to the ABI tags as consequence of TS experience; calls for polls to change the status quo.
\item Add template parameter \code{T} to \code{simd_abi::fixed_size}.
\item Remove \code{simd_abi::compatible}.
\item Add (but ask for removal) \code{simd_abi::abi_stable}.
\item Mention TS implementation in GCC releases.
\item Add more references to related papers.
\item Adjust the clause number for [numbers] to latest draft.
\item Add open question: what is the correct clause for [simd]?
\item Add open question: integration with ranges.
\item Add \code{simd_mask} generator constructor.
\item Consistently add simd and simd_mask to headings.
\item Remove experimental and parallelism_v2 namespaces.
\item Present the wording twice: with and without diff against N4808 (Parallelism TS 2).
\item Default load/store flags to \code{element_aligned}.
\item Generalize casts: conditionally \code{explicit} converting constructors.
\item Remove named cast functions.
\end{revision}

\begin{revision}
\item Add floating-point conversion rank to condition of \code{explicit} for converting constructors.
\item Call out different or equal semantics of the new ABI tags.
\item Update introductory paragraph of \sect{sec:changes}; R1 incorrectly kept the text from R0.
\item Define simd::size as a \code{constexpr} static data-member of type \code{integral_constant<size_t, N>}. This simplifies passing the size via function arguments and still be useable as a constant expression in the function body.
\item Document addition of \code{constexpr} to the API.
\item Add \code{constexpr} to the wording.
\item Removed ABI tag for passing \code{simd} over ABI boundaries.
\item Apply cast interface changes to the wording.
\item Explain the plan: what this paper wants to merge vs. subsequent papers for additional features. With an aim of minimal removal/changes of wording after this paper.
\item Document rationale and design intent for \code{where} replacement.
\end{revision}

\begin{revision}
\item Propose alternative to \code{hmin} and \code{hmax}.
\item Discuss \code{simd_mask} reductions wrt. consistency with \code{<bit>}. Propose better names to avoid ambiguity.
\item Remove \code{some_of}.
\item Add unary \code{\~{}} to \code{simd_mask}.
\item Discuss and ask for confirmation of masked ``overloads'' names and argument order.
\item Resolve inconsistencies wrt. \code{int} and \code{size_t}: Change \code{fixed_size} and \code{resize_simd} NTTPs from \code{int} to \code{size_t} (for consistency).
\item Discuss conversions on loads and stores. (\sect{sec:convertingLoadsAndStores})
\item Point to \cite{P2509R0} as related paper.
\item Generalize load and store from pointer to \code{contiguous_iterator}. (\sect{sec:contiguousItLoadStore})
\item Moved ``\code{element_reference} is overspecified'' to ``Open questions''.
\end{revision}

\begin{revision}
\item Remove wording diff.
\item Add std::simd to the paper title.
\item Update ranges integration discussion and mention formatting support via
  ranges (\sect{sec:formatting}).
\item Fix: pass iterators by value not const-ref.
\item Add lvalue-ref qualifier to subscript operators (\sect{sec:lvalue-subscript}).
\item Constrain \code{simd} operators: require operator to be well-formed on objects of \code{value_type} (\ref{sec:simd.unary}, \ref{sec:simd.binary}).
\item Rename mask reductions as decided in Issaquah.
\item Remove R3 ABI discussion and add follow-up question (\sect{sec:simplifyfixedsize}).
\item Add open question on first template parameter of \code{simd_mask} (\sect{sec:basicsimdmask}).
  \todo Replace \code{where} wording.
  \todo Apply the new library specification style from P0788R3.
  \todo Add \code{numeric_limits} / numeric traits specializations since behavior of e.g. \code{simd<float>} and \code{float} may differ for reasonable implementations.
  \todo Consider adding a note that recommends implementations to let simd primary operations behave like operations of arithmetic types to never be function calls. (cf. GCC PR108030)
\end{revision}
