\def\rSec#1[#2]#3{%
  \ifcase#1\wgSubsection[subsection]{#3}{#2}
  \or\wgSubsubsection[subsubsection]{#3}{#2}
  \or\wgSubsubsubsection[paragraph]{#3}{#2}
  \or\error
\fi}

\renewcommand\foralli[1][]{for all $i$ in the range of \range{0}{#1size()}}
\newcommand\Foralli[1][]{For all $i$ in the range of \range{0}{#1size()}}
\renewcommand\forallmaskedi{for all selected indices $i$ of \tcode{mask}}

\newcommand\op{\textrm{\textit{op}}}

\newcommand\ConstraintUnaryOperatorWellFormed[2][const ]{%
  \constraints \tcode{requires (#1value_type a) \{ #2; \}} is \tcode{true}.
}

\newcommand\ConstraintOperatorTWellFormed{%
  \constraints \tcode{requires (value_type a, value_type b) \{ a \op{} b; \}} is \tcode{true}.
}

\newcommand\ConversionsToIntrinsics[1]{
\recommended
Implementations should support explicit conversions between specializations of \tcode{#1} and
appropriate implementation-defined types.
\begin{note}
  The appropriate vector types which are available in the implementation.
\end{note}

%\begin{example}
  %Consider an implementation that supports the type \tcode{__vec4f} and the function \tcode{__vec4f
  %_vec4f_addsub(__vec4f, __vec4f)} for the architecture of the execution environment.
  %A user may require the use of \tcode{_vec4f_addsub} for maximum performance and thus writes:
  %\begin{codeblock}
    %using V = basic_simd<float, simd_abi::__simd128>;
    %V addsub(V a, V b) {
      %return static_cast<V>(_vec4f_addsub(static_cast<__vec4f>(a), static_cast<__vec4f>(b)));
    %}
  %\end{codeblock}
%\end{example}
}


\rSec1[simd.syn]{Header \texorpdfstring{\tcode{<simd>}}{<simd>} synopsis}

%\indexhdr{simd}
\begin{codeblock}
namespace std {

  // \ref{simd.reductions}, \tcode{basic_simd} reductions
  template<class T, class BinaryOperation = plus<>>
    constexpr T reduce(const T&, BinaryOperation = {});
  template<class T, class BinaryOperation = plus<>>
    constexpr T reduce(
      const T& x, same_as<bool> mask,
      BinaryOperation binary_op = {}, type_identity_t<T> identity_element = @\seebelow@);

  template<class T>
    constexpr T reduce_min(const T&) noexcept;
  template<class T>
    constexpr T reduce_min(const T&,
                           same_as<bool>) noexcept;
  template<class T>
    constexpr T reduce_max(const T&) noexcept;
  template<class T>
    constexpr T reduce_max(const T&,
                           same_as<bool>) noexcept;

}
\end{codeblock}

\rSec2[simd.reductions]{\tcode{basic_simd} reductions}

\begin{itemdecl}
template<class T, class BinaryOperation = plus<>>
  constexpr T reduce(const T& x, BinaryOperation binary_op = {});
\end{itemdecl}

\begin{itemdescr}
  \pnum\constraints
  \begin{itemize}
    \item \tcode{T} is vectorizable.

    \item \tcode{BinaryOperation} models \tcode{\reductionoperation<T>}.
  \end{itemize}
  \pnum\expects
  \tcode{binary_op} does not modify \tcode{x}.

  \pnum\returns \tcode{x}

  \pnum\throws
  Any exception thrown from \tcode{binary_op}.
\end{itemdescr}

\begin{itemdecl}
template<class T, class BinaryOperation = plus<>>
  constexpr T reduce(
    const T& x, same_as<bool> mask,
    BinaryOperation binary_op = {}, type_identity_t<T> identity_element = @\seebelow@);
\end{itemdecl}

\begin{itemdescr}
  \pnum\constraints
  \begin{itemize}
    \item \tcode{T} is vectorizable.

    \item \tcode{BinaryOperation} models \tcode{\reductionoperation<T>}.

    \item An argument for \tcode{identity_element} is provided for the invocation, unless
      \tcode{BinaryOperation} is one of \code{plus<>}, \code{multiplies<>}, \code{bit_and<>},
      \code{bit_or<>}, or \code{bit_xor<>}.
  \end{itemize}

  \pnum\expects
  \begin{itemize}
    \item \tcode{binary_op} does not modify \tcode{x}.

    \item For all finite values \tcode{y} representable by \tcode{T}, the results of
      \tcode{y == binary_op(simd<T, 1>(identity_element), simd<T, 1>(y))[0]} and
      \tcode{y == binary_op(simd<T, 1>(y), simd<T, 1>(identity_element))[0]} are \tcode{true}.
  \end{itemize}

  \pnum\returns
  If \tcode{mask} is \tcode{false}, returns \tcode{identity_element}.
  Otherwise, returns \tcode{x}.

  \pnum\throws
  Any exception thrown from \tcode{binary_op}.

  \pnum\remarks
  The default argument for \code{identity_element} is equal to
  \begin{itemize}
    \item \tcode{T()} if \code{BinaryOperation} is \code{plus<>},
    \item \tcode{T(1)} if \code{BinaryOperation} is \code{multiplies<>},
    \item \tcode{T(\~{}T())} if \code{BinaryOperation} is \code{bit_and<>},
    \item \tcode{T()} if \code{BinaryOperation} is \code{bit_or<>}, or
    \item \tcode{T()} if \code{BinaryOperation} is \code{bit_xor<>}.
  \end{itemize}
\end{itemdescr}

\begin{itemdecl}
template<class T> constexpr T reduce_min(const T& x) noexcept;
\end{itemdecl}

\begin{itemdescr}
  \pnum\constraints
  \begin{itemize}
    \item \tcode{T} is vectorizable.

    \item \tcode{T} models \tcode{totally_ordered}.
  \end{itemize}

  \pnum\returns \tcode{x}.
\end{itemdescr}

\begin{itemdecl}
template<class T>
  constexpr T reduce_min(
    const T& x, same_as<bool> mask) noexcept;
\end{itemdecl}

\begin{itemdescr}
  \pnum\constraints
  \begin{itemize}
    \item \tcode{T} is vectorizable.

    \item \tcode{T} models \tcode{totally_ordered}.
  \end{itemize}

  \pnum\returns
  If \tcode{mask} is \tcode{false}, returns \tcode{numeric_limits<T>::max()}.
  Otherwise, returns \tcode{x}.
\end{itemdescr}

\begin{itemdecl}
template<class T> constexpr T reduce_max(const T& x) noexcept;
\end{itemdecl}

\begin{itemdescr}
  \pnum\constraints
  \begin{itemize}
    \item \tcode{T} is vectorizable.

    \item \tcode{T} models \tcode{totally_ordered}.
  \end{itemize}

  \pnum\returns \tcode{x}.
\end{itemdescr}

\begin{itemdecl}
template<class T>
  constexpr T reduce_max(
    const T& x, same_as<bool> mask) noexcept;
\end{itemdecl}

\begin{itemdescr}
  \pnum\constraints
  \begin{itemize}
    \item \tcode{T} is vectorizable.

    \item \tcode{T} models \tcode{totally_ordered}.
  \end{itemize}

  \pnum\returns
  If \tcode{mask} is \tcode{false}, returns \tcode{numeric_limits<T>::lowest()}.
  Otherwise, returns \tcode{x}.
\end{itemdescr}

% vim: tw=100 cc=99
