\newcommand\wgTitle{Missing deduction guide from \code{simd::mask} to \code{simd::vec}}
\newcommand\wgName{Matthias Kretz <m.kretz@gsi.de>}
\newcommand\wgDocumentNumber{P3922R1}
\newcommand\wgGroup{LEWG, LWG}
\newcommand\wgTarget{\CC{}26}
%\newcommand\wgAcknowledgements{Daniel Towner and Ruslan Arutyunyan contributed to this paper via discussions / reviews. Thanks also to Jeff Garland for reviewing.}

\usepackage{mymacros}
\usepackage{wg21}
\setcounter{tocdepth}{2} % show sections and subsections in TOC
\hypersetup{bookmarksdepth=5}
\usepackage{changelog}
\usepackage{underscore}
\usepackage{multirow}
\usepackage{vectorpictures}

\addbibresource{extra.bib}

\newcommand\simd[1][]{\type{ba\-sic\_vec#1}\xspace}
\newcommand\simdT{\type{ba\-sic\_vec\MayBreak<\MayBreak{}T>}\xspace}
\newcommand\valuetype{\type{val\-ue\_type}\xspace}
\newcommand\referencetype{\type{ref\-er\-ence}\xspace}
\newcommand\mask[1][]{\type{ba\-sic\_mask#1}\xspace}
\newcommand\maskT{\type{ba\-sic\_mask\MayBreak<\MayBreak{}T>}\xspace}
\newcommand\wglink[1]{\href{https://wg21.link/#1}{#1}}

\newcommand\nativeabi{\UNSP{native-abi}}
\newcommand\deducet{\UNSP{deduce-t}}
\newcommand\simdsizev{\UNSP{simd-size-v}}
\newcommand\simdsizetype{\UNSP{simd-size-type}}
\newcommand\simdselect{\UNSP{simd-select-impl}}
\newcommand\maskelementsize{\UNSP{mask-element-size}}
\newcommand\integerfrom{\UNSP{integer-from}}
\newcommand\constexprwrapperlike{\UNSP{constexpr-wrapper-like}}
\newcommand\convertflag{\UNSP{convert-flag}}
\newcommand\alignedflag{\UNSP{aligned-flag}}
\newcommand\overalignedflag{\UNSP{overaligned-flag}}
\newcommand\reductionoperation{\UNSP{reduction-binary-operation}}
\newcommand\simdfloatingpoint{\UNSP{simd-floating-point}}
\newcommand\multisimdfloatingpoint{\UNSP{multi-arg-simd-floating-point}}

\renewcommand{\lst}[1]{Listing~\ref{#1}}
\renewcommand{\sect}[1]{Section~\ref{#1}}
\renewcommand{\ttref}[1]{Tony~Table~\ref{#1}}
\renewcommand{\tabref}[1]{Table~\ref{#1}}

\begin{document}
\selectlanguage{american}
\begin{wgTitlepage}
  After a fix to the return type of the unary operators of \code{basic_mask}, the natural
  corresponding integral \code{basic_vec} cannot be determined from the specification anymore.
  The implementation has some freedom to choose an integer type and an ABI tag, that the user can
  only determine via \code{decltype} on the result of a unary operator.
  To match the intended analogous usage in scalar code, it would be helpful if CTAD from
  \code{basic_mask} to \code{basic_vec} just works.
\end{wgTitlepage}

\pagestyle{scrheadings}

\section{Changelog}
(placeholder)
\begin{revision}
\item Add a simple example to the motivation section.
\item Expand the “Generalization” section to clearly define the feature rather
  than just sketching it.
  Also add a discussion of initial value and step.
\item Discuss why reusing the existing \code{iota} algorithm/view does not
  work/suffice for the \code{simd} use case.
\item Discuss why \code{iota_v} is the right name.
%  \todo
\end{revision}

\section{Straw Polls}
\subsection{SG1 at Kona 2022}
\wgPoll{After significant experience with the TS, we recommend that the next
version (the TS version with improvements) of \code{std::simd} target the IS (\CC{}26)}
{10&8&0&0&0}

\wgUnanimous{We like all of the recommended changes to \code{std::simd} proposed in p1928r1
(Includes making all of \code{std::simd} \code{constexpr}, and dropping an ABI stable type)}

\wgPoll{Future papers and future revisions of existing papers that target
\code{std::simd} should go directly to LEWG.
(We do not believe there are SG1 issues with \code{std::simd} today.)}
{9&8&0&0&0}


\pagebreak
\section{Motivation}

In scalar code we sometimes convert booleans into integers (typically to implement branch-less
algorithms). E.g.
\medskip\begin{lstlisting}[style=Vc]
template <typename T>
void f(T x, T y) {
  int b = x < y;
  // ...
}
\end{lstlisting}
The equivalent is possible with std::simd:
\medskip\begin{lstlisting}[style=Vc]
template <simd_vec_type V>
void f(V x, V y) {
  vec<int> b = x < y;
  // ...
}
\end{lstlisting}
Except that the type of \code{b} is probably wrong and so the code is ill-formed.
But what is the correct type?
\medskip\begin{lstlisting}[style=Vc]
template <simd_vec_type V>
void f(V x, V y) {
  auto b = x < y;
  // ...
}
\end{lstlisting}
This doesn't do the conversion to integral \code{basic_vec}, but rather keeps \code{b} of type
\code{basic_mask}.
What we can do is:
\medskip\begin{lstlisting}[style=Vc]
template <simd_vec_type V>
void f(V x, V y) {
  auto b = +(x < y);
  // ...
}
\end{lstlisting}
The unary plus operator will convert the mask to an integral \code{basic_vec}, analogous to unary
plus on \code{bool}.
But it would be less cryptic if the user could write instead:
\medskip\begin{lstlisting}[style=Vc]
template <simd_vec_type V>
void f(V x, V y) {
  basic_vec b = x < y;
  // ...
}
\end{lstlisting}
However, this CTAD requires a deduction guide, which has long been part of my implementation but
somehow didn't make it into the wording.

\section{Proposed Poll}
\wgPoll{Add the deduction guide from \code{basic_mask} to \code{basic_vec} from
\wgDocumentNumber{} to \CC{}26, resolving DE-287}{&&&&}

\section{Wording}
\subsection{Feature test macro}

In [version.syn] bump the \code{__cpp_lib_simd} version.

\subsection{Modify [simd.syn] and [simd.expos]}

In [simd.overview], insert:
\begin{wgText}[{[simd.syn]}]
\begin{codeblock}
  template<class R, class... Ts>
    basic_vec(R&& r, Ts...) -> @\seebelow@;

  @\wgAdd{template<size_t Bytes, class Abi>}@
    @\wgAdd{basic_vec(basic_mask<Bytes, Abi>) -> \mbox{\seebelow};}@
\end{codeblock}
\end{wgText}

In [simd.ctor], insert:
\begin{wgText}[{[simd.ctor]}]
\begin{itemdecl}
template<class R, class... Ts>
  basic_vec(R&& r, Ts...) -> @\seebelow@;
\end{itemdecl}

\begin{itemdescr}
\pnum
\constraints
\begin{itemize}
\item \tcode{R} models \tcode{ranges::\libconcept{contiguous_range}} and
  \tcode{ranges::\libconcept{sized_range}}, and
\item \tcode{ranges::size(r)} is a constant expression.
\end{itemize}

\pnum
\remarks
The deduced type is equivalent to \tcode{vec<ranges::range_value_t<R>,
ranges::size(r)>}.
\end{itemdescr}

\begin{itemdecl}
@\wgAdd{template<size_t Bytes, class Abi>}@
  @\wgAdd{basic_vec(basic_mask<Bytes, Abi> k) -> \mbox{\seebelow};}@
\end{itemdecl}

\begin{itemdescr}
\pnumAdd
\wgAdd{\constraints
\tcode{basic_mask<Bytes, Abi>} is an enabled specialization of \tcode{basic_mask} and
\tcode{decltype(+k)} is a valid type.}

\pnumAdd
\wgAdd{\remarks
The deduced type is equivalent to \tcode{decltype(+k)}.}
\end{itemdescr}
\end{wgText}


\end{document}
% vim: sw=2 sts=2 ai et tw=100
