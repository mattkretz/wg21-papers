\newcommand\wgTitle{Let \code{simd::vec} determine the \code{simd::mask} type}
\newcommand\wgName{Matthias Kretz <m.kretz@gsi.de>}
\newcommand\wgDocumentNumber{DXXXXR0}
\newcommand\wgGroup{LEWG,LWG}
\newcommand\wgTarget{\CC{}26}
%\newcommand\wgAcknowledgements{Daniel Towner and Ruslan Arutyunyan contributed to this paper via discussions / reviews. Thanks also to Jeff Garland for reviewing.}

\usepackage{mymacros}
\usepackage{wg21}
\setcounter{tocdepth}{2} % show sections and subsections in TOC
\hypersetup{bookmarksdepth=5}
\usepackage{changelog}
\usepackage{underscore}
\usepackage{multirow}
\usepackage{vectorpictures}

\addbibresource{extra.bib}

\newcommand\simd[1][]{\type{ba\-sic\_vec#1}\xspace}
\newcommand\simdT{\type{ba\-sic\_vec\MayBreak<\MayBreak{}T>}\xspace}
\newcommand\valuetype{\type{val\-ue\_type}\xspace}
\newcommand\referencetype{\type{ref\-er\-ence}\xspace}
\newcommand\mask[1][]{\type{ba\-sic\_mask#1}\xspace}
\newcommand\maskT{\type{ba\-sic\_mask\MayBreak<\MayBreak{}T>}\xspace}
\newcommand\wglink[1]{\href{https://wg21.link/#1}{#1}}

\newcommand\nativeabi{\UNSP{native-abi}}
\newcommand\deducet{\UNSP{deduce-t}}
\newcommand\simdsizev{\UNSP{simd-size-v}}
\newcommand\simdsizetype{\UNSP{simd-size-type}}
\newcommand\simdselect{\UNSP{simd-select-impl}}
\newcommand\maskelementsize{\UNSP{mask-element-size}}
\newcommand\integerfrom{\UNSP{integer-from}}
\newcommand\constexprwrapperlike{\UNSP{constexpr-wrapper-like}}
\newcommand\convertflag{\UNSP{convert-flag}}
\newcommand\alignedflag{\UNSP{aligned-flag}}
\newcommand\overalignedflag{\UNSP{overaligned-flag}}
\newcommand\reductionoperation{\UNSP{reduction-binary-operation}}
\newcommand\simdfloatingpoint{\UNSP{simd-floating-point}}
\newcommand\multisimdfloatingpoint{\UNSP{multi-arg-simd-floating-point}}

\renewcommand{\lst}[1]{Listing~\ref{#1}}
\renewcommand{\sect}[1]{Section~\ref{#1}}
\renewcommand{\ttref}[1]{Tony~Table~\ref{#1}}
\renewcommand{\tabref}[1]{Table~\ref{#1}}

\begin{document}
\selectlanguage{american}
\begin{wgTitlepage}
  The \stdsimd\code{mask<T, N>} alias template currently determines the \stdsimd\mask[<Bytes, Abi>]
  specialization using \code{Bytes = sizeof(T)} and a deduction of \code{Abi} using the same
  expression that \stdsimd\code{vec} uses.
  This could be simplified to \code{vec<T, N>::mask_type}, which additionally provides more freedom
  in choosing a value for \code{Bytes}, which is important for a non-interleaved implementation of
  \code{vec<complex<T>>}.
  Consequences of such a change are explored in this paper.
\end{wgTitlepage}

\pagestyle{scrheadings}

\section{Changelog}
(placeholder)
\begin{revision}
\item Add a simple example to the motivation section.
\item Expand the “Generalization” section to clearly define the feature rather
  than just sketching it.
  Also add a discussion of initial value and step.
\item Discuss why reusing the existing \code{iota} algorithm/view does not
  work/suffice for the \code{simd} use case.
\item Discuss why \code{iota_v} is the right name.
%  \todo
\end{revision}

\section{Straw Polls}
\subsection{SG1 at Kona 2022}
\wgPoll{After significant experience with the TS, we recommend that the next
version (the TS version with improvements) of \code{std::simd} target the IS (\CC{}26)}
{10&8&0&0&0}

\wgUnanimous{We like all of the recommended changes to \code{std::simd} proposed in p1928r1
(Includes making all of \code{std::simd} \code{constexpr}, and dropping an ABI stable type)}

\wgPoll{Future papers and future revisions of existing papers that target
\code{std::simd} should go directly to LEWG.
(We do not believe there are SG1 issues with \code{std::simd} today.)}
{9&8&0&0&0}


\pagebreak
\section{Motivation}

The CD defines \code{simd::vec} and \code{simd::mask} as follows:

\begin{wgText}[{[simd.syn]}]
\begin{codeblock}
// [simd.class], class template basic_vec
template<class T, class Abi = @\exposid{native-abi}@<T>> class basic_vec;
template<class T, @\exposid{simd-size-type}@ N = @\exposid{simd-size-v}@<T, @\exposid{native-abi}@<T>>>
  using vec = basic_vec<T, @\exposid{deduce-abi-t}@<T, N>>;

// [simd.mask.class], class template basic_mask
template<size_t Bytes, class Abi = @\exposid{native-abi}<\exposid{integer-from}@<Bytes>>> class basic_mask;
template<class T, @\exposid{simd-size-type}@ N = @\exposid{simd-size-v}@<T, @\exposid{native-abi}@<T>>>
  using mask = basic_mask<sizeof(T), @\exposid{deduce-abi-t}@<T, N>>;
\end{codeblock}
\end{wgText}

In addition, \code{basic_vec} has a type member called \code{mask_type}, which is the type returned
from comparison and used for masked operations.
The expectation has always been that \stdsimd\code{vec<T, N>::mask_type} is the same type as
\stdsimd\code{mask<T, N>}.
When \std\code{complex<T>} became a vectorizable type this equality became harder to implement and
for some targets (ABI tags) requires an inefficient implementation.
But even before \code{complex} was added, the situation for x86 AVX (without AVX2) targets was
tricky.

\subsection{The AVX without AVX2 challenge}
(Why this matters:
  SandyBridge and IvyBridge CPUs are fairly ancient now, so we could ignore the issue.
  However, the point is that we don't know what future hardware will look like and that we might see
similar issues with modern hardware.)

A sensible default ABI tag setup for x86 with AVX but without AVX2 follows instruction support.
AVX supports 256-bit floating-point instructions, but provides basically no 256-bit integer
instructions.
The latter was added with AVX2.

Thus, for the libstdc++ implementation \code{vec<float>::size() == 8} and \code{vec<int>::size() ==
4} with \code{-march=ivybridge}.
The \code{vec<int, 8>} type is stored, passed, and manipulated as two 128-bit xmm registers.
The implication for masks, which lose information about the value-type by specializing
\code{basic_mask} on the \code{sizeof}, is that \code{mask<float>::size()} must be 8 and
\code{mask<int>::size()} must be 4.
\code{mask<int, 8>} is thus made up of two 128-bit xmm registers while \code{mask<float, 8>} is a
single 256-bit ymm register.

The \code{basic_mask} template, however has trouble to do the right thing.
With \code{sizeof(float) == sizeof(int) == 4}, what should a user expect for \code{basic_mask<4>}?
Is it 1 ymm, 2 xmm, or 1 xmm register?
The wording is clear, since the ABI tag is deduced via the \exposid{integer-from} trait, that we need
to get the equivalent of \code{mask<int>} and \emph{not} \code{mask<float>}.

\medskip
\begin{tabular}{lll}
  Type & \code{size()} & stored as \\\hline
  \code{vec<float>}       & 8 & $1\times$ \code{ymm} \\
  \code{basic_vec<float>} & 8 & $1\times$ \code{ymm} \\
  \code{mask<float>}      & 8 & $1\times$ \code{ymm} \\
  \code{basic_mask<sizeof(float)>} & 4 & $1\times$ \code{xmm} \\
  \hline
  \code{vec<int>}       & 4 & $1\times$ \code{xmm} \\
  \code{basic_vec<int>} & 4 & $1\times$ \code{xmm} \\
  \code{mask<int>}      & 4 & $1\times$ \code{xmm} \\
  \code{basic_mask<sizeof(int)>} & 4 & $1\times$ \code{xmm} \\
  \hline
  \code{vec<int, 8>}    & 8 & $2\times$ \code{xmm} \\
  \code{mask<int, 8>}   & 8 & $2\times$ \code{xmm} \\
\end{tabular}

\subsection{\code{mask<complex<T>>} is surprisingly complex}

The previsouly mentioned \exposid{integer-from} trait is problematic on its own for
\code{complex<double>} as discussed for
LWG4238\footnote{\url{https://cplusplus.github.io/LWG/issue4238}\\Discussion:
\url{https://lists.isocpp.org/lib/2025/04/31171.php}}.

But let's look at vector-masks for \code{complex<T>} first (for simplicity I'll assume
x86\footnote{Nothing is really simple about x86, though.}):\\
\code{mask<complex<double>>} is an alias for \code{basic_mask<16,
\exposid{deduce-abi-t}<complex<double>, N>>}, where \code{N} depends on compiler flags and can be 1,
2, or 4 (SSE, AVX, or AVX512).
See \fig{sse.complex} for a graphical representation of the following discussion.
With AVX, \code{vec<complex<double>>} thus holds two \code{complex<double>} values ($x_0$ and $x_1$;
(1) in \fig{sse.complex}).
Interleaved storage stores it as four values of type \code{double} (\code{vec<double>}: $x_0^r$,
$x_0^i$, $x_1^r$, and $x_1^i$; (2) in \fig{sse.complex}) in one ymm register.

\begin{figure}[htb]
\centering
\begin{tikzpicture}
  \vInit
  \setlength\vNodeHeight{6ex}
  \vNode{$x_0$, $x_1$}
  \setlength\vNodeHeight{3ex}
  \addtocounter{vectornodeindexinner}{2}\vCaption{(1)}
  \vNode{$x_0^r$, $x_0^i$, $x_1^r$, $x_1^i$}
  \vCaption{(2)}
  \setlength\vNodeHeight{6ex}
  \vNode{$k_0$, $k_1$}
  \setlength\vNodeHeight{3ex}
  \addtocounter{vectornodeindexinner}{2}\vCaption{(3)}
  \vNode{$k_0^r$, $k_0^i$, $k_1^r$, $k_1^i$}
  \vCaption{(4)}
\end{tikzpicture}
\caption{SSE implementation of interleaved \code{complex}}
\label{fig:sse.complex}
\end{figure}

A \code{mask<complex<double>>} object is stored in just the same way, with two 128-bit values
($k_0$, $k_1$; (3) in \fig{sse.complex}) actually stored as four 64-bit values ($k_0^r$, $k_0^i$,
$k_1^r$, $k_1^i$; (4) in \fig{sse.complex}).
Comparing two objects of type \code{vec<complex<double>>} is translated into an AVX compare
instruction on four \code{double} values, producing four 64-bit results in a vector mask.
The implementation needs to fix up the result by 

\section{Implications}

A contiguous implementation of \code{basic_vec<complex<T>>} would use \code{basic_vec<sizeof(T),
SomeAbi>} as its \code{mask_type}.
An implementation that defaults to interleaved on one target and contiguous on a different target
would therefore treat \code{-mask<complex<double>>()} as ill-formed for the first target and produce
a \code{vec<long long>} on the other target.
\code{convertible_to<mask<complex<double>>, mask<double>>} could be implemented to be
\code{true}.

\section{Wording}

\subsection{Feature test macro}

Bump.

\subsection{Modify [simd.syn] and [simd.creation]}

In [simd.syn], change:
\begin{wgText}[{[simd.syn]}]
\begin{codeblock}
  template<class T, class Abi>
    constexpr auto chunk(const basic_simd_mask<@\exposid{mask-element-size}@<T>, Abi>& x) noexcept;

  template<@\wgChange{size_t}{\mbox{\exposid{simd-size-type}}}@ N, class T, class Abi>
    constexpr auto chunk(const basic_simd<T, Abi>& x) noexcept;
  template<@\wgChange{size_t}{\mbox{\exposid{simd-size-type}}}@ N, size_t Bytes, class Abi>
    constexpr auto chunk(const basic_simd_mask<Bytes, Abi>& x) noexcept;

  template<class T, class... Abis>
    constexpr basic_simd<T, @\exposid{deduce-abi-t}@<T, (basic_simd<T, Abis>::size() + ...)>>
      cat(const basic_simd<T, Abis>&...) noexcept;
\end{codeblock}
\end{wgText}

In [simd.creation], change:
\begin{wgText}[{[simd.creation]}]
\setcounter{WGClause}{29}
\setcounter{WGSubSection}{10}
\setcounter{WGSubSubSection}{7}
\setcounter{WGSubSubSubSection}{8}
\setcounter{Paras}{3}%
\begin{itemdecl}
template<@\wgChange{size_t}{\mbox{\exposid{simd-size-type}}}@ N, class T, class Abi>
  constexpr auto chunk(const basic_simd<T, Abi>& x) noexcept;
\end{itemdecl}

\begin{itemdescr}
\pnum
\effects
Equivalent to: \tcode{return chunk<resize_t<N, basic_simd<T, Abi>>>(x);}
\end{itemdescr}

\begin{itemdecl}
template<@\wgChange{size_t}{\mbox{\exposid{simd-size-type}}}@ N, size_t Bytes, class Abi>
  constexpr auto chunk(const basic_simd_mask<Bytes, Abi>& x) noexcept;
\end{itemdecl}

\begin{itemdescr}
\pnum
\effects
Equivalent to: \tcode{return chunk<resize_t<N, basic_simd_mask<Bytes, Abi>>>(\brk{}x);}
\end{itemdescr}
\end{wgText}

\end{document}
% vim: sw=2 sts=2 ai et tw=100
