\newcommand\wgTitle{std::simd is a range}
\newcommand\wgName{Matthias Kretz <m.kretz@gsi.de>}
\newcommand\wgDocumentNumber{P3480R4}
\newcommand\wgGroup{LEWG}
\newcommand\wgTarget{\CC{}26}
%\newcommand\wgAcknowledgements{}

\usepackage{mymacros}
\usepackage{wg21}
\setcounter{tocdepth}{2} % show sections and subsections in TOC
\hypersetup{bookmarksdepth=5}
\usepackage{changelog}
\usepackage{underscore}
\usepackage{multirow}

\addbibresource{extra.bib}

\newcommand\simd[1][]{\type{ba\-sic\_simd#1}\xspace}
\newcommand\simdT{\type{ba\-sic\_simd\MayBreak<\MayBreak{}T>}\xspace}
\newcommand\valuetype{\type{val\-ue\_type}\xspace}
\newcommand\referencetype{\type{ref\-er\-ence}\xspace}
\newcommand\mask[1][]{\type{ba\-sic\_simd\_mask#1}\xspace}
\newcommand\maskT{\type{ba\-sic\_simd\_mask\MayBreak<\MayBreak{}T>}\xspace}
\newcommand\wglink[1]{\href{https://wg21.link/#1}{#1}}

\newcommand\nativeabi{\UNSP{native-abi}}
\newcommand\deducet{\UNSP{deduce-t}}
\newcommand\simdsizev{\UNSP{simd-size-v}}
\newcommand\simdsizetype{\UNSP{simd-size-type}}
\newcommand\simdselect{\UNSP{simd-select-impl}}
\newcommand\maskelementsize{\UNSP{mask-element-size}}
\newcommand\integerfrom{\UNSP{integer-from}}
\newcommand\constexprwrapperlike{\UNSP{constexpr-wrapper-like}}
\newcommand\convertflag{\UNSP{convert-flag}}
\newcommand\alignedflag{\UNSP{aligned-flag}}
\newcommand\overalignedflag{\UNSP{overaligned-flag}}
\newcommand\reductionoperation{\UNSP{reduction-binary-operation}}
\newcommand\simdfloatingpoint{\UNSP{simd-floating-point}}
\newcommand\multisimdfloatingpoint{\UNSP{multi-arg-simd-floating-point}}
\newcommand\simditerator{\UNSP{simd-iterator}}

\renewcommand{\lst}[1]{Listing~\ref{#1}}
\renewcommand{\sect}[1]{Section~\ref{#1}}
\renewcommand{\ttref}[1]{Tony~Table~\ref{#1}}
\renewcommand{\tabref}[1]{Table~\ref{#1}}

\begin{document}
\selectlanguage{american}
\begin{wgTitlepage}
  P1928 “std::simd --- merge data-parallel types from the Parallelism TS 2” promised a paper on
  making \code{simd} a range. This paper explores the addition of iterators to \simd and \mask.
\end{wgTitlepage}

\pagestyle{scrheadings}

\section{Changelog}
(placeholder)
\begin{revision}
\item Add a simple example to the motivation section.
\item Expand the “Generalization” section to clearly define the feature rather
  than just sketching it.
  Also add a discussion of initial value and step.
\item Discuss why reusing the existing \code{iota} algorithm/view does not
  work/suffice for the \code{simd} use case.
\item Discuss why \code{iota_v} is the right name.
%  \todo
\end{revision}

\section{Straw Polls}
\subsection{SG1 at Kona 2022}
\wgPoll{After significant experience with the TS, we recommend that the next
version (the TS version with improvements) of \code{std::simd} target the IS (\CC{}26)}
{10&8&0&0&0}

\wgUnanimous{We like all of the recommended changes to \code{std::simd} proposed in p1928r1
(Includes making all of \code{std::simd} \code{constexpr}, and dropping an ABI stable type)}

\wgPoll{Future papers and future revisions of existing papers that target
\code{std::simd} should go directly to LEWG.
(We do not believe there are SG1 issues with \code{std::simd} today.)}
{9&8&0&0&0}


\section{Introduction, or why \code{simd} wasn't a range in the TS}

The Parallelism TS 2 was based on \CC{}17.
Ranges were added in \CC{}20.
Before ranges, an iterator category was tied to whether \code{operator*} of iterators returned an
lvalue reference.
Since \simd and \mask objects are not composed of sub-objects (in other words, a \code{simd<int>}
contains no \code{int} \emph{objects}), \code{operator[]} returns prvalues (or a proxy reference
in the TS for the non-const case).
An iterator needs to do the same and thus never could be in any other iterator category than
\emph{Cpp17InputIterator}.
In reality, the iterator category always was “random access” (never contiguous; because while \simd
is a contiguous range in memory it isn't one in the object model of \CC{}).
In order to not cement that mismatch, it was never proposed to make \simd/\mask a range for the TS.

Now that the iterator concepts don't require an lvalue reference anymore we can easily make
\simd/\mask a read-only range.
Iterator dereference would return a prvalue (a copy of the value stored in the \simd/\mask object).
In addition, the abstraction of a sentinel instead of an iterator pointing beyond the last value of
the \simd seems like a useful tool for \simd.

\section{Motivation}

After the technical reasons for \emph{not} adding iterators to \simd/\mask are resolved, we still
need to consider why \simd should be a range in the first place.

\section{Integration with the standard library}

We can improve integration of \simd/\mask with the rest of the standard library.
By making \simd/\mask a range many of the existing facilities in the standard library become
easily accessible.
All of these facilities do work as intended — in other words: presenting \simd/\mask as a range
matches on the semantic level, not only syntactically.

\subsection{Read-only subscript should imply read-only iteration}

With the latest WD we can write
\medskip\begin{lstlisting}
std::datapar::simd<int> v = ...;
for (int i = 0; i < v.size(); ++i) {
  do_something(v[i]);
}
\end{lstlisting}

Why then, can we not also write
\medskip\begin{lstlisting}
for (auto x : v) {
  do_something(x);
}
\end{lstlisting}
and
\medskip\begin{lstlisting}
std::ranges::for_each(v.begin(), v.end(), [](auto x) {
  do_something(x);
});
\end{lstlisting}
and
\medskip\begin{lstlisting}
v | std::views::filter([](auto x) { return x > 0; }) | std::ranges::to<std::vector>();
\end{lstlisting}

\CC{} users have learned that whenever a for loop with subscript does what they need to do, then a
ranged for loop, standard algorithm, or range adaptor are valid alternatives.
This expectation should not get an exception with \simd and \mask.

\subsection{Present a range of simd as a range of simd's value-type}

In some applications it is more efficient (and simpler) to work with \simd objects internally,
instead of constantly doing loads and stores.
Thus a fairly simple container that comes up in applications could be
\code{std::vector<std::datapar::simd<float>>}.
On I/O such an application typically cannot communicate in \simd objects anymore.
Instead it needs to present a range of \code{float}s.
Read-only iterators on \simd do not help with the input side.
But for output we can easily turn the \code{vector<simd<float>>} into a range of \code{float}:
\medskip\begin{lstlisting}
std::vector<std::datapar::simd<float>> data;
auto range_of_float = data | std::views::join;
\end{lstlisting}

\section{Downsides of making \code{simd} a range}

Really, I can't think of any downsides of making \simd/\mask a range.
In principle one could argue that \simd/\mask is not a container \cite{P0851R0}.
Consequently, it shouldn't have a container interface and thus no iterators.
But then we should probably remove the subscript operator as well.

\section{Design choice: sentinel}

The \simd iterator type must have a reference/pointer to the \simd object it is iterating together
with an offset, where into the \simd it is pointing.
Because of these two members (and their type), the iterator already knows the complete bounds of the
range it is pointing into.
Consequently, a single \simd iterator can always determine whether it points at the beginning or end
of the range, it doesn't need to compare against another offset.
A sentinel type allows asking that question via \code{operator==}.
Thus, instead of comparing two runtime offset members on \code{operator==}, a compare against a
sentinel is implemented as a compare against a compile-time constant.
This makes it easier for the compiler to optimize and reduces the size of the \code{end()} sentinel
to a single byte (empty type).


\section{Open Questions}

\subsection{Make \code{iterator} convertible to \code{const_iterator}}
After SG9 voted to make \code{basic_simd::const_iterator} and \code{basic_simd::iterator} different
types, interaction between the two types needs to be considered.
Since iterators model pointers and non-const pointers convert to const pointers, it seems we need
conversion from \code{\simditerator<V>} to \code{\simditerator<\brk{}const V>} to be implicit.
In the wording below, “\code{\#define LEWG_WANTS_CONVERSION 1}” to make
\lstinline@v.begin() == v.cbegin()@ a valid expression (evaluating to \code{true}).

\subsection{Add tuple interface}

\code{std::array} implements the tuple interface.
Should \code{std::simd} also implement \code{tuple_size}, \code{tuple_element}, and \code{get}?

\section{Wording}

\subsection{Feature Test Macro}
In [version.syn] bump the \code{__cpp_lib_simd} version.

\subsection{Add [simd.iterator]}
Add a new subclause before §29.10.6 [simd.class]:

\begin{wgText}[{[simd]}]
  \setcounter{WGClause}{29}
  \setcounter{WGSubSection}{10}
  \setcounter{WGSubSubSection}{5}
  \rSec2[simd.iterator]{Class \exposid{simd-iterator}}
  \begin{codeblock}
namespace std::datapar {
  template <class V>
  class @\exposidnc{simd-iterator}@ {         // \expos
    V* @\exposidnc{data_}@ = nullptr;         // \expos
    @\exposidnc{simd-size-type} \exposidnc{offset_}@ = 0; // \expos

  public:
    using value_type = typename V::value_type;
    using iterator_category = std::random_access_iterator_tag;
    using difference_type = @\exposid{simd-size-type}@;

    constexpr @\exposid{simd-iterator}@() = default;
    constexpr @\exposid{simd-iterator}@(V& d, int off);

    constexpr @\exposid{simd-iterator}@(const @\exposid{simd-iterator}@&) = default;
    constexpr @\exposid{simd-iterator}@& operator=(const @\exposid{simd-iterator}@&) = default;

#if LEWG_WANTS_CONVERSION
    constexpr @\exposid{simd-iterator}@(const @\exposid{simd-iterator}@<remove_const_t<V>>&) requires is_const_v<V>;
#endif

    constexpr value_type operator*() const;

    constexpr @\exposid{simd-iterator}@& operator++();
    constexpr @\exposid{simd-iterator}@ operator++(int);
    constexpr @\exposid{simd-iterator}@& operator--();
    constexpr @\exposid{simd-iterator}@ operator--(int);

    constexpr @\exposid{simd-iterator}@& operator+=(difference_type n);
    constexpr @\exposid{simd-iterator}@& operator-=(difference_type n);

    constexpr value_type operator[](difference_type n) const;

    constexpr friend bool operator==(@\exposid{simd-iterator}@ a, @\exposid{simd-iterator}@ b) = default;
    constexpr friend bool operator==(@\exposid{simd-iterator}@ a, default_sentinel_t);
    constexpr friend auto operator<=>(@\exposid{simd-iterator}@ a, @\exposid{simd-iterator}@ b);
    constexpr friend auto operator<=>(@\exposid{simd-iterator}@ a, default_sentinel_t);

    constexpr friend @\exposid{simd-iterator}@ operator+(const @\exposid{simd-iterator}@& i, difference_type n);
    constexpr friend @\exposid{simd-iterator}@ operator+(difference_type n, const @\exposid{simd-iterator}@& i);
    constexpr friend @\exposid{simd-iterator}@ operator-(const @\exposid{simd-iterator}@& i, difference_type n);

    constexpr friend difference_type operator-(@\exposid{simd-iterator}@ a, @\exposid{simd-iterator}@ b);
    constexpr friend difference_type operator-(@\exposid{simd-iterator}@ i, default_sentinel_t);
    constexpr friend difference_type operator-(default_sentinel_t, @\exposid{simd-iterator}@ i);
  };
}
  \end{codeblock}

\begin{itemdecl}
constexpr @\exposid{simd-iterator}@(V& d, int off);
\end{itemdecl}

\begin{itemdescr}
\pnum
\effects
Initializes \exposid{data_} with \tcode{d} and \exposid{offset_} with \tcode{off}.
\end{itemdescr}

\begin{itemdecl}
#if LEWG_WANTS_CONVERSION
constexpr @\exposid{simd-iterator}@(const @\exposid{simd-iterator}@<remove_const_t<V>>& i) requires is_const_v<V>;
#endif
\end{itemdecl}

\begin{itemdescr}
\pnum
\effects
Initializes \exposid{data_} with \tcode{i.\exposid{data_}} and \exposid{offset_} with \tcode{i.\exposid{offset_}}.
\end{itemdescr}

\begin{itemdecl}
constexpr value_type operator*() const;
\end{itemdecl}

\begin{itemdescr}
\pnum
\expects
\begin{itemize}
  \item \exposid{data_} is a valid pointer, and
  \item \exposid{offset_} is in the range \range{0}{V::size()}.
\end{itemize}
\FIXME{\expects already implied by \effects?}

\pnum
\effects
Equivalent to: \tcode{return (*\exposid{data_})[\exposid{offset_}];}
\end{itemdescr}

\begin{itemdecl}
constexpr @\exposid{simd-iterator}@& operator++();
\end{itemdecl}

\begin{itemdescr}
\pnum
\effects
Equivalent to:
\begin{codeblock}
++@\exposid{offset_}@;
return *this;
\end{codeblock}
\end{itemdescr}

\begin{itemdecl}
constexpr @\exposid{simd-iterator}@ operator++(int);
\end{itemdecl}

\begin{itemdescr}
\pnum
\effects
Equivalent to:
\begin{codeblock}
@\exposid{simd-iterator}@ tmp = *this;
++@\exposid{offset_}@;
return tmp;
\end{codeblock}
\end{itemdescr}

\begin{itemdecl}
constexpr @\exposid{simd-iterator}@& operator--();
\end{itemdecl}

\begin{itemdescr}
\pnum
\effects
Equivalent to:
\begin{codeblock}
--@\exposid{offset_}@;
return *this;
\end{codeblock}
\end{itemdescr}

\begin{itemdecl}
constexpr @\exposid{simd-iterator}@ operator--(int);
\end{itemdecl}

\begin{itemdescr}
\pnum
\effects
Equivalent to:
\begin{codeblock}
@\exposid{simd-iterator}@ tmp = *this;
--@\exposid{offset_}@;
return tmp;
\end{codeblock}
\end{itemdescr}

\begin{itemdecl}
constexpr @\exposid{simd-iterator}@& operator+=(difference_type n);
\end{itemdecl}

\begin{itemdescr}
\pnum
\effects
Equivalent to:
\begin{codeblock}
@\exposid{offset_}@ += n;
return *this;
\end{codeblock}
\end{itemdescr}

\begin{itemdecl}
constexpr @\exposid{simd-iterator}@& operator-=(difference_type n);
\end{itemdecl}

\begin{itemdescr}
\pnum
\effects
Equivalent to:
\begin{codeblock}
@\exposid{offset_}@ -= n;
return *this;
\end{codeblock}
\end{itemdescr}

\begin{itemdecl}
constexpr value_type operator[](difference_type n) const;
\end{itemdecl}

\begin{itemdescr}
\pnum
\effects
Equivalent to: \tcode{return (*\exposid{data_})[\exposid{offset_} + n];}
\end{itemdescr}

\begin{itemdecl}
constexpr friend bool operator==(@\exposid{simd-iterator}@ i, default_sentinel_t);
\end{itemdecl}

\begin{itemdescr}
\pnum
\effects
Equivalent to: \tcode{return i.\exposid{offset_} == V::size();}
\end{itemdescr}

\begin{itemdecl}
constexpr friend auto operator<=>(@\exposid{simd-iterator}@ a, @\exposid{simd-iterator}@ b);
\end{itemdecl}

\begin{itemdescr}
\pnum
\expects
\tcode{a.\exposid{data_} == b.\exposid{data_}} is \tcode{true}.

\pnum
\effects
Equivalent to: \tcode{return a.\exposid{offset_} <=> b.\exposid{offset_};}
\end{itemdescr}

\begin{itemdecl}
constexpr friend auto operator<=>(@\exposid{simd-iterator}@ i, default_sentinel_t);
\end{itemdecl}

\begin{itemdescr}
\pnum
\effects
Equivalent to: \tcode{return i.\exposid{offset_} <=> V::size();}
\end{itemdescr}

\begin{itemdecl}
constexpr friend @\exposid{simd-iterator}@ operator+(const @\exposid{simd-iterator}@& i, difference_type n);
constexpr friend @\exposid{simd-iterator}@ operator+(difference_type n, const @\exposid{simd-iterator}@& i);
\end{itemdecl}

\begin{itemdescr}
\pnum
\effects
Equivalent to: \tcode{return \exposid{simd-iterator}(*i.\exposid{data_}, i.\exposid{offset_} + x);}
\end{itemdescr}

\begin{itemdecl}
constexpr friend @\exposid{simd-iterator}@ operator-(const @\exposid{simd-iterator}@& i, difference_type n);
\end{itemdecl}

\begin{itemdescr}
\pnum
\effects
Equivalent to: \tcode{return \exposid{simd-iterator}(*i.\exposid{data_}, i.\exposid{offset_} - x);}
\end{itemdescr}

\begin{itemdecl}
constexpr friend difference_type operator-(@\exposid{simd-iterator}@ a, @\exposid{simd-iterator}@ b);
\end{itemdecl}

\begin{itemdescr}
\pnum
\expects
\tcode{a.\exposid{data_} == b.\exposid{data_}} is \tcode{true}.

\pnum
\effects
Equivalent to: \tcode{return a.\exposid{offset_} - b.\exposid{offset_};}
\end{itemdescr}

\begin{itemdecl}
constexpr friend difference_type operator-(@\exposid{simd-iterator}@ i, default_sentinel_t);
\end{itemdecl}

\begin{itemdescr}
\pnum
\effects
Equivalent to: \tcode{return i.\exposid{offset_} - V::size();}
\end{itemdescr}


\begin{itemdecl}
constexpr friend difference_type operator-(default_sentinel_t, @\exposid{simd-iterator}@ i);
\end{itemdecl}

\begin{itemdescr}
\pnum
\effects
Equivalent to: \tcode{return V::size() - i.\exposid{offset_};}
\end{itemdescr}
\end{wgText}

\subsection{Modify [simd.overview]}
\begin{wgText}[{[simd.overview]}]
\setcounter{Paras}{0}
\begin{codeblock}
template<class T, class Abi> class basic_simd {
public:
  using value_type = T;
  using mask_type = basic_simd_mask<sizeof(T), Abi>;
  using abi_type = Abi;
  @\wgAdd{using iterator = \mbox{\exposid{simd-iterator}}<basic_simd>;}@
  @\wgAdd{using const_iterator = \mbox{\exposid{simd-iterator}}<const basic_simd>;}@

  @\wgAdd{constexpr iterator begin();}@
  @\wgAdd{constexpr const_iterator begin() const;}@
  @\wgAdd{constexpr const_iterator cbegin() const;}@
  @\wgAdd{constexpr default_sentinel_t end() const;}@
  @\wgAdd{constexpr default_sentinel_t cend() const;}@
\end{codeblock}
\end{wgText}

\subsection{Modify [simd.mask.overview]}
\begin{wgText}[{[simd.mask.overview]}]
\setcounter{Paras}{0}
\begin{codeblock}
template<size_t Bytes, class Abi> class basic_simd_mask {
public:
  using value_type = bool;
  using abi_type = Abi;
  @\wgAdd{using iterator = \mbox{\exposid{simd-iterator}}<basic_simd_mask>;}@
  @\wgAdd{using const_iterator = \mbox{\exposid{simd-iterator}}<const basic_simd_mask>;}@

  @\wgAdd{constexpr iterator begin();}@
  @\wgAdd{constexpr const_iterator begin() const;}@
  @\wgAdd{constexpr const_iterator cbegin() const;}@
  @\wgAdd{constexpr default_sentinel_t end() const;}@
  @\wgAdd{constexpr default_sentinel_t cend() const;}@
\end{codeblock}
\end{wgText}

\begin{wgText}[{[simd.tuple]}]
\end{wgText}

\end{document}
% vim: sw=2 sts=2 ai et tw=100
