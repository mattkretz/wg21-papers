\newcommand\wgTitle{Reconsider naming of the namespace for “std::simd”}
\newcommand\wgName{
  Matthias Kretz <m.kretz@gsi.de>,\\
  Abhilash Majumder <abmajumder@nvidia.com>,\\
  Bryce Adelstein Lelbach <brycelelbach@gmail.com>,\\
  Daniel Towner <daniel.towner@intel.com>,\\
  Ilya Burylov <iburylov@nvidia.com>,\\
  Mark Hoemmen <mhoemmen@nvidia.com>,\\
  Ruslan Arutyunyan <ruslan.arutyunyan@intel.com>
}
\newcommand\wgDocumentNumber{P3691R1}
\newcommand\wgGroup{LWG}
\newcommand\wgTarget{\CC{}26}
%\newcommand\wgAcknowledgements{}

\usepackage{mymacros}
\usepackage{wg21}
\setcounter{tocdepth}{1} % show only sections in TOC
\hypersetup{bookmarksdepth=5}
\usepackage{changelog}
\usepackage{underscore}
\usepackage{makecell}

\addbibresource{extra.bib}

\DeclareRobustCommand\stdsimd{\code{std\MayBreak::\MayBreak{}simd\MayBreak::\MayBreak}}
\newcommand\simd[1][]{\type{ba\-sic\_simd#1}\xspace}
\newcommand\simdT{\type{ba\-sic\_simd\MayBreak<\MayBreak{}T>}\xspace}
\newcommand\valuetype{\type{val\-ue\_type}\xspace}
\newcommand\referencetype{\type{ref\-er\-ence}\xspace}
\newcommand\mask[1][]{\type{ba\-sic\_simd\_mask#1}\xspace}
\newcommand\maskT{\type{ba\-sic\_simd\_mask\MayBreak<\MayBreak{}T>}\xspace}
\newcommand\wglink[1]{\href{https://wg21.link/#1}{#1}}

\newcommand\nativeabi{\UNSP{native-abi}}
\newcommand\deducet{\UNSP{deduce-t}}
\newcommand\simdsizev{\UNSP{simd-size-v}}
\newcommand\simdsizetype{\UNSP{simd-size-type}}
\newcommand\simdselect{\UNSP{simd-select-impl}}
\newcommand\maskelementsize{\UNSP{mask-element-size}}
\newcommand\integerfrom{\UNSP{integer-from}}
\newcommand\constexprwrapperlike{\UNSP{constexpr-wrapper-like}}
\newcommand\convertflag{\UNSP{convert-flag}}
\newcommand\alignedflag{\UNSP{aligned-flag}}
\newcommand\overalignedflag{\UNSP{overaligned-flag}}
\newcommand\reductionoperation{\UNSP{reduction-binary-operation}}
\newcommand\simdfloatingpoint{\UNSP{simd-floating-point}}
\newcommand\multisimdfloatingpoint{\UNSP{multi-arg-simd-floating-point}}
\newcommand\simditerator{\UNSP{simd-iterator}}

\renewcommand{\lst}[1]{Listing~\ref{#1}}
\renewcommand{\sect}[1]{Section~\ref{#1}}
\renewcommand{\ttref}[1]{Tony~Table~\ref{#1}}
\renewcommand{\tabref}[1]{Table~\ref{#1}}

\begin{document}
\selectlanguage{american}
\begin{wgTitlepage}
  There has been negative feedback on the names we chose for \stddatapar\simd and \stddatapar\mask.
  The pushback is against the name of the namespace.
  This paper does \emph{not reconsider the namespace itself}, it only reconsiders our naming
  options.
  Much of the discussion revolves around the question of whether it is acceptable to repeat a
  namespace name for an entity inside that namespace.
  This paper looks at this question and aims for a “policy-like” LEWG answer (independent of SIMD).
  Among authors there was no clear consensus for one better alternative.
  We therefore invite reflector discussion, as we expect many strong opinions on the matter (naming
  is hard and most importantly not always objective).
  For now, the paper recommends to rename the namespace to \code{simd} and subsequently drop the
  \code{simd_} part from \code{basic_simd_mask} and \code{simd_mask}.
\end{wgTitlepage}

\pagestyle{scrheadings}

\section{Changelog}
\begin{revision}
\item Target \CC{}26, addressing SG1 and LEWG.
\item Call for a merge of the (improved \& adjusted) TS specification to the IS.
\item Discuss changes to the ABI tags as consequence of TS experience; calls for polls to change the status quo.
\item Add template parameter \code{T} to \code{simd_abi::fixed_size}.
\item Remove \code{simd_abi::compatible}.
\item Add (but ask for removal) \code{simd_abi::abi_stable}.
\item Mention TS implementation in GCC releases.
\item Add more references to related papers.
\item Adjust the clause number for [numbers] to latest draft.
\item Add open question: what is the correct clause for [simd]?
\item Add open question: integration with ranges.
\item Add \code{simd_mask} generator constructor.
\item Consistently add simd and simd_mask to headings.
\item Remove experimental and parallelism_v2 namespaces.
\item Present the wording twice: with and without diff against N4808 (Parallelism TS 2).
\item Default load/store flags to \code{element_aligned}.
\item Generalize casts: conditionally \code{explicit} converting constructors.
\item Remove named cast functions.
\end{revision}

\begin{revision}
\item Add floating-point conversion rank to condition of \code{explicit} for converting constructors.
\item Call out different or equal semantics of the new ABI tags.
\item Update introductory paragraph of \sect{sec:changes}; R1 incorrectly kept the text from R0.
\item Define simd::size as a \code{constexpr} static data-member of type \code{integral_constant<size_t, N>}. This simplifies passing the size via function arguments and still be useable as a constant expression in the function body.
\item Document addition of \code{constexpr} to the API.
\item Add \code{constexpr} to the wording.
\item Removed ABI tag for passing \code{simd} over ABI boundaries.
\item Apply cast interface changes to the wording.
\item Explain the plan: what this paper wants to merge vs. subsequent papers for additional features. With an aim of minimal removal/changes of wording after this paper.
\item Document rationale and design intent for \code{where} replacement.
\end{revision}

\begin{revision}
\item Propose alternative to \code{hmin} and \code{hmax}.
\item Discuss \code{simd_mask} reductions wrt. consistency with \code{<bit>}. Propose better names to avoid ambiguity.
\item Remove \code{some_of}.
\item Add unary \code{\~{}} to \code{simd_mask}.
\item Discuss and ask for confirmation of masked ``overloads'' names and argument order.
\item Resolve inconsistencies wrt. \code{int} and \code{size_t}: Change \code{fixed_size} and \code{resize_simd} NTTPs from \code{int} to \code{size_t} (for consistency).
\item Discuss conversions on loads and stores.
\item Point to \cite{P2509R0} as related paper.
\item Generalize load and store from pointer to \code{contiguous_iterator}. (\sect{sec:contiguousItLoadStore})
\item Moved ``\code{element_reference} is overspecified'' to ``Open questions''.
\end{revision}

\begin{revision}
\item Remove wording diff.
\item Add std::simd to the paper title.
\item Update ranges integration discussion and mention formatting support via
  ranges (\sect{sec:formatting}).
\item Fix: pass iterators by value not const-ref.
\item Add lvalue-ref qualifier to subscript operators (\sect{sec:lvalue-subscript}).
\item Constrain \code{simd} operators: require operator to be well-formed on objects of \code{value_type} (\ref{sec:simd.unary}, \ref{sec:simd.binary}).
\item Rename mask reductions as decided in Issaquah.
\item Remove R3 ABI discussion and add follow-up question.
\item Add open question on first template parameter of \code{simd_mask} (\sect{sec:basicsimdmask}).
\item Overload loads and stores with mask argument (\ref{sec:simd.ctor}, \ref{sec:simd.copy}, \ref{sec:simd.mask.ctor}, \ref{sec:simd.mask.copy}).
\item Respecify \simd reductions to use a \mask argument instead of \code{const_where_expression} (\ref{sec:simd.reductions}).
\item Add \mask operators returning a \simd (\ref{sec:simd.mask.unary}, \ref{sec:simd.mask.conv})
\item Add conditional operator overloads as hidden friends to \simd and \mask
  (\ref{sec:simd.cond}, \ref{sec:simd.mask.cond}).
\item Discuss \std\code{hash} for \simd (\sect{sec:hash}).
\item Constrain some functions (e.g., min, max, clamp) to be \code{totally_ordered} (\ref{sec:simd.reductions}, \ref{sec:simd.alg}).
\item Asking for reconsideration of conversion rules.
\item Rename load/store flags (\sect{sec:renameandextendflags}).
\item Extend load/store flags with a new flag for conversions on load/store. (\sect{sec:renameandextendflags}).
\item Update \code{hmin}/\code{hmax} discussion with more extensive naming discussion (\sect{sec:hminhmax}).
\item Discuss freestanding \simd (\sect{sec:freestanding}).
\item Discuss \code{split} and \code{concat} (\sect{sec:splitandconcat}).
\item Apply the new library specification style from P0788R3.
\end{revision}

\begin{revision}
\item Added \code{simd_select} discussion.
\end{revision}

\begin{revision}
\item Updated the wording for changes discussed in and requested by LEWG in Varna.
\item Rename to \code{simd_cat} and \code{simd_split}.
\item Drop \code{simd_cat(array)} overload.
\item Replace \code{simd_split} by \code{simd_split} as proposed in P1928R4.
\item Use \code{indirectly_writable} instead of \code{output_iterator}.
\item Replace most \code{size_t} and \code{int} uses by \code{\textit{simd-size-type}} signed integer type.
\item Remove everything in \code{simd_abi} and the namespace itself.
\item Reword section on ABI tags using exposition-only ABI tag aliases.
\item Guarantee generator ctor calls callable exactly once per index.
\item Remove \code{int}/\code{unsigned int} exception from conversion rules of broadcast ctor.
\item Rename \code{loadstore_flags} to \code{simd_flags}.
\item Make \code{simd_flags::operator|} \code{consteval}.
\item Remove \code{simd_flags::operator\&} and \code{simd_flags::operator\^}.
\item Increase minimum SIMD width to 64.
\item Rename \code{hmin}/\code{hmax} to \code{reduce_min} and \code{reduce_max}.
\item Refactor \code{simd_mask<T, Abi>} to \code{basic_simd_mask<Bytes, Abi>} and replace all occurrences accordingly.
\item Rename \code{simd<T, Abi>} to \code{basic_simd<Bytes, Abi>} and replace all occurrences accordingly.
\item Remove \code{long double} from the set of vectorizable types.
\item Remove \code{is_abi_tag}, \code{is_simd}, and \code{is_simd_mask} traits.
\item Make \code{simd_size} exposition-only.
\end{revision}

\begin{revision}
\item Remove mask reduction precondition but ask LEWG for reversal of that decision (\sect{sec:removemaskreductionprecondition}).
\item Fix return type of \mask unary operators.
\item Fix \code{bool} overload of \simdselect (\sect{sec:simdselectwording}).
\item Remove unnecessary implementation freedom in \code{simd_split} (\sect{sec:bettersimdsplitwording}).
\item Use \code{class} instead of \code{typename} in template heads.
\item Implement LEWG decision to SFINAE on \emph{values} of
  constexpr-wrapper-like arguments to the broadcast ctor (\ref{sec:simd.ctor}).
\item Add relational operators to \mask as directed by LEWG (\ref{sec:simd.mask.comparison}).
\item Update section on \code{size_t} vs. \code{int} usage (\sect{sec:simdsizetype}).
\item Remove all open design questions, leaving LWG / wording questions.
\item Add LWG question on implementation note (\sect{sec:implnote}).
\item Add constraint for \code{BinaryOperation} to \code{reduce} overloads (\ref{sec:simd.reductions}).
%  \todo Add \code{numeric_limits} / numeric traits specializations since behavior of e.g. \code{simd<float>} and \code{float} may differ for reasonable implementations.
\end{revision}

\begin{revision}
\item Include \code{std::optional} return value from \code{reduce_min_index} and \code{reduce_max_index} in the exploration.
\item Fix \LaTeX{} markup errors.
\item Remove repetitive mention of “exposition-only” before \deducet.
\item Replace “TU” with “translation unit”.
\item Reorder first paragraphs in the wording, especially reducing the note on compiling down to SIMD instructions.
\item Replace cv-unqualified arithmetic types with a more precise list of types.
\item Move the place where “supported” is defined.
\end{revision}

\begin{revision}
\item Improve wording that includes the \CC{}23 extended floating-point types in the set of vectorizable types (\ref{wording.vectorizable.types}).
\item Improve wording that defines “selected indices” and “selected elements” (\ref{wording.selected.indices}).
\item Remove superfluous introduction paragraph.
\item Improve wording introducing the intent of ABI tags (\ref{wording.ABI.tag})
\item Consistently use \code{size} as a callable in the wording.
\item Add missing \code{type_identity_t} for \code{reduce} (\ref{sec:simd.syn}, \ref{sec:simd.reductions}).
\item Spell out “iff” (\ref{wording.deducet}).
\item Fixed template argument to \nativeabi\ in the default template argument of \code{basic_simd_mask} (\ref{sec:simd.syn}).
\item Fixed default template argument to \code{simd_mask} to be consistent with \code{simd} (\ref{sec:simd.syn}).
\item Add instructions to add \code{<simd>} to the table of headers in [headers].
\item Add instructions to add a new subclause to the table in [numerics.general].
\item Add instructions to add \code{<simd>} [diff.23.library].
\item Add \simdsizev to the wording and replace \code{simd_size_v} to actually implement “Make \code{simd_size} exposition-only.”
\item Restored precondition (and removed \code{noexcept}) on
  \code{reduce_min_index} and \code{reduce_max_index} as directed by LEWG.
\end{revision}

\begin{revision}
\item Strike through wording removed by P3275 (non-const \code{operator[]}).
\item Remove “exposition only” from detailed prose, it's already marked as such in the synopsis.
\item Reorder defintion of \emph{vectorizable type} above its first use.
\item Commas, de-duplication, word order, \code{s/may/can/} in a note.
\item Use text font for “[)” when defining a range of integers.
\item Several small changes from LWG review on 2024-06-26.
\item Reword \code{rebind_simd} and \code{resize_simd}.
\item Remove mention of implementation-defined load/store flags.
\item Remove paragraph about default initialization of \simd.
\item Reword all constructor \emph{Effects} from “Constructs an object \ldots”
  to “Initializes \ldots”.
\item Instead of writing “satisfies X” in \emph{Constraints} and “models X” in
  \emph{Preconditions}, say only “models X” in \emph{Constraints}.
\item Replace \code{is_trivial_v} with “is trivially copyable”.
\item First shot at improving generator function constraints.
\item Reword constraints on unary and binary operators.
\item Add missing/inconsistent \code{explicit} on load constructors.
\item Fix preconditions of subscript operators.
\item Reword effects of compound assignment operators.
\item Add that \code{BinaryOperation} may not modify input \simd.
\item Fix definition of GENERALIZED_SUMs.
\end{revision}

\begin{revision}
\item Say “\textit{op}” instead of “the indicated operator”
\item Fix constraints on shift operators with \simdsizetype{} on the right operand.
\item Remove wording removed by P3275 (non-const \code{operator[]}).
\item Make intrinsics conversion recommended practice.
\item Make \code{simd_flags} template arguments exposition-only.
\item Make \code{simd_alignment} \emph{not} implementation-defined.
\item Reword “supported” to “enabled or disabled”.
\item Apply improved wording from \ref{sec:simd.overview} to \ref{sec:simd.mask.overview}.
\item Add comments for LWG to address to broadcast ctor (\ref{sec:simd.ctor}).
\item Respecify generator ctor to not reuse broadcast constraint (\ref{sec:simd.ctor}).
\item Use \code{to_address} on contiguous iterators (\ref{sec:simd.ctor} and \ref{sec:simd.copy}).
  This is more explicit about allowing memcpy on the complete range rather than
  having to iterate the range per element.
\end{revision}

\begin{revision}
\item Fix default size of \code{simd} and \code{simd_mask} aliases
  (\ref{sec:simd.syn}, necessary for
  \std\code{destructible<\MayBreak{}\std{}simd<\MayBreak\std{}string>>} to be well-formed).
\item Extend value-preserving to encompass conversions from all arithmetic
  types. Use this new freedom in \ref{sec:simd.ctor} to fully constrain the
  generator constructor and to plug a specification hole in the broadcast
  constructor.
\item Fix broadcast constructor wording by constraining \constexprwrapperlike
  arguments to arithmetic types.
  %\todo Reorder \code{simd} and \code{simd_mask} specification in the wording (mask first).
\end{revision}

\section{Straw Polls}


\section{Criticism and discussion on the current name}\label{sec:criticism}
It seems that the namespace name \code{datapar} doesn't appeal to some because “datapar” appears
as a newly \emph{invented name} and that's not a good fit for a namespace in the standard library.
A namespace name should describe the contents of the namespace without requiring an explanation.

“Datapar” was orignally invented (and used for some time) for the \code{simd<T>} type leading up to
the Parallelism TS 2.
The rationale behind the name “datpar” was that the \code{simd<T>} type does more than just SIMD: it
is an abstraction for expressing data-parallelism.
As such “datapar” describes a superset of what SIMD is.
This is why the combination \code{datapar::simd<T>} may look strange to some:
SIMD is one concrete implementation of data parallelism, but the namespace name implies there is
more.
The combination of “datapar” and “SIMD” is the strange part about the status quo.
Saying “data-parallel vector [of \code{int}]” / “data-parallel \code{int}” or a saying “SIMD vector
[of \code{int}]” / “SIMD \code{int}” is fine.
But a “data-parallel SIMD [vector] [of \code{int}]” is a strange thing to say/hear/read.

In the last LEWG discussion on \cite{P3287R2}, an alternative spelling for \code{datapar},
\code{dataparallel} was raised and voted on (with an even split on either).
So what if we were to spell it out to \code{dataparallel} or \code{data_parallel}?
Then we still have the issue from above, that “data-parallel SIMD” sounds strange because SIMD that
is not data-parallel doesn't exist, and data parallelism is more than just SIMD.
But more importantly, the term “data-parallel” is an adjective.
We only ever used an adjective for “experimental” APIs:\\
\begin{tabular}{lll}
  \code{std::chrono} &
  \code{std::contracts} &
  \code{std::execution} \\
  \code{std::experimental} (the exception) &
  \code{std::filesystem} &
  \code{std::linalg} \\
  \code{std::literals} &
  \code{std::numbers} &
  \code{std::placeholders} \\
  \code{std::pmr} &
  \code{std::ranges} &
  \code{std::regex_constants} \\
  \code{std::rel_ops} (deprecated) &
  \code{std::this_thread} &
  \code{std::views} \\
\end{tabular}

However, the intent for \code{datapar} (in \cite{P3287R2}) was to be an abbreviation of
“Data-parallel types”, the subclause heading in the current working draft.
There is precedent with e.g. the \code{std::pmr} and \code{std::linalg} namespaces of using an
abbreviation.
The name “pmr” also needs an explanation.
Does that make “pmr” a bad name or does it help to justify “datapar”?

Consequently, the argument that \std\code{datapar} has to be spelled out in a longer form as the
adjective \std\code{data_parallel}, isn't necessarily correct.
If instead we were to spell it out as \std\code{data_parallel_types}\footnote{
  Other variations: \code{std::data_parallel_execution}, \code{std::data_parallel_algorithms},
  \code{std::data_parallel_numerics}, \code{std::data_parallelism}
} or \std\code{dpt}, that argument doesn't hold anymore.
%If those names are too long then you can argue against them from that position but I don't think you can argue that an adjective is a bad name for a namespace, without considering a non-adjective namespace name as well.

Thus, it appears that any argument for replacing the \code{datapar} status quo with \code{simd} has
to be that \emph{\code{datapar} is unclear and the clearer alternatives are too long}.

One argument that was used against \stdsimd\code{simd} (and thus the only reason we considered a
different namespace name) is the ambiguity between namespace and class
name and whether one is looking at a constructor (\code{T::T}).
What had not been mentioned in these discussions is that \code{simd} is actually not a class
template name but alias template.\footnote{Granted, this makes little difference for some of us.}
The default constructor, for example, therefore is spelled
\stdsimd\code{basic_simd::basic_simd()}.\footnote{In diagnostic output it is typically spelled
\stdsimd\code{basic_simd<T, Abi>::basic_simd() [with T = ..., Abi = ...]}}
Nevertheless, the repetition of “simd” looks odd and might be confusing to humans and simple tools.
Examples of libraries that do something similar are Boost.Flyweight and Boost.Histogram:
\medskip\begin{lstlisting}[style=Vc]
namespace boost
{
  namespace flyweights
  {
    template</*...*/> class flyweight;
  }
  using flyweights::flyweight;

  namespace histogram
  {
    template<typename Axes, typename Storage> class histogram;
  }
}
\end{lstlisting}
So this has been done before and passed through Boost review.
Some of us believe the fear, uncertainty, and doubts about repeating the name of the namespace in
the alias template are possibly no more than that; and in reality the “problem” is little to
non-existent.

\section{A survey of existing SIMD libraries}

Interestingly, no \CC{} library uses the term “SIMD” in the name of the (unqualified) type.
There are\footnote{“were”, \code{boost::simd} doesn't exist anymore} two libraries that used “SIMD”
in the namespace.
Rust, however, uses \code{Simd} for the type name and just plain \code{Mask} for the mask type.
Rust has no need for an additional namespace scope because all operations on \code{Simd} are
implemented as operators or member functions\footnote{Rust does not enable SIMD-generic
programming}.

\subsection{Agner Fog's C++ vector class library}

The library does not define class templates but classes like \code{Vec4f}, \code{Vec16c},
\code{Vec2uq}, \ldots
The corresponding mask types have an additional \code{b} suffix in their name.

\medskip\begin{lstlisting}
class Vecf4;   // ~simd<float, 4>
class Vecf4b;  // ~simd_mask<float, 4>
\end{lstlisting}


\subsection{“boost” simd}

\medskip\begin{lstlisting}
namespace simd
{
  template <class T> struct pack; // ~simd
  template <> struct pack<bool>;  // ~simd_mask
}
\end{lstlisting}

\subsection{E.V.E}

\medskip\begin{lstlisting}
namespace eve
{
  template <class T, /*Cardinal*/> struct wide; // ~ simd
  template <class T> struct logical;            // ~ simd_mask
}
\end{lstlisting}

\subsection{Highway}

\medskip\begin{lstlisting}[style=Vc]
namespace hwy
{
  template <class T, size_t N> class Vec128;  // ~ simd
  template <class T, size_t N> class Mask128; // ~ simd_mask
  // ...
}
\end{lstlisting}


\subsection{Vc}

\medskip\begin{lstlisting}
namespace Vc
{
  template <class T, class Abi> class Vector;  // ~ basic_simd
  template <class T, class Abi> class Mask;    // ~ basic_simd_mask
  using double_v = Vector<double>;
  using double_m = Mask<double>;
  using int_v = Vector<int>;
  using int_m = Mask<int>;
  // ...
}
\end{lstlisting}

\subsection{xsimd}

\medskip\begin{lstlisting}
namespace xsimd
{
  template <class T, class A> class batch;       // ~ basic_simd
  template <class T, class A> class batch_bool;  // ~ basic_simd_mask
}
\end{lstlisting}

\subsection{Other programming languages}

\begin{itemize}
  \item Rust's experimental std::simd:
    \medskip\begin{lstlisting}
    pub struct Simd<T, const N: usize> ...; // ~ simd
    pub struct Mask<T, const N: usize> ...; // ~ simd_mask
    \end{lstlisting}

  \item C\#: \code{System.Numerics.Vector<T>}

  \item Java Panama Vector API:\\
    \code{jdk.incubator.vector.VectorSpecies<Float>}\\
    \code{jdk.incubator.vector.VectorMask<Float>}

  \item Swift SIMD module:\\
    \code{protocol SIMD<Scalar>}\\
    \code{struct SIMD8<Scalar>}\\
    \code{struct SIMDMask<Storage>}

  \item Julia SIMD.jl: \code{xs = Vec{4,Float64}(1)}

\end{itemize}




\section{Names considered}

The name of the namespace and the name of the class (and alias) templates need to be considered as a
whole.
It is not helpful to consider a namespace name or class name in isolation.
The names of the class/alias templates also basically never appear without at least one template
argument in user code.
Consequently, the following presentation will include the namespace, class template, and first
template argument.

\subsection{Options for \simd}
\begingroup%
\smaller%
\newcommand\optionrow[3]{\hline\makecell[cl]{\code{std::#1::basic_#2<int>}\\\code{std::#1::#2<int>}} & \makecell[cl]{#3}\\}%
\begin{tabular}{l|l}
  \thead{class template \\ alias template} & \thead{obvious criticism} \\
  \optionrow{simd}{simd}{repetitive and human-ambiguous}
  \optionrow{dpt}{simd}{what is “dpt”? (like what is “pmr”?)}
  \optionrow{datapar}{simd}{see \sect{sec:criticism}}
  \optionrow{dataparallel}{simd}{see \sect{sec:criticism}}
  \optionrow{data_parallelism}{simd}{too long? also see \sect{sec:criticism}}
  \optionrow{simd}{batch}{}
  \optionrow{simd}{number}{it's not a number, but set of numbers}
  \optionrow{simd}{numbers}{we already have \code{std::numbers::*}\\what are “SIMD numbers”?}
  \optionrow{simd}{pack}{\CC{} already has parameter packs; \\ another existing meaning: packed structs}
  \optionrow{simd}{value}{too many variables are named \code{value}}
  \optionrow{simd}{vec}{sounds like a container}
  \optionrow{simd}{vector}{sounds even more like a container}
  \optionrow{simd}{wide}{like in \code{wchar_t}?\\“SIMD wide” and “SIMD basic wide” need an explanation}
  \optionrow{simd}{chunk}{this is not a chunk out of a SIMD register}
  \optionrow{datapar}{chunk}{where did the “SIMD” name go?}
  \optionrow{dataparallel}{chunk}{ditto}
  \optionrow{dataparallel}{numbers}{long; any abbreviation becomes unclear}
\end{tabular}
\endgroup

\subsection{Options for \mask}
\nopagebreak
\begingroup%
\smaller%
\newcommand\optionrow[3]{\hline\makecell[cl]{\code{std::#1::basic_#2<4>}\\\code{std::#1::#2<int>}} & \makecell[cl]{#3}\\}%
\begin{tabular}{l|l}
  \thead{class template \\ alias template} & \thead{obvious criticism} \\
  \optionrow{simd}{mask}{}
  \optionrow{simd}{simd_mask}{repetitive}
  \optionrow{dpt}{mask}{relation to \code{dpt::simd} only via namespace}
  \optionrow{dpt}{simd_mask}{repetitive like \code{dpt::simd}}
  \optionrow{datapar}{mask}{LEWG already voted against this}
  \optionrow{datapar}{simd_mask}{repetitive in a different way}
  \optionrow{dataparallel}{simd_mask}{ditto}
  \optionrow{simd}{logical}{}
  \optionrow{simd}{batch_bool}{}
  \optionrow{simd}{simd_bool}{}
  \optionrow{simd}{boolean}{}
\end{tabular}
\endgroup

\section{Discussion}

There seems to be a strong desire to have “SIMD” in the name somewhere.
This desire seems to be about findability and about trying not to invent a new name where the
industry already recognizes an existing name.
(Potentially also about buzzword compliance and delivering on a name we have been communicating
before?)

If we were to use a different name for the namespace and class/alias templates, is the namespace
or the class/alias name more important and thus needs to use the “SIMD” name?
The class/alias name is the identifier that will always appear in the source code so choosing a
class name which is ambiguous once the namespace is removed (e.g., with a using statement) may obscure
the codes intent.
We would end up with variables with types like \code{pack<int>}, \code{vec<int>}, \code{batch<int>},
\code{wide<int>}, and so on, and all hint of their SIMD behaviour is lost.
The only viable alternative term that could potentially stand on its own (in terms of hinting at
behavior) is “vector”.
But that train has left the station decades ago.

On the other hand, if we consider the group of types and functions in the “SIMD” namespace to be a
“SIMD-only” library (i.e., no other data-parallelism abstractions unrelated to the \code{simd}
type), then shouldn't the namespace have the “SIMD” name?
If we expect common practice to use fully qualified names or use of a namespace alias
(\code{namespace simd = std::simd}; there's no other conceivable abbreviation other than maybe
\code{namespace ss = std::simd}), then \code{simd} is already always part of the name.
Consequently, we could then choose a name “SIMD <something>” to avoid the perceived ambiguity of
\code{simd::simd}.

On the topic of ambiguity, does anyone feel one of the following functions is problematic?
Is there any example that can create “visual confusion” to a human reader?
I specifically combined \code{reduce} and the static data member \code{size} into one line of
(non-sensical) code because I suspect that's the most contentious syntax similarity.
\medskip\begin{lstlisting}[style=Vc]
int f1() {
  std::simd::simd v = std::array {1, 2, 3, 4};
  return std::simd::reduce(v) + std::simd::simd<int>::size();
}

int f2() {
  namespace simd = std::simd;
  simd::simd v = std::array {1, 2, 3, 4};
  return simd::reduce(v) + simd::simd<int>::size();
}

int f3() {
  using std::simd::simd;
  simd v = std::array {1, 2, 3, 4};
  return std::simd::reduce(v) + simd<int>::size();
}

int f4() {
  using std::simd::simd;
  simd v = std::array {1, 2, 3, 4};
  return reduce(v) + simd<int>::size();
}

int f5() {
  using namespace std;
  simd::simd v = std::array {1, 2, 3, 4};
  return simd::reduce(v) + simd::simd<int>::size();
}

int f6() {
  using namespace std::simd;
  simd v = std::array {1, 2, 3, 4};
  return reduce(v) + simd<int>::size();
}
\end{lstlisting}
Naming the namespace of the \code{reduce} function after the type reinforces the
connection between function and type it operates on, which improves clarity and cohesion.
Whereas the status-quo of \stddatapar\code{reduce} used in isolation conveys less about its intended
argument.
The counter-argument here is that \code{simd::reduce} looks like a call to a static member function
inside \code{simd}. (It can't be, though, because \code{simd} is an alias \emph{template} and thus
requires a template argument.)

\section{Main positions}

There seem to be three main positions that we were able to identify.
These positions inform where the “over my dead body” and “any of these is fine” opinions originate.

\subsection{Acceptable to repeat name in namespace and type}
People who take no issue\footnote{though, typically there's still a preference for different names
--- if there's an obvious set of names} with repeating the name of the namespace for a type inside
that namespace tend to favor \stdsimd\code{simd<T>} / \stdsimd\code{mask<T>}.
Those who take issue with such repetition fall into one of the following two categories.

\subsection{The namespace name carries more weight}
People on this position expect the namespace name to always be visible in code and diagnostics and
therefore do the heavy lifting.
Since there seems to be consensus on using the term SIMD, the namespcae name thus should be “simd”.
The name of the class/alias consequently needs to be something that reads as “SIMD <foo>”, where
“<foo>” must not contain “SIMD”.
Favored outcomes are \stdsimd\code{pack}, \stdsimd\code{vec}, \stdsimd\code{batch},
\stdsimd\code{wide}, \ldots

\subsection{The class/alias name carries more weight}
Poeple on this position consider the namespace name optional in code.
Therefore, the class/alias name must stand on its own.
Again, with the consensus on using “SIMD”, the class/alias name should be “simd” and the namespace
name should be something else.
Favored outcomes are \stddatapar\code{simd}, \std\code{dpt::simd},
\std\code{data_parallelism::simd}, \ldots

\section{Conclusion}
\label{sec:conclusion}

After LEWG discussion and polls we converged at the following conclusion:
Go back to the advertised “std::simd” name by naming the namespace \code{std::\MayBreak{}simd}.
In order to avoid the unnecessary duplication in \stdsimd\code{simd_mask}, rename
\stdsimd\code{basic_simd_mask} to \stdsimd\code{basic_mask} and \stdsimd\code{simd_mask} to
\stdsimd\code{mask}.
To avoid the repetition in \stdsimd\code{simd}, rename \stdsimd\code{basic_simd} to
\stdsimd\code{basic_vec} and \stdsimd\code{simd} to \stdsimd\code{vec}.

\section{Wording}

\subsection{Feature test macro}

No feature test macro is added or bumped.

\subsection{Ordering constraints on LWG motions}

All other motions against [simd] need to be applied first.

\subsection{Instructions to the editor}

In 29.10 Data-parallel types [simd],

\begin{enumerate}
  \item change every occurrence of \code{basic_\wgRem{simd_}mask};
  \item change every occurrence of \code{\wgRem{simd_}mask};
  \item change every occurrence of \code{basic_\wgChange{simd}{vec}};
  \item change every occurrence of \code{\wgChange{simd}{vec}};
  \item change every occurrence of \code{namespace std::\wgChange{datapar}{simd} \{};
  \item change every occurrence of \code{using \wgChange{datapar}{simd}::}.
\end{enumerate}

\end{document}
% vim: sw=2 sts=2 ai et tw=100
