\newcommand\wgTitle{Consistency fix: Make \code{simd} reductions SIMD-generic}
\newcommand\wgName{Olaf Krzikalla <olaf.krzikalla@dlr.de>,
Matthias Kretz <m.kretz@gsi.de>}
\newcommand\wgDocumentNumber{D3690R1}
\newcommand\wgGroup{LWG}
\newcommand\wgTarget{\CC{}26}
%\newcommand\wgAcknowledgements{}

\usepackage{mymacros}
\usepackage{wg21}
\setcounter{tocdepth}{2} % show sections and subsections in TOC
\hypersetup{bookmarksdepth=5}
\usepackage{changelog}
\usepackage{underscore}
\usepackage{multirow}

\addbibresource{extra.bib}

\newcommand\simd[1][]{\type{basic\_simd#1}\xspace}
\newcommand\simdT{\type{basic\_simd<T>}\xspace}
\newcommand\valuetype{\type{value\_type}\xspace}
\newcommand\referencetype{\type{reference}\xspace}
\newcommand\simdcast{\code{simd\_cast}\xspace}
\newcommand\mask[1][]{\type{basic\_simd\_mask#1}\xspace}
\newcommand\maskT{\type{basic\_simd\_mask<T>}\xspace}
\newcommand\fixedsizeN{\type{simd\_abi::fixed\_size<N>}\xspace}
\newcommand\fixedsizescoped{\type{simd\_abi::fixed\_size}\xspace}
\newcommand\fixedsize{\type{fixed\_size}\xspace}
\newcommand\wglink[1]{\href{https://wg21.link/#1}{#1}}
\DeclareRobustCommand\simdabi{\code{simd\_abi\MayBreak::\MayBreak}}

\newcommand\nativeabi{\UNSP{native-abi}}
\newcommand\deducet{\UNSP{deduce-abi-t}}
\newcommand\simdsizev{\UNSP{simd-size-v}}
\newcommand\simdsizetype{\UNSP{simd-size-type}}
\newcommand\simdselect{\UNSP{simd-select-impl}}
\newcommand\maskelementsize{\UNSP{mask-element-size}}
\newcommand\integerfrom{\UNSP{integer-from}}
\newcommand\constexprwrapperlike{\UNSP{constexpr-wrapper-like}}
\newcommand\convertflag{\UNSP{convert-flag}}
\newcommand\alignedflag{\UNSP{aligned-flag}}
\newcommand\overalignedflag{\UNSP{overaligned-flag}}
\newcommand\reductionoperation{\UNSP{reduction-binary-operation}}
\newcommand\simdfloatingpoint{\UNSP{simd-floating-point}}
\newcommand\deducedsimd{\UNSP{deduced-simd-t}}
\newcommand\makecompatiblesimdt{\UNSP{make-compatible-simd-t}}
\newcommand\mathfloatingpoint{\UNSP{math-floating-point}}
\newcommand\mathcommonsimd{\UNSP{math-common-simd-t}}

\renewcommand{\lst}[1]{Listing~\ref{#1}}
\renewcommand{\sect}[1]{Section~\ref{#1}}
\renewcommand{\ttref}[1]{Tony~Table~\ref{#1}}
\renewcommand{\tabref}[1]{Table~\ref{#1}}

\begin{document}
\selectlanguage{american}
\begin{wgTitlepage}
  One design goal of the \code{simd} interface is to enable SIMD-generic code.
  This is why all arithmetic operators and functions have corresponding overloads.
  However, arithmetic reductions are still missing overloads for
  non-\code{simd}, vectorizable types.
\end{wgTitlepage}

\pagestyle{scrheadings}

\section{Changelog}
(placeholder)
\begin{revision}
\item Add a simple example to the motivation section.
\item Expand the “Generalization” section to clearly define the feature rather
  than just sketching it.
  Also add a discussion of initial value and step.
\item Discuss why reusing the existing \code{iota} algorithm/view does not
  work/suffice for the \code{simd} use case.
\item Discuss why \code{iota_v} is the right name.
%  \todo
\end{revision}

\section{Straw Polls}
\subsection{SG1 at Kona 2022}
\wgPoll{After significant experience with the TS, we recommend that the next
version (the TS version with improvements) of \code{std::simd} target the IS (\CC{}26)}
{10&8&0&0&0}

\wgUnanimous{We like all of the recommended changes to \code{std::simd} proposed in p1928r1
(Includes making all of \code{std::simd} \code{constexpr}, and dropping an ABI stable type)}

\wgPoll{Future papers and future revisions of existing papers that target
\code{std::simd} should go directly to LEWG.
(We do not believe there are SG1 issues with \code{std::simd} today.)}
{9&8&0&0&0}


\section{Introduction}
\cite{P1928R15} introduced \code{std::simd} and related types and functions.
It enables the programmer to write \emph{SIMD-generic} code, i.e. function templates instantiable
with scalar types as well as vector types.
\medskip\begin{lstlisting}[style=Vc]
template<class T>
auto f(const T& x, const T& y, const T& z)
{
  return x + std::sqrt(y) * std::pow(z);
}
\end{lstlisting}
This function template can be instantiated with scalar floating types, with complex types, or with their
vectorized counterparts (e.g. \code{std::simd<double>} or \code{std::simd<std::complex<double>>}).

SIMD-generic code is also possible for boolean reductions.
\medskip\begin{lstlisting}[style=Vc]
template<class T>
bool all_lt(const T& x, const T& y)
{
  return std::datapar::all_of(x < y);
}
\end{lstlisting}
Such code is possible because~\cite{P1928R15} explicitly includes overloads for \code{all_of(bool)} aso.

On the other hand, we believe, that~\cite{P1928R15} just forgot to provide scalar overloads for arithmetic reductions.
\medskip\begin{lstlisting}[style=Vc]
template<class T>
auto calc_contribution(const T& x, const T& y)
{
  return std::datapar::reduce(x * y);
}
\end{lstlisting}
With just~\cite{P1928R15} this code is not SIMD-generic yet.
It cannot be called with scalar types, as there is no scalar overload for \code{std::datapar::reduce} yet.
That makes this part of the simd interface incoherent with the rest that does work.

\section{Proposal}

We propose an introduction of scalar overloads for all arithmetic reduce functions introduced in~\cite{P1928R15}.
This applies to the functions in 29.10.7.5: \code{reduce}, \code{reduce_min}, and \code{reduce_max}.
The semantic of the functions is mostly trivial: they just return the passed argument.
The masked functions take a scalar \code{bool} as mask argument.
If the value of that argument is \code{false}, then the functions behave like their vectorized counterparts
if \code{none_of(mask) == true} applies to them.

\section{Discussion}

When the function was still called \code{std::reduce}
there was some doubt whether such an overload of the name could be too much.
But now that reduce moved into the subnamespace \code{std::datapar}
there is no apparent reason to avoid a scalar overload of reduce.

\section{Wording}\label{sec:wording}
\subsection{Feature test macro}

In [version.syn] bump the \code{__cpp_lib_simd} version.

\subsection{Changes to {[simd]}}
\def\rSec#1[#2]#3{%
  \ifcase#1\wgSubsection[subsection]{#3}{#2}
  \or\wgSubsubsection[subsubsection]{#3}{#2}
  \or\wgSubsubsubsection[paragraph]{#3}{#2}
  \or\error
\fi}

Add the following to \iref{simd.syn}, after the declaration of \code{cat}:
\begin{wgText}[{[simd.syn]}]
\begin{codeblock}
  template<size_t Bs, class... Abis>
    constexpr basic_simd_mask<Bs, @\deducet@<@\integerfrom@<Bs>,
                              (basic_simd_mask<Bs, Abis>::size() + ...)>>
      cat(const basic_simd_mask<Bs, Abis>&...) noexcept;

  @\wgAdd{template<class T> inline constexpr T iota = \mbox{\seebelow};}@

  // [simd.mask.reductions], \tcode{basic_simd_mask} reductions
\end{codeblock}
\end{wgText}

Add the following at the end of \iref{simd.creation}:
\begin{wgText}[{[simd.creation]}]
  \setcounter{WGClause}{29}
  \setcounter{WGSubSection}{10}
  \setcounter{WGSubSubSection}{7}
  \setcounter{WGSubSubSubSection}{6}
  \setcounter{Paras}{4}
\begin{itemdescr}
  \pnum\returns
  A data-parallel object initialized with the concatenated values in the \tcode{xs} pack of
  data-parallel objects: The $i^\text{th}$ \tcode{basic_simd}/\tcode{basic_simd_mask} element of the
  $j^\text{th}$ parameter in the \tcode{xs} pack is copied to the return value's element with index
  $i$ + the sum of the width of the first $j$ parameters in the \tcode{xs} pack.
\end{itemdescr}

\begin{wgBAdd}
\begin{itemdecl}
@\wgAdd{template<class T> inline constexpr T iota = \mbox{\seebelow};}@
\end{itemdecl}

\begin{itemdescr}
\pnum
\wgAdd{\constraints \tcode{is_arithmetic_v<T>} is \tcode{true} or \tcode{T}
is an enabled specialization of \code{basic_simd}.}

\pnum
\wgAdd{\mandates \tcode{is_arithmetic_v<T>} is \tcode{true} or
\tcode{T::size() - 1} $\le$ \tcode{numeric_limits<typename T::value_type>:: max()}.}

  \pnum
  \wgAdd{\effects
    If \tcode{is_arithmetic_v<T>} is \tcode{true} the value of
    \tcode{iota<T>} is equal to \tcode{T()}.
    Otherwise, the value of \tcode{iota<T>} is equal to \tcode{T([](typename
    T::value_type i) \{ return i; \})}.
  }
\end{itemdescr}
\end{wgBAdd}

\wgSubsubsubsection[paragraph]{Algorithms}{simd.alg}
\end{wgText}


\end{document}
% vim: sw=2 sts=2 ai et tw=0
