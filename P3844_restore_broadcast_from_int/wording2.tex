\clearpage
\section{Wording}

\subsection{Feature test macro}

In [version.syn] bump the \code{__cpp_lib_simd} version.

\subsection{Modify [simd.expos]}

In [simd.expos], insert:
\begin{wgText}[{[simd.expos]}]
\begin{codeblock}
@\wgRem{template<class V, class T> using \exposid{make-compatible-simd-t} = \seebelow; \rmfamily\itshape// \expos}@

@\wgAdd{template<class From, class To>}@
  @\wgAdd{concept \defexposconceptnc{simd-consteval-broadcast-arg} = \seebelownc;}@                  @\wgAdd{\rmfamily\itshape// \expos}@

template<class V>
  concept @\defexposconceptnc{simd-vec-type}@ =                                            // \expos
\end{codeblock}
\end{wgText}

\subsection{Modify [simd.expos.defn]}

In [simd.expos.defn], change:
\begin{wgText}[{[simd.expos.defn]}]
\begin{itemdecl}
template<class T> using @\exposid{deduced-vec-t}@ = @\seebelow@;
\end{itemdecl}

\begin{itemdescr}
\pnum
Let \tcode{x} denote an lvalue of type \tcode{const T}.

\pnum
\tcode{\exposid{deduced-vec-t}<T>} is an alias for
\begin{itemize}
  \wgItemRem\color{black}
   \tcode{decltype(x + x)}, if the type of \tcode{x + x} is an enabled
   specialization of \tcode{basic_vec}\wgRem{; otherwise}
   %TODO: convertible_to<const T&, const decltype(x + x)&> ?
  \wgItemRem[\tcode{void}]\color{black}.
\end{itemize}
\end{itemdescr}

\begin{itemdecl}
@\wgRem{template<class V, class T> using \exposid{make-compatible-simd-t} = \seebelow;}@
\end{itemdecl}

\begin{itemdescr}
\pnumRem
\wgRem{Let \tcode{x} denote an lvalue of type \tcode{const T}.}

\pnumRem
\wgRem{\tcode{\exposid{make-compatible-simd-t}<V, T>} is an alias for}
\begin{itemize}
 \wgItemRem[
   \tcode{\exposid{deduced-vec-t}<T>}, if that type is not \tcode{void},
 otherwise]
 \wgItemRem[
 \tcode{vec<decltype(x + x), V::size()>}.]
\end{itemize}
\end{itemdescr}

\begin{itemdecl}
@\wgAdd{template<class From, class To> concept \defexposconceptnc{simd-consteval-broadcast-arg} = \seebelownc;}@
\end{itemdecl}

\begin{itemdescr}
\pnumAdd
\wgAdd{%
\exposconcept{simd-consteval-broadcast-arg} subsumes
\exposconcept{explicitly-convertible-to}.
}

\pnumAdd
\wgAdd{%
\tcode{From} satisfies
\tcode{\exposconcept{simd-consteval-broadcast-arg}<To>} only if}
\begin{itemize}
  \wgItemAdd\wgAdd{\tcode{remove_cvref_t<From>} is an arithmetic type,}
  \wgItemAdd\wgAdd{\tcode{From} satisfies \tcode{\libconcept{convertible_to}<To>},}
  \wgItemAdd\wgAdd{the conversion from \tcode{remove_cvref_t<From>} to \tcode{To} is not value-preserving, and}
  \wgItemAdd\wgAdd{either}
  \begin{itemize}
    \wgItemAdd\wgAdd{\tcode{common_type_t<From, To>} is \tcode{To},}
    \wgItemAdd\wgAdd{\tcode{To} is integral and \tcode{remove_cvref_t<From>} is \tcode{int}, or}
    \wgItemAdd\wgAdd{\tcode{To} satisfies \tcode{unsigned_integral} and \tcode{remove_cvref_t<From>} is \tcode{unsigned int}.}
  \end{itemize}
\end{itemize}
\end{itemdescr}
\end{wgText}

\subsection{Modify [simd.syn]}

In [simd.syn], change:
\begin{wgText}[{[simd.syn]}]
\begin{codeblock}
  template<size_t Bytes, class Abi, class T, class U>
    constexpr auto select(const basic_mask<Bytes, Abi>& c, const T& a, const U& b)
    noexcept -> decltype(@\exposid{simd-select-impl}@(c, a, b));

  // \iref{simd.math}, mathematical functions
  template<@\exposconcept{math-floating-point}@ V> constexpr @\exposid{deduced-vec-t}@<V> acos(const V& x);
  template<@\exposconcept{math-floating-point}@ V> constexpr @\exposid{deduced-vec-t}@<V> asin(const V& x);
  template<@\exposconcept{math-floating-point}@ V> constexpr @\exposid{deduced-vec-t}@<V> atan(const V& x);
  template<@\wgChange{class V0, class V1}{\exposid{math-floating-point} V}@>
    constexpr @\wgChange{\exposid{math-common-simd-t}<V0, V1>}{\exposid{deduced-vec-t}<V>}@ atan2(const V@\wgRem{0}@& y, const V@\wgRem{1}@& x);
  template<@\exposconcept{math-floating-point}@ V> constexpr @\exposid{deduced-vec-t}@<V> cos(const V& x);
  template<@\exposconcept{math-floating-point}@ V> constexpr @\exposid{deduced-vec-t}@<V> sin(const V& x);
  template<@\exposconcept{math-floating-point}@ V> constexpr @\exposid{deduced-vec-t}@<V> tan(const V& x);
  template<@\exposconcept{math-floating-point}@ V> constexpr @\exposid{deduced-vec-t}@<V> acosh(const V& x);
  template<@\exposconcept{math-floating-point}@ V> constexpr @\exposid{deduced-vec-t}@<V> asinh(const V& x);
  template<@\exposconcept{math-floating-point}@ V> constexpr @\exposid{deduced-vec-t}@<V> atanh(const V& x);
  template<@\exposconcept{math-floating-point}@ V> constexpr @\exposid{deduced-vec-t}@<V> cosh(const V& x);
  template<@\exposconcept{math-floating-point}@ V> constexpr @\exposid{deduced-vec-t}@<V> sinh(const V& x);
  template<@\exposconcept{math-floating-point}@ V> constexpr @\exposid{deduced-vec-t}@<V> tanh(const V& x);
  template<@\exposconcept{math-floating-point}@ V> constexpr @\exposid{deduced-vec-t}@<V> exp(const V& x);
  template<@\exposconcept{math-floating-point}@ V> constexpr @\exposid{deduced-vec-t}@<V> exp2(const V& x);
  template<@\exposconcept{math-floating-point}@ V> constexpr @\exposid{deduced-vec-t}@<V> expm1(const V& x);
  template<@\exposconcept{math-floating-point}@ V>
    constexpr @\exposid{deduced-vec-t}@<V>
      frexp(const V& value, rebind_t<int, @\exposid{deduced-vec-t}@<V>>* exp);
  template<@\exposconcept{math-floating-point}@ V>
    constexpr rebind_t<int, @\exposid{deduced-vec-t}@<V>> ilogb(const V& x);
  template<@\exposconcept{math-floating-point}@ V>
    constexpr @\exposid{deduced-vec-t}@<V> ldexp(const V& x, const rebind_t<int, @\exposid{deduced-vec-t}@<V>>& exp);
  template<@\exposconcept{math-floating-point}@ V> constexpr @\exposid{deduced-vec-t}@<V> log(const V& x);
  template<@\exposconcept{math-floating-point}@ V> constexpr @\exposid{deduced-vec-t}@<V> log10(const V& x);
  template<@\exposconcept{math-floating-point}@ V> constexpr @\exposid{deduced-vec-t}@<V> log1p(const V& x);
  template<@\exposconcept{math-floating-point}@ V> constexpr @\exposid{deduced-vec-t}@<V> log2(const V& x);
  template<@\exposconcept{math-floating-point}@ V> constexpr @\exposid{deduced-vec-t}@<V> logb(const V& x);
  template<class T, class Abi>
    constexpr basic_vec<T, Abi>
      modf(const type_identity_t<basic_vec<T, Abi>>& value, basic_vec<T, Abi>* iptr);
  template<@\exposconcept{math-floating-point}@ V>
    constexpr @\exposid{deduced-vec-t}@<V> scalbn(const V& x, const rebind_t<int, @\exposid{deduced-vec-t}@<V>>& n);
  template<@\exposconcept{math-floating-point}@ V>
    constexpr @\exposid{deduced-vec-t}@<V> scalbln(
      const V& x, const rebind_t<long int, @\exposid{deduced-vec-t}@<V>>& n);
  template<@\exposconcept{math-floating-point}@ V> constexpr @\exposid{deduced-vec-t}@<V> cbrt(const V& x);
  template<@\libconcept{signed_integral}@ T, class Abi>
    constexpr basic_vec<T, Abi> abs(const basic_vec<T, Abi>& j);
  template<@\exposconcept{math-floating-point}@ V> constexpr @\exposid{deduced-vec-t}@<V> abs(const V& j);
  template<@\exposconcept{math-floating-point}@ V> constexpr @\exposid{deduced-vec-t}@<V> fabs(const V& x);
  template<@\wgChange{class V0, class V1}{\exposid{math-floating-point} V}@>
    constexpr @\wgChange{\exposid{math-common-simd-t}<V0, V1>}{\exposid{deduced-vec-t}<V>}@ hypot(const V@\wgRem{0}@& x, const V@\wgRem{1}@& y);
  template<@\wgChange{class V0, class V1, class V2}{\exposid{math-floating-point} V}@>
    constexpr @\wgChange{\exposid{math-common-simd-t}<V0, V1, V2>}{\exposid{deduced-vec-t}<V>}@ hypot(const V@\wgRem{0}@& x, const V@\wgRem{1}@& y, const V@\wgRem{2}@& z);
  template<@\wgChange{class V0, class V1}{\exposid{math-floating-point} V}@>
    constexpr @\wgChange{\exposid{math-common-simd-t}<V0, V1>}{\exposid{deduced-vec-t}<V>}@ pow(const V@\wgRem{0}@& x, const V@\wgRem{1}@& y);
  template<@\exposconcept{math-floating-point}@ V> constexpr @\exposid{deduced-vec-t}@<V> sqrt(const V& x);
  template<@\exposconcept{math-floating-point}@ V> constexpr @\exposid{deduced-vec-t}@<V> erf(const V& x);
  template<@\exposconcept{math-floating-point}@ V> constexpr @\exposid{deduced-vec-t}@<V> erfc(const V& x);
  template<@\exposconcept{math-floating-point}@ V> constexpr @\exposid{deduced-vec-t}@<V> lgamma(const V& x);
  template<@\exposconcept{math-floating-point}@ V> constexpr @\exposid{deduced-vec-t}@<V> tgamma(const V& x);
  template<@\exposconcept{math-floating-point}@ V> constexpr @\exposid{deduced-vec-t}@<V> ceil(const V& x);
  template<@\exposconcept{math-floating-point}@ V> constexpr @\exposid{deduced-vec-t}@<V> floor(const V& x);
  template<@\exposconcept{math-floating-point}@ V> @\exposid{deduced-vec-t}@<V> nearbyint(const V& x);
  template<@\exposconcept{math-floating-point}@ V> @\exposid{deduced-vec-t}@<V> rint(const V& x);
  template<@\exposconcept{math-floating-point}@ V>
    rebind_t<long int, @\exposid{deduced-vec-t}@<V>> lrint(const V& x);
  template<@\exposconcept{math-floating-point}@ V>
    rebind_t<long long int, V> llrint(const @\exposid{deduced-vec-t}@<V>& x);
  template<@\exposconcept{math-floating-point}@ V>
    constexpr @\exposid{deduced-vec-t}@<V> round(const V& x);
  template<@\exposconcept{math-floating-point}@ V>
    constexpr rebind_t<long int, @\exposid{deduced-vec-t}@<V>> lround(const V& x);
  template<@\exposconcept{math-floating-point}@ V>
    constexpr rebind_t<long long int, @\exposid{deduced-vec-t}@<V>> llround(const V& x);
  template<@\exposconcept{math-floating-point}@ V>
    constexpr @\exposid{deduced-vec-t}@<V> trunc(const V& x);
  template<@\wgChange{class V0, class V1}{\exposid{math-floating-point} V}@>
    constexpr @\wgChange{\exposid{math-common-simd-t}<V0, V1>}{\exposid{deduced-vec-t}<V>}@ fmod(const V@\wgRem{0}@& x, const V@\wgRem{1}@& y);
  template<@\wgChange{class V0, class V1}{\exposid{math-floating-point} V}@>
    constexpr @\wgChange{\exposid{math-common-simd-t}<V0, V1>}{\exposid{deduced-vec-t}<V>}@ remainder(const V@\wgRem{0}@& x, const V@\wgRem{1}@& y);
  template<@\wgChange{class V0, class V1}{\exposid{math-floating-point} V}@>
    constexpr @\wgChange{\exposid{math-common-simd-t}<V0, V1>}{\exposid{deduced-vec-t}<V>}@
      remquo(const V@\wgRem{0}@& x, const V@\wgRem{1}@& y, rebind_t<int, @\wgChange{\exposid{math-common-simd-t}<V0, V1>}{\exposid{deduced-vec-t}<V>}@>* quo);
  template<@\wgChange{class V0, class V1}{\exposid{math-floating-point} V}@>
    constexpr @\wgChange{\exposid{math-common-simd-t}<V0, V1>}{\exposid{deduced-vec-t}<V>}@ copysign(const V@\wgRem{0}@& x, const V@\wgRem{1}@& y);
  template<@\wgChange{class V0, class V1}{\exposid{math-floating-point} V}@>
    constexpr @\wgChange{\exposid{math-common-simd-t}<V0, V1>}{\exposid{deduced-vec-t}<V>}@ nextafter(const V@\wgRem{0}@& x, const V@\wgRem{1}@& y);
  template<@\wgChange{class V0, class V1}{\exposid{math-floating-point} V}@>
    constexpr @\wgChange{\exposid{math-common-simd-t}<V0, V1>}{\exposid{deduced-vec-t}<V>}@ fdim(const V@\wgRem{0}@& x, const V@\wgRem{1}@& y);
  template<@\wgChange{class V0, class V1}{\exposid{math-floating-point} V}@>
    constexpr @\wgChange{\exposid{math-common-simd-t}<V0, V1>}{\exposid{deduced-vec-t}<V>}@ fmax(const V@\wgRem{0}@& x, const V@\wgRem{1}@& y);
  template<@\wgChange{class V0, class V1}{\exposid{math-floating-point} V}@>
    constexpr @\wgChange{\exposid{math-common-simd-t}<V0, V1>}{\exposid{deduced-vec-t}<V>}@ fmin(const V@\wgRem{0}@& x, const V@\wgRem{1}@& y);
  template<@\wgChange{class V0, class V1, class V2}{\exposid{math-floating-point} V}@>
    constexpr @\wgChange{\exposid{math-common-simd-t}<V0, V1, V2>}{\exposid{deduced-vec-t}<V>}@ fma(const V@\wgRem{0}@& x, const V@\wgRem{1}@& y, const V@\wgRem{2}@& z);
  template<@\wgChange{class V0, class V1, class V2}{\exposid{math-floating-point} V}@>
    constexpr @\wgChange{\exposid{math-common-simd-t}<V0, V1, V2>}{\exposid{deduced-vec-t}<V>}@
      lerp(const V@\wgRem{0}@& a, const V@\wgRem{1}@& b, const V@\wgRem{2}@& t) noexcept;
  template<@\exposconcept{math-floating-point}@ V>
    constexpr rebind_t<int, @\exposid{deduced-vec-t}@<V>> fpclassify(const V& x);
  template<@\exposconcept{math-floating-point}@ V>
    constexpr typename @\exposid{deduced-vec-t}@<V>::mask_type isfinite(const V& x);
  template<@\exposconcept{math-floating-point}@ V>
    constexpr typename @\exposid{deduced-vec-t}@<V>::mask_type isinf(const V& x);
  template<@\exposconcept{math-floating-point}@ V>
    constexpr typename @\exposid{deduced-vec-t}@<V>::mask_type isnan(const V& x);
  template<@\exposconcept{math-floating-point}@ V>
    constexpr typename @\exposid{deduced-vec-t}@<V>::mask_type isnormal(const V& x);
  template<@\exposconcept{math-floating-point}@ V>
    constexpr typename @\exposid{deduced-vec-t}@<V>::mask_type signbit(const V& x);
  template<@\wgChange{class V0, class V1}{\exposid{math-floating-point} V}@>
    constexpr typename @\wgChange{\exposid{math-common-simd-t}<V0, V1>}{\exposid{deduced-vec-t}<V>}@::mask_type
      isgreater(const V@\wgRem{0}@& x, const V@\wgRem{1}@& y);
  template<@\wgChange{class V0, class V1}{\exposid{math-floating-point} V}@>
    constexpr typename @\wgChange{\exposid{math-common-simd-t}<V0, V1>}{\exposid{deduced-vec-t}<V>}@::mask_type
      isgreaterequal(const V@\wgRem{0}@& x, const V@\wgRem{1}@& y);
  template<@\wgChange{class V0, class V1}{\exposid{math-floating-point} V}@>
    constexpr typename @\wgChange{\exposid{math-common-simd-t}<V0, V1>}{\exposid{deduced-vec-t}<V>}@::mask_type
      isless(const V@\wgRem{0}@& x, const V@\wgRem{1}@& y);
  template<@\wgChange{class V0, class V1}{\exposid{math-floating-point} V}@>
    constexpr typename @\wgChange{\exposid{math-common-simd-t}<V0, V1>}{\exposid{deduced-vec-t}<V>}@::mask_type
      islessequal(const V@\wgRem{0}@& x, const V@\wgRem{1}@& y);
  template<@\wgChange{class V0, class V1}{\exposid{math-floating-point} V}@>
    constexpr typename @\wgChange{\exposid{math-common-simd-t}<V0, V1>}{\exposid{deduced-vec-t}<V>}@::mask_type
      islessgreater(const V@\wgRem{0}@& x, const V@\wgRem{1}@& y);
  template<@\wgChange{class V0, class V1}{\exposid{math-floating-point} V}@>
    constexpr typename @\wgChange{\exposid{math-common-simd-t}<V0, V1>}{\exposid{deduced-vec-t}<V>}@::mask_type
      isunordered(const V@\wgRem{0}@& x, const V@\wgRem{1}@& y);
  template<@\exposconcept{math-floating-point}@ V>
    @\exposid{deduced-vec-t}@<V> assoc_laguerre(const rebind_t<unsigned, @\exposid{deduced-vec-t}@<V>>& n,
                                    const rebind_t<unsigned, @\exposid{deduced-vec-t}@<V>>& m, const V& x);
  template<@\exposconcept{math-floating-point}@ V>
    @\exposid{deduced-vec-t}@<V> assoc_legendre(const rebind_t<unsigned, @\exposid{deduced-vec-t}@<V>>& l,
                                    const rebind_t<unsigned, @\exposid{deduced-vec-t}@<V>>& m, const V& x);
  template<@\wgChange{class V0, class V1}{\exposid{math-floating-point} V}@>
    @\wgChange{\exposid{math-common-simd-t}<V0, V1>}{\exposid{deduced-vec-t}<V>}@ beta(const V@\wgRem{0}@& x, const V@\wgRem{1}@& y);
  template<@\exposconcept{math-floating-point}@ V> @\exposid{deduced-vec-t}@<V> comp_ellint_1(const V& k);
  template<@\exposconcept{math-floating-point}@ V> @\exposid{deduced-vec-t}@<V> comp_ellint_2(const V& k);
  template<@\wgChange{class V0, class V1}{\exposid{math-floating-point} V}@>
    @\wgChange{\exposid{math-common-simd-t}<V0, V1>}{\exposid{deduced-vec-t}<V>}@ comp_ellint_3(const V@\wgRem{0}@& k, const V@\wgRem{1}@& nu);
  template<@\wgChange{class V0, class V1}{\exposid{math-floating-point} V}@>
    @\wgChange{\exposid{math-common-simd-t}<V0, V1>}{\exposid{deduced-vec-t}<V>}@ cyl_bessel_i(const V@\wgRem{0}@& nu, const V@\wgRem{1}@& x);
  template<@\wgChange{class V0, class V1}{\exposid{math-floating-point} V}@>
    @\wgChange{\exposid{math-common-simd-t}<V0, V1>}{\exposid{deduced-vec-t}<V>}@ cyl_bessel_j(const V@\wgRem{0}@& nu, const V@\wgRem{1}@& x);
  template<@\wgChange{class V0, class V1}{\exposid{math-floating-point} V}@>
    @\wgChange{\exposid{math-common-simd-t}<V0, V1>}{\exposid{deduced-vec-t}<V>}@ cyl_bessel_k(const V@\wgRem{0}@& nu, const V@\wgRem{1}@& x);
  template<@\wgChange{class V0, class V1}{\exposid{math-floating-point} V}@>
    @\wgChange{\exposid{math-common-simd-t}<V0, V1>}{\exposid{deduced-vec-t}<V>}@ cyl_neumann(const V@\wgRem{0}@& nu, const V@\wgRem{1}@& x);
  template<@\wgChange{class V0, class V1}{\exposid{math-floating-point} V}@>
    @\wgChange{\exposid{math-common-simd-t}<V0, V1>}{\exposid{deduced-vec-t}<V>}@ ellint_1(const V@\wgRem{0}@& k, const V@\wgRem{1}@& phi);
  template<@\wgChange{class V0, class V1}{\exposid{math-floating-point} V}@>
    @\wgChange{\exposid{math-common-simd-t}<V0, V1>}{\exposid{deduced-vec-t}<V>}@ ellint_2(const V@\wgRem{0}@& k, const V@\wgRem{1}@& phi);
  template<@\wgChange{class V0, class V1, class V2}{\exposid{math-floating-point} V}@>
    @\wgChange{\exposid{math-common-simd-t}<V0, V1, V2>}{\exposid{deduced-vec-t}<V>}@ ellint_3(const V@\wgRem{0}@& k, const V@\wgRem{1}@& nu, const V@\wgRem{2}@& phi);
  template<@\exposconcept{math-floating-point}@ V> @\exposid{deduced-vec-t}@<V> expint(const V& x);
  template<@\exposconcept{math-floating-point}@ V>
    @\exposid{deduced-vec-t}@<V> hermite(const rebind_t<unsigned, @\exposid{deduced-vec-t}@<V>>& n, const V& x);
  template<@\exposconcept{math-floating-point}@ V>
    @\exposid{deduced-vec-t}@<V> laguerre(const rebind_t<unsigned, @\exposid{deduced-vec-t}@<V>>& n, const V& x);
  template<@\exposconcept{math-floating-point}@ V>
    @\exposid{deduced-vec-t}@<V> legendre(const rebind_t<unsigned, @\exposid{deduced-vec-t}@<V>>& l, const V& x);
  template<@\exposconcept{math-floating-point}@ V>
    @\exposid{deduced-vec-t}@<V> riemann_zeta(const V& x);
  template<@\exposconcept{math-floating-point}@ V>
    @\exposid{deduced-vec-t}@<V> sph_bessel(
      const rebind_t<unsigned, @\exposid{deduced-vec-t}@<V>>& n, const V& x);
  template<@\exposconcept{math-floating-point}@ V>
    @\exposid{deduced-vec-t}@<V> sph_legendre(const rebind_t<unsigned, @\exposid{deduced-vec-t}@<V>>& l,
      const rebind_t<unsigned, @\exposid{deduced-vec-t}@<V>>& m, const V& theta);
  template<@\exposconcept{math-floating-point}@ V>
    @\exposid{deduced-vec-t}@<V>
      sph_neumann(const rebind_t<unsigned, @\exposid{deduced-vec-t}@<V>>& n, const V& x);

  @\wgAdd{template<\exposconcept{math-floating-point} V>}@
    @\wgAdd{constexpr \exposid{deduced-vec-t}<V> fmod(const \exposid{deduced-vec-t}<V>\& x, const V\& y);}@
  @\wgAdd{template<\exposconcept{math-floating-point} V>}@
    @\wgAdd{constexpr \exposid{deduced-vec-t}<V> fmod(const V\& x, const \exposid{deduced-vec-t}<V>\& y);}@
  @\wgAdd{template<\exposconcept{math-floating-point} V>}@
    @\wgAdd{constexpr \exposid{deduced-vec-t}<V> remainder(const \exposid{deduced-vec-t}<V>\& x, const V\& y);}@
  @\wgAdd{template<\exposconcept{math-floating-point} V>}@
    @\wgAdd{constexpr \exposid{deduced-vec-t}<V> remainder(const V\& x, const \exposid{deduced-vec-t}<V>\& y);}@
  @\wgAdd{template<\exposconcept{math-floating-point} V>}@
    @\wgAdd{constexpr \exposid{deduced-vec-t}<V> copysign(const \exposid{deduced-vec-t}<V>\& x, const V\& y);}@
  @\wgAdd{template<\exposconcept{math-floating-point} V>}@
    @\wgAdd{constexpr \exposid{deduced-vec-t}<V> copysign(const V\& x, const \exposid{deduced-vec-t}<V>\& y);}@
  @\wgAdd{template<\exposconcept{math-floating-point} V>}@
    @\wgAdd{constexpr \exposid{deduced-vec-t}<V> nextafter(const \exposid{deduced-vec-t}<V>\& x, const V\& y);}@
  @\wgAdd{template<\exposconcept{math-floating-point} V>}@
    @\wgAdd{constexpr \exposid{deduced-vec-t}<V> nextafter(const V\& x, const \exposid{deduced-vec-t}<V>\& y);}@
  @\wgAdd{template<\exposconcept{math-floating-point} V>}@
    @\wgAdd{constexpr \exposid{deduced-vec-t}<V> fdim(const \exposid{deduced-vec-t}<V>\& x, const V\& y);}@
  @\wgAdd{template<\exposconcept{math-floating-point} V>}@
    @\wgAdd{constexpr \exposid{deduced-vec-t}<V> fdim(const V\& x, const \exposid{deduced-vec-t}<V>\& y);}@
  @\wgAdd{template<\exposconcept{math-floating-point} V>}@
    @\wgAdd{constexpr \exposid{deduced-vec-t}<V> fmax(const \exposid{deduced-vec-t}<V>\& x, const V\& y);}@
  @\wgAdd{template<\exposconcept{math-floating-point} V>}@
    @\wgAdd{constexpr \exposid{deduced-vec-t}<V> fmax(const V\& x, const \exposid{deduced-vec-t}<V>\& y);}@
  @\wgAdd{template<\exposconcept{math-floating-point} V>}@
    @\wgAdd{constexpr \exposid{deduced-vec-t}<V> fmin(const \exposid{deduced-vec-t}<V>\& x, const V\& y);}@
  @\wgAdd{template<\exposconcept{math-floating-point} V>}@
    @\wgAdd{constexpr \exposid{deduced-vec-t}<V> fmin(const V\& x, const \exposid{deduced-vec-t}<V>\& y);}@
  @\wgAdd{template<\exposconcept{math-floating-point} V>}@
    @\wgAdd{constexpr \exposid{deduced-vec-t}<V> isgreater(const \exposid{deduced-vec-t}<V>\& x, const V\& y);}@
  @\wgAdd{template<\exposconcept{math-floating-point} V>}@
    @\wgAdd{constexpr \exposid{deduced-vec-t}<V> isgreater(const V\& x, const \exposid{deduced-vec-t}<V>\& y);}@
  @\wgAdd{template<\exposconcept{math-floating-point} V>}@
    @\wgAdd{constexpr \exposid{deduced-vec-t}<V> islessequal(const \exposid{deduced-vec-t}<V>\& x, const V\& y);}@
  @\wgAdd{template<\exposconcept{math-floating-point} V>}@
    @\wgAdd{constexpr \exposid{deduced-vec-t}<V> islessequal(const V\& x, const \exposid{deduced-vec-t}<V>\& y);}@
  @\wgAdd{template<\exposconcept{math-floating-point} V>}@
    @\wgAdd{constexpr \exposid{deduced-vec-t}<V> islessgreater(const \exposid{deduced-vec-t}<V>\& x, const V\& y);}@
  @\wgAdd{template<\exposconcept{math-floating-point} V>}@
    @\wgAdd{constexpr \exposid{deduced-vec-t}<V> islessgreater(const V\& x, const \exposid{deduced-vec-t}<V>\& y);}@
  @\wgAdd{template<\exposconcept{math-floating-point} V>}@
    @\wgAdd{constexpr \exposid{deduced-vec-t}<V> isunordered(const \exposid{deduced-vec-t}<V>\& x, const V\& y);}@
  @\wgAdd{template<\exposconcept{math-floating-point} V>}@
    @\wgAdd{constexpr \exposid{deduced-vec-t}<V> isunordered(const V\& x, const \exposid{deduced-vec-t}<V>\& y);}@
  @\wgAdd{template<\exposconcept{math-floating-point} V>}@
    @\wgAdd{constexpr \exposid{deduced-vec-t}<V> atan2(const \exposid{deduced-vec-t}<V>\& x, const V\& y);}@
  @\wgAdd{template<\exposconcept{math-floating-point} V>}@
    @\wgAdd{constexpr \exposid{deduced-vec-t}<V> atan2(const V\& x, const \exposid{deduced-vec-t}<V>\& y);}@
  @\wgAdd{template<\exposconcept{math-floating-point} V>}@
    @\wgAdd{constexpr \exposid{deduced-vec-t}<V> hypot(const \exposid{deduced-vec-t}<V>\& x, const V\& y);}@
  @\wgAdd{template<\exposconcept{math-floating-point} V>}@
    @\wgAdd{constexpr \exposid{deduced-vec-t}<V> hypot(const V\& x, const \exposid{deduced-vec-t}<V>\& y);}@
  @\wgAdd{template<\exposconcept{math-floating-point} V>}@
    @\wgAdd{constexpr \exposid{deduced-vec-t}<V> pow(const \exposid{deduced-vec-t}<V>\& x, const V\& y);}@
  @\wgAdd{template<\exposconcept{math-floating-point} V>}@
    @\wgAdd{constexpr \exposid{deduced-vec-t}<V> pow(const V\& x, const \exposid{deduced-vec-t}<V>\& y);}@
  @\wgAdd{template<\exposconcept{math-floating-point} V>}@
    @\wgAdd{constexpr \exposid{deduced-vec-t}<V> beta(const \exposid{deduced-vec-t}<V>\& x, const V\& y);}@
  @\wgAdd{template<\exposconcept{math-floating-point} V>}@
    @\wgAdd{constexpr \exposid{deduced-vec-t}<V> beta(const V\& x, const \exposid{deduced-vec-t}<V>\& y);}@
  @\wgAdd{template<\exposconcept{math-floating-point} V>}@
    @\wgAdd{constexpr \exposid{deduced-vec-t}<V> comp_ellint_3(const \exposid{deduced-vec-t}<V>\& x, const V\& y);}@
  @\wgAdd{template<\exposconcept{math-floating-point} V>}@
    @\wgAdd{constexpr \exposid{deduced-vec-t}<V> comp_ellint_3(const V\& x, const \exposid{deduced-vec-t}<V>\& y);}@
  @\wgAdd{template<\exposconcept{math-floating-point} V>}@
    @\wgAdd{constexpr \exposid{deduced-vec-t}<V> cyl_bessel_i(const \exposid{deduced-vec-t}<V>\& x, const V\& y);}@
  @\wgAdd{template<\exposconcept{math-floating-point} V>}@
    @\wgAdd{constexpr \exposid{deduced-vec-t}<V> cyl_bessel_i(const V\& x, const \exposid{deduced-vec-t}<V>\& y);}@
  @\wgAdd{template<\exposconcept{math-floating-point} V>}@
    @\wgAdd{constexpr \exposid{deduced-vec-t}<V> cyl_bessel_j(const \exposid{deduced-vec-t}<V>\& x, const V\& y);}@
  @\wgAdd{template<\exposconcept{math-floating-point} V>}@
    @\wgAdd{constexpr \exposid{deduced-vec-t}<V> cyl_bessel_j(const V\& x, const \exposid{deduced-vec-t}<V>\& y);}@
  @\wgAdd{template<\exposconcept{math-floating-point} V>}@
    @\wgAdd{constexpr \exposid{deduced-vec-t}<V> cyl_bessel_k(const \exposid{deduced-vec-t}<V>\& x, const V\& y);}@
  @\wgAdd{template<\exposconcept{math-floating-point} V>}@
    @\wgAdd{constexpr \exposid{deduced-vec-t}<V> cyl_bessel_k(const V\& x, const \exposid{deduced-vec-t}<V>\& y);}@
  @\wgAdd{template<\exposconcept{math-floating-point} V>}@
    @\wgAdd{constexpr \exposid{deduced-vec-t}<V> cyl_neumann(const \exposid{deduced-vec-t}<V>\& x, const V\& y);}@
  @\wgAdd{template<\exposconcept{math-floating-point} V>}@
    @\wgAdd{constexpr \exposid{deduced-vec-t}<V> cyl_neumann(const V\& x, const \exposid{deduced-vec-t}<V>\& y);}@
  @\wgAdd{template<\exposconcept{math-floating-point} V>}@
    @\wgAdd{constexpr \exposid{deduced-vec-t}<V> ellint_1(const \exposid{deduced-vec-t}<V>\& x, const V\& y);}@
  @\wgAdd{template<\exposconcept{math-floating-point} V>}@
    @\wgAdd{constexpr \exposid{deduced-vec-t}<V> ellint_1(const V\& x, const \exposid{deduced-vec-t}<V>\& y);}@
  @\wgAdd{template<\exposconcept{math-floating-point} V>}@
    @\wgAdd{constexpr \exposid{deduced-vec-t}<V> ellint_2(const \exposid{deduced-vec-t}<V>\& x, const V\& y);}@
  @\wgAdd{template<\exposconcept{math-floating-point} V>}@
    @\wgAdd{constexpr \exposid{deduced-vec-t}<V> ellint_2(const V\& x, const \exposid{deduced-vec-t}<V>\& y);}@
  @\wgAdd{template<\exposconcept{math-floating-point} V>}@
    @\wgAdd{constexpr \exposid{deduced-vec-t}<V>}@
      @\wgAdd{remquo(const \exposid{deduced-vec-t}<V>\& x, const V\& y, rebind_t<int, \exposid{deduced-vec-t}<V>>* quo);}@
  @\wgAdd{template<\exposconcept{math-floating-point} V>}@
    @\wgAdd{constexpr \exposid{deduced-vec-t}<V>}@
      @\wgAdd{remquo(const V\& x, const \exposid{deduced-vec-t}<V>\& y, rebind_t<int, \exposid{deduced-vec-t}<V>>* quo);}@
  @\wgAdd{template<\exposconcept{math-floating-point} V>}@
    @\wgAdd{constexpr \exposid{deduced-vec-t}<V> fma(const \exposid{deduced-vec-t}<V>\& x, const V\& y, const V\& z);}@
  @\wgAdd{template<\exposconcept{math-floating-point} V>}@
    @\wgAdd{constexpr \exposid{deduced-vec-t}<V> fma(const V\& x, const \exposid{deduced-vec-t}<V>\& y, const V\& z);}@
  @\wgAdd{template<\exposconcept{math-floating-point} V>}@
    @\wgAdd{constexpr \exposid{deduced-vec-t}<V> fma(const V\& x, const V\& y, const \exposid{deduced-vec-t}<V>\& z);}@
  @\wgAdd{template<\exposconcept{math-floating-point} V>}@
    @\wgAdd{constexpr \exposid{deduced-vec-t}<V> fma(const \exposid{deduced-vec-t}<V>\& x, const \exposid{deduced-vec-t}<V>\& y}@
                                   @\wgAdd{const V\& z);}@
  @\wgAdd{template<\exposconcept{math-floating-point} V>}@
    @\wgAdd{constexpr \exposid{deduced-vec-t}<V> fma(const \exposid{deduced-vec-t}<V>\& x, const V\& y}@
                                   @\wgAdd{const \exposid{deduced-vec-t}<V>\& z);}@
  @\wgAdd{template<\exposconcept{math-floating-point} V>}@
    @\wgAdd{constexpr \exposid{deduced-vec-t}<V> fma(const V\& x, const \exposid{deduced-vec-t}<V>\& y}@
                                   @\wgAdd{const \exposid{deduced-vec-t}<V>\& z);}@
  @\wgAdd{template<\exposconcept{math-floating-point} V>}@
    @\wgAdd{constexpr \exposid{deduced-vec-t}<V> hypot(const \exposid{deduced-vec-t}<V>\& x, const V\& y, const V\& z);}@
  @\wgAdd{template<\exposconcept{math-floating-point} V>}@
    @\wgAdd{constexpr \exposid{deduced-vec-t}<V> hypot(const V\& x, const \exposid{deduced-vec-t}<V>\& y, const V\& z);}@
  @\wgAdd{template<\exposconcept{math-floating-point} V>}@
    @\wgAdd{constexpr \exposid{deduced-vec-t}<V> hypot(const V\& x, const V\& y, const \exposid{deduced-vec-t}<V>\& z);}@
  @\wgAdd{template<\exposconcept{math-floating-point} V>}@
    @\wgAdd{constexpr \exposid{deduced-vec-t}<V> hypot(const \exposid{deduced-vec-t}<V>\& x, const \exposid{deduced-vec-t}<V>\& y}@
                                     @\wgAdd{const V\& z);}@
  @\wgAdd{template<\exposconcept{math-floating-point} V>}@
    @\wgAdd{constexpr \exposid{deduced-vec-t}<V> hypot(const \exposid{deduced-vec-t}<V>\& x, const V\& y}@
                                     @\wgAdd{const \exposid{deduced-vec-t}<V>\& z);}@
  @\wgAdd{template<\exposconcept{math-floating-point} V>}@
    @\wgAdd{constexpr \exposid{deduced-vec-t}<V> hypot(const V\& x, const \exposid{deduced-vec-t}<V>\& y}@
                                     @\wgAdd{const \exposid{deduced-vec-t}<V>\& z);}@
  @\wgAdd{template<\exposconcept{math-floating-point} V>}@
    @\wgAdd{constexpr \exposid{deduced-vec-t}<V> lerp(const \exposid{deduced-vec-t}<V>\& x, const V\& y, const V\& z);}@
  @\wgAdd{template<\exposconcept{math-floating-point} V>}@
    @\wgAdd{constexpr \exposid{deduced-vec-t}<V> lerp(const V\& x, const \exposid{deduced-vec-t}<V>\& y, const V\& z);}@
  @\wgAdd{template<\exposconcept{math-floating-point} V>}@
    @\wgAdd{constexpr \exposid{deduced-vec-t}<V> lerp(const V\& x, const V\& y, const \exposid{deduced-vec-t}<V>\& z);}@
  @\wgAdd{template<\exposconcept{math-floating-point} V>}@
    @\wgAdd{constexpr \exposid{deduced-vec-t}<V> lerp(const \exposid{deduced-vec-t}<V>\& x, const \exposid{deduced-vec-t}<V>\& y}@
                                    @\wgAdd{const V\& z);}@
  @\wgAdd{template<\exposconcept{math-floating-point} V>}@
    @\wgAdd{constexpr \exposid{deduced-vec-t}<V> lerp(const \exposid{deduced-vec-t}<V>\& x, const V\& y}@
                                    @\wgAdd{const \exposid{deduced-vec-t}<V>\& z);}@
  @\wgAdd{template<\exposconcept{math-floating-point} V>}@
    @\wgAdd{constexpr \exposid{deduced-vec-t}<V> lerp(const V\& x, const \exposid{deduced-vec-t}<V>\& y}@
                                    @\wgAdd{const \exposid{deduced-vec-t}<V>\& z);}@
  @\wgAdd{template<\exposconcept{math-floating-point} V>}@
    @\wgAdd{constexpr \exposid{deduced-vec-t}<V> ellint_3(const \exposid{deduced-vec-t}<V>\& x, const V\& y, const V\& z);}@
  @\wgAdd{template<\exposconcept{math-floating-point} V>}@
    @\wgAdd{constexpr \exposid{deduced-vec-t}<V> ellint_3(const V\& x, const \exposid{deduced-vec-t}<V>\& y, const V\& z);}@
  @\wgAdd{template<\exposconcept{math-floating-point} V>}@
    @\wgAdd{constexpr \exposid{deduced-vec-t}<V> ellint_3(const V\& x, const V\& y, const \exposid{deduced-vec-t}<V>\& z);}@
  @\wgAdd{template<\exposconcept{math-floating-point} V>}@
    @\wgAdd{constexpr \exposid{deduced-vec-t}<V> ellint_3(const \exposid{deduced-vec-t}<V>\& x, const \exposid{deduced-vec-t}<V>\& y,}@
                                        @\wgAdd{const V\& z);}@
  @\wgAdd{template<\exposconcept{math-floating-point} V>}@
    @\wgAdd{constexpr \exposid{deduced-vec-t}<V> ellint_3(const \exposid{deduced-vec-t}<V>\& x, const V\& y,}@
                                        @\wgAdd{const \exposid{deduced-vec-t}<V>\& z);}@
  @\wgAdd{template<\exposconcept{math-floating-point} V>}@
    @\wgAdd{constexpr \exposid{deduced-vec-t}<V> ellint_3(const V\& x, const \exposid{deduced-vec-t}<V>\& y,}@
                                        @\wgAdd{const \exposid{deduced-vec-t}<V>\& z);}@

  // \iref{simd.bit}, bit manipulation
  template<@\exposconcept{simd-vec-type}@ V> constexpr V byteswap(const V& v) noexcept;
\end{codeblock}
\end{wgText}

\subsection{Modify [simd.overview]}

In [simd.overview], insert:
\begin{wgText}[{[simd.overview]}]
\begin{codeblock}
    // \iref{simd.ctor}, \tcode{basic_vec} constructors
    template<@\wgChange{class}{\exposconcept{explicitly-convertible-to}<value_type>}@ U>
      constexpr explicit(@\seebelow@) basic_vec(U&& value) noexcept;
    @\wgAdd{template<\exposconcept{simd-consteval-broadcast-arg}<value_type> U>}@
      @\wgAdd{consteval basic_vec(U\&\& x)}@
    template<class U, class UAbi>
      constexpr explicit(@\seebelow@) basic_vec(const basic_vec<U, UAbi>&) noexcept;
\end{codeblock}
\end{wgText}

\subsection{Modify [simd.ctor]}

\begin{wgText}[{[simd.ctor]}]
\begin{itemdecl}
template<@\wgChange{class}{\exposconcept{explicitly-convertible-to}<value_type>}@ U>
  constexpr explicit(@\seebelow@) basic_vec(U&& value) noexcept;
\end{itemdecl}

\begin{itemdescr}
\pnum
Let \tcode{From} denote the type \tcode{remove_cvref_t<U>}.

\pnumRem
\wgRem{\constraints
\tcode{value_type} satisfies \tcode{\libconcept{constructible_from}<U>}.}

\pnum
\effects
Initializes each element to the value of the argument after conversion to
\tcode{value_type}.

\pnum
\remarks
The expression inside \tcode{explicit} evaluates to \tcode{false} if and only if
\tcode{U} satisfies \tcode{\libconcept{convertible_to}<value_type>}, and either
\begin{itemize}
 \item
   \tcode{From} is not an arithmetic type and does not satisfy
   \exposconcept{constexpr-wrapper-like},
 \item
   \tcode{From} is an arithmetic type and the conversion from \tcode{From} to
   \tcode{value_type} is value-preserving\iref{simd.general}, or
 \item
   \tcode{From} satisfies \exposconcept{constexpr-wrapper-like},
   \tcode{remove_const_t<decltype(From::value)>} is an arithmetic type, and
   \tcode{From::value} is representable by \tcode{value_type}.
\end{itemize}
\end{itemdescr}

\begin{itemdecl}
@\wgAdd{template<\exposconcept{simd-consteval-broadcast-arg}<value_type> U> consteval basic_vec(U\&\& x)}@
\end{itemdecl}

\begin{itemdescr}
\pnumAdd
\wgAdd{%
\expects
The value of \tcode{x} is equal to the value of \tcode{x} after conversion to \tcode{value_type}.
}

\pnumAdd
\wgAdd{\effects}
\wgAdd{Initializes each element to the value of the argument after conversion to
\tcode{value_type}.}

\pnumAdd
\wgAdd{%
\remarks
An expression that violates the precondition in the \expects element is not a core constant
expression\iref{expr.const}.
}
\end{itemdescr}
\end{wgText}

\subsection{Modify [simd.math]}

\begin{wgText}[{[simd.math]}]
\begin{itemdecl}
template<@\exposconcept{math-floating-point}@ V>
  constexpr rebind_t<int, @\exposid{deduced-vec-t}@<V>> ilogb(const V& x);
@\wgRem{template<\exposconcept{math-floating-point} V>}@
  @\wgRem{constexpr \exposid{deduced-vec-t}<V> ldexp(const V\& x, const rebind_t<int, \exposid{deduced-vec-t}<V>}\wgRem{>\& exp);}@
@\wgRem{template<\exposconcept{math-floating-point} V>}@
  @\wgRem{constexpr \exposid{deduced-vec-t}<V> scalbn(const V\& x, const rebind_t<int, \exposid{deduced-vec-t}<V>}\wgRem{>\& n);}@
@\wgRem{template<\exposconcept{math-floating-point} V>}@
  @\wgRem{constexpr \exposid{deduced-vec-t}<V>}@
    @\wgRem{scalbln(const V\& x, const rebind_t<long int, \exposid{deduced-vec-t}<V>}\wgRem{>\& n);}@
@\wgRem{template<\libconcept{signed_integral} T, class Abi>}@
  @\wgRem{constexpr basic_vec<T, Abi> abs(const basic_vec<T, Abi>\& j);}@
template<@\exposconcept{math-floating-point}@ V>
  constexpr @\exposid{deduced-vec-t}@<V> abs(const V& j);
template<@\exposconcept{math-floating-point}@ V>
  constexpr @\exposid{deduced-vec-t}@<V> fabs(const V& x);
template<@\exposconcept{math-floating-point}@ V>
  constexpr @\exposid{deduced-vec-t}@<V> ceil(const V& x);
template<@\exposconcept{math-floating-point}@ V>
  constexpr @\exposid{deduced-vec-t}@<V> floor(const V& x);
template<@\exposconcept{math-floating-point}@ V>
  @\exposid{deduced-vec-t}@<V> nearbyint(const V& x);
template<@\exposconcept{math-floating-point}@ V>
  @\exposid{deduced-vec-t}@<V> rint(const V& x);
template<@\exposconcept{math-floating-point}@ V>
  rebind_t<long int, @\exposid{deduced-vec-t}@<V>> lrint(const V& x);
template<@\exposconcept{math-floating-point}@ V>
  rebind_t<long long int, @\exposid{deduced-vec-t}@<V>> llrint(const V& x);
template<@\exposconcept{math-floating-point}@ V>
  constexpr @\exposid{deduced-vec-t}@<V> round(const V& x);
template<@\exposconcept{math-floating-point}@ V>
  constexpr rebind_t<long int, @\exposid{deduced-vec-t}@<V>> lround(const V& x);
template<@\exposconcept{math-floating-point}@ V>
  constexpr rebind_t<long long int, @\exposid{deduced-vec-t}@<V>> llround(const V& x);
template<@\wgChange{class V0, class V1}{\exposid{math-floating-point} V}@>
  constexpr @\wgChange{\exposid{math-common-simd-t}<V0, V1>}{\exposid{deduced-vec-t}<V>}@ fmod(const V@\wgRem{0}@& x, const V@\wgRem{1}@& y);
template<@\exposconcept{math-floating-point}@ V>
  constexpr @\exposid{deduced-vec-t}@<V> trunc(const V& x);
template<@\wgChange{class V0, class V1}{\exposid{math-floating-point} V}@>
  constexpr @\wgChange{\exposid{math-common-simd-t}<V0, V1>}{\exposid{deduced-vec-t}<V>}@ remainder(const V@\wgRem{0}@& x, const V@\wgRem{1}@& y);
template<@\wgChange{class V0, class V1}{\exposid{math-floating-point} V}@>
  constexpr @\wgChange{\exposid{math-common-simd-t}<V0, V1>}{\exposid{deduced-vec-t}<V>}@ copysign(const V@\wgRem{0}@& x, const V@\wgRem{1}@& y);
template<@\wgChange{class V0, class V1}{\exposid{math-floating-point} V}@>
  constexpr @\wgChange{\exposid{math-common-simd-t}<V0, V1>}{\exposid{deduced-vec-t}<V>}@ nextafter(const V@\wgRem{0}@& x, const V@\wgRem{1}@& y);
template<@\wgChange{class V0, class V1}{\exposid{math-floating-point} V}@>
  constexpr @\wgChange{\exposid{math-common-simd-t}<V0, V1>}{\exposid{deduced-vec-t}<V>}@ fdim(const V@\wgRem{0}@& x, const V@\wgRem{1}@& y);
template<@\wgChange{class V0, class V1}{\exposid{math-floating-point} V}@>
  constexpr @\wgChange{\exposid{math-common-simd-t}<V0, V1>}{\exposid{deduced-vec-t}<V>}@ fmax(const V@\wgRem{0}@& x, const V@\wgRem{1}@& y);
template<@\wgChange{class V0, class V1}{\exposid{math-floating-point} V}@>
  constexpr @\wgChange{\exposid{math-common-simd-t}<V0, V1>}{\exposid{deduced-vec-t}<V>}@ fmin(const V@\wgRem{0}@& x, const V@\wgRem{1}@& y);
template<@\wgChange{class V0, class V1, class V2}{\exposconcept{math-floating-point} V}@>
  constexpr @\wgChange{\exposid{math-common-simd-t}<V0, V1, V2>}{\exposid{deduced-vec-t}<V>}@ fma(const V@\wgRem{0}@& x, const V@\wgRem{1}@& y, const V@\wgRem{2}@& z);
template<@\exposconcept{math-floating-point}@ V>
  constexpr rebind_t<int, @\exposid{deduced-vec-t}@<V>> fpclassify(const V& x);
template<@\exposconcept{math-floating-point}@ V>
  constexpr typename @\exposid{deduced-vec-t}@<V>::mask_type isfinite(const V& x);
template<@\exposconcept{math-floating-point}@ V>
  constexpr typename @\exposid{deduced-vec-t}@<V>::mask_type isinf(const V& x);
template<@\exposconcept{math-floating-point}@ V>
  constexpr typename @\exposid{deduced-vec-t}@<V>::mask_type isnan(const V& x);
template<@\exposconcept{math-floating-point}@ V>
  constexpr typename @\exposid{deduced-vec-t}@<V>::mask_type isnormal(const V& x);
template<@\exposconcept{math-floating-point}@ V>
  constexpr typename @\exposid{deduced-vec-t}@<V>::mask_type signbit(const V& x);
template<@\wgChange{class V0, class V1}{\exposid{math-floating-point} V}@>
  constexpr typename @\wgChange{\exposid{math-common-simd-t}<V0, V1>}{\exposid{deduced-vec-t}<V>}@::mask_type isgreater(const V@\wgRem{0}@& x, const V@\wgRem{1}@& y);
template<@\wgChange{class V0, class V1}{\exposid{math-floating-point} V}@>
  constexpr typename @\wgChange{\exposid{math-common-simd-t}<V0, V1>}{\exposid{deduced-vec-t}<V>}@::mask_type
    isgreaterequal(const V@\wgRem{0}@& x, const V@\wgRem{1}@& y);
template<@\wgChange{class V0, class V1}{\exposid{math-floating-point} V}@>
  constexpr typename @\wgChange{\exposid{math-common-simd-t}<V0, V1>}{\exposid{deduced-vec-t}<V>}@::mask_type isless(const V@\wgRem{0}@& x, const V@\wgRem{1}@& y);
template<@\wgChange{class V0, class V1}{\exposid{math-floating-point} V}@>
  constexpr typename @\wgChange{\exposid{math-common-simd-t}<V0, V1>}{\exposid{deduced-vec-t}<V>}@::mask_type islessequal(const V@\wgRem{0}@& x, const V@\wgRem{1}@& y);
template<@\wgChange{class V0, class V1}{\exposid{math-floating-point} V}@>
  constexpr typename @\wgChange{\exposid{math-common-simd-t}<V0, V1>}{\exposid{deduced-vec-t}<V>}@::mask_type islessgreater(const V@\wgRem{0}@& x, const V@\wgRem{1}@& y);
template<@\wgChange{class V0, class V1}{\exposid{math-floating-point} V}@>
  constexpr typename @\wgChange{\exposid{math-common-simd-t}<V0, V1>}{\exposid{deduced-vec-t}<V>}@::mask_type isunordered(const V@\wgRem{0}@& x, const V@\wgRem{1}@& y);
\end{itemdecl}

\begin{itemdescr}
\pnum
Let \tcode{Ret} denote the return type of the specialization of a function
template with the name \tcode{\placeholder{math-func}}.
Let \tcode{\placeholder{math-func-vec}} denote:
\begin{codeblock}
template<class... Args>
Ret @\placeholder{math-func-vec}@(Args... args) {
  return Ret([&](@\exposid{simd-size-type}@ i) {
    return @\placeholder{math-func}(\wgChange{\exposid{make-compatible-simd-t}<Ret, Args>}{static_cast<const \exposid{deduced-vec-t}<V>\&>}@(args)[i]...);
  });
}
\end{codeblock}

\pnum
\returns
A value \tcode{ret} of type \tcode{Ret}, that is element-wise equal to the
result of calling \tcode{\placeholder{math-func-vec}} with the arguments of the above
functions.
If in an invocation of a scalar overload of \tcode{\placeholder{math-func}} for index
\tcode{i} in \tcode{\placeholder{math-func-vec}} a domain, pole, or range error would
occur, the value of \tcode{ret[i]} is unspecified.

\pnum
\remarks
It is unspecified whether \tcode{errno}\iref{errno} is accessed.
\end{itemdescr}

\begin{itemdecl}
@\wgAdd{template<\exposconcept{math-floating-point} V>}@
  @\wgAdd{constexpr \exposid{deduced-vec-t}<V> ldexp(const V\& x, const rebind_t<int, \exposid{deduced-vec-t}<V>}\wgAdd{>\& exp);}@
@\wgAdd{template<\exposconcept{math-floating-point} V>}@
  @\wgAdd{constexpr \exposid{deduced-vec-t}<V> scalbn(const V\& x, const rebind_t<int, \exposid{deduced-vec-t}<V>}\wgAdd{>\& n);}@
@\wgAdd{template<\exposconcept{math-floating-point} V>}@
  @\wgAdd{constexpr \exposid{deduced-vec-t}<V>}@
    @\wgAdd{scalbln(const V\& x, const rebind_t<long int, \exposid{deduced-vec-t}<V>}\wgAdd{>\& n);}@
\end{itemdecl}

\begin{itemdescr}
\pnumAdd
\wgAdd{%
Let \tcode{Ret} be \tcode{\exposid{deduced-vec-t}<V>}.
Let \tcode{\placeholder{math-func}} denote the name of the function template.
Let \tcode{\placeholder{math-func-vec}} denote:}
\begin{codeblock}
@\wgAdd{Ret \placeholder{math-func-vec}(const \exposid{deduced-vec-t}<V>\& a, const auto\& b) \{}@
  @\wgAdd{return Ret([\&](\exposid{simd-size-type} i) \{}@
    @\wgAdd{return \placeholder{math-func}(a[i], b[i]);}@
  @\wgAdd{\});}@
@\wgAdd{\}}@
\end{codeblock}

\pnumAdd
\wgAdd{%
\returns
A value \tcode{ret} of type \tcode{Ret}, that is element-wise equal to the
result of calling \tcode{\placeholder{math-func-vec}} with the arguments of the above
functions.
If in an invocation of a scalar overload of \tcode{\placeholder{math-func}} for index
\tcode{i} in \tcode{\placeholder{math-func-vec}} a domain, pole, or range error would
occur, the value of \tcode{ret[i]} is unspecified.}

\pnumAdd
\wgAdd{%
\remarks
It is unspecified whether \tcode{errno}\iref{errno} is accessed.}
\end{itemdescr}

\begin{itemdecl}
@\wgAdd{template<\libconcept{signed_integral} T, class Abi>}@
  @\wgAdd{constexpr basic_vec<T, Abi> abs(const basic_vec<T, Abi>\& j);}@
\end{itemdecl}

\begin{itemdescr}
\pnumAdd
\wgAdd{%
\expects
\tcode{all_of(j >= -numeric_limits<T>::max())} is \tcode{true}.}

\pnumAdd
\wgAdd{%
\returns
An object where the $i^{\textrm{th}}$ element is initialized to the result of
\tcode{std::abs(j[$i$])} for all $i$ in the range \mbox{\range{0}{\tcode{j.size()}}}.}

\end{itemdescr}

\begin{itemdecl}
template<@\exposconcept{math-floating-point}@ V> constexpr @\exposid{deduced-vec-t}@<V> acos(const V& x);
template<@\exposconcept{math-floating-point}@ V> constexpr @\exposid{deduced-vec-t}@<V> asin(const V& x);
template<@\exposconcept{math-floating-point}@ V> constexpr @\exposid{deduced-vec-t}@<V> atan(const V& x);
template<@\wgChange{class V0, class V1}{\exposid{math-floating-point} V}@>
  constexpr @\wgChange{\exposid{math-common-simd-t}<V0, V1>}{\exposid{deduced-vec-t}<V>}@ atan2(const V@\wgRem{0}@& y, const V@\wgRem{1}@& x);
template<@\exposconcept{math-floating-point}@ V> constexpr @\exposid{deduced-vec-t}@<V> cos(const V& x);
template<@\exposconcept{math-floating-point}@ V> constexpr @\exposid{deduced-vec-t}@<V> sin(const V& x);
template<@\exposconcept{math-floating-point}@ V> constexpr @\exposid{deduced-vec-t}@<V> tan(const V& x);
template<@\exposconcept{math-floating-point}@ V> constexpr @\exposid{deduced-vec-t}@<V> acosh(const V& x);
template<@\exposconcept{math-floating-point}@ V> constexpr @\exposid{deduced-vec-t}@<V> asinh(const V& x);
template<@\exposconcept{math-floating-point}@ V> constexpr @\exposid{deduced-vec-t}@<V> atanh(const V& x);
template<@\exposconcept{math-floating-point}@ V> constexpr @\exposid{deduced-vec-t}@<V> cosh(const V& x);
template<@\exposconcept{math-floating-point}@ V> constexpr @\exposid{deduced-vec-t}@<V> sinh(const V& x);
template<@\exposconcept{math-floating-point}@ V> constexpr @\exposid{deduced-vec-t}@<V> tanh(const V& x);
template<@\exposconcept{math-floating-point}@ V> constexpr @\exposid{deduced-vec-t}@<V> exp(const V& x);
template<@\exposconcept{math-floating-point}@ V> constexpr @\exposid{deduced-vec-t}@<V> exp2(const V& x);
template<@\exposconcept{math-floating-point}@ V> constexpr @\exposid{deduced-vec-t}@<V> expm1(const V& x);
template<@\exposconcept{math-floating-point}@ V> constexpr @\exposid{deduced-vec-t}@<V> log(const V& x);
template<@\exposconcept{math-floating-point}@ V> constexpr @\exposid{deduced-vec-t}@<V> log10(const V& x);
template<@\exposconcept{math-floating-point}@ V> constexpr @\exposid{deduced-vec-t}@<V> log1p(const V& x);
template<@\exposconcept{math-floating-point}@ V> constexpr @\exposid{deduced-vec-t}@<V> log2(const V& x);
template<@\exposconcept{math-floating-point}@ V> constexpr @\exposid{deduced-vec-t}@<V> logb(const V& x);
template<@\exposconcept{math-floating-point}@ V> constexpr @\exposid{deduced-vec-t}@<V> cbrt(const V& x);
template<@\wgChange{class V0, class V1}{\exposid{math-floating-point} V}@>
  constexpr @\wgChange{\exposid{math-common-simd-t}<V0, V1>}{\exposid{deduced-vec-t}<V>}@ hypot(const V@\wgRem{0}@& x, const V@\wgRem{1}@& y);
template<@\wgChange{class V0, class V1, class V2}{\exposid{math-floating-point} V}@>
  constexpr @\wgChange{\exposid{math-common-simd-t}<V0, V1, V2>}{\exposid{deduced-vec-t}<V>}@ hypot(const V@\wgRem{0}@& x, const V@\wgRem{1}@& y, const V@\wgRem{2}@& z);
template<@\wgChange{class V0, class V1}{\exposid{math-floating-point} V}@>
  constexpr @\wgChange{\exposid{math-common-simd-t}<V0, V1>}{\exposid{deduced-vec-t}<V>}@ pow(const V@\wgRem{0}@& x, const V@\wgRem{1}@& y);
template<@\exposconcept{math-floating-point}@ V> constexpr @\exposid{deduced-vec-t}@<V> sqrt(const V& x);
template<@\exposconcept{math-floating-point}@ V> constexpr @\exposid{deduced-vec-t}@<V> erf(const V& x);
template<@\exposconcept{math-floating-point}@ V> constexpr @\exposid{deduced-vec-t}@<V> erfc(const V& x);
template<@\exposconcept{math-floating-point}@ V> constexpr @\exposid{deduced-vec-t}@<V> lgamma(const V& x);
template<@\exposconcept{math-floating-point}@ V> constexpr @\exposid{deduced-vec-t}@<V> tgamma(const V& x);
template<@\wgChange{class V0, class V1, class V2}{\exposid{math-floating-point} V}@>
  constexpr @\wgChange{\exposid{math-common-simd-t}<V0, V1, V2>}{\exposid{deduced-vec-t}<V>}@ lerp(const V@\wgRem{0}@& a, const V@\wgRem{1}@& b, const V@\wgRem{2}@& t) noexcept;
@\wgRem{template<\exposconcept{math-floating-point} V>}@
  @\wgRem{\exposid{deduced-vec-t}<V> assoc_laguerre(const rebind_t<unsigned, \exposid{deduced-vec-t}<V>>\& n, const}@
    @\wgRem{rebind_t<unsigned, \exposid{deduced-vec-t}<V>>\& m, const V\& x);}@
@\wgRem{template<\exposconcept{math-floating-point} V>}@
  @\wgRem{\exposid{deduced-vec-t}<V> assoc_legendre(const rebind_t<unsigned, \exposid{deduced-vec-t}<V>>\& l, const}@
    @\wgRem{rebind_t<unsigned, \exposid{deduced-vec-t}<V>>\& m, const V\& x);}@
template<@\wgChange{class V0, class V1}{\exposid{math-floating-point} V}@>
  @\wgChange{\exposid{math-common-simd-t}<V0, V1>}{\exposid{deduced-vec-t}<V>}@ beta(const V@\wgRem{0}@& x, const V@\wgRem{1}@& y);
template<@\exposconcept{math-floating-point}@ V> @\exposid{deduced-vec-t}@<V> comp_ellint_1(const V& k);
template<@\exposconcept{math-floating-point}@ V> @\exposid{deduced-vec-t}@<V> comp_ellint_2(const V& k);
template<@\wgChange{class V0, class V1}{\exposid{math-floating-point} V}@>
  @\wgChange{\exposid{math-common-simd-t}<V0, V1>}{\exposid{deduced-vec-t}<V>}@ comp_ellint_3(const V@\wgRem{0}@& k, const V@\wgRem{1}@& nu);
template<@\wgChange{class V0, class V1}{\exposid{math-floating-point} V}@>
  @\wgChange{\exposid{math-common-simd-t}<V0, V1>}{\exposid{deduced-vec-t}<V>}@ cyl_bessel_i(const V@\wgRem{0}@& nu, const V@\wgRem{1}@& x);
template<@\wgChange{class V0, class V1}{\exposid{math-floating-point} V}@>
  @\wgChange{\exposid{math-common-simd-t}<V0, V1>}{\exposid{deduced-vec-t}<V>}@ cyl_bessel_j(const V@\wgRem{0}@& nu, const V@\wgRem{1}@& x);
template<@\wgChange{class V0, class V1}{\exposid{math-floating-point} V}@>
  @\wgChange{\exposid{math-common-simd-t}<V0, V1>}{\exposid{deduced-vec-t}<V>}@ cyl_bessel_k(const V@\wgRem{0}@& nu, const V@\wgRem{1}@& x);
template<@\wgChange{class V0, class V1}{\exposid{math-floating-point} V}@>
  @\wgChange{\exposid{math-common-simd-t}<V0, V1>}{\exposid{deduced-vec-t}<V>}@ cyl_neumann(const V@\wgRem{0}@& nu, const V@\wgRem{1}@& x);
template<@\wgChange{class V0, class V1}{\exposid{math-floating-point} V}@>
  @\wgChange{\exposid{math-common-simd-t}<V0, V1>}{\exposid{deduced-vec-t}<V>}@ ellint_1(const V@\wgRem{0}@& k, const V@\wgRem{1}@& phi);
template<@\wgChange{class V0, class V1}{\exposid{math-floating-point} V}@>
  @\wgChange{\exposid{math-common-simd-t}<V0, V1>}{\exposid{deduced-vec-t}<V>}@ ellint_2(const V@\wgRem{0}@& k, const V@\wgRem{1}@& phi);
template<@\wgChange{class V0, class V1, class V2}{\exposid{math-floating-point} V}@>
  @\wgChange{\exposid{math-common-simd-t}<V0, V1, V2>}{\exposid{deduced-vec-t}<V>}@ ellint_3(const V@\wgRem{0}@& k, const V@\wgRem{1}@& nu, const V@\wgRem{2}@& phi);
template<@\exposconcept{math-floating-point}@ V> @\exposid{deduced-vec-t}@<V> expint(const V& x);
@\wgRem{template<\exposconcept{math-floating-point} V> \exposid{deduced-vec-t}<V> hermite(const rebind_t<unsigned,}@
@\wgRem{\exposid{deduced-vec-t}<V>>\& n, const V\& x);}@
@\wgRem{template<\exposconcept{math-floating-point} V> \exposid{deduced-vec-t}<V> laguerre(const rebind_t<unsigned,}@
@\wgRem{\exposid{deduced-vec-t}<V>>\& n, const V\& x);}@
@\wgRem{template<\exposconcept{math-floating-point} V> \exposid{deduced-vec-t}<V> legendre(const rebind_t<unsigned,}@
@\wgRem{\exposid{deduced-vec-t}<V>>\& l, const V\& x);}@
template<@\exposconcept{math-floating-point}@ V> @\exposid{deduced-vec-t}@<V> riemann_zeta(const V& x);
@\wgRem{template<\exposconcept{math-floating-point} V> \exposid{deduced-vec-t}<V> sph_bessel(const rebind_t<unsigned,}@
@\wgRem{\exposid{deduced-vec-t}<V>>\& n, const V\& x);}@
@\wgRem{template<\exposconcept{math-floating-point} V>}@
  @\wgRem{\exposid{deduced-vec-t}<V> sph_legendre(const rebind_t<unsigned, \exposid{deduced-vec-t}<V>>\& l,}@
                                 @\wgRem{const rebind_t<unsigned, \exposid{deduced-vec-t}<V>>\& m,}@
                                 @\wgRem{const V\& theta);}@
@\wgRem{template<\exposconcept{math-floating-point} V> \exposid{deduced-vec-t}<V> sph_neumann(const rebind_t<unsigned,}@
@\wgRem{\exposid{deduced-vec-t}<V>>\& n, const V\& x);}@
\end{itemdecl}

\begin{itemdescr}
\pnum
Let \tcode{Ret} denote the return type of the specialization of a function
template with the name \tcode{\placeholder{math-func}}.
Let \tcode{\placeholder{math-func-vec}} denote:
\begin{codeblock}
template<class... Args>
Ret @\placeholder{math-func-vec}(\wgAdd{const }Args\wgAdd{\&}@... args) {
  return Ret([&](@\exposid{simd-size-type}@ i) {
    return @\placeholder{math-func}(\wgChange{\exposid{make-compatible-simd-t}<Ret, Args>}{static_cast<const \exposid{deduced-vec-t}<V>\&>}@(args)[i]...);
  });
}
\end{codeblock}

\pnum
\returns
A value \tcode{ret} of type \tcode{Ret}, that is element-wise approximately
equal to the result of calling \tcode{\placeholder{math-func-vec}} with the arguments
of the above functions.
If in an invocation of a scalar overload of \tcode{\placeholder{math-func}} for index
\tcode{i} in \tcode{\placeholder{math-func-vec}} a domain, pole, or range error would
occur, the value of \tcode{ret[i]} is unspecified.

\pnum
\remarks
It is unspecified whether \tcode{errno}\iref{errno} is accessed.
\end{itemdescr}

\begin{itemdecl}
@\wgAdd{template<\exposconcept{math-floating-point} V>}@
  @\wgAdd{\exposid{deduced-vec-t}<V> assoc_laguerre(const rebind_t<unsigned, \exposid{deduced-vec-t}<V>}\wgAdd{>\& n,}@
                                  @\wgAdd{const rebind_t<unsigned, \exposid{deduced-vec-t}<V>}\wgAdd{>\& m, const V\& x);}@
@\wgAdd{template<\exposconcept{math-floating-point} V>}@
  @\wgAdd{\exposid{deduced-vec-t}<V> assoc_legendre(const rebind_t<unsigned, \exposid{deduced-vec-t}<V>}\wgAdd{>\& l,}@
                                  @\wgAdd{const rebind_t<unsigned, \exposid{deduced-vec-t}<V>}\wgAdd{>\& m, const V\& x);}@
@\wgAdd{template<\exposconcept{math-floating-point} V>}@
  @\wgAdd{\exposid{deduced-vec-t}<V> sph_legendre(const rebind_t<unsigned, \exposid{deduced-vec-t}<V>}\wgAdd{>\& l,}@
                                @\wgAdd{const rebind_t<unsigned, \exposid{deduced-vec-t}<V>}\wgAdd{>\& m, const V\& theta);}@
\end{itemdecl}

\begin{itemdescr}
\pnumAdd
\wgAdd{%
Let \tcode{\placeholder{math-func}} denote the name of the function template.
Let \tcode{\placeholder{math-func-vec}} denote:}
\begin{codeblock}
@\wgAdd{auto \placeholder{math-func-vec}(const auto\& a, const auto\&b, const \exposid{deduced-vec-t}<V>\& c) \{}@
  @\wgAdd{return \exposid{deduced-vec-t}<V>([\&](\exposid{simd-size-type} i) \{}@
    @\wgAdd{return std::\placeholder{math-func}(a[i], b[i], c[i]);}@
  @\wgAdd{\});}@
@\wgAdd{\}}@
\end{codeblock}

\pnumAdd
\wgAdd{%
\returns
An object that is element-wise approximately equal to the result of calling
\tcode{\placeholder{math-func-vec}} with the arguments of the above functions.}
\end{itemdescr}

\begin{itemdecl}
@\wgAdd{template<\exposconcept{math-floating-point} V>}@
  @\wgAdd{\exposid{deduced-vec-t}<V> hermite(const rebind_t<unsigned, \exposid{deduced-vec-t}<V>}\wgAdd{>\& n, const V\& x);}@
@\wgAdd{template<\exposconcept{math-floating-point} V>}@
  @\wgAdd{\exposid{deduced-vec-t}<V> laguerre(const rebind_t<unsigned, \exposid{deduced-vec-t}<V>}\wgAdd{>\& n, const V\& x);}@
@\wgAdd{template<\exposconcept{math-floating-point} V>}@
  @\wgAdd{\exposid{deduced-vec-t}<V> legendre(const rebind_t<unsigned, \exposid{deduced-vec-t}<V>}\wgAdd{>\& l, const V\& x);}@
@\wgAdd{template<\exposconcept{math-floating-point} V>}@
  @\wgAdd{\exposid{deduced-vec-t}<V> sph_bessel(const rebind_t<unsigned, \exposid{deduced-vec-t}<V>}\wgAdd{>\& n, const V\& x);}@
@\wgAdd{template<\exposconcept{math-floating-point} V>}@
  @\wgAdd{\exposid{deduced-vec-t}<V> sph_neumann(const rebind_t<unsigned, \exposid{deduced-vec-t}<V>}\wgAdd{>\& n, const V\& x);}@
\end{itemdecl}

\begin{itemdescr}
\pnumAdd
\wgAdd{%
Let \tcode{\placeholder{math-func}} denote the name of the function template.
Let \tcode{\placeholder{math-func-vec}} denote:}
\begin{codeblock}
@\wgAdd{auto \placeholder{math-func-vec}(const auto\& a, const \exposid{deduced-vec-t}<V>\& b) \{}@
  @\wgAdd{return \exposid{deduced-vec-t}<V>([\&](\exposid{simd-size-type} i) \{ return std::\placeholder{math-func}(a[i], b[i]); \});}@
@\wgAdd{\}}@
\end{codeblock}

\pnumAdd
\wgAdd{%
\returns
An object that is element-wise approximately equal to the result of calling
\tcode{\placeholder{math-func-vec}} with the arguments of the above functions.}
\end{itemdescr}

\begin{itemdecl}
template<@\exposconcept{math-floating-point}@ V>
  constexpr @\exposid{deduced-vec-t}@<V> frexp(const V& value, rebind_t<int, @\exposid{deduced-vec-t}@<V>>* exp);
\end{itemdecl}

\begin{itemdescr}
\pnum
Let \tcode{Ret} be \tcode{\exposid{deduced-vec-t}<V>}.
Let \tcode{\placeholder{frexp-vec}} denote:
\begin{codeblock}
template<class V>
pair<Ret, rebind_t<int, Ret>> @\placeholder{frexp-vec}@(const @\wgAdd{\exposid{deduced-vec-t}<}V\wgAdd{>}@& x) {
  int r1[Ret::size()];
  Ret r0([&](@\exposid{simd-size-type}@ i) {
    return frexp(@\wgChange{\exposid{make-compatible-simd-t}<Ret, V>(x)}{x}@[i], &r1[i]);
  });
  return {r0, rebind_t<int, Ret>(r1)};
}
\end{codeblock}
Let \tcode{ret} be a value of type \tcode{pair<Ret, rebind_t<int, Ret>>}
that is the same value as the result of calling
\tcode{\placeholder{frexp-vec}(x)}.

\pnum
\effects
Sets \tcode{*exp} to \tcode{ret.second}.

\pnum
\returns
\tcode{ret.first}.
\end{itemdescr}

\begin{itemdecl}
template<@\wgChange{class V0, class V1}{\exposid{math-floating-point} V}@>
  constexpr @\wgChange{\exposid{math-common-simd-t}<V0, V1>}{\exposid{deduced-vec-t}<V>}@ remquo(const V@\wgRem{0}@& x, const V@\wgRem{1}@& y,
                          rebind_t<int, @\wgChange{\exposid{math-common-simd-t}<V0, V1>}{\exposid{deduced-vec-t}<V>}@>* quo);
\end{itemdecl}

\begin{itemdescr}
\pnum
Let \wgChange{\tcode{Ret} be \tcode{\exposid{math-common-simd-t}<V0, V1>}}{\tcode{V0} be \tcode{\exposid{deduced-vec-t}<V>}}.
Let \tcode{\placeholder{remquo-vec}} denote:
\begin{codeblock}
@\wgRem{template<class V0, class V1>}@
pair<@\wgChange{Ret}{V0}, rebind_t<int, \wgChange{Ret}{V0}>> \placeholder{remquo-vec}@(const V0& x, const @\wgChange{V1}{V0}@& y) {
  int r1[@\wgChange{Ret}{V0}@::size()];
  V0 r0([&](@\exposid{simd-size-type}@ i) {
      return remquo(@\wgChange{\exposid{make-compatible-simd-t}<Ret, V0>(x)}{x}@[i],
      @\wgChange{\exposid{make-compatible-simd-t}<Ret, V1>(y)}{y}@[i], &r1[i]);
  });
  return {r0, rebind_t<int, @\wgChange{Ret}{V0}@>(r1)};
}
\end{codeblock}
Let \tcode{ret} be a value of type \wgChange{\tcode{pair<Ret, rebind_t<int, Ret>>}}{\tcode{pair<V0, rebind_t<int, V0>>}}
that is the same value as the result of calling
\tcode{\placeholder{remquo-vec}(x, y)}.
If in an invocation of a scalar overload of \tcode{remquo} for index \tcode{i}
in \tcode{\placeholder{remquo-vec}} a domain, pole, or range error would occur, the
value of \tcode{ret[i]} is unspecified.

\pnum
\effects
Sets \tcode{*quo} to \tcode{ret.second}.

\pnum
\returns
\tcode{ret.first}.

\pnum
\remarks
It is unspecified whether \tcode{errno}\iref{errno} is accessed.
\end{itemdescr}

\begin{itemdecl}
template<class T, class Abi>
  constexpr basic_vec<T, Abi> modf(const type_identity_t<basic_vec<T, Abi>>& value,
                                   basic_vec<T, Abi>* iptr);
\end{itemdecl}

\begin{itemdescr}
\pnum
Let \tcode{V} be \tcode{basic_vec<T, Abi>}.
Let \tcode{\placeholder{modf-vec}} denote:
\begin{codeblock}
pair<V, V> @\placeholder{modf-vec}@(const V& x) {
  T r1[Ret::size()];
  V r0([&](@\exposid{simd-size-type}@ i) {
    return modf(V(x)[i], &r1[i]);
  });
  return {r0, V(r1)};
}
\end{codeblock}
Let \tcode{ret} be a value of type \tcode{pair<V, V>} that is the same value as
the result of calling \tcode{\placeholder{modf-vec}(value)}.

\pnum
\effects
Sets \tcode{*iptr} to \tcode{ret.second}.

\pnum
\returns
\tcode{ret.first}.
\end{itemdescr}

\begin{itemdecl}
@\wgAdd{template<\exposconcept{math-floating-point} V>}@
  @\wgAdd{constexpr \exposid{deduced-vec-t}<V> fmod(const \exposid{deduced-vec-t}<V>\& x, const V\& y);}@
@\wgAdd{template<\exposconcept{math-floating-point} V>}@
  @\wgAdd{constexpr \exposid{deduced-vec-t}<V> fmod(const V\& x, const \exposid{deduced-vec-t}<V>\& y);}@
@\wgAdd{template<\exposconcept{math-floating-point} V>}@
  @\wgAdd{constexpr \exposid{deduced-vec-t}<V> remainder(const \exposid{deduced-vec-t}<V>\& x, const V\& y);}@
@\wgAdd{template<\exposconcept{math-floating-point} V>}@
  @\wgAdd{constexpr \exposid{deduced-vec-t}<V> remainder(const V\& x, const \exposid{deduced-vec-t}<V>\& y);}@
@\wgAdd{template<\exposconcept{math-floating-point} V>}@
  @\wgAdd{constexpr \exposid{deduced-vec-t}<V> copysign(const \exposid{deduced-vec-t}<V>\& x, const V\& y);}@
@\wgAdd{template<\exposconcept{math-floating-point} V>}@
  @\wgAdd{constexpr \exposid{deduced-vec-t}<V> copysign(const V\& x, const \exposid{deduced-vec-t}<V>\& y);}@
@\wgAdd{template<\exposconcept{math-floating-point} V>}@
  @\wgAdd{constexpr \exposid{deduced-vec-t}<V> nextafter(const \exposid{deduced-vec-t}<V>\& x, const V\& y);}@
@\wgAdd{template<\exposconcept{math-floating-point} V>}@
  @\wgAdd{constexpr \exposid{deduced-vec-t}<V> nextafter(const V\& x, const \exposid{deduced-vec-t}<V>\& y);}@
@\wgAdd{template<\exposconcept{math-floating-point} V>}@
  @\wgAdd{constexpr \exposid{deduced-vec-t}<V> fdim(const \exposid{deduced-vec-t}<V>\& x, const V\& y);}@
@\wgAdd{template<\exposconcept{math-floating-point} V>}@
  @\wgAdd{constexpr \exposid{deduced-vec-t}<V> fdim(const V\& x, const \exposid{deduced-vec-t}<V>\& y);}@
@\wgAdd{template<\exposconcept{math-floating-point} V>}@
  @\wgAdd{constexpr \exposid{deduced-vec-t}<V> fmax(const \exposid{deduced-vec-t}<V>\& x, const V\& y);}@
@\wgAdd{template<\exposconcept{math-floating-point} V>}@
  @\wgAdd{constexpr \exposid{deduced-vec-t}<V> fmax(const V\& x, const \exposid{deduced-vec-t}<V>\& y);}@
@\wgAdd{template<\exposconcept{math-floating-point} V>}@
  @\wgAdd{constexpr \exposid{deduced-vec-t}<V> fmin(const \exposid{deduced-vec-t}<V>\& x, const V\& y);}@
@\wgAdd{template<\exposconcept{math-floating-point} V>}@
  @\wgAdd{constexpr \exposid{deduced-vec-t}<V> fmin(const V\& x, const \exposid{deduced-vec-t}<V>\& y);}@
@\wgAdd{template<\exposconcept{math-floating-point} V>}@
  @\wgAdd{constexpr \exposid{deduced-vec-t}<V> isgreater(const \exposid{deduced-vec-t}<V>\& x, const V\& y);}@
@\wgAdd{template<\exposconcept{math-floating-point} V>}@
  @\wgAdd{constexpr \exposid{deduced-vec-t}<V> isgreater(const V\& x, const \exposid{deduced-vec-t}<V>\& y);}@
@\wgAdd{template<\exposconcept{math-floating-point} V>}@
  @\wgAdd{constexpr \exposid{deduced-vec-t}<V> islessequal(const \exposid{deduced-vec-t}<V>\& x, const V\& y);}@
@\wgAdd{template<\exposconcept{math-floating-point} V>}@
  @\wgAdd{constexpr \exposid{deduced-vec-t}<V> islessequal(const V\& x, const \exposid{deduced-vec-t}<V>\& y);}@
@\wgAdd{template<\exposconcept{math-floating-point} V>}@
  @\wgAdd{constexpr \exposid{deduced-vec-t}<V> islessgreater(const \exposid{deduced-vec-t}<V>\& x, const V\& y);}@
@\wgAdd{template<\exposconcept{math-floating-point} V>}@
  @\wgAdd{constexpr \exposid{deduced-vec-t}<V> islessgreater(const V\& x, const \exposid{deduced-vec-t}<V>\& y);}@
@\wgAdd{template<\exposconcept{math-floating-point} V>}@
  @\wgAdd{constexpr \exposid{deduced-vec-t}<V> isunordered(const \exposid{deduced-vec-t}<V>\& x, const V\& y);}@
@\wgAdd{template<\exposconcept{math-floating-point} V>}@
  @\wgAdd{constexpr \exposid{deduced-vec-t}<V> isunordered(const V\& x, const \exposid{deduced-vec-t}<V>\& y);}@
@\wgAdd{template<\exposconcept{math-floating-point} V>}@
  @\wgAdd{constexpr \exposid{deduced-vec-t}<V> atan2(const \exposid{deduced-vec-t}<V>\& x, const V\& y);}@
@\wgAdd{template<\exposconcept{math-floating-point} V>}@
  @\wgAdd{constexpr \exposid{deduced-vec-t}<V> atan2(const V\& x, const \exposid{deduced-vec-t}<V>\& y);}@
@\wgAdd{template<\exposconcept{math-floating-point} V>}@
  @\wgAdd{constexpr \exposid{deduced-vec-t}<V> hypot(const \exposid{deduced-vec-t}<V>\& x, const V\& y);}@
@\wgAdd{template<\exposconcept{math-floating-point} V>}@
  @\wgAdd{constexpr \exposid{deduced-vec-t}<V> hypot(const V\& x, const \exposid{deduced-vec-t}<V>\& y);}@
@\wgAdd{template<\exposconcept{math-floating-point} V>}@
  @\wgAdd{constexpr \exposid{deduced-vec-t}<V> pow(const \exposid{deduced-vec-t}<V>\& x, const V\& y);}@
@\wgAdd{template<\exposconcept{math-floating-point} V>}@
  @\wgAdd{constexpr \exposid{deduced-vec-t}<V> pow(const V\& x, const \exposid{deduced-vec-t}<V>\& y);}@
@\wgAdd{template<\exposconcept{math-floating-point} V>}@
  @\wgAdd{constexpr \exposid{deduced-vec-t}<V> beta(const \exposid{deduced-vec-t}<V>\& x, const V\& y);}@
@\wgAdd{template<\exposconcept{math-floating-point} V>}@
  @\wgAdd{constexpr \exposid{deduced-vec-t}<V> beta(const V\& x, const \exposid{deduced-vec-t}<V>\& y);}@
@\wgAdd{template<\exposconcept{math-floating-point} V>}@
  @\wgAdd{constexpr \exposid{deduced-vec-t}<V> comp_ellint_3(const \exposid{deduced-vec-t}<V>\& x, const V\& y);}@
@\wgAdd{template<\exposconcept{math-floating-point} V>}@
  @\wgAdd{constexpr \exposid{deduced-vec-t}<V> comp_ellint_3(const V\& x, const \exposid{deduced-vec-t}<V>\& y);}@
@\wgAdd{template<\exposconcept{math-floating-point} V>}@
  @\wgAdd{constexpr \exposid{deduced-vec-t}<V> cyl_bessel_i(const \exposid{deduced-vec-t}<V>\& x, const V\& y);}@
@\wgAdd{template<\exposconcept{math-floating-point} V>}@
  @\wgAdd{constexpr \exposid{deduced-vec-t}<V> cyl_bessel_i(const V\& x, const \exposid{deduced-vec-t}<V>\& y);}@
@\wgAdd{template<\exposconcept{math-floating-point} V>}@
  @\wgAdd{constexpr \exposid{deduced-vec-t}<V> cyl_bessel_j(const \exposid{deduced-vec-t}<V>\& x, const V\& y);}@
@\wgAdd{template<\exposconcept{math-floating-point} V>}@
  @\wgAdd{constexpr \exposid{deduced-vec-t}<V> cyl_bessel_j(const V\& x, const \exposid{deduced-vec-t}<V>\& y);}@
@\wgAdd{template<\exposconcept{math-floating-point} V>}@
  @\wgAdd{constexpr \exposid{deduced-vec-t}<V> cyl_bessel_k(const \exposid{deduced-vec-t}<V>\& x, const V\& y);}@
@\wgAdd{template<\exposconcept{math-floating-point} V>}@
  @\wgAdd{constexpr \exposid{deduced-vec-t}<V> cyl_bessel_k(const V\& x, const \exposid{deduced-vec-t}<V>\& y);}@
@\wgAdd{template<\exposconcept{math-floating-point} V>}@
  @\wgAdd{constexpr \exposid{deduced-vec-t}<V> cyl_neumann(const \exposid{deduced-vec-t}<V>\& x, const V\& y);}@
@\wgAdd{template<\exposconcept{math-floating-point} V>}@
  @\wgAdd{constexpr \exposid{deduced-vec-t}<V> cyl_neumann(const V\& x, const \exposid{deduced-vec-t}<V>\& y);}@
@\wgAdd{template<\exposconcept{math-floating-point} V>}@
  @\wgAdd{constexpr \exposid{deduced-vec-t}<V> ellint_1(const \exposid{deduced-vec-t}<V>\& x, const V\& y);}@
@\wgAdd{template<\exposconcept{math-floating-point} V>}@
  @\wgAdd{constexpr \exposid{deduced-vec-t}<V> ellint_1(const V\& x, const \exposid{deduced-vec-t}<V>\& y);}@
@\wgAdd{template<\exposconcept{math-floating-point} V>}@
  @\wgAdd{constexpr \exposid{deduced-vec-t}<V> ellint_2(const \exposid{deduced-vec-t}<V>\& x, const V\& y);}@
@\wgAdd{template<\exposconcept{math-floating-point} V>}@
  @\wgAdd{constexpr \exposid{deduced-vec-t}<V> ellint_2(const V\& x, const \exposid{deduced-vec-t}<V>\& y);}@
@\wgAdd{template<\exposconcept{math-floating-point} V>}@
  @\wgAdd{constexpr \exposid{deduced-vec-t}<V>}@
    @\wgAdd{remquo(const \exposid{deduced-vec-t}<V>\& x, const V\& y, rebind_t<int, \exposid{deduced-vec-t}<V>>* quo);}@
@\wgAdd{template<\exposconcept{math-floating-point} V>}@
  @\wgAdd{constexpr \exposid{deduced-vec-t}<V>}@
    @\wgAdd{remquo(const V\& x, const \exposid{deduced-vec-t}<V>\& y, rebind_t<int, \exposid{deduced-vec-t}<V>>* quo);}@
@\wgAdd{template<\exposconcept{math-floating-point} V>}@
  @\wgAdd{constexpr \exposid{deduced-vec-t}<V> fma(const \exposid{deduced-vec-t}<V>\& x, const V\& y, const V\& z);}@
@\wgAdd{template<\exposconcept{math-floating-point} V>}@
  @\wgAdd{constexpr \exposid{deduced-vec-t}<V> fma(const V\& x, const \exposid{deduced-vec-t}<V>\& y, const V\& z);}@
@\wgAdd{template<\exposconcept{math-floating-point} V>}@
  @\wgAdd{constexpr \exposid{deduced-vec-t}<V> fma(const V\& x, const V\& y, const \exposid{deduced-vec-t}<V>\& z);}@
@\wgAdd{template<\exposconcept{math-floating-point} V>}@
  @\wgAdd{constexpr \exposid{deduced-vec-t}<V> fma(const \exposid{deduced-vec-t}<V>\& x, const \exposid{deduced-vec-t}<V>\& y, const V\& z);}@
@\wgAdd{template<\exposconcept{math-floating-point} V>}@
  @\wgAdd{constexpr \exposid{deduced-vec-t}<V> fma(const \exposid{deduced-vec-t}<V>\& x, const V\& y, const \exposid{deduced-vec-t}<V>\& z);}@
@\wgAdd{template<\exposconcept{math-floating-point} V>}@
  @\wgAdd{constexpr \exposid{deduced-vec-t}<V> fma(const V\& x, const \exposid{deduced-vec-t}<V>\& y, const \exposid{deduced-vec-t}<V>\& z);}@
@\wgAdd{template<\exposconcept{math-floating-point} V>}@
  @\wgAdd{constexpr \exposid{deduced-vec-t}<V> hypot(const \exposid{deduced-vec-t}<V>\& x, const V\& y, const V\& z);}@
@\wgAdd{template<\exposconcept{math-floating-point} V>}@
  @\wgAdd{constexpr \exposid{deduced-vec-t}<V> hypot(const V\& x, const \exposid{deduced-vec-t}<V>\& y, const V\& z);}@
@\wgAdd{template<\exposconcept{math-floating-point} V>}@
  @\wgAdd{constexpr \exposid{deduced-vec-t}<V> hypot(const V\& x, const V\& y, const \exposid{deduced-vec-t}<V>\& z);}@
@\wgAdd{template<\exposconcept{math-floating-point} V>}@
  @\wgAdd{constexpr \exposid{deduced-vec-t}<V> hypot(const \exposid{deduced-vec-t}<V>\& x, const \exposid{deduced-vec-t}<V>\& y, const V\& z);}@
@\wgAdd{template<\exposconcept{math-floating-point} V>}@
  @\wgAdd{constexpr \exposid{deduced-vec-t}<V> hypot(const \exposid{deduced-vec-t}<V>\& x, const V\& y, const \exposid{deduced-vec-t}<V>\& z);}@
@\wgAdd{template<\exposconcept{math-floating-point} V>}@
  @\wgAdd{constexpr \exposid{deduced-vec-t}<V> hypot(const V\& x, const \exposid{deduced-vec-t}<V>\& y, const \exposid{deduced-vec-t}<V>\& z);}@
@\wgAdd{template<\exposconcept{math-floating-point} V>}@
  @\wgAdd{constexpr \exposid{deduced-vec-t}<V> lerp(const \exposid{deduced-vec-t}<V>\& x, const V\& y, const V\& z);}@
@\wgAdd{template<\exposconcept{math-floating-point} V>}@
  @\wgAdd{constexpr \exposid{deduced-vec-t}<V> lerp(const V\& x, const \exposid{deduced-vec-t}<V>\& y, const V\& z);}@
@\wgAdd{template<\exposconcept{math-floating-point} V>}@
  @\wgAdd{constexpr \exposid{deduced-vec-t}<V> lerp(const V\& x, const V\& y, const \exposid{deduced-vec-t}<V>\& z);}@
@\wgAdd{template<\exposconcept{math-floating-point} V>}@
  @\wgAdd{constexpr \exposid{deduced-vec-t}<V> lerp(const \exposid{deduced-vec-t}<V>\& x, const \exposid{deduced-vec-t}<V>\& y, const V\& z);}@
@\wgAdd{template<\exposconcept{math-floating-point} V>}@
  @\wgAdd{constexpr \exposid{deduced-vec-t}<V> lerp(const \exposid{deduced-vec-t}<V>\& x, const V\& y, const \exposid{deduced-vec-t}<V>\& z);}@
@\wgAdd{template<\exposconcept{math-floating-point} V>}@
  @\wgAdd{constexpr \exposid{deduced-vec-t}<V> lerp(const V\& x, const \exposid{deduced-vec-t}<V>\& y, const \exposid{deduced-vec-t}<V>\& z);}@
@\wgAdd{template<\exposconcept{math-floating-point} V>}@
  @\wgAdd{constexpr \exposid{deduced-vec-t}<V> ellint_3(const \exposid{deduced-vec-t}<V>\& x, const V\& y, const V\& z);}@
@\wgAdd{template<\exposconcept{math-floating-point} V>}@
  @\wgAdd{constexpr \exposid{deduced-vec-t}<V> ellint_3(const V\& x, const \exposid{deduced-vec-t}<V>\& y, const V\& z);}@
@\wgAdd{template<\exposconcept{math-floating-point} V>}@
  @\wgAdd{constexpr \exposid{deduced-vec-t}<V> ellint_3(const V\& x, const V\& y, const \exposid{deduced-vec-t}<V>\& z);}@
@\wgAdd{template<\exposconcept{math-floating-point} V>}@
  @\wgAdd{constexpr \exposid{deduced-vec-t}<V> ellint_3(const \exposid{deduced-vec-t}<V>\& x, const \exposid{deduced-vec-t}<V>\& y,}@
                                      @\wgAdd{const V\& z);}@
@\wgAdd{template<\exposconcept{math-floating-point} V>}@
  @\wgAdd{constexpr \exposid{deduced-vec-t}<V> ellint_3(const \exposid{deduced-vec-t}<V>\& x, const V\& y,}@
                                      @\wgAdd{const \exposid{deduced-vec-t}<V>\& z);}@
@\wgAdd{template<\exposconcept{math-floating-point} V>}@
  @\wgAdd{constexpr \exposid{deduced-vec-t}<V> ellint_3(const V\& x, const \exposid{deduced-vec-t}<V>\& y,}@
                                      @\wgAdd{const \exposid{deduced-vec-t}<V>\& z);}@
\end{itemdecl}

\begin{itemdescr}
\pnumAdd
\wgAdd{Let}
\begin{itemize}
  \wgItemAdd[\placeholder{math-func} denote the name of the function template;]
  \wgItemAdd[\tcode{args...} be \tcode{x} and \tcode{y}, or \tcode{x}, \tcode{y}, and \tcode{z};]
  \wgItemAdd[\tcode{rest...} be all remaining arguments besides \tcode{x}, \tcode{y}, and \tcode{z}.]
\end{itemize}

\pnumAdd
\wgAdd{\effects}
\wgAdd{Equivalent to: \tcode{\placeholder{math-func}(static_cast<const \exposid{deduced-vec-t}<V>\&>(args)..., rest...)}}
\end{itemdescr}
\end{wgText}
