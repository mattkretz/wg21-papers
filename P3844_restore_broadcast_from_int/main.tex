\newcommand\wgTitle{Restore \code{simd::vec} broadcast from \code{int}}
\newcommand\wgName{Matthias Kretz <m.kretz@gsi.de>}
\newcommand\wgDocumentNumber{D3844R0}
\newcommand\wgGroup{LEWG}
\newcommand\wgTarget{\CC{}26}
%\newcommand\wgAcknowledgements{Daniel Towner and Ruslan Arutyunyan contributed to this paper via discussions / reviews. Thanks also to Jeff Garland for reviewing.}

\usepackage{mymacros}
\usepackage{wg21}
\setcounter{tocdepth}{2} % show sections and subsections in TOC
\hypersetup{bookmarksdepth=5}
\usepackage{changelog}
\usepackage{underscore}
\usepackage{multirow}
\usepackage{vectorpictures}

\addbibresource{extra.bib}

\newcommand\simd[1][]{\type{ba\-sic\_vec#1}\xspace}
\newcommand\simdT{\type{ba\-sic\_vec\MayBreak<\MayBreak{}T>}\xspace}
\newcommand\valuetype{\type{val\-ue\_type}\xspace}
\newcommand\referencetype{\type{ref\-er\-ence}\xspace}
\newcommand\mask[1][]{\type{ba\-sic\_mask#1}\xspace}
\newcommand\maskT{\type{ba\-sic\_mask\MayBreak<\MayBreak{}T>}\xspace}
\newcommand\wglink[1]{\href{https://wg21.link/#1}{#1}}

\newcommand\nativeabi{\UNSP{native-abi}}
\newcommand\deducet{\UNSP{deduce-t}}
\newcommand\simdsizev{\UNSP{simd-size-v}}
\newcommand\simdsizetype{\UNSP{simd-size-type}}
\newcommand\simdselect{\UNSP{simd-select-impl}}
\newcommand\maskelementsize{\UNSP{mask-element-size}}
\newcommand\integerfrom{\UNSP{integer-from}}
\newcommand\constexprwrapperlike{\UNSP{constexpr-wrapper-like}}
\newcommand\convertflag{\UNSP{convert-flag}}
\newcommand\alignedflag{\UNSP{aligned-flag}}
\newcommand\overalignedflag{\UNSP{overaligned-flag}}
\newcommand\reductionoperation{\UNSP{reduction-binary-operation}}
\newcommand\simdfloatingpoint{\UNSP{simd-floating-point}}
\newcommand\multisimdfloatingpoint{\UNSP{multi-arg-simd-floating-point}}

\renewcommand{\lst}[1]{Listing~\ref{#1}}
\renewcommand{\sect}[1]{Section~\ref{#1}}
\renewcommand{\ttref}[1]{Tony~Table~\ref{#1}}
\renewcommand{\tabref}[1]{Table~\ref{#1}}

\begin{document}
\selectlanguage{american}
\begin{wgTitlepage}
  The broadcast constructor in the Parallelism 2 TS allowed construction from (\code{unsigned})
  \code{int}, allowing e.g. \code{vec<float>() + 1}, which is ill-formed in the CD.
  This breaks existing code that gets ported from the TS to \code{std::simd}.
  The design intent behind \code{std::simd} was for this to work.
  However, the understanding in LEWG appeared to be that we can't get this right without constexpr
  function arguments getting added to the language.
  This paper shows that a \code{consteval} constructor overload together with \code{constexpr}
  exceptions can resolve the issue for \CC{}26 and is a better solution than constexpr function
  arguments would be.
\end{wgTitlepage}

\pagestyle{scrheadings}

\section{Changelog}
(placeholder)
\begin{revision}
\item Add a simple example to the motivation section.
\item Expand the “Generalization” section to clearly define the feature rather
  than just sketching it.
  Also add a discussion of initial value and step.
\item Discuss why reusing the existing \code{iota} algorithm/view does not
  work/suffice for the \code{simd} use case.
\item Discuss why \code{iota_v} is the right name.
%  \todo
\end{revision}

\section{Straw Polls}
\subsection{SG1 at Kona 2022}
\wgPoll{After significant experience with the TS, we recommend that the next
version (the TS version with improvements) of \code{std::simd} target the IS (\CC{}26)}
{10&8&0&0&0}

\wgUnanimous{We like all of the recommended changes to \code{std::simd} proposed in p1928r1
(Includes making all of \code{std::simd} \code{constexpr}, and dropping an ABI stable type)}

\wgPoll{Future papers and future revisions of existing papers that target
\code{std::simd} should go directly to LEWG.
(We do not believe there are SG1 issues with \code{std::simd} today.)}
{9&8&0&0&0}


\pagebreak
\section{Motivation}

It is very common in floating-point code to simply write e.g. \code{* 2} rather than \code{* 2.f}
when multiplying a \code{float} with a constant:
\medskip\begin{lstlisting}[style=Vc]
float f(float x) { return x * 2; } // converts 2 to float (at compile time)

float g(float x) { return x * 2.; } // converts x to double (at run time)

float h(float x) { return x * 2.f; } // no conversions
\end{lstlisting}

More importantly, using \code{* 2} works reliably in generic code, where the type of \code{x} could
be any arithmetic type.

Since this is so common, \code{std::experimental::simd<T>} made an exception for \code{int} in the
broadcast constructor to not require value-preserving conversions.
Consequently, the TS allows:
\medskip\begin{lstlisting}[style=Vc]
using floatv = std::experimental::native_simd<float>;

floatv f(floatv x) { return x * 2; } // converts 2 to float and broadcasts (at compile time)

floatv g(floatv x) { return x * 2.; } // ill-formed

floatv h(floatv x) { return x * 2.f; } // broadcasts 2.f to floatv
\end{lstlisting}

When porting existing code written against the TS to \CC{}26, the first step is to adjust the types:
\medskip\begin{lstlisting}
using floatv = std::@\wgChange{experimental::native_simd}{simd::vec}@<float>;
\end{lstlisting}
Except for uses of \code{std::experimental::where}, which need to be refactored to use
\code{simd::select}, the remaining code should work.
The one place where it doesn't work is code such as in function \code{f}, where \code{2} needs to be
replaced:
\medskip\begin{lstlisting}[style=Vc]
floatv f(floatv x) { return x * @\wgChange{2}{std::cw<2>}@; }
\end{lstlisting}

Since we don't have \code{constexpr} function arguments in the language, \code{std::simd} works
around it by recognizing \exposid{integral-constant-like} / \code{constant-wrapper-like} types, that
encode a \emph{value} into a type.
This, however, comes at a compile-time cost.
Every different value leads to a template specialization of both \code{constant_wrapper} and a
\code{basic_vec} broadcast constructor (with it's helper types/concepts to determine whether the
specialization is allowed).
Consequently, for \code{vec<float>}, I would recommend to always use an \code{f} suffix rather than
\code{std::cw}.

But that solution is fairly limited, since we don't have literals for 8-bit and 16-bit integers in
the language.
A function template like
\medskip\begin{lstlisting}[style=Vc]
template <simd_integral V>
V f(V x) {
  return x + 1; // ill-formed for V::value_type = (u)int8_t, (un)int16_t, and uint32_t
}
\end{lstlisting}
needs to use \code{x + V(1)}\footnote{explicit conversion to \code{basic_vec} allows conversions
that are not value-preserving}.
A clever user might write \lstinline@x + '\1'@ instead.
But that fails for the \code{char} type with different signedness.

Consequently, users would need to get used to writing explicit conversions for the constants they
use in \code{std::simd} code.
That's not only verbose and ugly, it is also error-prone.
Whenever we coerce our users into writing explicit conversions, then value-changing conversions
cannot be diagnosed as erroneous anymore.
An explicit \code{static_cast<uint64_t>(-1)} means \code{0xffff'ffff'ffff'ffff}, whereas
\code{uint64_t x = -1} could have been intended to mean \code{0x0000'0000'ffff'ffff} or is a result
of a logic flaw in the code.
E.g., GCC's \code{-Wsign-conversion} diagnoses the latter, but not the former\footnote{And that's
useful, because the former says “I'm intentionally doing this conversion, no need to warn.”}.

If, with \CC{}26, our users are starting to explicitly convert their \code{int} constants to
\code{basic_vec}, then the interface of \code{basic_vec} is at least in part guilty for introducing
harder to find bugs.

\ttref{tt:add-an-offset} presents an example of the solution\footnote{I got bitten by this in my
\code{std::simd} unit tests}.
Note that the code on the left will never warn about the value-changing conversion, even with all
conversion related warnings enabled.
This is due to the explicit conversion, which is telling the compiler “I know what I'm doing; no
need to warn me about it”.
\begin{tonytable}{Add an offset}\label{tt:add-an-offset}
  \begin{lstlisting}
template <simd_floating_point V>
V f(V x) {
  return x + V(0x5EAF00D);
}

f(vec<double>()); // OK

// compiles but
// adds 99282960 instead of 99282957
f(vec<float>());

// compiles but
// adds infinity instead of 99282957
f(vec<std::float16_t>());
  \end{lstlisting}
  &
  \begin{lstlisting}
template <simd_floating_point V>
V f(V x) {
  return x + 0x5EAF00D;
}

f(vec<double>()); // OK

// ill-formed: uncaught exception on
// value-changing conversion
f(vec<float>());

// ill-formed: uncaught exception on
// value-changing conversion
f(vec<std::float16_t>());
  \end{lstlisting}
\end{tonytable}%

A safer implementation of the code on the left side of \ttref{tt:add-an-offset} (without this paper)
would have been to write \code{x + std::cw<0x5EAF00D>} instead.
Then the value-changing conversion would have resulted in a constraint failure on the broadcast
constructor.
However, \code{V(0x5EAF00D)} is shorter and needs fewer template instantations.
I expect most users (including myself) will/do not use \code{std::cw} all over the place.

\section{Design space}
In the design review of this issue of the broadcast constructor it was overlooked (and never
discussed) that a \code{consteval} overload of the broadcast constructor could solve this problem.
Before \code{constexpr} exceptions, we would have worded it to be ill-formed (by unspecified means)
if the value changes on conversion to the \code{basic_vec}'s value-type.
Now that we have \code{constexpr} exceptions, we can specify a \code{consteval} broadcast overload
that throws on value-changing conversion.
If the caller cares, the exception can even be handled at compile time.
(I believe it should not throw in \CC{}26, for a minimal change to the WD.)

Ordering the overloads for overload resolution is tricky, which is another reason why we should
consider this issue before \CC{}26 ships and potentially take action even if we don't add a
\code{consteval} overload.
Overload resolution does not take \code{consteval} into account.
The process of finding candidate functions%
\href{https://eel.is/c++draft/over.match#funcs.general-8}{\iref{over.match.funcs.general}}, however,
does remove explicit constructors from the candidate set if the
context does not allow the explicit constructor to be called.

\subsection{Status quo}
The following code shows the properties of the current broadcast constructor.

See Appendix \ref{sec:really-convertible} for the definition of the \code{really_convertible_to}
concept.

\bigskip
\begin{lstlisting}[style=Vc]
using V = simd::vec<float>;

template <typename T> struct X { explicit operator T() const; };

static_assert(not   std::convertible_to<X<float>, V>);
static_assert(      std::convertible_to<float, V>);
static_assert(      std::convertible_to<short, V>);
static_assert(    really_convertible_to<short, V>);
static_assert(not   std::convertible_to<int, V>);
static_assert(not really_convertible_to<int, V>);

static_assert(    std::constructible_from<V, X<float>>);
static_assert(not std::constructible_from<V, X<short>>);
static_assert(    std::constructible_from<V, float>);
static_assert(    std::constructible_from<V, short>);
static_assert(    std::constructible_from<V, int>);

V f(int n, short m, std::reference_wrapper<int> l, std::reference_wrapper<float> f)
{
  V x = '\1'; // OK
  x = 1;      // ill-formed
  x = 0x5EAF00D; // ill-formed
  x = V(n);   // OK
  x = m;      // OK
  x = l;      // OK (because convertible_to<decltype(l), float> is true)
  x = f;      // OK
  x = X<float>(); // ill-formed: no match for operator= (no known conversion […])
  x = float(X<float>()); // OK (obvious)
  x = V(X<float>()); // OK
}
\end{lstlisting}

\subsection{More constrained \code{constexpr} overload}\label{sec:explicit}
A possible solution selects the existing (\code{constexpr}) broadcast constructor for
everything but the cases where the value of the argument needs to be checked.
Thus, we need the existing constructor to always be \emph{more constrained} \iref{temp.constr.order}
than the \code{consteval} constructor.
The \code{consteval} constructor can then only be selected if the other constructor is not part of
the candidate set at all (via \code{explicit}).

Sketch:
\medskip\begin{lstlisting}[style=Vc]
template <class From, class To>
  concept @\exposconcept{simd-consteval-broadcast-arg}@ = constructible_from<To, From>;

template <class From, class To>
  concept @\exposconcept{simd-broadcast-arg}@ = @\exposconcept{simd-consteval-broadcast-arg}@<From, To> and true;

template <class T>
class basic_vec
{
public:
  template <@\exposconcept{simd-broadcast-arg}@<T> U>
    constexpr explicit(@\seebelow@) basic_vec(U&&); // #1
  template <@\exposconcept{simd-consteval-broadcast-arg}@<T> U>
    consteval basic_vec(U&&);                    // #2
    // Mandates: convertible_to<U, value_type> && is_arithmetic_v<remove_cvref_t<U>>
};
\end{lstlisting}

Now every explicit call to the broadcast constructor will always select \code{\#1}.
Implicit calls to the broadcast constructor will select \code{\#1} if the condition in the
\code{explicit} specifier is \code{false}.
Otherwise, \code{\#1} is not part of the candidate set and \code{\#2} is called.
Thus, the condition on the \code{explicit} specifier determines whether the \code{consteval}
overload is chosen or not.

\bigskip
\begin{lstlisting}[style=Vc]
static_assert(      std::convertible_to<X<float>, V>); // different to status quo
static_assert(      std::convertible_to<float, V>);
static_assert(      std::convertible_to<short, V>);
static_assert(    really_convertible_to<short, V>);
static_assert(      std::convertible_to<int, V>);      // different to status quo
static_assert(not really_convertible_to<int, V>);

static_assert(    std::constructible_from<V, X<float>>);
static_assert(not std::constructible_from<V, X<short>>);
static_assert(    std::constructible_from<V, float>);
static_assert(    std::constructible_from<V, short>);
static_assert(    std::constructible_from<V, int>);

V f(int n, short m, std::reference_wrapper<int> l, std::reference_wrapper<float> f)
{
  V x = '\1'; // OK
  x = 1;      // OK (different to status quo)
  x = 0x5EAF00D; // ill-formed
  x = V(n);   // OK
  x = m;      // OK
  x = V(l);   // OK
  x = f;      // OK
  x = X<float>(); // ill-formed: static_assert failed (different reason to status quo)
  x = float(X<float>()); // OK (obvious)
  x = V(X<float>()); // OK
}
\end{lstlisting}

\subsection{More constrained \code{consteval} overload}\label{sec:consteval}
A viable alternative involves the removal of explicit conversions from arithmetic types to
\code{basic_vec}.
The \code{consteval} constructor is declared with additional constraints over the existing
constructor (satisfies \code{convertible_to}, \code{is_arithmetic_v}, and not value-preserving
conversion).
This way the \code{consteval} constructor is always chosen if the conversion of the given
(arithmetic) type to \code{value_type} is not value-preserving.
Otherwise, the \code{constexpr} overload is used.

Sketch:
\medskip\begin{lstlisting}[style=Vc]
template <class From, class To>
  concept @\exposconcept{simd-broadcast-arg}@ = constructible_from<To, From>;

template <class From, class To>
  concept @\exposconcept{simd-consteval-broadcast-arg}@
    = @\exposconcept{simd-broadcast-arg}@<From, To> && convertible_to<From, To>
        && !@\exposconcept{value-preserving-convertible-to}@<From, To>;

template <class T>
class basic_vec
{
public:
  template <@\exposconcept{simd-broadcast-arg}@<T> U>
    constexpr explicit(@\seebelow@) basic_vec(U&&); // #1
  template <@\exposconcept{simd-consteval-broadcast-arg}@<T> U>
    consteval basic_vec(U&&); // #2
};
\end{lstlisting}

Here, every explicit call to the broadcast constructor with an arithmetic type is equivalent to an
implicit conversion, since the \code{consteval} overload is viable and more constrained.
Every type with a value-preserving conversion to \code{T} will select \code{\#1} (because of the
constraint on \code{\#2}).
Every non-arithmetic type (notably, user-defined types with conversion operator to some arithmetic
type) will continue to work as today, since \code{\#2} is not viable.

\bigskip
\begin{lstlisting}[style=Vc]
static_assert(not   std::convertible_to<X<float>, V>); // equal to status quo / different to above
static_assert(      std::convertible_to<float, V>);
static_assert(      std::convertible_to<short, V>);
static_assert(    really_convertible_to<short, V>);
static_assert(      std::convertible_to<int, V>);      // different to status quo / equal to above
static_assert(not really_convertible_to<int, V>);

static_assert(    std::constructible_from<V, X<float>>);
static_assert(not std::constructible_from<V, X<short>>);
static_assert(    std::constructible_from<V, float>);
static_assert(    std::constructible_from<V, short>);
static_assert(    std::constructible_from<V, int>);

V f(int n, short m, std::reference_wrapper<int> l, std::reference_wrapper<float> f)
{
  V x = '\1'; // OK
  x = 1;      // OK (different to status quo / equal to above)
  x = 0x5EAF00D; // ill-formed
  x = V(n);   // ill-formed (different to both)
  x = m;      // OK
  x = V(l);   // OK
  x = f;      // OK
  x = X<float>(); // ill-formed: no match for operator= (no known conversion […])
  x = float(X<float>()); // OK (obvious)
  x = V(X<float>()); // OK
}
\end{lstlisting}

\subsection{How to handle bad value-preserving casts}\label{sec:exceptions}

The \code{consteval} broadcast overload needs to be ill-formed if the argument value cannot be
converted to the value type without changing the value.
This can be achieved via the mechanism used in \iref{simd.bit} for \code{bit_ceil}.
The constructor would spell out a precondition followed by \remarks An expression that violates
the precondition in the \expects element is not a core constant expression\iref{expr.const}.

The alternative that was mentioned before is to throw an exception (at compile time).
Since in basically all cases such an exception would not be caught at compile time, the program
becomes ill-formed.
The ability to catch the exception allowed me to hack up a \code{really_convertible_to} concept.
But otherwise, the utility of using an exception here seems fairly limited.
The main reason for using an exception is better diagnostics on ill-formed programs.
If we decide to add the \code{consteval} constructor for \CC{}26, then we might want to delay the
new exception type for \CC{}29, though.

\section{Differences}

\newcommand\good{\color[rgb]{0,.4,.2}\ding{52} }
\newcommand\bad{\color[rgb]{.5,0,0}\ding{56} }

Differences between the status quo and the two alternatives above:\\
\begin{tabular}{l|rrr}
  & status quo & \sect{sec:explicit} & \sect{sec:consteval} \\ \hline
  \code{convertible_to<X<float>, V>}     & \false & \true & \false \\
  \code{convertible_to<int, V>}          & \false & \true & \true \\
  \code{x = 1;}                          & \bad   & \good & \good \\
  \code{x = V(n);}                       & \good  & \good & \bad \\
\end{tabular}

\medskip
Note that \code{X<float>} is never implicitly convertible to \code{vec<float>}, so the solution in
\sect{sec:explicit} lies about that.
Also while some values of constant expressions of type \code{int} are convertible to
\code{vec<float>}, it is not true in general that \code{int} is convertible to \code{vec<float>}.

\section{Recommendation}

My recommendation is to go with the solution presented in \sect{sec:consteval} without a new
exception type (\sect{sec:exceptions}) for \CC{}26.
This would roll back a small part of a recent change done by \cite{P3430R3}.

Rationale for my preference:
\begin{enumerate}
  \item The explicit conversion from arithmetic types via broadcast constructor is significantly
    less important after implicit conversion from constant expressions becomes possible.
  \item The new \code{consteval} overload cannot be fully constrained in the solution presented in
    \sect{sec:explicit}, leading to incorrect answers on traits or in requires expressions.
  \item This should be part of \CC{}26 because it helps avoiding bugs in user code.
  \item A new exception type is not important enough to add it to \CC{}26 and it can easily be added
    later.
\end{enumerate}

\section{Implementation experience}
Both solutions (and a lot more variants that were discarded) have been implemented and tested in my
implementation.

\section{Proposed polls}

\wgPoll{Something needs to be done for \CC{}26. (If we do nothing, the design space is constrained
and \sect{sec:consteval} would be a breaking change.)}{&&&&}

\wgPoll{Add a \code{consteval} broadcast overload for value-preserving conversions to
\CC{}26.}{&&&&}

\noindent If the vote for \CC{}26 failed:\\
\wgPoll{Add a \code{consteval} broadcast overload for value-preserving conversions to
\CC{}29.}{&&&&}

\noindent Otherwise \emph{maybe} poll:\\
\wgPoll{It is better to add a \code{consteval} broadcast overload for \CC{}29 rather than
\CC{}26.}{&&&&}

\noindent Note: The next poll makes sense for \CC{}26, even if we only intend to add the new
overload for \CC{}29.\\
\wgPoll{The new constructor overload should be fully constrained, which requires the removal of
\code{explicit} (not value-preserving conversions) from the existing broadcast constructor.}{&&&&}

\wgPoll{Add a new exception type to \CC{}26 that is thrown from the new \code{consteval} broadcast
constructor.}{&&&&}

\wgPoll{Encourage a paper targeting \CC{}29 on a new exception type that is thrown from the new
\code{consteval} broadcast constructor.}{&&&&}

\clearpage
\section{Wording for \sect{sec:explicit}}

\subsection{Feature test macro}

In [version.syn] bump the \code{__cpp_lib_simd} version.

\subsection{Modify [simd.expos]}

In [simd.expos], insert:
\begin{wgText}[{[simd.expos]}]
\begin{codeblock}
template<class V, class T> using @\exposidnc{make-compatible-simd-t} = \seebelownc@; // \expos

@\wgAdd{template<class From, class To>}@
  @\wgAdd{\mbox{concept \defexposconceptnc{simd-consteval-broadcast-arg} = \exposconcept{explicitly-convertible-to}}<From, To>; \mbox{\rmfamily\itshape// \expos}}@

@\wgAdd{template<class From, class To>}@
  @\wgAdd{\mbox{concept \defexposconceptnc{simd-broadcast-arg} =}}@                                       @\wgAdd{\mbox{\rmfamily\itshape// \expos}}@
    @\wgAdd{\mbox{\exposconcept{simd-consteval-broadcast-arg}<From, To> \&\& true;}}@

template<class V>
  concept @\defexposconceptnc{simd-vec-type}@ =                                            // \expos
\end{codeblock}
\end{wgText}

\subsection{Modify [simd.overview]}

In [simd.overview], insert:
\begin{wgText}[{[simd.overview]}]
\begin{codeblock}
    // \iref{simd.ctor}, \tcode{basic_vec} constructors
    template<@\wgChange{class}{\mbox{\exposconcept{simd-broadcast-arg}}}@ U>
      constexpr explicit(@\seebelow@) basic_vec(U&& value) noexcept;
    @\wgAdd{template<\mbox{\exposconcept{simd-consteval-broadcast-arg}} U>}@
      @\wgAdd{consteval basic_vec(U\&\& x)}@
    template<class U, class UAbi>
      constexpr explicit(@\seebelow@) basic_vec(const basic_vec<U, UAbi>&) noexcept;
\end{codeblock}
\end{wgText}

\subsection{Modify [simd.ctor]}

\begin{wgText}[{[simd.ctor]}]
\begin{itemdecl}
template<@\wgChange{class}{\mbox{\exposconcept{simd-broadcast-arg}}}@ U> constexpr explicit(@\seebelow@) basic_vec(U&& value) noexcept;
\end{itemdecl}

\begin{itemdescr}
\pnum
Let \tcode{From} denote the type \tcode{remove_cvref_t<U>}.

\pnumRem
\wgRem{\constraints
\tcode{value_type} satisfies \mbox{\tcode{\libconcept{constructible_from}<U>}}.}

\pnum
\effects
Initializes each element to the value of the argument after conversion to
\tcode{value_type}.

\pnum
\remarks
The expression inside \tcode{explicit} evaluates to \tcode{false} if and only if
\tcode{U} satisfies \tcode{\libconcept{convertible_to}<value_type>}, and either
\begin{itemize}
 \item
   \tcode{From} is not an arithmetic type and does not satisfy
   \exposconcept{constexpr-wrapper-like},
 \item
   \tcode{From} is an arithmetic type and the conversion from \tcode{From} to
   \tcode{value_type} is value-preserving\iref{simd.general}, or
 \item
   \tcode{From} satisfies \exposconcept{constexpr-wrapper-like},
   \tcode{remove_const_t<decltype(From::value)>} is an arithmetic type, and
   \tcode{From::value} is representable by \tcode{value_type}.
\end{itemize}
\end{itemdescr}

\begin{itemdecl}
@\wgAdd{template<\mbox{\exposconcept{simd-consteval-broadcast-arg}} U> consteval basic_vec(U\&\& x)}@
\end{itemdecl}

\begin{itemdescr}
\pnumAdd
\wgAdd{\mandates}
\begin{itemize}
  \wgItemAdd\wgAdd{\tcode{remove_cvref_t<U>} is an arithmetic type,}
  \wgItemAdd\wgAdd{\tcode{U} satisfies \mbox{\tcode{\libconcept{convertible_to}<value_type>}},}
  \wgItemAdd\wgAdd{the conversion from \tcode{remove_cvref_t<U>} to \tcode{value_type} is not value-preserving, and}
  \wgItemAdd\wgAdd{either}
  \begin{itemize}
    \wgItemAdd\wgAdd{\tcode{common_type_t<U, value_type>} is \tcode{value_type},}
    \wgItemAdd\wgAdd{\tcode{value_type} is integral and \tcode{remove_cvref_t<U>} is \tcode{int}, or}
    \wgItemAdd\wgAdd{\tcode{value_type} satisfies \tcode{unsigned_integral} and \tcode{remove_cvref_t<U>} is \tcode{unsigned int}.}
  \end{itemize}
\end{itemize}

\pnumAdd
\wgAdd{%
\expects
The value of \tcode{x} is equal to the value of \tcode{x} after conversion to \tcode{value_type}.
}

\pnumAdd
\wgAdd{\effects}
\wgAdd{Initializes each element to the value of the argument after conversion to
\tcode{value_type}.}

\pnumAdd
\wgAdd{%
\remarks
An expression that violates the precondition in the \mbox{\expects} element is not a core constant
expression\mbox{\iref{expr.const}}.
}
\end{itemdescr}
\end{wgText}


\clearpage
\section{Wording}

\subsection{Feature test macro}

In [version.syn] bump the \code{__cpp_lib_simd} version.

\subsection{Modify [simd.expos]}

In [simd.expos], insert:
\begin{wgText}[{[simd.expos]}]
\begin{codeblock}
@\wgRem{template<class V, class T> using \exposid{make-compatible-simd-t} = \seebelow; \rmfamily\itshape// \expos}@

@\wgAdd{template<class From, class To>}@
  @\wgAdd{concept \defexposconceptnc{simd-consteval-broadcast-arg} = \seebelownc;}@                  @\wgAdd{\rmfamily\itshape// \expos}@

template<class V>
  concept @\defexposconceptnc{simd-vec-type}@ =                                            // \expos
\end{codeblock}
\end{wgText}

\subsection{Modify [simd.expos.defn]}

In [simd.expos.defn], change:
\begin{wgText}[{[simd.expos.defn]}]
\begin{itemdecl}
template<class T> using @\exposid{deduced-vec-t}@ = @\seebelow@;
\end{itemdecl}

\begin{itemdescr}
\pnum
Let \tcode{x} denote an lvalue of type \tcode{const T}.

\pnum
\tcode{\exposid{deduced-vec-t}<T>} is an alias for
\begin{itemize}
  \wgItemRem\color{black}
   \tcode{decltype(x + x)}, if the type of \tcode{x + x} is an enabled
   specialization of \tcode{basic_vec}\wgRem{; otherwise}
   %TODO: convertible_to<const T&, const decltype(x + x)&> ?
  \wgItemRem[\tcode{void}]\color{black}.
\end{itemize}
\end{itemdescr}

\begin{itemdecl}
@\wgRem{template<class V, class T> using \exposid{make-compatible-simd-t} = \seebelow;}@
\end{itemdecl}

\begin{itemdescr}
\pnumRem
\wgRem{Let \tcode{x} denote an lvalue of type \tcode{const T}.}

\pnumRem
\wgRem{\tcode{\exposid{make-compatible-simd-t}<V, T>} is an alias for}
\begin{itemize}
 \wgItemRem[
   \tcode{\exposid{deduced-vec-t}<T>}, if that type is not \tcode{void},
 otherwise]
 \wgItemRem[
 \tcode{vec<decltype(x + x), V::size()>}.]
\end{itemize}
\end{itemdescr}

\begin{itemdecl}
@\wgAdd{template<class From, class To> concept \defexposconceptnc{simd-consteval-broadcast-arg} = \seebelownc;}@
\end{itemdecl}

\begin{itemdescr}
\pnumAdd
\wgAdd{%
\exposconcept{simd-consteval-broadcast-arg} subsumes
\exposconcept{explicitly-convertible-to}.
}

\pnumAdd
\wgAdd{%
\tcode{From} satisfies
\tcode{\exposconcept{simd-consteval-broadcast-arg}<To>} only if}
\begin{itemize}
  \wgItemAdd\wgAdd{\tcode{remove_cvref_t<From>} is an arithmetic type,}
  \wgItemAdd\wgAdd{\tcode{From} satisfies \tcode{\libconcept{convertible_to}<To>},}
  \wgItemAdd\wgAdd{the conversion from \tcode{remove_cvref_t<From>} to \tcode{To} is not value-preserving, and}
  \wgItemAdd\wgAdd{either}
  \begin{itemize}
    \wgItemAdd\wgAdd{\tcode{common_type_t<From, To>} is \tcode{To},}
    \wgItemAdd\wgAdd{\tcode{To} is integral and \tcode{remove_cvref_t<From>} is \tcode{int}, or}
    \wgItemAdd\wgAdd{\tcode{To} satisfies \tcode{unsigned_integral} and \tcode{remove_cvref_t<From>} is \tcode{unsigned int}.}
  \end{itemize}
\end{itemize}
\end{itemdescr}
\end{wgText}

\subsection{Modify [simd.syn]}

In [simd.syn], change:
\begin{wgText}[{[simd.syn]}]
\begin{codeblock}
  template<size_t Bytes, class Abi, class T, class U>
    constexpr auto select(const basic_mask<Bytes, Abi>& c, const T& a, const U& b)
    noexcept -> decltype(@\exposid{simd-select-impl}@(c, a, b));

  // \iref{simd.math}, mathematical functions
  template<@\exposconcept{math-floating-point}@ V> constexpr @\exposid{deduced-vec-t}@<V> acos(const V& x);
  template<@\exposconcept{math-floating-point}@ V> constexpr @\exposid{deduced-vec-t}@<V> asin(const V& x);
  template<@\exposconcept{math-floating-point}@ V> constexpr @\exposid{deduced-vec-t}@<V> atan(const V& x);
  template<@\wgChange{class V0, class V1}{\exposid{math-floating-point} V}@>
    constexpr @\wgChange{\exposid{math-common-simd-t}<V0, V1>}{\exposid{deduced-vec-t}<V>}@ atan2(const V@\wgRem{0}@& y, const V@\wgRem{1}@& x);
  template<@\exposconcept{math-floating-point}@ V> constexpr @\exposid{deduced-vec-t}@<V> cos(const V& x);
  template<@\exposconcept{math-floating-point}@ V> constexpr @\exposid{deduced-vec-t}@<V> sin(const V& x);
  template<@\exposconcept{math-floating-point}@ V> constexpr @\exposid{deduced-vec-t}@<V> tan(const V& x);
  template<@\exposconcept{math-floating-point}@ V> constexpr @\exposid{deduced-vec-t}@<V> acosh(const V& x);
  template<@\exposconcept{math-floating-point}@ V> constexpr @\exposid{deduced-vec-t}@<V> asinh(const V& x);
  template<@\exposconcept{math-floating-point}@ V> constexpr @\exposid{deduced-vec-t}@<V> atanh(const V& x);
  template<@\exposconcept{math-floating-point}@ V> constexpr @\exposid{deduced-vec-t}@<V> cosh(const V& x);
  template<@\exposconcept{math-floating-point}@ V> constexpr @\exposid{deduced-vec-t}@<V> sinh(const V& x);
  template<@\exposconcept{math-floating-point}@ V> constexpr @\exposid{deduced-vec-t}@<V> tanh(const V& x);
  template<@\exposconcept{math-floating-point}@ V> constexpr @\exposid{deduced-vec-t}@<V> exp(const V& x);
  template<@\exposconcept{math-floating-point}@ V> constexpr @\exposid{deduced-vec-t}@<V> exp2(const V& x);
  template<@\exposconcept{math-floating-point}@ V> constexpr @\exposid{deduced-vec-t}@<V> expm1(const V& x);
  template<@\exposconcept{math-floating-point}@ V>
    constexpr @\exposid{deduced-vec-t}@<V>
      frexp(const V& value, rebind_t<int, @\exposid{deduced-vec-t}@<V>>* exp);
  template<@\exposconcept{math-floating-point}@ V>
    constexpr rebind_t<int, @\exposid{deduced-vec-t}@<V>> ilogb(const V& x);
  template<@\exposconcept{math-floating-point}@ V>
    constexpr @\exposid{deduced-vec-t}@<V> ldexp(const V& x, const rebind_t<int, @\exposid{deduced-vec-t}@<V>>& exp);
  template<@\exposconcept{math-floating-point}@ V> constexpr @\exposid{deduced-vec-t}@<V> log(const V& x);
  template<@\exposconcept{math-floating-point}@ V> constexpr @\exposid{deduced-vec-t}@<V> log10(const V& x);
  template<@\exposconcept{math-floating-point}@ V> constexpr @\exposid{deduced-vec-t}@<V> log1p(const V& x);
  template<@\exposconcept{math-floating-point}@ V> constexpr @\exposid{deduced-vec-t}@<V> log2(const V& x);
  template<@\exposconcept{math-floating-point}@ V> constexpr @\exposid{deduced-vec-t}@<V> logb(const V& x);
  template<class T, class Abi>
    constexpr basic_vec<T, Abi>
      modf(const type_identity_t<basic_vec<T, Abi>>& value, basic_vec<T, Abi>* iptr);
  template<@\exposconcept{math-floating-point}@ V>
    constexpr @\exposid{deduced-vec-t}@<V> scalbn(const V& x, const rebind_t<int, @\exposid{deduced-vec-t}@<V>>& n);
  template<@\exposconcept{math-floating-point}@ V>
    constexpr @\exposid{deduced-vec-t}@<V> scalbln(
      const V& x, const rebind_t<long int, @\exposid{deduced-vec-t}@<V>>& n);
  template<@\exposconcept{math-floating-point}@ V> constexpr @\exposid{deduced-vec-t}@<V> cbrt(const V& x);
  template<@\libconcept{signed_integral}@ T, class Abi>
    constexpr basic_vec<T, Abi> abs(const basic_vec<T, Abi>& j);
  template<@\exposconcept{math-floating-point}@ V> constexpr @\exposid{deduced-vec-t}@<V> abs(const V& j);
  template<@\exposconcept{math-floating-point}@ V> constexpr @\exposid{deduced-vec-t}@<V> fabs(const V& x);
  template<@\wgChange{class V0, class V1}{\exposid{math-floating-point} V}@>
    constexpr @\wgChange{\exposid{math-common-simd-t}<V0, V1>}{\exposid{deduced-vec-t}<V>}@ hypot(const V@\wgRem{0}@& x, const V@\wgRem{1}@& y);
  template<@\wgChange{class V0, class V1, class V2}{\exposid{math-floating-point} V}@>
    constexpr @\wgChange{\exposid{math-common-simd-t}<V0, V1, V2>}{\exposid{deduced-vec-t}<V>}@ hypot(const V@\wgRem{0}@& x, const V@\wgRem{1}@& y, const V@\wgRem{2}@& z);
  template<@\wgChange{class V0, class V1}{\exposid{math-floating-point} V}@>
    constexpr @\wgChange{\exposid{math-common-simd-t}<V0, V1>}{\exposid{deduced-vec-t}<V>}@ pow(const V@\wgRem{0}@& x, const V@\wgRem{1}@& y);
  template<@\exposconcept{math-floating-point}@ V> constexpr @\exposid{deduced-vec-t}@<V> sqrt(const V& x);
  template<@\exposconcept{math-floating-point}@ V> constexpr @\exposid{deduced-vec-t}@<V> erf(const V& x);
  template<@\exposconcept{math-floating-point}@ V> constexpr @\exposid{deduced-vec-t}@<V> erfc(const V& x);
  template<@\exposconcept{math-floating-point}@ V> constexpr @\exposid{deduced-vec-t}@<V> lgamma(const V& x);
  template<@\exposconcept{math-floating-point}@ V> constexpr @\exposid{deduced-vec-t}@<V> tgamma(const V& x);
  template<@\exposconcept{math-floating-point}@ V> constexpr @\exposid{deduced-vec-t}@<V> ceil(const V& x);
  template<@\exposconcept{math-floating-point}@ V> constexpr @\exposid{deduced-vec-t}@<V> floor(const V& x);
  template<@\exposconcept{math-floating-point}@ V> @\exposid{deduced-vec-t}@<V> nearbyint(const V& x);
  template<@\exposconcept{math-floating-point}@ V> @\exposid{deduced-vec-t}@<V> rint(const V& x);
  template<@\exposconcept{math-floating-point}@ V>
    rebind_t<long int, @\exposid{deduced-vec-t}@<V>> lrint(const V& x);
  template<@\exposconcept{math-floating-point}@ V>
    rebind_t<long long int, V> llrint(const @\exposid{deduced-vec-t}@<V>& x);
  template<@\exposconcept{math-floating-point}@ V>
    constexpr @\exposid{deduced-vec-t}@<V> round(const V& x);
  template<@\exposconcept{math-floating-point}@ V>
    constexpr rebind_t<long int, @\exposid{deduced-vec-t}@<V>> lround(const V& x);
  template<@\exposconcept{math-floating-point}@ V>
    constexpr rebind_t<long long int, @\exposid{deduced-vec-t}@<V>> llround(const V& x);
  template<@\exposconcept{math-floating-point}@ V>
    constexpr @\exposid{deduced-vec-t}@<V> trunc(const V& x);
  template<@\wgChange{class V0, class V1}{\exposid{math-floating-point} V}@>
    constexpr @\wgChange{\exposid{math-common-simd-t}<V0, V1>}{\exposid{deduced-vec-t}<V>}@ fmod(const V@\wgRem{0}@& x, const V@\wgRem{1}@& y);
  template<@\wgChange{class V0, class V1}{\exposid{math-floating-point} V}@>
    constexpr @\wgChange{\exposid{math-common-simd-t}<V0, V1>}{\exposid{deduced-vec-t}<V>}@ remainder(const V@\wgRem{0}@& x, const V@\wgRem{1}@& y);
  template<@\wgChange{class V0, class V1}{\exposid{math-floating-point} V}@>
    constexpr @\wgChange{\exposid{math-common-simd-t}<V0, V1>}{\exposid{deduced-vec-t}<V>}@
      remquo(const V@\wgRem{0}@& x, const V@\wgRem{1}@& y, rebind_t<int, @\wgChange{\exposid{math-common-simd-t}<V0, V1>}{\exposid{deduced-vec-t}<V>}@>* quo);
  template<@\wgChange{class V0, class V1}{\exposid{math-floating-point} V}@>
    constexpr @\wgChange{\exposid{math-common-simd-t}<V0, V1>}{\exposid{deduced-vec-t}<V>}@ copysign(const V@\wgRem{0}@& x, const V@\wgRem{1}@& y);
  template<@\wgChange{class V0, class V1}{\exposid{math-floating-point} V}@>
    constexpr @\wgChange{\exposid{math-common-simd-t}<V0, V1>}{\exposid{deduced-vec-t}<V>}@ nextafter(const V@\wgRem{0}@& x, const V@\wgRem{1}@& y);
  template<@\wgChange{class V0, class V1}{\exposid{math-floating-point} V}@>
    constexpr @\wgChange{\exposid{math-common-simd-t}<V0, V1>}{\exposid{deduced-vec-t}<V>}@ fdim(const V@\wgRem{0}@& x, const V@\wgRem{1}@& y);
  template<@\wgChange{class V0, class V1}{\exposid{math-floating-point} V}@>
    constexpr @\wgChange{\exposid{math-common-simd-t}<V0, V1>}{\exposid{deduced-vec-t}<V>}@ fmax(const V@\wgRem{0}@& x, const V@\wgRem{1}@& y);
  template<@\wgChange{class V0, class V1}{\exposid{math-floating-point} V}@>
    constexpr @\wgChange{\exposid{math-common-simd-t}<V0, V1>}{\exposid{deduced-vec-t}<V>}@ fmin(const V@\wgRem{0}@& x, const V@\wgRem{1}@& y);
  template<@\wgChange{class V0, class V1, class V2}{\exposid{math-floating-point} V}@>
    constexpr @\wgChange{\exposid{math-common-simd-t}<V0, V1, V2>}{\exposid{deduced-vec-t}<V>}@ fma(const V@\wgRem{0}@& x, const V@\wgRem{1}@& y, const V@\wgRem{2}@& z);
  template<@\wgChange{class V0, class V1, class V2}{\exposid{math-floating-point} V}@>
    constexpr @\wgChange{\exposid{math-common-simd-t}<V0, V1, V2>}{\exposid{deduced-vec-t}<V>}@
      lerp(const V@\wgRem{0}@& a, const V@\wgRem{1}@& b, const V@\wgRem{2}@& t) noexcept;
  template<@\exposconcept{math-floating-point}@ V>
    constexpr rebind_t<int, @\exposid{deduced-vec-t}@<V>> fpclassify(const V& x);
  template<@\exposconcept{math-floating-point}@ V>
    constexpr typename @\exposid{deduced-vec-t}@<V>::mask_type isfinite(const V& x);
  template<@\exposconcept{math-floating-point}@ V>
    constexpr typename @\exposid{deduced-vec-t}@<V>::mask_type isinf(const V& x);
  template<@\exposconcept{math-floating-point}@ V>
    constexpr typename @\exposid{deduced-vec-t}@<V>::mask_type isnan(const V& x);
  template<@\exposconcept{math-floating-point}@ V>
    constexpr typename @\exposid{deduced-vec-t}@<V>::mask_type isnormal(const V& x);
  template<@\exposconcept{math-floating-point}@ V>
    constexpr typename @\exposid{deduced-vec-t}@<V>::mask_type signbit(const V& x);
  template<@\wgChange{class V0, class V1}{\exposid{math-floating-point} V}@>
    constexpr typename @\wgChange{\exposid{math-common-simd-t}<V0, V1>}{\exposid{deduced-vec-t}<V>}@::mask_type
      isgreater(const V@\wgRem{0}@& x, const V@\wgRem{1}@& y);
  template<@\wgChange{class V0, class V1}{\exposid{math-floating-point} V}@>
    constexpr typename @\wgChange{\exposid{math-common-simd-t}<V0, V1>}{\exposid{deduced-vec-t}<V>}@::mask_type
      isgreaterequal(const V@\wgRem{0}@& x, const V@\wgRem{1}@& y);
  template<@\wgChange{class V0, class V1}{\exposid{math-floating-point} V}@>
    constexpr typename @\wgChange{\exposid{math-common-simd-t}<V0, V1>}{\exposid{deduced-vec-t}<V>}@::mask_type
      isless(const V@\wgRem{0}@& x, const V@\wgRem{1}@& y);
  template<@\wgChange{class V0, class V1}{\exposid{math-floating-point} V}@>
    constexpr typename @\wgChange{\exposid{math-common-simd-t}<V0, V1>}{\exposid{deduced-vec-t}<V>}@::mask_type
      islessequal(const V@\wgRem{0}@& x, const V@\wgRem{1}@& y);
  template<@\wgChange{class V0, class V1}{\exposid{math-floating-point} V}@>
    constexpr typename @\wgChange{\exposid{math-common-simd-t}<V0, V1>}{\exposid{deduced-vec-t}<V>}@::mask_type
      islessgreater(const V@\wgRem{0}@& x, const V@\wgRem{1}@& y);
  template<@\wgChange{class V0, class V1}{\exposid{math-floating-point} V}@>
    constexpr typename @\wgChange{\exposid{math-common-simd-t}<V0, V1>}{\exposid{deduced-vec-t}<V>}@::mask_type
      isunordered(const V@\wgRem{0}@& x, const V@\wgRem{1}@& y);
  template<@\exposconcept{math-floating-point}@ V>
    @\exposid{deduced-vec-t}@<V> assoc_laguerre(const rebind_t<unsigned, @\exposid{deduced-vec-t}@<V>>& n,
                                    const rebind_t<unsigned, @\exposid{deduced-vec-t}@<V>>& m, const V& x);
  template<@\exposconcept{math-floating-point}@ V>
    @\exposid{deduced-vec-t}@<V> assoc_legendre(const rebind_t<unsigned, @\exposid{deduced-vec-t}@<V>>& l,
                                    const rebind_t<unsigned, @\exposid{deduced-vec-t}@<V>>& m, const V& x);
  template<@\wgChange{class V0, class V1}{\exposid{math-floating-point} V}@>
    @\wgChange{\exposid{math-common-simd-t}<V0, V1>}{\exposid{deduced-vec-t}<V>}@ beta(const V@\wgRem{0}@& x, const V@\wgRem{1}@& y);
  template<@\exposconcept{math-floating-point}@ V> @\exposid{deduced-vec-t}@<V> comp_ellint_1(const V& k);
  template<@\exposconcept{math-floating-point}@ V> @\exposid{deduced-vec-t}@<V> comp_ellint_2(const V& k);
  template<@\wgChange{class V0, class V1}{\exposid{math-floating-point} V}@>
    @\wgChange{\exposid{math-common-simd-t}<V0, V1>}{\exposid{deduced-vec-t}<V>}@ comp_ellint_3(const V@\wgRem{0}@& k, const V@\wgRem{1}@& nu);
  template<@\wgChange{class V0, class V1}{\exposid{math-floating-point} V}@>
    @\wgChange{\exposid{math-common-simd-t}<V0, V1>}{\exposid{deduced-vec-t}<V>}@ cyl_bessel_i(const V@\wgRem{0}@& nu, const V@\wgRem{1}@& x);
  template<@\wgChange{class V0, class V1}{\exposid{math-floating-point} V}@>
    @\wgChange{\exposid{math-common-simd-t}<V0, V1>}{\exposid{deduced-vec-t}<V>}@ cyl_bessel_j(const V@\wgRem{0}@& nu, const V@\wgRem{1}@& x);
  template<@\wgChange{class V0, class V1}{\exposid{math-floating-point} V}@>
    @\wgChange{\exposid{math-common-simd-t}<V0, V1>}{\exposid{deduced-vec-t}<V>}@ cyl_bessel_k(const V@\wgRem{0}@& nu, const V@\wgRem{1}@& x);
  template<@\wgChange{class V0, class V1}{\exposid{math-floating-point} V}@>
    @\wgChange{\exposid{math-common-simd-t}<V0, V1>}{\exposid{deduced-vec-t}<V>}@ cyl_neumann(const V@\wgRem{0}@& nu, const V@\wgRem{1}@& x);
  template<@\wgChange{class V0, class V1}{\exposid{math-floating-point} V}@>
    @\wgChange{\exposid{math-common-simd-t}<V0, V1>}{\exposid{deduced-vec-t}<V>}@ ellint_1(const V@\wgRem{0}@& k, const V@\wgRem{1}@& phi);
  template<@\wgChange{class V0, class V1}{\exposid{math-floating-point} V}@>
    @\wgChange{\exposid{math-common-simd-t}<V0, V1>}{\exposid{deduced-vec-t}<V>}@ ellint_2(const V@\wgRem{0}@& k, const V@\wgRem{1}@& phi);
  template<@\wgChange{class V0, class V1, class V2}{\exposid{math-floating-point} V}@>
    @\wgChange{\exposid{math-common-simd-t}<V0, V1, V2>}{\exposid{deduced-vec-t}<V>}@ ellint_3(const V@\wgRem{0}@& k, const V@\wgRem{1}@& nu, const V@\wgRem{2}@& phi);
  template<@\exposconcept{math-floating-point}@ V> @\exposid{deduced-vec-t}@<V> expint(const V& x);
  template<@\exposconcept{math-floating-point}@ V>
    @\exposid{deduced-vec-t}@<V> hermite(const rebind_t<unsigned, @\exposid{deduced-vec-t}@<V>>& n, const V& x);
  template<@\exposconcept{math-floating-point}@ V>
    @\exposid{deduced-vec-t}@<V> laguerre(const rebind_t<unsigned, @\exposid{deduced-vec-t}@<V>>& n, const V& x);
  template<@\exposconcept{math-floating-point}@ V>
    @\exposid{deduced-vec-t}@<V> legendre(const rebind_t<unsigned, @\exposid{deduced-vec-t}@<V>>& l, const V& x);
  template<@\exposconcept{math-floating-point}@ V>
    @\exposid{deduced-vec-t}@<V> riemann_zeta(const V& x);
  template<@\exposconcept{math-floating-point}@ V>
    @\exposid{deduced-vec-t}@<V> sph_bessel(
      const rebind_t<unsigned, @\exposid{deduced-vec-t}@<V>>& n, const V& x);
  template<@\exposconcept{math-floating-point}@ V>
    @\exposid{deduced-vec-t}@<V> sph_legendre(const rebind_t<unsigned, @\exposid{deduced-vec-t}@<V>>& l,
      const rebind_t<unsigned, @\exposid{deduced-vec-t}@<V>>& m, const V& theta);
  template<@\exposconcept{math-floating-point}@ V>
    @\exposid{deduced-vec-t}@<V>
      sph_neumann(const rebind_t<unsigned, @\exposid{deduced-vec-t}@<V>>& n, const V& x);

  @\wgAdd{template<\exposconcept{math-floating-point} V>}@
    @\wgAdd{constexpr \exposid{deduced-vec-t}<V> fmod(const \exposid{deduced-vec-t}<V>\& x, const V\& y);}@
  @\wgAdd{template<\exposconcept{math-floating-point} V>}@
    @\wgAdd{constexpr \exposid{deduced-vec-t}<V> fmod(const V\& x, const \exposid{deduced-vec-t}<V>\& y);}@
  @\wgAdd{template<\exposconcept{math-floating-point} V>}@
    @\wgAdd{constexpr \exposid{deduced-vec-t}<V> remainder(const \exposid{deduced-vec-t}<V>\& x, const V\& y);}@
  @\wgAdd{template<\exposconcept{math-floating-point} V>}@
    @\wgAdd{constexpr \exposid{deduced-vec-t}<V> remainder(const V\& x, const \exposid{deduced-vec-t}<V>\& y);}@
  @\wgAdd{template<\exposconcept{math-floating-point} V>}@
    @\wgAdd{constexpr \exposid{deduced-vec-t}<V> copysign(const \exposid{deduced-vec-t}<V>\& x, const V\& y);}@
  @\wgAdd{template<\exposconcept{math-floating-point} V>}@
    @\wgAdd{constexpr \exposid{deduced-vec-t}<V> copysign(const V\& x, const \exposid{deduced-vec-t}<V>\& y);}@
  @\wgAdd{template<\exposconcept{math-floating-point} V>}@
    @\wgAdd{constexpr \exposid{deduced-vec-t}<V> nextafter(const \exposid{deduced-vec-t}<V>\& x, const V\& y);}@
  @\wgAdd{template<\exposconcept{math-floating-point} V>}@
    @\wgAdd{constexpr \exposid{deduced-vec-t}<V> nextafter(const V\& x, const \exposid{deduced-vec-t}<V>\& y);}@
  @\wgAdd{template<\exposconcept{math-floating-point} V>}@
    @\wgAdd{constexpr \exposid{deduced-vec-t}<V> fdim(const \exposid{deduced-vec-t}<V>\& x, const V\& y);}@
  @\wgAdd{template<\exposconcept{math-floating-point} V>}@
    @\wgAdd{constexpr \exposid{deduced-vec-t}<V> fdim(const V\& x, const \exposid{deduced-vec-t}<V>\& y);}@
  @\wgAdd{template<\exposconcept{math-floating-point} V>}@
    @\wgAdd{constexpr \exposid{deduced-vec-t}<V> fmax(const \exposid{deduced-vec-t}<V>\& x, const V\& y);}@
  @\wgAdd{template<\exposconcept{math-floating-point} V>}@
    @\wgAdd{constexpr \exposid{deduced-vec-t}<V> fmax(const V\& x, const \exposid{deduced-vec-t}<V>\& y);}@
  @\wgAdd{template<\exposconcept{math-floating-point} V>}@
    @\wgAdd{constexpr \exposid{deduced-vec-t}<V> fmin(const \exposid{deduced-vec-t}<V>\& x, const V\& y);}@
  @\wgAdd{template<\exposconcept{math-floating-point} V>}@
    @\wgAdd{constexpr \exposid{deduced-vec-t}<V> fmin(const V\& x, const \exposid{deduced-vec-t}<V>\& y);}@
  @\wgAdd{template<\exposconcept{math-floating-point} V>}@
    @\wgAdd{constexpr \exposid{deduced-vec-t}<V> isgreater(const \exposid{deduced-vec-t}<V>\& x, const V\& y);}@
  @\wgAdd{template<\exposconcept{math-floating-point} V>}@
    @\wgAdd{constexpr \exposid{deduced-vec-t}<V> isgreater(const V\& x, const \exposid{deduced-vec-t}<V>\& y);}@
  @\wgAdd{template<\exposconcept{math-floating-point} V>}@
    @\wgAdd{constexpr \exposid{deduced-vec-t}<V> islessequal(const \exposid{deduced-vec-t}<V>\& x, const V\& y);}@
  @\wgAdd{template<\exposconcept{math-floating-point} V>}@
    @\wgAdd{constexpr \exposid{deduced-vec-t}<V> islessequal(const V\& x, const \exposid{deduced-vec-t}<V>\& y);}@
  @\wgAdd{template<\exposconcept{math-floating-point} V>}@
    @\wgAdd{constexpr \exposid{deduced-vec-t}<V> islessgreater(const \exposid{deduced-vec-t}<V>\& x, const V\& y);}@
  @\wgAdd{template<\exposconcept{math-floating-point} V>}@
    @\wgAdd{constexpr \exposid{deduced-vec-t}<V> islessgreater(const V\& x, const \exposid{deduced-vec-t}<V>\& y);}@
  @\wgAdd{template<\exposconcept{math-floating-point} V>}@
    @\wgAdd{constexpr \exposid{deduced-vec-t}<V> isunordered(const \exposid{deduced-vec-t}<V>\& x, const V\& y);}@
  @\wgAdd{template<\exposconcept{math-floating-point} V>}@
    @\wgAdd{constexpr \exposid{deduced-vec-t}<V> isunordered(const V\& x, const \exposid{deduced-vec-t}<V>\& y);}@
  @\wgAdd{template<\exposconcept{math-floating-point} V>}@
    @\wgAdd{constexpr \exposid{deduced-vec-t}<V> atan2(const \exposid{deduced-vec-t}<V>\& x, const V\& y);}@
  @\wgAdd{template<\exposconcept{math-floating-point} V>}@
    @\wgAdd{constexpr \exposid{deduced-vec-t}<V> atan2(const V\& x, const \exposid{deduced-vec-t}<V>\& y);}@
  @\wgAdd{template<\exposconcept{math-floating-point} V>}@
    @\wgAdd{constexpr \exposid{deduced-vec-t}<V> hypot(const \exposid{deduced-vec-t}<V>\& x, const V\& y);}@
  @\wgAdd{template<\exposconcept{math-floating-point} V>}@
    @\wgAdd{constexpr \exposid{deduced-vec-t}<V> hypot(const V\& x, const \exposid{deduced-vec-t}<V>\& y);}@
  @\wgAdd{template<\exposconcept{math-floating-point} V>}@
    @\wgAdd{constexpr \exposid{deduced-vec-t}<V> pow(const \exposid{deduced-vec-t}<V>\& x, const V\& y);}@
  @\wgAdd{template<\exposconcept{math-floating-point} V>}@
    @\wgAdd{constexpr \exposid{deduced-vec-t}<V> pow(const V\& x, const \exposid{deduced-vec-t}<V>\& y);}@
  @\wgAdd{template<\exposconcept{math-floating-point} V>}@
    @\wgAdd{constexpr \exposid{deduced-vec-t}<V> beta(const \exposid{deduced-vec-t}<V>\& x, const V\& y);}@
  @\wgAdd{template<\exposconcept{math-floating-point} V>}@
    @\wgAdd{constexpr \exposid{deduced-vec-t}<V> beta(const V\& x, const \exposid{deduced-vec-t}<V>\& y);}@
  @\wgAdd{template<\exposconcept{math-floating-point} V>}@
    @\wgAdd{constexpr \exposid{deduced-vec-t}<V> comp_ellint_3(const \exposid{deduced-vec-t}<V>\& x, const V\& y);}@
  @\wgAdd{template<\exposconcept{math-floating-point} V>}@
    @\wgAdd{constexpr \exposid{deduced-vec-t}<V> comp_ellint_3(const V\& x, const \exposid{deduced-vec-t}<V>\& y);}@
  @\wgAdd{template<\exposconcept{math-floating-point} V>}@
    @\wgAdd{constexpr \exposid{deduced-vec-t}<V> cyl_bessel_i(const \exposid{deduced-vec-t}<V>\& x, const V\& y);}@
  @\wgAdd{template<\exposconcept{math-floating-point} V>}@
    @\wgAdd{constexpr \exposid{deduced-vec-t}<V> cyl_bessel_i(const V\& x, const \exposid{deduced-vec-t}<V>\& y);}@
  @\wgAdd{template<\exposconcept{math-floating-point} V>}@
    @\wgAdd{constexpr \exposid{deduced-vec-t}<V> cyl_bessel_j(const \exposid{deduced-vec-t}<V>\& x, const V\& y);}@
  @\wgAdd{template<\exposconcept{math-floating-point} V>}@
    @\wgAdd{constexpr \exposid{deduced-vec-t}<V> cyl_bessel_j(const V\& x, const \exposid{deduced-vec-t}<V>\& y);}@
  @\wgAdd{template<\exposconcept{math-floating-point} V>}@
    @\wgAdd{constexpr \exposid{deduced-vec-t}<V> cyl_bessel_k(const \exposid{deduced-vec-t}<V>\& x, const V\& y);}@
  @\wgAdd{template<\exposconcept{math-floating-point} V>}@
    @\wgAdd{constexpr \exposid{deduced-vec-t}<V> cyl_bessel_k(const V\& x, const \exposid{deduced-vec-t}<V>\& y);}@
  @\wgAdd{template<\exposconcept{math-floating-point} V>}@
    @\wgAdd{constexpr \exposid{deduced-vec-t}<V> cyl_neumann(const \exposid{deduced-vec-t}<V>\& x, const V\& y);}@
  @\wgAdd{template<\exposconcept{math-floating-point} V>}@
    @\wgAdd{constexpr \exposid{deduced-vec-t}<V> cyl_neumann(const V\& x, const \exposid{deduced-vec-t}<V>\& y);}@
  @\wgAdd{template<\exposconcept{math-floating-point} V>}@
    @\wgAdd{constexpr \exposid{deduced-vec-t}<V> ellint_1(const \exposid{deduced-vec-t}<V>\& x, const V\& y);}@
  @\wgAdd{template<\exposconcept{math-floating-point} V>}@
    @\wgAdd{constexpr \exposid{deduced-vec-t}<V> ellint_1(const V\& x, const \exposid{deduced-vec-t}<V>\& y);}@
  @\wgAdd{template<\exposconcept{math-floating-point} V>}@
    @\wgAdd{constexpr \exposid{deduced-vec-t}<V> ellint_2(const \exposid{deduced-vec-t}<V>\& x, const V\& y);}@
  @\wgAdd{template<\exposconcept{math-floating-point} V>}@
    @\wgAdd{constexpr \exposid{deduced-vec-t}<V> ellint_2(const V\& x, const \exposid{deduced-vec-t}<V>\& y);}@
  @\wgAdd{template<\exposconcept{math-floating-point} V>}@
    @\wgAdd{constexpr \exposid{deduced-vec-t}<V>}@
      @\wgAdd{remquo(const \exposid{deduced-vec-t}<V>\& x, const V\& y, rebind_t<int, \exposid{deduced-vec-t}<V>>* quo);}@
  @\wgAdd{template<\exposconcept{math-floating-point} V>}@
    @\wgAdd{constexpr \exposid{deduced-vec-t}<V>}@
      @\wgAdd{remquo(const V\& x, const \exposid{deduced-vec-t}<V>\& y, rebind_t<int, \exposid{deduced-vec-t}<V>>* quo);}@
  @\wgAdd{template<\exposconcept{math-floating-point} V>}@
    @\wgAdd{constexpr \exposid{deduced-vec-t}<V> fma(const \exposid{deduced-vec-t}<V>\& x, const V\& y, const V\& z);}@
  @\wgAdd{template<\exposconcept{math-floating-point} V>}@
    @\wgAdd{constexpr \exposid{deduced-vec-t}<V> fma(const V\& x, const \exposid{deduced-vec-t}<V>\& y, const V\& z);}@
  @\wgAdd{template<\exposconcept{math-floating-point} V>}@
    @\wgAdd{constexpr \exposid{deduced-vec-t}<V> fma(const V\& x, const V\& y, const \exposid{deduced-vec-t}<V>\& z);}@
  @\wgAdd{template<\exposconcept{math-floating-point} V>}@
    @\wgAdd{constexpr \exposid{deduced-vec-t}<V> fma(const \exposid{deduced-vec-t}<V>\& x, const \exposid{deduced-vec-t}<V>\& y}@
                                   @\wgAdd{const V\& z);}@
  @\wgAdd{template<\exposconcept{math-floating-point} V>}@
    @\wgAdd{constexpr \exposid{deduced-vec-t}<V> fma(const \exposid{deduced-vec-t}<V>\& x, const V\& y}@
                                   @\wgAdd{const \exposid{deduced-vec-t}<V>\& z);}@
  @\wgAdd{template<\exposconcept{math-floating-point} V>}@
    @\wgAdd{constexpr \exposid{deduced-vec-t}<V> fma(const V\& x, const \exposid{deduced-vec-t}<V>\& y}@
                                   @\wgAdd{const \exposid{deduced-vec-t}<V>\& z);}@
  @\wgAdd{template<\exposconcept{math-floating-point} V>}@
    @\wgAdd{constexpr \exposid{deduced-vec-t}<V> hypot(const \exposid{deduced-vec-t}<V>\& x, const V\& y, const V\& z);}@
  @\wgAdd{template<\exposconcept{math-floating-point} V>}@
    @\wgAdd{constexpr \exposid{deduced-vec-t}<V> hypot(const V\& x, const \exposid{deduced-vec-t}<V>\& y, const V\& z);}@
  @\wgAdd{template<\exposconcept{math-floating-point} V>}@
    @\wgAdd{constexpr \exposid{deduced-vec-t}<V> hypot(const V\& x, const V\& y, const \exposid{deduced-vec-t}<V>\& z);}@
  @\wgAdd{template<\exposconcept{math-floating-point} V>}@
    @\wgAdd{constexpr \exposid{deduced-vec-t}<V> hypot(const \exposid{deduced-vec-t}<V>\& x, const \exposid{deduced-vec-t}<V>\& y}@
                                     @\wgAdd{const V\& z);}@
  @\wgAdd{template<\exposconcept{math-floating-point} V>}@
    @\wgAdd{constexpr \exposid{deduced-vec-t}<V> hypot(const \exposid{deduced-vec-t}<V>\& x, const V\& y}@
                                     @\wgAdd{const \exposid{deduced-vec-t}<V>\& z);}@
  @\wgAdd{template<\exposconcept{math-floating-point} V>}@
    @\wgAdd{constexpr \exposid{deduced-vec-t}<V> hypot(const V\& x, const \exposid{deduced-vec-t}<V>\& y}@
                                     @\wgAdd{const \exposid{deduced-vec-t}<V>\& z);}@
  @\wgAdd{template<\exposconcept{math-floating-point} V>}@
    @\wgAdd{constexpr \exposid{deduced-vec-t}<V> lerp(const \exposid{deduced-vec-t}<V>\& x, const V\& y, const V\& z);}@
  @\wgAdd{template<\exposconcept{math-floating-point} V>}@
    @\wgAdd{constexpr \exposid{deduced-vec-t}<V> lerp(const V\& x, const \exposid{deduced-vec-t}<V>\& y, const V\& z);}@
  @\wgAdd{template<\exposconcept{math-floating-point} V>}@
    @\wgAdd{constexpr \exposid{deduced-vec-t}<V> lerp(const V\& x, const V\& y, const \exposid{deduced-vec-t}<V>\& z);}@
  @\wgAdd{template<\exposconcept{math-floating-point} V>}@
    @\wgAdd{constexpr \exposid{deduced-vec-t}<V> lerp(const \exposid{deduced-vec-t}<V>\& x, const \exposid{deduced-vec-t}<V>\& y}@
                                    @\wgAdd{const V\& z);}@
  @\wgAdd{template<\exposconcept{math-floating-point} V>}@
    @\wgAdd{constexpr \exposid{deduced-vec-t}<V> lerp(const \exposid{deduced-vec-t}<V>\& x, const V\& y}@
                                    @\wgAdd{const \exposid{deduced-vec-t}<V>\& z);}@
  @\wgAdd{template<\exposconcept{math-floating-point} V>}@
    @\wgAdd{constexpr \exposid{deduced-vec-t}<V> lerp(const V\& x, const \exposid{deduced-vec-t}<V>\& y}@
                                    @\wgAdd{const \exposid{deduced-vec-t}<V>\& z);}@
  @\wgAdd{template<\exposconcept{math-floating-point} V>}@
    @\wgAdd{constexpr \exposid{deduced-vec-t}<V> ellint_3(const \exposid{deduced-vec-t}<V>\& x, const V\& y, const V\& z);}@
  @\wgAdd{template<\exposconcept{math-floating-point} V>}@
    @\wgAdd{constexpr \exposid{deduced-vec-t}<V> ellint_3(const V\& x, const \exposid{deduced-vec-t}<V>\& y, const V\& z);}@
  @\wgAdd{template<\exposconcept{math-floating-point} V>}@
    @\wgAdd{constexpr \exposid{deduced-vec-t}<V> ellint_3(const V\& x, const V\& y, const \exposid{deduced-vec-t}<V>\& z);}@
  @\wgAdd{template<\exposconcept{math-floating-point} V>}@
    @\wgAdd{constexpr \exposid{deduced-vec-t}<V> ellint_3(const \exposid{deduced-vec-t}<V>\& x, const \exposid{deduced-vec-t}<V>\& y,}@
                                        @\wgAdd{const V\& z);}@
  @\wgAdd{template<\exposconcept{math-floating-point} V>}@
    @\wgAdd{constexpr \exposid{deduced-vec-t}<V> ellint_3(const \exposid{deduced-vec-t}<V>\& x, const V\& y,}@
                                        @\wgAdd{const \exposid{deduced-vec-t}<V>\& z);}@
  @\wgAdd{template<\exposconcept{math-floating-point} V>}@
    @\wgAdd{constexpr \exposid{deduced-vec-t}<V> ellint_3(const V\& x, const \exposid{deduced-vec-t}<V>\& y,}@
                                        @\wgAdd{const \exposid{deduced-vec-t}<V>\& z);}@

  // \iref{simd.bit}, bit manipulation
  template<@\exposconcept{simd-vec-type}@ V> constexpr V byteswap(const V& v) noexcept;
\end{codeblock}
\end{wgText}

\subsection{Modify [simd.overview]}

In [simd.overview], insert:
\begin{wgText}[{[simd.overview]}]
\begin{codeblock}
    // \iref{simd.ctor}, \tcode{basic_vec} constructors
    template<@\wgChange{class}{\exposconcept{explicitly-convertible-to}<value_type>}@ U>
      constexpr explicit(@\seebelow@) basic_vec(U&& value) noexcept;
    @\wgAdd{template<\exposconcept{simd-consteval-broadcast-arg}<value_type> U>}@
      @\wgAdd{consteval basic_vec(U\&\& x)}@
    template<class U, class UAbi>
      constexpr explicit(@\seebelow@) basic_vec(const basic_vec<U, UAbi>&) noexcept;
\end{codeblock}
\end{wgText}

\subsection{Modify [simd.ctor]}

\begin{wgText}[{[simd.ctor]}]
\begin{itemdecl}
template<@\wgChange{class}{\exposconcept{explicitly-convertible-to}<value_type>}@ U>
  constexpr explicit(@\seebelow@) basic_vec(U&& value) noexcept;
\end{itemdecl}

\begin{itemdescr}
\pnum
Let \tcode{From} denote the type \tcode{remove_cvref_t<U>}.

\pnumRem
\wgRem{\constraints
\tcode{value_type} satisfies \tcode{\libconcept{constructible_from}<U>}.}

\pnum
\effects
Initializes each element to the value of the argument after conversion to
\tcode{value_type}.

\pnum
\remarks
The expression inside \tcode{explicit} evaluates to \tcode{false} if and only if
\tcode{U} satisfies \tcode{\libconcept{convertible_to}<value_type>}, and either
\begin{itemize}
 \item
   \tcode{From} is not an arithmetic type and does not satisfy
   \exposconcept{constexpr-wrapper-like},
 \item
   \tcode{From} is an arithmetic type and the conversion from \tcode{From} to
   \tcode{value_type} is value-preserving\iref{simd.general}, or
 \item
   \tcode{From} satisfies \exposconcept{constexpr-wrapper-like},
   \tcode{remove_const_t<decltype(From::value)>} is an arithmetic type, and
   \tcode{From::value} is representable by \tcode{value_type}.
\end{itemize}
\end{itemdescr}

\begin{itemdecl}
@\wgAdd{template<\exposconcept{simd-consteval-broadcast-arg}<value_type> U> consteval basic_vec(U\&\& x)}@
\end{itemdecl}

\begin{itemdescr}
\pnumAdd
\wgAdd{%
\expects
The value of \tcode{x} is equal to the value of \tcode{x} after conversion to \tcode{value_type}.
}

\pnumAdd
\wgAdd{\effects}
\wgAdd{Initializes each element to the value of the argument after conversion to
\tcode{value_type}.}

\pnumAdd
\wgAdd{%
\remarks
An expression that violates the precondition in the \expects element is not a core constant
expression\iref{expr.const}.
}
\end{itemdescr}
\end{wgText}

\subsection{Modify [simd.math]}

\begin{wgText}[{[simd.math]}]
\begin{itemdecl}
template<@\exposconcept{math-floating-point}@ V>
  constexpr rebind_t<int, @\exposid{deduced-vec-t}@<V>> ilogb(const V& x);
@\wgRem{template<\exposconcept{math-floating-point} V>}@
  @\wgRem{constexpr \exposid{deduced-vec-t}<V> ldexp(const V\& x, const rebind_t<int, \exposid{deduced-vec-t}<V>}\wgRem{>\& exp);}@
@\wgRem{template<\exposconcept{math-floating-point} V>}@
  @\wgRem{constexpr \exposid{deduced-vec-t}<V> scalbn(const V\& x, const rebind_t<int, \exposid{deduced-vec-t}<V>}\wgRem{>\& n);}@
@\wgRem{template<\exposconcept{math-floating-point} V>}@
  @\wgRem{constexpr \exposid{deduced-vec-t}<V>}@
    @\wgRem{scalbln(const V\& x, const rebind_t<long int, \exposid{deduced-vec-t}<V>}\wgRem{>\& n);}@
@\wgRem{template<\libconcept{signed_integral} T, class Abi>}@
  @\wgRem{constexpr basic_vec<T, Abi> abs(const basic_vec<T, Abi>\& j);}@
template<@\exposconcept{math-floating-point}@ V>
  constexpr @\exposid{deduced-vec-t}@<V> abs(const V& j);
template<@\exposconcept{math-floating-point}@ V>
  constexpr @\exposid{deduced-vec-t}@<V> fabs(const V& x);
template<@\exposconcept{math-floating-point}@ V>
  constexpr @\exposid{deduced-vec-t}@<V> ceil(const V& x);
template<@\exposconcept{math-floating-point}@ V>
  constexpr @\exposid{deduced-vec-t}@<V> floor(const V& x);
template<@\exposconcept{math-floating-point}@ V>
  @\exposid{deduced-vec-t}@<V> nearbyint(const V& x);
template<@\exposconcept{math-floating-point}@ V>
  @\exposid{deduced-vec-t}@<V> rint(const V& x);
template<@\exposconcept{math-floating-point}@ V>
  rebind_t<long int, @\exposid{deduced-vec-t}@<V>> lrint(const V& x);
template<@\exposconcept{math-floating-point}@ V>
  rebind_t<long long int, @\exposid{deduced-vec-t}@<V>> llrint(const V& x);
template<@\exposconcept{math-floating-point}@ V>
  constexpr @\exposid{deduced-vec-t}@<V> round(const V& x);
template<@\exposconcept{math-floating-point}@ V>
  constexpr rebind_t<long int, @\exposid{deduced-vec-t}@<V>> lround(const V& x);
template<@\exposconcept{math-floating-point}@ V>
  constexpr rebind_t<long long int, @\exposid{deduced-vec-t}@<V>> llround(const V& x);
template<@\wgChange{class V0, class V1}{\exposid{math-floating-point} V}@>
  constexpr @\wgChange{\exposid{math-common-simd-t}<V0, V1>}{\exposid{deduced-vec-t}<V>}@ fmod(const V@\wgRem{0}@& x, const V@\wgRem{1}@& y);
template<@\exposconcept{math-floating-point}@ V>
  constexpr @\exposid{deduced-vec-t}@<V> trunc(const V& x);
template<@\wgChange{class V0, class V1}{\exposid{math-floating-point} V}@>
  constexpr @\wgChange{\exposid{math-common-simd-t}<V0, V1>}{\exposid{deduced-vec-t}<V>}@ remainder(const V@\wgRem{0}@& x, const V@\wgRem{1}@& y);
template<@\wgChange{class V0, class V1}{\exposid{math-floating-point} V}@>
  constexpr @\wgChange{\exposid{math-common-simd-t}<V0, V1>}{\exposid{deduced-vec-t}<V>}@ copysign(const V@\wgRem{0}@& x, const V@\wgRem{1}@& y);
template<@\wgChange{class V0, class V1}{\exposid{math-floating-point} V}@>
  constexpr @\wgChange{\exposid{math-common-simd-t}<V0, V1>}{\exposid{deduced-vec-t}<V>}@ nextafter(const V@\wgRem{0}@& x, const V@\wgRem{1}@& y);
template<@\wgChange{class V0, class V1}{\exposid{math-floating-point} V}@>
  constexpr @\wgChange{\exposid{math-common-simd-t}<V0, V1>}{\exposid{deduced-vec-t}<V>}@ fdim(const V@\wgRem{0}@& x, const V@\wgRem{1}@& y);
template<@\wgChange{class V0, class V1}{\exposid{math-floating-point} V}@>
  constexpr @\wgChange{\exposid{math-common-simd-t}<V0, V1>}{\exposid{deduced-vec-t}<V>}@ fmax(const V@\wgRem{0}@& x, const V@\wgRem{1}@& y);
template<@\wgChange{class V0, class V1}{\exposid{math-floating-point} V}@>
  constexpr @\wgChange{\exposid{math-common-simd-t}<V0, V1>}{\exposid{deduced-vec-t}<V>}@ fmin(const V@\wgRem{0}@& x, const V@\wgRem{1}@& y);
template<@\wgChange{class V0, class V1, class V2}{\exposconcept{math-floating-point} V}@>
  constexpr @\wgChange{\exposid{math-common-simd-t}<V0, V1, V2>}{\exposid{deduced-vec-t}<V>}@ fma(const V@\wgRem{0}@& x, const V@\wgRem{1}@& y, const V@\wgRem{2}@& z);
template<@\exposconcept{math-floating-point}@ V>
  constexpr rebind_t<int, @\exposid{deduced-vec-t}@<V>> fpclassify(const V& x);
template<@\exposconcept{math-floating-point}@ V>
  constexpr typename @\exposid{deduced-vec-t}@<V>::mask_type isfinite(const V& x);
template<@\exposconcept{math-floating-point}@ V>
  constexpr typename @\exposid{deduced-vec-t}@<V>::mask_type isinf(const V& x);
template<@\exposconcept{math-floating-point}@ V>
  constexpr typename @\exposid{deduced-vec-t}@<V>::mask_type isnan(const V& x);
template<@\exposconcept{math-floating-point}@ V>
  constexpr typename @\exposid{deduced-vec-t}@<V>::mask_type isnormal(const V& x);
template<@\exposconcept{math-floating-point}@ V>
  constexpr typename @\exposid{deduced-vec-t}@<V>::mask_type signbit(const V& x);
template<@\wgChange{class V0, class V1}{\exposid{math-floating-point} V}@>
  constexpr typename @\wgChange{\exposid{math-common-simd-t}<V0, V1>}{\exposid{deduced-vec-t}<V>}@::mask_type isgreater(const V@\wgRem{0}@& x, const V@\wgRem{1}@& y);
template<@\wgChange{class V0, class V1}{\exposid{math-floating-point} V}@>
  constexpr typename @\wgChange{\exposid{math-common-simd-t}<V0, V1>}{\exposid{deduced-vec-t}<V>}@::mask_type
    isgreaterequal(const V@\wgRem{0}@& x, const V@\wgRem{1}@& y);
template<@\wgChange{class V0, class V1}{\exposid{math-floating-point} V}@>
  constexpr typename @\wgChange{\exposid{math-common-simd-t}<V0, V1>}{\exposid{deduced-vec-t}<V>}@::mask_type isless(const V@\wgRem{0}@& x, const V@\wgRem{1}@& y);
template<@\wgChange{class V0, class V1}{\exposid{math-floating-point} V}@>
  constexpr typename @\wgChange{\exposid{math-common-simd-t}<V0, V1>}{\exposid{deduced-vec-t}<V>}@::mask_type islessequal(const V@\wgRem{0}@& x, const V@\wgRem{1}@& y);
template<@\wgChange{class V0, class V1}{\exposid{math-floating-point} V}@>
  constexpr typename @\wgChange{\exposid{math-common-simd-t}<V0, V1>}{\exposid{deduced-vec-t}<V>}@::mask_type islessgreater(const V@\wgRem{0}@& x, const V@\wgRem{1}@& y);
template<@\wgChange{class V0, class V1}{\exposid{math-floating-point} V}@>
  constexpr typename @\wgChange{\exposid{math-common-simd-t}<V0, V1>}{\exposid{deduced-vec-t}<V>}@::mask_type isunordered(const V@\wgRem{0}@& x, const V@\wgRem{1}@& y);
\end{itemdecl}

\begin{itemdescr}
\pnum
Let \tcode{Ret} denote the return type of the specialization of a function
template with the name \tcode{\placeholder{math-func}}.
Let \tcode{\placeholder{math-func-vec}} denote:
\begin{codeblock}
template<class... Args>
Ret @\placeholder{math-func-vec}@(Args... args) {
  return Ret([&](@\exposid{simd-size-type}@ i) {
    return @\placeholder{math-func}(\wgChange{\exposid{make-compatible-simd-t}<Ret, Args>}{static_cast<const \exposid{deduced-vec-t}<V>\&>}@(args)[i]...);
  });
}
\end{codeblock}

\pnum
\returns
A value \tcode{ret} of type \tcode{Ret}, that is element-wise equal to the
result of calling \tcode{\placeholder{math-func-vec}} with the arguments of the above
functions.
If in an invocation of a scalar overload of \tcode{\placeholder{math-func}} for index
\tcode{i} in \tcode{\placeholder{math-func-vec}} a domain, pole, or range error would
occur, the value of \tcode{ret[i]} is unspecified.

\pnum
\remarks
It is unspecified whether \tcode{errno}\iref{errno} is accessed.
\end{itemdescr}

\begin{itemdecl}
@\wgAdd{template<\exposconcept{math-floating-point} V>}@
  @\wgAdd{constexpr \exposid{deduced-vec-t}<V> ldexp(const V\& x, const rebind_t<int, \exposid{deduced-vec-t}<V>}\wgAdd{>\& exp);}@
@\wgAdd{template<\exposconcept{math-floating-point} V>}@
  @\wgAdd{constexpr \exposid{deduced-vec-t}<V> scalbn(const V\& x, const rebind_t<int, \exposid{deduced-vec-t}<V>}\wgAdd{>\& n);}@
@\wgAdd{template<\exposconcept{math-floating-point} V>}@
  @\wgAdd{constexpr \exposid{deduced-vec-t}<V>}@
    @\wgAdd{scalbln(const V\& x, const rebind_t<long int, \exposid{deduced-vec-t}<V>}\wgAdd{>\& n);}@
\end{itemdecl}

\begin{itemdescr}
\pnumAdd
\wgAdd{%
Let \tcode{Ret} be \tcode{\exposid{deduced-vec-t}<V>}.
Let \tcode{\placeholder{math-func}} denote the name of the function template.
Let \tcode{\placeholder{math-func-vec}} denote:}
\begin{codeblock}
@\wgAdd{Ret \placeholder{math-func-vec}(const \exposid{deduced-vec-t}<V>\& a, const auto\& b) \{}@
  @\wgAdd{return Ret([\&](\exposid{simd-size-type} i) \{}@
    @\wgAdd{return \placeholder{math-func}(a[i], b[i]);}@
  @\wgAdd{\});}@
@\wgAdd{\}}@
\end{codeblock}

\pnumAdd
\wgAdd{%
\returns
A value \tcode{ret} of type \tcode{Ret}, that is element-wise equal to the
result of calling \tcode{\placeholder{math-func-vec}} with the arguments of the above
functions.
If in an invocation of a scalar overload of \tcode{\placeholder{math-func}} for index
\tcode{i} in \tcode{\placeholder{math-func-vec}} a domain, pole, or range error would
occur, the value of \tcode{ret[i]} is unspecified.}

\pnumAdd
\wgAdd{%
\remarks
It is unspecified whether \tcode{errno}\iref{errno} is accessed.}
\end{itemdescr}

\begin{itemdecl}
@\wgAdd{template<\libconcept{signed_integral} T, class Abi>}@
  @\wgAdd{constexpr basic_vec<T, Abi> abs(const basic_vec<T, Abi>\& j);}@
\end{itemdecl}

\begin{itemdescr}
\pnumAdd
\wgAdd{%
\expects
\tcode{all_of(j >= -numeric_limits<T>::max())} is \tcode{true}.}

\pnumAdd
\wgAdd{%
\returns
An object where the $i^{\textrm{th}}$ element is initialized to the result of
\tcode{std::abs(j[$i$])} for all $i$ in the range \mbox{\range{0}{\tcode{j.size()}}}.}

\end{itemdescr}

\begin{itemdecl}
template<@\exposconcept{math-floating-point}@ V> constexpr @\exposid{deduced-vec-t}@<V> acos(const V& x);
template<@\exposconcept{math-floating-point}@ V> constexpr @\exposid{deduced-vec-t}@<V> asin(const V& x);
template<@\exposconcept{math-floating-point}@ V> constexpr @\exposid{deduced-vec-t}@<V> atan(const V& x);
template<@\wgChange{class V0, class V1}{\exposid{math-floating-point} V}@>
  constexpr @\wgChange{\exposid{math-common-simd-t}<V0, V1>}{\exposid{deduced-vec-t}<V>}@ atan2(const V@\wgRem{0}@& y, const V@\wgRem{1}@& x);
template<@\exposconcept{math-floating-point}@ V> constexpr @\exposid{deduced-vec-t}@<V> cos(const V& x);
template<@\exposconcept{math-floating-point}@ V> constexpr @\exposid{deduced-vec-t}@<V> sin(const V& x);
template<@\exposconcept{math-floating-point}@ V> constexpr @\exposid{deduced-vec-t}@<V> tan(const V& x);
template<@\exposconcept{math-floating-point}@ V> constexpr @\exposid{deduced-vec-t}@<V> acosh(const V& x);
template<@\exposconcept{math-floating-point}@ V> constexpr @\exposid{deduced-vec-t}@<V> asinh(const V& x);
template<@\exposconcept{math-floating-point}@ V> constexpr @\exposid{deduced-vec-t}@<V> atanh(const V& x);
template<@\exposconcept{math-floating-point}@ V> constexpr @\exposid{deduced-vec-t}@<V> cosh(const V& x);
template<@\exposconcept{math-floating-point}@ V> constexpr @\exposid{deduced-vec-t}@<V> sinh(const V& x);
template<@\exposconcept{math-floating-point}@ V> constexpr @\exposid{deduced-vec-t}@<V> tanh(const V& x);
template<@\exposconcept{math-floating-point}@ V> constexpr @\exposid{deduced-vec-t}@<V> exp(const V& x);
template<@\exposconcept{math-floating-point}@ V> constexpr @\exposid{deduced-vec-t}@<V> exp2(const V& x);
template<@\exposconcept{math-floating-point}@ V> constexpr @\exposid{deduced-vec-t}@<V> expm1(const V& x);
template<@\exposconcept{math-floating-point}@ V> constexpr @\exposid{deduced-vec-t}@<V> log(const V& x);
template<@\exposconcept{math-floating-point}@ V> constexpr @\exposid{deduced-vec-t}@<V> log10(const V& x);
template<@\exposconcept{math-floating-point}@ V> constexpr @\exposid{deduced-vec-t}@<V> log1p(const V& x);
template<@\exposconcept{math-floating-point}@ V> constexpr @\exposid{deduced-vec-t}@<V> log2(const V& x);
template<@\exposconcept{math-floating-point}@ V> constexpr @\exposid{deduced-vec-t}@<V> logb(const V& x);
template<@\exposconcept{math-floating-point}@ V> constexpr @\exposid{deduced-vec-t}@<V> cbrt(const V& x);
template<@\wgChange{class V0, class V1}{\exposid{math-floating-point} V}@>
  constexpr @\wgChange{\exposid{math-common-simd-t}<V0, V1>}{\exposid{deduced-vec-t}<V>}@ hypot(const V@\wgRem{0}@& x, const V@\wgRem{1}@& y);
template<@\wgChange{class V0, class V1, class V2}{\exposid{math-floating-point} V}@>
  constexpr @\wgChange{\exposid{math-common-simd-t}<V0, V1, V2>}{\exposid{deduced-vec-t}<V>}@ hypot(const V@\wgRem{0}@& x, const V@\wgRem{1}@& y, const V@\wgRem{2}@& z);
template<@\wgChange{class V0, class V1}{\exposid{math-floating-point} V}@>
  constexpr @\wgChange{\exposid{math-common-simd-t}<V0, V1>}{\exposid{deduced-vec-t}<V>}@ pow(const V@\wgRem{0}@& x, const V@\wgRem{1}@& y);
template<@\exposconcept{math-floating-point}@ V> constexpr @\exposid{deduced-vec-t}@<V> sqrt(const V& x);
template<@\exposconcept{math-floating-point}@ V> constexpr @\exposid{deduced-vec-t}@<V> erf(const V& x);
template<@\exposconcept{math-floating-point}@ V> constexpr @\exposid{deduced-vec-t}@<V> erfc(const V& x);
template<@\exposconcept{math-floating-point}@ V> constexpr @\exposid{deduced-vec-t}@<V> lgamma(const V& x);
template<@\exposconcept{math-floating-point}@ V> constexpr @\exposid{deduced-vec-t}@<V> tgamma(const V& x);
template<@\wgChange{class V0, class V1, class V2}{\exposid{math-floating-point} V}@>
  constexpr @\wgChange{\exposid{math-common-simd-t}<V0, V1, V2>}{\exposid{deduced-vec-t}<V>}@ lerp(const V@\wgRem{0}@& a, const V@\wgRem{1}@& b, const V@\wgRem{2}@& t) noexcept;
@\wgRem{template<\exposconcept{math-floating-point} V>}@
  @\wgRem{\exposid{deduced-vec-t}<V> assoc_laguerre(const rebind_t<unsigned, \exposid{deduced-vec-t}<V>>\& n, const}@
    @\wgRem{rebind_t<unsigned, \exposid{deduced-vec-t}<V>>\& m, const V\& x);}@
@\wgRem{template<\exposconcept{math-floating-point} V>}@
  @\wgRem{\exposid{deduced-vec-t}<V> assoc_legendre(const rebind_t<unsigned, \exposid{deduced-vec-t}<V>>\& l, const}@
    @\wgRem{rebind_t<unsigned, \exposid{deduced-vec-t}<V>>\& m, const V\& x);}@
template<@\wgChange{class V0, class V1}{\exposid{math-floating-point} V}@>
  @\wgChange{\exposid{math-common-simd-t}<V0, V1>}{\exposid{deduced-vec-t}<V>}@ beta(const V@\wgRem{0}@& x, const V@\wgRem{1}@& y);
template<@\exposconcept{math-floating-point}@ V> @\exposid{deduced-vec-t}@<V> comp_ellint_1(const V& k);
template<@\exposconcept{math-floating-point}@ V> @\exposid{deduced-vec-t}@<V> comp_ellint_2(const V& k);
template<@\wgChange{class V0, class V1}{\exposid{math-floating-point} V}@>
  @\wgChange{\exposid{math-common-simd-t}<V0, V1>}{\exposid{deduced-vec-t}<V>}@ comp_ellint_3(const V@\wgRem{0}@& k, const V@\wgRem{1}@& nu);
template<@\wgChange{class V0, class V1}{\exposid{math-floating-point} V}@>
  @\wgChange{\exposid{math-common-simd-t}<V0, V1>}{\exposid{deduced-vec-t}<V>}@ cyl_bessel_i(const V@\wgRem{0}@& nu, const V@\wgRem{1}@& x);
template<@\wgChange{class V0, class V1}{\exposid{math-floating-point} V}@>
  @\wgChange{\exposid{math-common-simd-t}<V0, V1>}{\exposid{deduced-vec-t}<V>}@ cyl_bessel_j(const V@\wgRem{0}@& nu, const V@\wgRem{1}@& x);
template<@\wgChange{class V0, class V1}{\exposid{math-floating-point} V}@>
  @\wgChange{\exposid{math-common-simd-t}<V0, V1>}{\exposid{deduced-vec-t}<V>}@ cyl_bessel_k(const V@\wgRem{0}@& nu, const V@\wgRem{1}@& x);
template<@\wgChange{class V0, class V1}{\exposid{math-floating-point} V}@>
  @\wgChange{\exposid{math-common-simd-t}<V0, V1>}{\exposid{deduced-vec-t}<V>}@ cyl_neumann(const V@\wgRem{0}@& nu, const V@\wgRem{1}@& x);
template<@\wgChange{class V0, class V1}{\exposid{math-floating-point} V}@>
  @\wgChange{\exposid{math-common-simd-t}<V0, V1>}{\exposid{deduced-vec-t}<V>}@ ellint_1(const V@\wgRem{0}@& k, const V@\wgRem{1}@& phi);
template<@\wgChange{class V0, class V1}{\exposid{math-floating-point} V}@>
  @\wgChange{\exposid{math-common-simd-t}<V0, V1>}{\exposid{deduced-vec-t}<V>}@ ellint_2(const V@\wgRem{0}@& k, const V@\wgRem{1}@& phi);
template<@\wgChange{class V0, class V1, class V2}{\exposid{math-floating-point} V}@>
  @\wgChange{\exposid{math-common-simd-t}<V0, V1, V2>}{\exposid{deduced-vec-t}<V>}@ ellint_3(const V@\wgRem{0}@& k, const V@\wgRem{1}@& nu, const V@\wgRem{2}@& phi);
template<@\exposconcept{math-floating-point}@ V> @\exposid{deduced-vec-t}@<V> expint(const V& x);
@\wgRem{template<\exposconcept{math-floating-point} V> \exposid{deduced-vec-t}<V> hermite(const rebind_t<unsigned,}@
@\wgRem{\exposid{deduced-vec-t}<V>>\& n, const V\& x);}@
@\wgRem{template<\exposconcept{math-floating-point} V> \exposid{deduced-vec-t}<V> laguerre(const rebind_t<unsigned,}@
@\wgRem{\exposid{deduced-vec-t}<V>>\& n, const V\& x);}@
@\wgRem{template<\exposconcept{math-floating-point} V> \exposid{deduced-vec-t}<V> legendre(const rebind_t<unsigned,}@
@\wgRem{\exposid{deduced-vec-t}<V>>\& l, const V\& x);}@
template<@\exposconcept{math-floating-point}@ V> @\exposid{deduced-vec-t}@<V> riemann_zeta(const V& x);
@\wgRem{template<\exposconcept{math-floating-point} V> \exposid{deduced-vec-t}<V> sph_bessel(const rebind_t<unsigned,}@
@\wgRem{\exposid{deduced-vec-t}<V>>\& n, const V\& x);}@
@\wgRem{template<\exposconcept{math-floating-point} V>}@
  @\wgRem{\exposid{deduced-vec-t}<V> sph_legendre(const rebind_t<unsigned, \exposid{deduced-vec-t}<V>>\& l,}@
                                 @\wgRem{const rebind_t<unsigned, \exposid{deduced-vec-t}<V>>\& m,}@
                                 @\wgRem{const V\& theta);}@
@\wgRem{template<\exposconcept{math-floating-point} V> \exposid{deduced-vec-t}<V> sph_neumann(const rebind_t<unsigned,}@
@\wgRem{\exposid{deduced-vec-t}<V>>\& n, const V\& x);}@
\end{itemdecl}

\begin{itemdescr}
\pnum
Let \tcode{Ret} denote the return type of the specialization of a function
template with the name \tcode{\placeholder{math-func}}.
Let \tcode{\placeholder{math-func-vec}} denote:
\begin{codeblock}
template<class... Args>
Ret @\placeholder{math-func-vec}(\wgAdd{const }Args\wgAdd{\&}@... args) {
  return Ret([&](@\exposid{simd-size-type}@ i) {
    return @\placeholder{math-func}(\wgChange{\exposid{make-compatible-simd-t}<Ret, Args>}{static_cast<const \exposid{deduced-vec-t}<V>\&>}@(args)[i]...);
  });
}
\end{codeblock}

\pnum
\returns
A value \tcode{ret} of type \tcode{Ret}, that is element-wise approximately
equal to the result of calling \tcode{\placeholder{math-func-vec}} with the arguments
of the above functions.
If in an invocation of a scalar overload of \tcode{\placeholder{math-func}} for index
\tcode{i} in \tcode{\placeholder{math-func-vec}} a domain, pole, or range error would
occur, the value of \tcode{ret[i]} is unspecified.

\pnum
\remarks
It is unspecified whether \tcode{errno}\iref{errno} is accessed.
\end{itemdescr}

\begin{itemdecl}
@\wgAdd{template<\exposconcept{math-floating-point} V>}@
  @\wgAdd{\exposid{deduced-vec-t}<V> assoc_laguerre(const rebind_t<unsigned, \exposid{deduced-vec-t}<V>}\wgAdd{>\& n,}@
                                  @\wgAdd{const rebind_t<unsigned, \exposid{deduced-vec-t}<V>}\wgAdd{>\& m, const V\& x);}@
@\wgAdd{template<\exposconcept{math-floating-point} V>}@
  @\wgAdd{\exposid{deduced-vec-t}<V> assoc_legendre(const rebind_t<unsigned, \exposid{deduced-vec-t}<V>}\wgAdd{>\& l,}@
                                  @\wgAdd{const rebind_t<unsigned, \exposid{deduced-vec-t}<V>}\wgAdd{>\& m, const V\& x);}@
@\wgAdd{template<\exposconcept{math-floating-point} V>}@
  @\wgAdd{\exposid{deduced-vec-t}<V> sph_legendre(const rebind_t<unsigned, \exposid{deduced-vec-t}<V>}\wgAdd{>\& l,}@
                                @\wgAdd{const rebind_t<unsigned, \exposid{deduced-vec-t}<V>}\wgAdd{>\& m, const V\& theta);}@
\end{itemdecl}

\begin{itemdescr}
\pnumAdd
\wgAdd{%
Let \tcode{\placeholder{math-func}} denote the name of the function template.
Let \tcode{\placeholder{math-func-vec}} denote:}
\begin{codeblock}
@\wgAdd{auto \placeholder{math-func-vec}(const auto\& a, const auto\&b, const \exposid{deduced-vec-t}<V>\& c) \{}@
  @\wgAdd{return \exposid{deduced-vec-t}<V>([\&](\exposid{simd-size-type} i) \{}@
    @\wgAdd{return std::\placeholder{math-func}(a[i], b[i], c[i]);}@
  @\wgAdd{\});}@
@\wgAdd{\}}@
\end{codeblock}

\pnumAdd
\wgAdd{%
\returns
An object that is element-wise approximately equal to the result of calling
\tcode{\placeholder{math-func-vec}} with the arguments of the above functions.}
\end{itemdescr}

\begin{itemdecl}
@\wgAdd{template<\exposconcept{math-floating-point} V>}@
  @\wgAdd{\exposid{deduced-vec-t}<V> hermite(const rebind_t<unsigned, \exposid{deduced-vec-t}<V>}\wgAdd{>\& n, const V\& x);}@
@\wgAdd{template<\exposconcept{math-floating-point} V>}@
  @\wgAdd{\exposid{deduced-vec-t}<V> laguerre(const rebind_t<unsigned, \exposid{deduced-vec-t}<V>}\wgAdd{>\& n, const V\& x);}@
@\wgAdd{template<\exposconcept{math-floating-point} V>}@
  @\wgAdd{\exposid{deduced-vec-t}<V> legendre(const rebind_t<unsigned, \exposid{deduced-vec-t}<V>}\wgAdd{>\& l, const V\& x);}@
@\wgAdd{template<\exposconcept{math-floating-point} V>}@
  @\wgAdd{\exposid{deduced-vec-t}<V> sph_bessel(const rebind_t<unsigned, \exposid{deduced-vec-t}<V>}\wgAdd{>\& n, const V\& x);}@
@\wgAdd{template<\exposconcept{math-floating-point} V>}@
  @\wgAdd{\exposid{deduced-vec-t}<V> sph_neumann(const rebind_t<unsigned, \exposid{deduced-vec-t}<V>}\wgAdd{>\& n, const V\& x);}@
\end{itemdecl}

\begin{itemdescr}
\pnumAdd
\wgAdd{%
Let \tcode{\placeholder{math-func}} denote the name of the function template.
Let \tcode{\placeholder{math-func-vec}} denote:}
\begin{codeblock}
@\wgAdd{auto \placeholder{math-func-vec}(const auto\& a, const \exposid{deduced-vec-t}<V>\& b) \{}@
  @\wgAdd{return \exposid{deduced-vec-t}<V>([\&](\exposid{simd-size-type} i) \{ return std::\placeholder{math-func}(a[i], b[i]); \});}@
@\wgAdd{\}}@
\end{codeblock}

\pnumAdd
\wgAdd{%
\returns
An object that is element-wise approximately equal to the result of calling
\tcode{\placeholder{math-func-vec}} with the arguments of the above functions.}
\end{itemdescr}

\begin{itemdecl}
template<@\exposconcept{math-floating-point}@ V>
  constexpr @\exposid{deduced-vec-t}@<V> frexp(const V& value, rebind_t<int, @\exposid{deduced-vec-t}@<V>>* exp);
\end{itemdecl}

\begin{itemdescr}
\pnum
Let \tcode{Ret} be \tcode{\exposid{deduced-vec-t}<V>}.
Let \tcode{\placeholder{frexp-vec}} denote:
\begin{codeblock}
template<class V>
pair<Ret, rebind_t<int, Ret>> @\placeholder{frexp-vec}@(const @\wgAdd{\exposid{deduced-vec-t}<}V\wgAdd{>}@& x) {
  int r1[Ret::size()];
  Ret r0([&](@\exposid{simd-size-type}@ i) {
    return frexp(@\wgChange{\exposid{make-compatible-simd-t}<Ret, V>(x)}{x}@[i], &r1[i]);
  });
  return {r0, rebind_t<int, Ret>(r1)};
}
\end{codeblock}
Let \tcode{ret} be a value of type \tcode{pair<Ret, rebind_t<int, Ret>>}
that is the same value as the result of calling
\tcode{\placeholder{frexp-vec}(x)}.

\pnum
\effects
Sets \tcode{*exp} to \tcode{ret.second}.

\pnum
\returns
\tcode{ret.first}.
\end{itemdescr}

\begin{itemdecl}
template<@\wgChange{class V0, class V1}{\exposid{math-floating-point} V}@>
  constexpr @\wgChange{\exposid{math-common-simd-t}<V0, V1>}{\exposid{deduced-vec-t}<V>}@ remquo(const V@\wgRem{0}@& x, const V@\wgRem{1}@& y,
                          rebind_t<int, @\wgChange{\exposid{math-common-simd-t}<V0, V1>}{\exposid{deduced-vec-t}<V>}@>* quo);
\end{itemdecl}

\begin{itemdescr}
\pnum
Let \wgChange{\tcode{Ret} be \tcode{\exposid{math-common-simd-t}<V0, V1>}}{\tcode{V0} be \tcode{\exposid{deduced-vec-t}<V>}}.
Let \tcode{\placeholder{remquo-vec}} denote:
\begin{codeblock}
@\wgRem{template<class V0, class V1>}@
pair<@\wgChange{Ret}{V0}, rebind_t<int, \wgChange{Ret}{V0}>> \placeholder{remquo-vec}@(const V0& x, const @\wgChange{V1}{V0}@& y) {
  int r1[@\wgChange{Ret}{V0}@::size()];
  V0 r0([&](@\exposid{simd-size-type}@ i) {
      return remquo(@\wgChange{\exposid{make-compatible-simd-t}<Ret, V0>(x)}{x}@[i],
      @\wgChange{\exposid{make-compatible-simd-t}<Ret, V1>(y)}{y}@[i], &r1[i]);
  });
  return {r0, rebind_t<int, @\wgChange{Ret}{V0}@>(r1)};
}
\end{codeblock}
Let \tcode{ret} be a value of type \wgChange{\tcode{pair<Ret, rebind_t<int, Ret>>}}{\tcode{pair<V0, rebind_t<int, V0>>}}
that is the same value as the result of calling
\tcode{\placeholder{remquo-vec}(x, y)}.
If in an invocation of a scalar overload of \tcode{remquo} for index \tcode{i}
in \tcode{\placeholder{remquo-vec}} a domain, pole, or range error would occur, the
value of \tcode{ret[i]} is unspecified.

\pnum
\effects
Sets \tcode{*quo} to \tcode{ret.second}.

\pnum
\returns
\tcode{ret.first}.

\pnum
\remarks
It is unspecified whether \tcode{errno}\iref{errno} is accessed.
\end{itemdescr}

\begin{itemdecl}
template<class T, class Abi>
  constexpr basic_vec<T, Abi> modf(const type_identity_t<basic_vec<T, Abi>>& value,
                                   basic_vec<T, Abi>* iptr);
\end{itemdecl}

\begin{itemdescr}
\pnum
Let \tcode{V} be \tcode{basic_vec<T, Abi>}.
Let \tcode{\placeholder{modf-vec}} denote:
\begin{codeblock}
pair<V, V> @\placeholder{modf-vec}@(const V& x) {
  T r1[Ret::size()];
  V r0([&](@\exposid{simd-size-type}@ i) {
    return modf(V(x)[i], &r1[i]);
  });
  return {r0, V(r1)};
}
\end{codeblock}
Let \tcode{ret} be a value of type \tcode{pair<V, V>} that is the same value as
the result of calling \tcode{\placeholder{modf-vec}(value)}.

\pnum
\effects
Sets \tcode{*iptr} to \tcode{ret.second}.

\pnum
\returns
\tcode{ret.first}.
\end{itemdescr}

\begin{itemdecl}
@\wgAdd{template<\exposconcept{math-floating-point} V>}@
  @\wgAdd{constexpr \exposid{deduced-vec-t}<V> fmod(const \exposid{deduced-vec-t}<V>\& x, const V\& y);}@
@\wgAdd{template<\exposconcept{math-floating-point} V>}@
  @\wgAdd{constexpr \exposid{deduced-vec-t}<V> fmod(const V\& x, const \exposid{deduced-vec-t}<V>\& y);}@
@\wgAdd{template<\exposconcept{math-floating-point} V>}@
  @\wgAdd{constexpr \exposid{deduced-vec-t}<V> remainder(const \exposid{deduced-vec-t}<V>\& x, const V\& y);}@
@\wgAdd{template<\exposconcept{math-floating-point} V>}@
  @\wgAdd{constexpr \exposid{deduced-vec-t}<V> remainder(const V\& x, const \exposid{deduced-vec-t}<V>\& y);}@
@\wgAdd{template<\exposconcept{math-floating-point} V>}@
  @\wgAdd{constexpr \exposid{deduced-vec-t}<V> copysign(const \exposid{deduced-vec-t}<V>\& x, const V\& y);}@
@\wgAdd{template<\exposconcept{math-floating-point} V>}@
  @\wgAdd{constexpr \exposid{deduced-vec-t}<V> copysign(const V\& x, const \exposid{deduced-vec-t}<V>\& y);}@
@\wgAdd{template<\exposconcept{math-floating-point} V>}@
  @\wgAdd{constexpr \exposid{deduced-vec-t}<V> nextafter(const \exposid{deduced-vec-t}<V>\& x, const V\& y);}@
@\wgAdd{template<\exposconcept{math-floating-point} V>}@
  @\wgAdd{constexpr \exposid{deduced-vec-t}<V> nextafter(const V\& x, const \exposid{deduced-vec-t}<V>\& y);}@
@\wgAdd{template<\exposconcept{math-floating-point} V>}@
  @\wgAdd{constexpr \exposid{deduced-vec-t}<V> fdim(const \exposid{deduced-vec-t}<V>\& x, const V\& y);}@
@\wgAdd{template<\exposconcept{math-floating-point} V>}@
  @\wgAdd{constexpr \exposid{deduced-vec-t}<V> fdim(const V\& x, const \exposid{deduced-vec-t}<V>\& y);}@
@\wgAdd{template<\exposconcept{math-floating-point} V>}@
  @\wgAdd{constexpr \exposid{deduced-vec-t}<V> fmax(const \exposid{deduced-vec-t}<V>\& x, const V\& y);}@
@\wgAdd{template<\exposconcept{math-floating-point} V>}@
  @\wgAdd{constexpr \exposid{deduced-vec-t}<V> fmax(const V\& x, const \exposid{deduced-vec-t}<V>\& y);}@
@\wgAdd{template<\exposconcept{math-floating-point} V>}@
  @\wgAdd{constexpr \exposid{deduced-vec-t}<V> fmin(const \exposid{deduced-vec-t}<V>\& x, const V\& y);}@
@\wgAdd{template<\exposconcept{math-floating-point} V>}@
  @\wgAdd{constexpr \exposid{deduced-vec-t}<V> fmin(const V\& x, const \exposid{deduced-vec-t}<V>\& y);}@
@\wgAdd{template<\exposconcept{math-floating-point} V>}@
  @\wgAdd{constexpr \exposid{deduced-vec-t}<V> isgreater(const \exposid{deduced-vec-t}<V>\& x, const V\& y);}@
@\wgAdd{template<\exposconcept{math-floating-point} V>}@
  @\wgAdd{constexpr \exposid{deduced-vec-t}<V> isgreater(const V\& x, const \exposid{deduced-vec-t}<V>\& y);}@
@\wgAdd{template<\exposconcept{math-floating-point} V>}@
  @\wgAdd{constexpr \exposid{deduced-vec-t}<V> islessequal(const \exposid{deduced-vec-t}<V>\& x, const V\& y);}@
@\wgAdd{template<\exposconcept{math-floating-point} V>}@
  @\wgAdd{constexpr \exposid{deduced-vec-t}<V> islessequal(const V\& x, const \exposid{deduced-vec-t}<V>\& y);}@
@\wgAdd{template<\exposconcept{math-floating-point} V>}@
  @\wgAdd{constexpr \exposid{deduced-vec-t}<V> islessgreater(const \exposid{deduced-vec-t}<V>\& x, const V\& y);}@
@\wgAdd{template<\exposconcept{math-floating-point} V>}@
  @\wgAdd{constexpr \exposid{deduced-vec-t}<V> islessgreater(const V\& x, const \exposid{deduced-vec-t}<V>\& y);}@
@\wgAdd{template<\exposconcept{math-floating-point} V>}@
  @\wgAdd{constexpr \exposid{deduced-vec-t}<V> isunordered(const \exposid{deduced-vec-t}<V>\& x, const V\& y);}@
@\wgAdd{template<\exposconcept{math-floating-point} V>}@
  @\wgAdd{constexpr \exposid{deduced-vec-t}<V> isunordered(const V\& x, const \exposid{deduced-vec-t}<V>\& y);}@
@\wgAdd{template<\exposconcept{math-floating-point} V>}@
  @\wgAdd{constexpr \exposid{deduced-vec-t}<V> atan2(const \exposid{deduced-vec-t}<V>\& x, const V\& y);}@
@\wgAdd{template<\exposconcept{math-floating-point} V>}@
  @\wgAdd{constexpr \exposid{deduced-vec-t}<V> atan2(const V\& x, const \exposid{deduced-vec-t}<V>\& y);}@
@\wgAdd{template<\exposconcept{math-floating-point} V>}@
  @\wgAdd{constexpr \exposid{deduced-vec-t}<V> hypot(const \exposid{deduced-vec-t}<V>\& x, const V\& y);}@
@\wgAdd{template<\exposconcept{math-floating-point} V>}@
  @\wgAdd{constexpr \exposid{deduced-vec-t}<V> hypot(const V\& x, const \exposid{deduced-vec-t}<V>\& y);}@
@\wgAdd{template<\exposconcept{math-floating-point} V>}@
  @\wgAdd{constexpr \exposid{deduced-vec-t}<V> pow(const \exposid{deduced-vec-t}<V>\& x, const V\& y);}@
@\wgAdd{template<\exposconcept{math-floating-point} V>}@
  @\wgAdd{constexpr \exposid{deduced-vec-t}<V> pow(const V\& x, const \exposid{deduced-vec-t}<V>\& y);}@
@\wgAdd{template<\exposconcept{math-floating-point} V>}@
  @\wgAdd{constexpr \exposid{deduced-vec-t}<V> beta(const \exposid{deduced-vec-t}<V>\& x, const V\& y);}@
@\wgAdd{template<\exposconcept{math-floating-point} V>}@
  @\wgAdd{constexpr \exposid{deduced-vec-t}<V> beta(const V\& x, const \exposid{deduced-vec-t}<V>\& y);}@
@\wgAdd{template<\exposconcept{math-floating-point} V>}@
  @\wgAdd{constexpr \exposid{deduced-vec-t}<V> comp_ellint_3(const \exposid{deduced-vec-t}<V>\& x, const V\& y);}@
@\wgAdd{template<\exposconcept{math-floating-point} V>}@
  @\wgAdd{constexpr \exposid{deduced-vec-t}<V> comp_ellint_3(const V\& x, const \exposid{deduced-vec-t}<V>\& y);}@
@\wgAdd{template<\exposconcept{math-floating-point} V>}@
  @\wgAdd{constexpr \exposid{deduced-vec-t}<V> cyl_bessel_i(const \exposid{deduced-vec-t}<V>\& x, const V\& y);}@
@\wgAdd{template<\exposconcept{math-floating-point} V>}@
  @\wgAdd{constexpr \exposid{deduced-vec-t}<V> cyl_bessel_i(const V\& x, const \exposid{deduced-vec-t}<V>\& y);}@
@\wgAdd{template<\exposconcept{math-floating-point} V>}@
  @\wgAdd{constexpr \exposid{deduced-vec-t}<V> cyl_bessel_j(const \exposid{deduced-vec-t}<V>\& x, const V\& y);}@
@\wgAdd{template<\exposconcept{math-floating-point} V>}@
  @\wgAdd{constexpr \exposid{deduced-vec-t}<V> cyl_bessel_j(const V\& x, const \exposid{deduced-vec-t}<V>\& y);}@
@\wgAdd{template<\exposconcept{math-floating-point} V>}@
  @\wgAdd{constexpr \exposid{deduced-vec-t}<V> cyl_bessel_k(const \exposid{deduced-vec-t}<V>\& x, const V\& y);}@
@\wgAdd{template<\exposconcept{math-floating-point} V>}@
  @\wgAdd{constexpr \exposid{deduced-vec-t}<V> cyl_bessel_k(const V\& x, const \exposid{deduced-vec-t}<V>\& y);}@
@\wgAdd{template<\exposconcept{math-floating-point} V>}@
  @\wgAdd{constexpr \exposid{deduced-vec-t}<V> cyl_neumann(const \exposid{deduced-vec-t}<V>\& x, const V\& y);}@
@\wgAdd{template<\exposconcept{math-floating-point} V>}@
  @\wgAdd{constexpr \exposid{deduced-vec-t}<V> cyl_neumann(const V\& x, const \exposid{deduced-vec-t}<V>\& y);}@
@\wgAdd{template<\exposconcept{math-floating-point} V>}@
  @\wgAdd{constexpr \exposid{deduced-vec-t}<V> ellint_1(const \exposid{deduced-vec-t}<V>\& x, const V\& y);}@
@\wgAdd{template<\exposconcept{math-floating-point} V>}@
  @\wgAdd{constexpr \exposid{deduced-vec-t}<V> ellint_1(const V\& x, const \exposid{deduced-vec-t}<V>\& y);}@
@\wgAdd{template<\exposconcept{math-floating-point} V>}@
  @\wgAdd{constexpr \exposid{deduced-vec-t}<V> ellint_2(const \exposid{deduced-vec-t}<V>\& x, const V\& y);}@
@\wgAdd{template<\exposconcept{math-floating-point} V>}@
  @\wgAdd{constexpr \exposid{deduced-vec-t}<V> ellint_2(const V\& x, const \exposid{deduced-vec-t}<V>\& y);}@
@\wgAdd{template<\exposconcept{math-floating-point} V>}@
  @\wgAdd{constexpr \exposid{deduced-vec-t}<V>}@
    @\wgAdd{remquo(const \exposid{deduced-vec-t}<V>\& x, const V\& y, rebind_t<int, \exposid{deduced-vec-t}<V>>* quo);}@
@\wgAdd{template<\exposconcept{math-floating-point} V>}@
  @\wgAdd{constexpr \exposid{deduced-vec-t}<V>}@
    @\wgAdd{remquo(const V\& x, const \exposid{deduced-vec-t}<V>\& y, rebind_t<int, \exposid{deduced-vec-t}<V>>* quo);}@
@\wgAdd{template<\exposconcept{math-floating-point} V>}@
  @\wgAdd{constexpr \exposid{deduced-vec-t}<V> fma(const \exposid{deduced-vec-t}<V>\& x, const V\& y, const V\& z);}@
@\wgAdd{template<\exposconcept{math-floating-point} V>}@
  @\wgAdd{constexpr \exposid{deduced-vec-t}<V> fma(const V\& x, const \exposid{deduced-vec-t}<V>\& y, const V\& z);}@
@\wgAdd{template<\exposconcept{math-floating-point} V>}@
  @\wgAdd{constexpr \exposid{deduced-vec-t}<V> fma(const V\& x, const V\& y, const \exposid{deduced-vec-t}<V>\& z);}@
@\wgAdd{template<\exposconcept{math-floating-point} V>}@
  @\wgAdd{constexpr \exposid{deduced-vec-t}<V> fma(const \exposid{deduced-vec-t}<V>\& x, const \exposid{deduced-vec-t}<V>\& y, const V\& z);}@
@\wgAdd{template<\exposconcept{math-floating-point} V>}@
  @\wgAdd{constexpr \exposid{deduced-vec-t}<V> fma(const \exposid{deduced-vec-t}<V>\& x, const V\& y, const \exposid{deduced-vec-t}<V>\& z);}@
@\wgAdd{template<\exposconcept{math-floating-point} V>}@
  @\wgAdd{constexpr \exposid{deduced-vec-t}<V> fma(const V\& x, const \exposid{deduced-vec-t}<V>\& y, const \exposid{deduced-vec-t}<V>\& z);}@
@\wgAdd{template<\exposconcept{math-floating-point} V>}@
  @\wgAdd{constexpr \exposid{deduced-vec-t}<V> hypot(const \exposid{deduced-vec-t}<V>\& x, const V\& y, const V\& z);}@
@\wgAdd{template<\exposconcept{math-floating-point} V>}@
  @\wgAdd{constexpr \exposid{deduced-vec-t}<V> hypot(const V\& x, const \exposid{deduced-vec-t}<V>\& y, const V\& z);}@
@\wgAdd{template<\exposconcept{math-floating-point} V>}@
  @\wgAdd{constexpr \exposid{deduced-vec-t}<V> hypot(const V\& x, const V\& y, const \exposid{deduced-vec-t}<V>\& z);}@
@\wgAdd{template<\exposconcept{math-floating-point} V>}@
  @\wgAdd{constexpr \exposid{deduced-vec-t}<V> hypot(const \exposid{deduced-vec-t}<V>\& x, const \exposid{deduced-vec-t}<V>\& y, const V\& z);}@
@\wgAdd{template<\exposconcept{math-floating-point} V>}@
  @\wgAdd{constexpr \exposid{deduced-vec-t}<V> hypot(const \exposid{deduced-vec-t}<V>\& x, const V\& y, const \exposid{deduced-vec-t}<V>\& z);}@
@\wgAdd{template<\exposconcept{math-floating-point} V>}@
  @\wgAdd{constexpr \exposid{deduced-vec-t}<V> hypot(const V\& x, const \exposid{deduced-vec-t}<V>\& y, const \exposid{deduced-vec-t}<V>\& z);}@
@\wgAdd{template<\exposconcept{math-floating-point} V>}@
  @\wgAdd{constexpr \exposid{deduced-vec-t}<V> lerp(const \exposid{deduced-vec-t}<V>\& x, const V\& y, const V\& z);}@
@\wgAdd{template<\exposconcept{math-floating-point} V>}@
  @\wgAdd{constexpr \exposid{deduced-vec-t}<V> lerp(const V\& x, const \exposid{deduced-vec-t}<V>\& y, const V\& z);}@
@\wgAdd{template<\exposconcept{math-floating-point} V>}@
  @\wgAdd{constexpr \exposid{deduced-vec-t}<V> lerp(const V\& x, const V\& y, const \exposid{deduced-vec-t}<V>\& z);}@
@\wgAdd{template<\exposconcept{math-floating-point} V>}@
  @\wgAdd{constexpr \exposid{deduced-vec-t}<V> lerp(const \exposid{deduced-vec-t}<V>\& x, const \exposid{deduced-vec-t}<V>\& y, const V\& z);}@
@\wgAdd{template<\exposconcept{math-floating-point} V>}@
  @\wgAdd{constexpr \exposid{deduced-vec-t}<V> lerp(const \exposid{deduced-vec-t}<V>\& x, const V\& y, const \exposid{deduced-vec-t}<V>\& z);}@
@\wgAdd{template<\exposconcept{math-floating-point} V>}@
  @\wgAdd{constexpr \exposid{deduced-vec-t}<V> lerp(const V\& x, const \exposid{deduced-vec-t}<V>\& y, const \exposid{deduced-vec-t}<V>\& z);}@
@\wgAdd{template<\exposconcept{math-floating-point} V>}@
  @\wgAdd{constexpr \exposid{deduced-vec-t}<V> ellint_3(const \exposid{deduced-vec-t}<V>\& x, const V\& y, const V\& z);}@
@\wgAdd{template<\exposconcept{math-floating-point} V>}@
  @\wgAdd{constexpr \exposid{deduced-vec-t}<V> ellint_3(const V\& x, const \exposid{deduced-vec-t}<V>\& y, const V\& z);}@
@\wgAdd{template<\exposconcept{math-floating-point} V>}@
  @\wgAdd{constexpr \exposid{deduced-vec-t}<V> ellint_3(const V\& x, const V\& y, const \exposid{deduced-vec-t}<V>\& z);}@
@\wgAdd{template<\exposconcept{math-floating-point} V>}@
  @\wgAdd{constexpr \exposid{deduced-vec-t}<V> ellint_3(const \exposid{deduced-vec-t}<V>\& x, const \exposid{deduced-vec-t}<V>\& y,}@
                                      @\wgAdd{const V\& z);}@
@\wgAdd{template<\exposconcept{math-floating-point} V>}@
  @\wgAdd{constexpr \exposid{deduced-vec-t}<V> ellint_3(const \exposid{deduced-vec-t}<V>\& x, const V\& y,}@
                                      @\wgAdd{const \exposid{deduced-vec-t}<V>\& z);}@
@\wgAdd{template<\exposconcept{math-floating-point} V>}@
  @\wgAdd{constexpr \exposid{deduced-vec-t}<V> ellint_3(const V\& x, const \exposid{deduced-vec-t}<V>\& y,}@
                                      @\wgAdd{const \exposid{deduced-vec-t}<V>\& z);}@
\end{itemdecl}

\begin{itemdescr}
\pnumAdd
\wgAdd{Let}
\begin{itemize}
  \wgItemAdd[\placeholder{math-func} denote the name of the function template;]
  \wgItemAdd[\tcode{args...} be \tcode{x} and \tcode{y}, or \tcode{x}, \tcode{y}, and \tcode{z};]
  \wgItemAdd[\tcode{rest...} be all remaining arguments besides \tcode{x}, \tcode{y}, and \tcode{z}.]
\end{itemize}

\pnumAdd
\wgAdd{\effects}
\wgAdd{Equivalent to: \tcode{\placeholder{math-func}(static_cast<const \exposid{deduced-vec-t}<V>\&>(args)..., rest...)}}
\end{itemdescr}
\end{wgText}


\appendix
\let\appendix=\relax
\section{\code{really_convertible_to} definition}\label{sec:really-convertible}
\medskip\begin{lstlisting}[style=Vc]
template <typename To, typename From>
  consteval bool converting_limits_throws()
  {
    try {
      using L = std::numeric_limits<From>;
      [[maybe_unused]] To x = L::max();
      x = L::min();
      x = L::lowest();
    } catch(...) {
      return true;
    }
    return false;
  }

template <typename From, typename To>
  concept really_convertible_to = std::convertible_to<From, To>
                                    and not converting_limits_throws<To, From>();
\end{lstlisting}


\end{document}
% vim: sw=2 sts=2 ai et tw=100
