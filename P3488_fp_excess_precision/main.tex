\newcommand\wgTitle{Floating-Point Excess Precision}
\newcommand\wgName{Matthias Kretz <m.kretz@gsi.de>}
\newcommand\wgDocumentNumber{D3488R0}
\newcommand\wgGroup{SG6, EWG}
\newcommand\wgTarget{\CC{}26}
%\newcommand\wgAcknowledgements{ }

\usepackage{mymacros}
\usepackage{wg21}
%\setcounter{tocdepth}{2} % show sections and subsections in TOC
%\hypersetup{bookmarksdepth=5}
\usepackage{changelog}
\usepackage{underscore}
%\usepackage{comment}

\addbibresource{extra.bib}

\renewcommand{\lst}[1]{Listing~\ref{#1}}
\renewcommand{\sect}[1]{Section~\ref{#1}}
\renewcommand{\ttref}[1]{Tony~Table~\ref{#1}}
\newcommand\fp{floating-point\xspace}
\newcommand\Fp{Floating-point\xspace}
\newcommand\discussionref[1]{\hyperref[d:#1]{\color{Headings}$\rightarrow$ Discussion}}

\begin{document}
\selectlanguage{american}
\begin{wgTitlepage}
  CWG2752 asks whether a conforming implementation can represent a \fp literal
  with excess precision.
  This issue was opened after GCC implemented excess precision for \CC{}.
  Notably, GCC also uses excess precision for evaluation at compile-time as
  shown in this paper.
  For a holistic answer this paper considers excess precision of constants and
  in evaluation.
  Therefore, the main question we need answered is whether literals must be
  rounded or can be stored with excess precision.
  The secondary question is the use of excess precision in constant expressions
  and in compile-time evaluation of floating-point operations.
  The goal is to find a consensus on what the design intent should be, without
  breaking performance or correctness requirements of \CC{} users.
  This paper lists possible design intent and discusses their implications on
  potential optimizations.
\end{wgTitlepage}

\pagestyle{scrheadings}

\section{Changelog}
(placeholder)
\begin{revision}
\item Add a simple example to the motivation section.
\item Expand the “Generalization” section to clearly define the feature rather
  than just sketching it.
  Also add a discussion of initial value and step.
\item Discuss why reusing the existing \code{iota} algorithm/view does not
  work/suffice for the \code{simd} use case.
\item Discuss why \code{iota_v} is the right name.
%  \todo
\end{revision}

\section{Straw Polls}
\subsection{SG1 at Kona 2022}
\wgPoll{After significant experience with the TS, we recommend that the next
version (the TS version with improvements) of \code{std::simd} target the IS (\CC{}26)}
{10&8&0&0&0}

\wgUnanimous{We like all of the recommended changes to \code{std::simd} proposed in p1928r1
(Includes making all of \code{std::simd} \code{constexpr}, and dropping an ABI stable type)}

\wgPoll{Future papers and future revisions of existing papers that target
\code{std::simd} should go directly to LEWG.
(We do not believe there are SG1 issues with \code{std::simd} today.)}
{9&8&0&0&0}


\section{Introduction}

(This is an extended/modified copy of CWG2752.)

Consider:

\begin{lstlisting}
int main()
{
  constexpr auto x = 3.14f;
  assert( x == 3.14f );         // can fail?
  static_assert( x == 3.14f );  // can fail?
}
\end{lstlisting}

Can a conforming implementation represent a floating-point literal with excess
precision, causing the comparisons to fail?

Subclause 5.13.4 [lex.fcon] paragraph 3 specifies:

\begin{wgText}[\CC{} {[lex.fcon]}]
  \setcounter{Paras}{2}\pnum
  If the scaled value is not in the range of representable values for its type,
  the program is ill-formed. Otherwise, the value of a floating-point-literal
  is the scaled value if representable, else the larger or smaller
  representable value nearest the scaled value, chosen in an
  implementation-defined manner.
\end{wgText}

This phrasing leaves little leeway for excess precision.
In contrast, C23 specifies:

\begin{wgText}[ISO/IEC 9899:2024 6.4.4.3 Floating constants]
  \setcounter{Paras}{5}\pnum
  The values of floating constants may be represented in greater range and
  precision than that required by the type (determined by the suffix); the
  types are not changed thereby. See 5.2.5.3.3 regarding evaluation
  formats.\footnote{Hexadecimal floating constants can be used to obtain exact
    values in the semantic type that are independent of the evaluation format.
    Casts produce values in the semantic type, though depend on the rounding
    mode and may raise the inexact floating-point exception.}
\end{wgText}

Subclause 7.1 [expr.pre] paragraph 6 uses very similar wording to allow excess
precision for floating-point computations (including their operands):

\begin{wgText}[\CC{} {[expr.pre]}]
  \setcounter{Paras}{5}\pnum
  The values of the floating-point operands and the results of floating-point
  expressions may be represented in greater precision and range than that
  required by the type; the types are not changed thereby.%
  \footnote{The cast and assignment operators must still perform their specific
    conversions as described in 7.6.1.4 [expr.type.conv], 7.6.3 [expr.cast],
    7.6.1.9 [expr.static.cast] and 7.6.19 [expr.ass].}
\end{wgText}

Taken together, that means that \code{314.f / 100.f} can be computed and
represented more precisely than \code{3.14f}, which is hard to justify.
The footnote appears to imply that \code{(float)3.14f} is required to yield a
value with \float precision, but that conversion (eventually) ends up at 9.4.1
[dcl.init.general] bullet 16.9:

\begin{wgText}[\CC{} {[dcl.init.general]}]
  \setcounter{Paras}{0}
  [\ldots]
  Otherwise, the initial value of the object being initialized is the (possibly
  converted) value of the initializer expression.
  [\ldots]
\end{wgText}

This phrasing leaves no permission to discard excess precision when converting
from a \float value to type \float ("[\ldots] is the value [\ldots]").

However, if initialization is intended to drop excess precision, then an
overloaded operator returning \float can never behave like a built-in operation
with excess precision, because returning a value means initializing the return
value.

The \CC{} standard library inherits the \code{FLT_EVAL_METHOD} macro from the C
standard library. C23 specifies it as follows:

\begin{wgText}[ISO/IEC 9899:2024 5.2.5.3.3 Characteristics of floating types \code{<float.h>}]
  \setcounter{Paras}{25}\pnum
  The values of floating type yielded by operators subject to the usual
  arithmetic conversions, including the values yielded by the implicit
  conversion of operands, and the values of floating constants are evaluated to
  a format whose range and precision may be greater than required by the type.
  Such a format is called an evaluation format.
  In all cases, assignment and cast operators yield values in the format of the
  type.
  The extent to which evaluation formats are used is characterized by the value
  of \code{FLT_EVAL_METHOD}:
  \begin{itemize}
    \item [-1] indeterminable;

    \item [0] evaluate all operations and constants just to the range and
      precision of the type;

    \item [1] evaluate operations and constants of type \float and \double to
      the range and precision of the \double type, evaluate \code{long double}
      operations and constants to the range and precision of the \code{long
      double} type;

    \item [2] evaluate all operations and constants to the range and precision
      of the \code{long double} type.
  \end{itemize}
  All other negative values for \code{FLT_EVAL_METHOD} characterize
  implementation-defined behavior.
  The value of \code{FLT_EVAL_METHOD} does not characterize values returned by
  function calls (see 6.8.7.5, F.6).
\end{wgText}

Taken together, a conforming \CC{} implementation cannot define
\code{FLT_EVAL_METHOD} to 1 or 2, because literals (= "constants") cannot be
represented with excess precision in \CC{}.

\subsection{Annex H of C23}

Annex H of C23 “specifies extension types for programming language C that have
the arithmetic interchange and extended floating-point formats specified in
ISO/IEC 60559”.

This annex modifies \code{FLT_EVAL_METHOD} and is relevant with regard to
discussion around evaluation of e.g. \code{std::float16_t} operations:
\begin{wgText}[ISO/IEC 9899:2024 H.3 Characteristics in \code{<float.h>}]
  \setcounter{Paras}{1}\pnum
  If \code{FLT_RADIX} is \code{2}, the value of \code{FLT_EVAL_METHOD}
  (5.2.5.3.3) characterizes the use of evaluation formats for standard floating
  types and for binary floating types:
  \begin{itemize}
    \item[\code{-1}] indeterminable;
    \item [\code 0] evaluate all operations and constants, whose semantic type
      comprises a set of values that is a strict subset of the values of
      \float, to the range and precision of \float; evaluate all other
      operations and constants to the range and precision of the semantic type;
    \item [\code 1] evaluate operations and constants, whose semantic type comprises
      a set of values that is a strict subset of the values of \double, to the
      range and precision of \double; evaluate all other operations and
      constants to the range and precision of the semantic type;
    \item [\code 2] evaluate operations and constants, whose semantic type comprises
      a set of values that is a strict subset of the values of \code{long
      double}, to the range and precision of \code{long double}; evaluate all
      other operations and constants to the range and precision of the semantic
      type;
    \item [$N$] where \code{_Float$N$} is a supported interchange floating
      type, evaluate operations and constants, whose semantic type comprises a
      set of values that is a strict subset of the values of \code{_Float$N$},
      to the range and precision of \code{_Float$N$}; evaluate all other
      operations and constants to the range and precision of the semantic type;
    \item [$N + \code{1}$] where \code{_Float$N$x} is a supported extended
      floating type, evaluate operations and constants, whose semantic type
      comprises a set of values that is a strict subset of the values of
      \code{_Float$N$x}, to the range and precision of \code{_Float$N$x};
      evaluate all other operations and constants to the range and precision of
      the semantic type.
  \end{itemize}
\end{wgText}

\subsection{Relevance of this issue}

This issue should be irrelevant for all environments where
\code{FLT_EVAL_METHOD} is \code{0}.
An example environment where \code{FLT_EVAL_METHOD} is non-zero is GCC
compiling with \code{-m32} or \code{-mfpmath=387}.
With GCC 13 or later and one of the mentioned compiler flags and e.g.
\code{-std=c++23} the above code example fails both the \code{static_assert}
and the runtime \code{assert}\footnote{\url{https://compiler-explorer.com/z/vrYoT5cer}}.

An example that exhibits different behavior for constant propagation /
expressions can also be constructed%
\footnote{\url{https://compiler-explorer.com/z/5KGoebo75}}:
\begin{lstlisting}
constexpr float a = 0x1.000003p0f; // this rounds to nearest
static_assert(a ==  0x1.000004p0f); // as expected

constexpr float b = 0x2.000005p0f; // this rounds to nearest
static_assert(b ==  0x2.000004p0f); // as expected

constexpr float b0 = 0x1.000002p0f + 0x1.000003p0f;
// -> without intermediate rounding: 0x2.000005p0f
//           -> subsequent rounding: 0x2.000004p0f (A)
// ->    with intermediate rounding: 0x2.000006p0f (B')
//           -> subsequent rounding: 0x2.000008p0f (B)
static_assert(b0 != 0x2.000004p0f); // (A)
static_assert(b0 == 0x2.000006p0f); // (B')
static_assert(b0 == 0x2.000008p0f); // (B)

constexpr float b1 = 0x1.000002p0f + a;
// same constants as 'b0' except rounding for 'a' is required
// -> 0x2.000006p0f -> subsequent rounding: 0x2.000008p0f
static_assert(b1 == 0x2.000008p0f);

constexpr float b2 = 0x1.000002p0f + a - 1.f;
// 0x2.000006p0f - 1 -> 0x1.000006p0f (C)
// 0x2.000006p0f rounds to 0x2.000008p0f -> subtract 1 -> 0x1.000008p0f (D)

static_assert(b2 != 0x1.000006p0f); // (C)
static_assert(b2 == 0x1.000008p0f); // (D)

constexpr float third = 1 / 3.f;
constexpr float five_third = 5 * third;
constexpr float five_third_ = 5 * (1 / 3.f);
static_assert(five_third == five_third_); // (E)
\end{lstlisting}
All of these static assertions hold on GCC, Clang, and MSVC as far as I tested
them, except when compiling with GCC 13 (and up) and the \code{-m32} flag
(targeting 32-bit x86).
There, the assertions marked \code{(A)}, \code{(B')} \code{(B)}, \code{(C)},
\code{(D)}, and \code{(E)} fail.
This is due to \code{FLT_EVAL_METHOD == 2} which GCC interprets as allowing /
requesting constants in \code{long double} precision.

\section{A plan on how to reach a conclusion}

Three steps:
\begin{enumerate}
  \item SG6 documents possible design intent and their implications.
    The group then makes a recommendation on how the issue should be resolved.
    Irrespective of whether a consensus is reached, the paper then progresses
    to EWG.

  \item EWG does what it does.
    Most importantly EWG is the group that has the final say in how this issue
    is resolved.

  \item CWG.
\end{enumerate}

\section{Choose a design intent}

This section only explains the options.
In other words, we want to be able to choose one of these and say “this is the
design intent”.
A discussion of the options follows in the next section.

\subsection{strictest: Disallow all excess precision}\label{o:1}

\begin{itemize}
  \item {[expr.pre]} must disallow greater precision / range in \fp expressions. (But not disallow floating-point contraction. See \sect{sec:fma}.)

  \item Consequently, \code{FLT_EVAL_METHOD} must always be \code{0}.
\end{itemize}
(This had strong SG6 support.)\\
\discussionref{1}

\subsection{compatible: Do exactly the same as C}\label{o:2}

\begin{itemize}
  \item \iref{lex.fcon} must allow representing \fp constants in greater range and
    precision.

  \item Evaluation of constant expressions and compile-time evaluation of
    expressions may use excess precision.

  \item Intermediate rounding in runtime and compile-time evaluation is
    reflected by \code{FLT_EVAL_METHOD}.
\end{itemize}
(This had no SG6 support.)\\
\discussionref{2}

\pagebreak
\subsection{like C but only for runtime evaluation}\label{o:3}

\begin{itemize}
  \item The value of a \fp literal is always rounded to the precision of its
    type (status quo of [lex.fcon]).

  \item Evaluation of floating-point expressions in constant expressions is not
    allowed to use excess precision.

  \item \code{FLT_EVAL_METHOD} only reflects on runtime evaluation of \fp
    expressions.

  \item Constant folding exhibits the same behavior as runtime evaluation.

  \item \Fp evaluation at runtime can use greater precision and range of
    a different \fp type and is only required to round to the precision and
    range of the \fp type on cast and assignment.
    The intermediate precision is exposed to the program via
    \code{FLT_EVAL_METHOD} (for now).

  \item We should consider adding a note to [expr.pre] saying that while excess
    precision in evaluation is allowed, it is only allowed for performance
    reasons and it is preferred that intermediate precision and range match the
    \fp type.
\end{itemize}
(This had some support in SG6.)\\
\discussionref{3}

\section{Discussion}

\subsection{Member Types}
The member types may not seem obvious.
Rationales:
\begin{typelist*}
  \item[value_type]
    In the spirit of the \valuetype member of STL containers, this type denotes the \emph{logical} type of the values in the vector.

  \item[register_value_type]
    On some targets it may be beneficial to implement \datapar instantiations of some \type T with a different type \type{register_value_type}, which has higher precision than \type T.
    This is mostly an implementation detail, but can be important to know in some situations, especially whenever \type{native_handle_type} is involved.

    \guidance{A better name might be \type{native_value_type}.}

  \item[native_handle_type]
    The type used for enabling access to an implementation-defined member object (via the \code{native_handle()} function).

  \item[reference]
    Used as the return type of the non-const scalar subscript operator.
    This may use implementation-defined means to solve possible type aliasing issues.

  \item[const_reference]
    Used as the return type of the const scalar subscript operator.
    From my experience with Vc, it is safest to actually not use a const lvalue reference here, but a temporary.

  \item[mask_type]
    The natural mask type for this \datapar instantiation.
    This type is used as return type of compares and write-mask on assignments.

  \item[size_type]
    Standard member type used for \code{size()} and \code{operator[]}.

  \item[target_type]
    The \type{Abi} template parameter to \datapar.

\end{typelist*}

\subsection{Conversions}
The \datapar conversion constructor only allows implicit conversion from \datapar template instantiations with the same \type{Abi} type and compatible \valuetype.
Discussion in SG1 showed clear preference for only allowing implicit conversion between integral types that only differ in signedness.
All other conversions could be implemented via an explicit conversion constructor.
The alternative (preferred) is to use \simdcast consistently for all other conversions.

\subsection{Broadcast Constructor}
The broadcast constructor is not declared as \code{explicit} to ease the use of scalar prvalues in expressions involving data-parallel operations.
The operations where such a conversion should not be implicit consequently need to use SFINAE / concepts to inhibit the conversion.

\subsection{Compound Assignment}
The semantics of compound assignment would allow less strict implicit conversion rules.
Consider \code{datapar<int>() *= double()}: the corresponding multiplication operator would not compile because the implicit conversion to \datapar[<float>] is non-portable.
Compound assignment, on the other hand, implies an implicit conversion back to the type of the expression on the left of the assignment operator.
Thus, it is possible to define compound operators that execute the operation correctly on the promoted type without sacrificing portability.
There are two arguments for not relaxing the rules for compound assignment, though:
\begin{enumerate}
  \item Consistency: The conversion of an expression with compound assignment to a binary operator suddenly would not compile anymore.
  \item The implicit conversion in the \code{int * double} case could be expensive and unintended.
    This is already a problem for builtin types where many developers multiply \float variables with \double prvalues.
\end{enumerate}

\subsection{Fundamental SIMD Type or Not?}
\subsubsection{The Issue}
There has been renewed discussion on the reflectors over the question whether \CC{} should define a fundamental, native SIMD type (let us call it \type{fundamental<T>}) and a generic data-parallel type on top which supports an arbitrary number of elements (call it \type{arbitrary<T, N>}).
The alternative to defining both types is to only define \type{arbitrary<T, N = default_size<T>>}, since it encompasses the \type{fundamental<T>} type.

With regard to this proposal this second approach would add a third template parameter to \datapar and \mask as shown in \lst{datapar N}.
\begin{lstlisting}[style=Vc,numbers=left,float,label=lst:datapar N,caption={
  Possible declaration of the class template parameters of a \datapar class with arbitrary width.
}]
template <class T, size_t N = datapar_size_v<T, datapar_abi::compatible>,
          class Abi = datapar_abi::compatible>
class datapar;
\end{lstlisting}

\subsubsection{Standpoints}
The controversy is about how the flexibility of a type with arbitrary \code N is presented to the users.
Is there a (clear) distinction between a “fundamental” type with target-dependent (i.e. fixed) \code N and a higher-level abstraction with arbitrary \code N which can potentially compile to inefficient machine code.
Or should the \CC{} standard only define \type{arbitrary} and set it to a default \code N value that corresponds to the target-dependent \code N.
Thus, the default \code N, of \type{arbitrary} would correspond to \type{fundamental}.

It is interesting to note that \type{arbitrary<T, 1>} is the class variant of \type T.
Consequently, if we say there is no need for a \type{fundamental} type then we could argue for the deprecation of the builtin arithmetic types, in favor of \type{arbitrary<T, 1>}. \wgNote{This is an academic discussion, of course.}

The author has implemented a library where a clear distinction is made between \type{fundamental<T, Abi>} and \type{arbitrary<T, N>}.
The documentation and all teaching material says that the user should program with \type{fundamental}.
The \type{arbitrary} type should be used in special circumstances, or wherever \type{fundamental} works with the \type{arbitrary} type in its interfaces (e.g. for gather \& scatter or the \code{ldexp} \& \code{frexp} functions).

\subsubsection{Issues}
The definition of two separate class templates can alleviate some source compatibility issues resulting from different \code N on different target systems.
Consider the simplest example of a multiplication of an \intt vector with a \float vector:
\smallskip\begin{lstlisting}[style=Vc]
arbitrary<float>() * arbitrary<int>();  // compiles for some targets, fails for others
fundamental<float>() * fundamental<int>();  // never compiles, requires explicit cast
\end{lstlisting}
The \datapar[<T>] operators specified in such a way that source compatibility is ensured.
For a type with user definable \code N, the binary operators should work slightly different with regard to implicit conversions.
Most importantly, \type{arbitrary<T, N>} solves the issue of portable code containing mixed integral and floating-point values.
A user would typically create aliases such as:
\smallskip\begin{lstlisting}[style=Vc]
using floatvec = datapar<float>;
using intvec = arbitrary<int, floatvec::size()>;
using doublevec = arbitrary<int, floatvec::size()>;
\end{lstlisting}
Objects of types \type{floatvec}, \type{intvec}, and \type{doublevec} will work together independent of the target system.

Obviously these type aliases are basically the same if the \code N parameter of \type{arbitrary} has a default value:
\smallskip\begin{lstlisting}[style=Vc]
using floatvec = arbitrary<float>;
using intvec = arbitrary<int, floatvec::size()>;
using doublevec = arbitrary<int, floatvec::size()>;
\end{lstlisting}
The ability to create these aliases is not the issue.
Seeing the need for using such a pattern is the issue.
Typically a developer will think no more of it if his code compiles on his machine.
If \code{arbitrary<float>() * arbitrary<int>()} just happens to compile (which is likely) then this is the code that will get checked in to the repository.
Note that with the existence of the \type{fundamental} class template, the \code N parameter of the \type{arbitrary} class would not have a default value and thus force the user to think a second longer about portability.

%Consider the \code{ldexp} function in \lst{ldexp}.
%\begin{lstlisting}[style=Vc,numbers=left,float,label=lst:ldexp,caption={
%  Declaration of \code{ldexp} for \type{fundamental} and \type{arbitrary} (only single-precision \float).
%}]
%template <class Abi>
%fundamental<float, Abi> ldexp(fundamental<float, Abi> x,
%                              arbitrary<int, fundamental<float, Abi>::size()> exp);
%template <int N>
%arbitrary<float, N> ldexp(arbitrary<float, N> x, arbitrary<int, N> exp);
%\end{lstlisting}
%%\lst{fundamental-ldexp}
%\begin{lstlisting}[style=Vc,numbers=left,float,label=lst:fundamental-ldexp,caption={
%  Calls to \code{ldexp} compile or fail to compile independent of the target.
%}]
%fundamental<float> a = ...;
%fundamental<int> exp = ...;
%a = ldexp(a, exp);  // compiles nowhere
%\end{lstlisting}
%%\lst{arbitrary-ldexp}
%\begin{lstlisting}[style=Vc,numbers=left,float,label=lst:arbitrary-ldexp,caption={
%  Calls to \code{ldexp} compile or fail to compile depending on the target.
%}]
%arbitrary<float> a = ...;
%arbitrary<int> exp = ...;
%a = ldexp(a, exp);  // compiles where a.size() == exp.size(), fails otherwise
%\end{lstlisting}
%
%
%The \type{fundamental} and \type{arbitrary} types have mostly the same interface.
%It is interesting to make \type{arbitrary} behave slightly differently, catering to the special use cases where \type{fundamental} is designed for safety.
%Consider implicit conversions (cf. \lst{implicit conversions}): \type{fundamental} must be very restrictive with regard to implicit conversions.
%\begin{lstlisting}[style=Vc,numbers=left,float,label=lst:implicit conversion,caption={
%  Implicit conversions behave differently for \type{fundamental} and \type{arbitrary}.
%}]
%auto x0 = fundamental<int>() * 1.f; // fails to compile since fundamental<int>::size()
%                                    // might be different to fundamental<float>::size()
%auto x1 = arbitrary<int>() * 1.f;   // x1 is of type arbitrary<float,
%                                    // arbitrary<int>::size()>
%\end{lstlisting}
%Implicit conversions are only allowed where \code{size()} is equal for the two types for each conceivable target systems.\footnote{
%  The safest approach is to disallow all implicit conversions.
%  Vc allows conversion between signed and unsigned integral types.
%}
%The \type{arbitrary} type, on the other hand, can allow implicit conversions to work much more alike to the fundamental arithmetic types.


\section{Floating-point contraction}

\Fp contraction is the transformation of \code{a * b + c} into
\code{std::fma(a, b, c)}.
This effectively increases the intermediate precision of the multiplication
result.
Thus, \fp contraction is related to this discussion.
[expr.pre] p6 appears to allow \fp contraction.

ISO/IEC 60559:2020 specifies
\begin{wgText}[ISO/IEC 60559:2020 10.4 Literal meaning and value-changing optimizations]
  A language standard should also define, and require implementations to
  provide, attributes that allow and disallow value-changing optimizations,
  separately or collectively, for a block.
  These optimizations might include, but are not limited to:
  \begin{itemize}
    \item Applying the associative or distributive laws.

    \item Synthesis of a \textbf{fusedMultiplyAdd} operation from a \textbf{multiplication} and
      an \textbf{addition}.

    \item Synthesis of a \textit{formatOf} operation from an operation and a conversion
      of the result of the operation.

    \item Use of wider intermediate results in expression evaluation.
  \end{itemize}
\end{wgText}

The fourth item is what this paper has been discussing so far.

The second item is considered a different optimization in the 60559 standard.
Therefore, we should also consider \fp contraction separately from
\code{FLT_EVAL_METHOD}.
It is unclear what the original intent for \fp contraction for \CC{} had been.
Existing practice is to default to \fp contraction as an optimization
independent of \code{FLT_EVAL_METHOD}.
Therefore, I suggest we ensure the wording matches existing practice.

Note that the 60559 wording talks about “attributes that allow and disallow
value-changing optimizations”.
\CC{} does not provide such attributes.
However, implementations typically provide them (e.g. as compiler flags
treating one complete translation unit as a “block”, but also as vendor
attributes that can be applied to functions).
This appears to follow the guidance in 60559 which says that if a language
standard doesn't define something it is implementation defined.

Consequently, I'd be wary of making \fp contraction non-conforming.
Rather we want to keep it as a conforming optimization and (for now) continue
to trust the implementations to provide the necessary “attributes” to control
\fp contraction.
Adding such an “attribute” to \CC{} itself is material for another paper, but
should not be done in a resolution to a core issue.

\subsection{Guaranteed opt-out of \fp contraction}

It appears that accoding to the footnote of [expr.pre] p6 the expression
\lstinline@a * b + c@ can be transformed into an FMA, whereas
\lstinline@auto(a * b) + c@ cannot.
Likewise \lstinline@auto ab = a * b; ab * c@ would not lead to \fp contraction.

It is unclear whether a simple \fp wrapper class would inhibit \fp contraction:
\medskip
\begin{lstlisting}
class Float
{
  float x;

public:
  Float(float xx) : x(xx) {}

  friend Float operator+(Float a, Float b) { return a.x. + b.x; }
  friend Float operator*(Float a, Float b) { return a.x. * b.x; }
};

Float test(Float a, Float b, Float c)
{ return a * b + c; } // is contraction allowed or not?
\end{lstlisting}

The copy constructor of \code{Float} implicitly assigns to the data member \code{x}.
But there is no assignment or cast expression.
The return statements in the binary operators of \code{Float} call the
\code{Float(float)} constructor which copies the \code{float} into \code{xx}
and subsequently into \code{x}.
Both copies are neither using a cast not assignment expression.
Consequently this wrapper class would still allow \fp contraction, correct?

With a minor change to the \code{Float(float)} constructor to
\medskip
\begin{lstlisting}
  Float(float xx) : x(float(xx)) {}
\end{lstlisting}
\fp contractions would be inhibited.

I believe we need to clarify whether this matches the intent and at least
add a note in the wording to explain this subtlety.



\section{Wording}

TBD.
But here's at least a sketch if we agree on adopting \ref{o:3}:

\begin{enumerate}
  \item Clarify [expr.pre] that it only provides this freedom for runtime
    evaluation.

  \item Clarify [expr.pre] that \fp contraction is a conforming transformation
    (but not required)

  \item Add the above \code{Float} class example to [expr.pre]?

  \item Stop inheriting \code{FLT_EVAL_METHOD} verbatim from C.
    We need to write our own wording that clarifies \code{FLT_EVAL_METHOD} only
    applies to runtime evaluation and not to constants.
    Also we need to consider adopting and adjusting the wording from Annex H,
    which is important for \code{std::float16_t} and \code{std::bfloat16_t}.
\end{enumerate}

\end{document}
% vim: sw=2 sts=2 ai et tw=0
