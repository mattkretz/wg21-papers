\newcommand\wgTitle{Floating-Point Excess Precision}
\newcommand\wgName{Matthias Kretz <m.kretz@gsi.de>}
\newcommand\wgDocumentNumber{P3488R0}
\newcommand\wgGroup{SG6, EWG}
\newcommand\wgTarget{\CC{}26}
%\newcommand\wgAcknowledgements{ }

\usepackage{mymacros}
\usepackage{wg21}
%\setcounter{tocdepth}{2} % show sections and subsections in TOC
%\hypersetup{bookmarksdepth=5}
\usepackage{changelog}
\usepackage{underscore}
%\usepackage{comment}

\addbibresource{extra.bib}

\renewcommand{\lst}[1]{Listing~\ref{#1}}
\renewcommand{\sect}[1]{Section~\ref{#1}}
\renewcommand{\ttref}[1]{Tony~Table~\ref{#1}}
\newcommand\fp{floating-point\xspace}
\newcommand\Fp{Floating-point\xspace}
\newcommand\discussionref[1]{\hyperref[d:#1]{\color{Headings}$\rightarrow$ Discussion}}

\begin{document}
\selectlanguage{american}
\begin{wgTitlepage}
  CWG2752 asks about the design intent for handling excess-precision
  floating-point computations in \CC{}.
  The main question we need answered is whether literals must be rounded or can
  be stored with excess precision.
  The secondary question is the use of excess precision in constant expressions
  and in compile-time evaluation of floating-point operations.
  The goal is to find a consensus on what the original design intent had been.
  This paper lists possible original intent and discusses their implications on
  potential optimizations.
\end{wgTitlepage}

\pagestyle{scrheadings}

\section{Changelog}
\begin{revision}
\item Target \CC{}26, addressing SG1 and LEWG.
\item Call for a merge of the (improved \& adjusted) TS specification to the IS.
\item Discuss changes to the ABI tags as consequence of TS experience; calls for polls to change the status quo.
\item Add template parameter \code{T} to \code{simd_abi::fixed_size}.
\item Remove \code{simd_abi::compatible}.
\item Add (but ask for removal) \code{simd_abi::abi_stable}.
\item Mention TS implementation in GCC releases.
\item Add more references to related papers.
\item Adjust the clause number for [numbers] to latest draft.
\item Add open question: what is the correct clause for [simd]?
\item Add open question: integration with ranges.
\item Add \code{simd_mask} generator constructor.
\item Consistently add simd and simd_mask to headings.
\item Remove experimental and parallelism_v2 namespaces.
\item Present the wording twice: with and without diff against N4808 (Parallelism TS 2).
\item Default load/store flags to \code{element_aligned}.
\item Generalize casts: conditionally \code{explicit} converting constructors.
\item Remove named cast functions.
\end{revision}

\begin{revision}
\item Add floating-point conversion rank to condition of \code{explicit} for converting constructors.
\item Call out different or equal semantics of the new ABI tags.
\item Update introductory paragraph of \sect{sec:changes}; R1 incorrectly kept the text from R0.
\item Define simd::size as a \code{constexpr} static data-member of type \code{integral_constant<size_t, N>}. This simplifies passing the size via function arguments and still be useable as a constant expression in the function body.
\item Document addition of \code{constexpr} to the API.
\item Add \code{constexpr} to the wording.
\item Removed ABI tag for passing \code{simd} over ABI boundaries.
\item Apply cast interface changes to the wording.
\item Explain the plan: what this paper wants to merge vs. subsequent papers for additional features. With an aim of minimal removal/changes of wording after this paper.
\item Document rationale and design intent for \code{where} replacement.
\end{revision}

\begin{revision}
\item Propose alternative to \code{hmin} and \code{hmax}.
\item Discuss \code{simd_mask} reductions wrt. consistency with \code{<bit>}. Propose better names to avoid ambiguity.
\item Remove \code{some_of}.
\item Add unary \code{\~{}} to \code{simd_mask}.
\item Discuss and ask for confirmation of masked ``overloads'' names and argument order.
\item Resolve inconsistencies wrt. \code{int} and \code{size_t}: Change \code{fixed_size} and \code{resize_simd} NTTPs from \code{int} to \code{size_t} (for consistency).
\item Discuss conversions on loads and stores.
\item Point to \cite{P2509R0} as related paper.
\item Generalize load and store from pointer to \code{contiguous_iterator}. (\sect{sec:contiguousItLoadStore})
\item Moved ``\code{element_reference} is overspecified'' to ``Open questions''.
\end{revision}

\begin{revision}
\item Remove wording diff.
\item Add std::simd to the paper title.
\item Update ranges integration discussion and mention formatting support via
  ranges (\sect{sec:formatting}).
\item Fix: pass iterators by value not const-ref.
\item Add lvalue-ref qualifier to subscript operators (\sect{sec:lvalue-subscript}).
\item Constrain \code{simd} operators: require operator to be well-formed on objects of \code{value_type} (\ref{sec:simd.unary}, \ref{sec:simd.binary}).
\item Rename mask reductions as decided in Issaquah.
\item Remove R3 ABI discussion and add follow-up question.
\item Add open question on first template parameter of \code{simd_mask} (\sect{sec:basicsimdmask}).
\item Overload loads and stores with mask argument (\ref{sec:simd.ctor}, \ref{sec:simd.copy}, \ref{sec:simd.mask.ctor}, \ref{sec:simd.mask.copy}).
\item Respecify \simd reductions to use a \mask argument instead of \code{const_where_expression} (\ref{sec:simd.reductions}).
\item Add \mask operators returning a \simd (\ref{sec:simd.mask.unary}, \ref{sec:simd.mask.conv})
\item Add conditional operator overloads as hidden friends to \simd and \mask
  (\ref{sec:simd.cond}, \ref{sec:simd.mask.cond}).
\item Discuss \std\code{hash} for \simd (\sect{sec:hash}).
\item Constrain some functions (e.g., min, max, clamp) to be \code{totally_ordered} (\ref{sec:simd.reductions}, \ref{sec:simd.alg}).
\item Asking for reconsideration of conversion rules.
\item Rename load/store flags (\sect{sec:renameandextendflags}).
\item Extend load/store flags with a new flag for conversions on load/store. (\sect{sec:renameandextendflags}).
\item Update \code{hmin}/\code{hmax} discussion with more extensive naming discussion (\sect{sec:hminhmax}).
\item Discuss freestanding \simd (\sect{sec:freestanding}).
\item Discuss \code{split} and \code{concat} (\sect{sec:splitandconcat}).
\item Apply the new library specification style from P0788R3.
\end{revision}

\begin{revision}
\item Added \code{simd_select} discussion.
\end{revision}

\begin{revision}
\item Updated the wording for changes discussed in and requested by LEWG in Varna.
\item Rename to \code{simd_cat} and \code{simd_split}.
\item Drop \code{simd_cat(array)} overload.
\item Replace \code{simd_split} by \code{simd_split} as proposed in P1928R4.
\item Use \code{indirectly_writable} instead of \code{output_iterator}.
\item Replace most \code{size_t} and \code{int} uses by \code{\textit{simd-size-type}} signed integer type.
\item Remove everything in \code{simd_abi} and the namespace itself.
\item Reword section on ABI tags using exposition-only ABI tag aliases.
\item Guarantee generator ctor calls callable exactly once per index.
\item Remove \code{int}/\code{unsigned int} exception from conversion rules of broadcast ctor.
\item Rename \code{loadstore_flags} to \code{simd_flags}.
\item Make \code{simd_flags::operator|} \code{consteval}.
\item Remove \code{simd_flags::operator\&} and \code{simd_flags::operator\^}.
\item Increase minimum SIMD width to 64.
\item Rename \code{hmin}/\code{hmax} to \code{reduce_min} and \code{reduce_max}.
\item Refactor \code{simd_mask<T, Abi>} to \code{basic_simd_mask<Bytes, Abi>} and replace all occurrences accordingly.
\item Rename \code{simd<T, Abi>} to \code{basic_simd<Bytes, Abi>} and replace all occurrences accordingly.
\item Remove \code{long double} from the set of vectorizable types.
\item Remove \code{is_abi_tag}, \code{is_simd}, and \code{is_simd_mask} traits.
\item Make \code{simd_size} exposition-only.
\end{revision}

\begin{revision}
\item Remove mask reduction precondition but ask LEWG for reversal of that decision (\sect{sec:removemaskreductionprecondition}).
\item Fix return type of \mask unary operators.
\item Fix \code{bool} overload of \simdselect (\sect{sec:simdselectwording}).
\item Remove unnecessary implementation freedom in \code{simd_split} (\sect{sec:bettersimdsplitwording}).
\item Use \code{class} instead of \code{typename} in template heads.
\item Implement LEWG decision to SFINAE on \emph{values} of
  constexpr-wrapper-like arguments to the broadcast ctor (\ref{sec:simd.ctor}).
\item Add relational operators to \mask as directed by LEWG (\ref{sec:simd.mask.comparison}).
\item Update section on \code{size_t} vs. \code{int} usage (\sect{sec:simdsizetype}).
\item Remove all open design questions, leaving LWG / wording questions.
\item Add LWG question on implementation note (\sect{sec:implnote}).
\item Add constraint for \code{BinaryOperation} to \code{reduce} overloads (\ref{sec:simd.reductions}).
%  \todo Add \code{numeric_limits} / numeric traits specializations since behavior of e.g. \code{simd<float>} and \code{float} may differ for reasonable implementations.
\end{revision}

\begin{revision}
\item Include \code{std::optional} return value from \code{reduce_min_index} and \code{reduce_max_index} in the exploration.
\item Fix \LaTeX{} markup errors.
\item Remove repetitive mention of “exposition-only” before \deducet.
\item Replace “TU” with “translation unit”.
\item Reorder first paragraphs in the wording, especially reducing the note on compiling down to SIMD instructions.
\item Replace cv-unqualified arithmetic types with a more precise list of types.
\item Move the place where “supported” is defined.
\end{revision}

\begin{revision}
\item Improve wording that includes the \CC{}23 extended floating-point types in the set of vectorizable types (\ref{wording.vectorizable.types}).
\item Improve wording that defines “selected indices” and “selected elements” (\ref{wording.selected.indices}).
\item Remove superfluous introduction paragraph.
\item Improve wording introducing the intent of ABI tags (\ref{wording.ABI.tag})
\item Consistently use \code{size} as a callable in the wording.
\item Add missing \code{type_identity_t} for \code{reduce} (\ref{sec:simd.syn}, \ref{sec:simd.reductions}).
\item Spell out “iff” (\ref{wording.deducet}).
\item Fixed template argument to \nativeabi\ in the default template argument of \code{basic_simd_mask} (\ref{sec:simd.syn}).
\item Fixed default template argument to \code{simd_mask} to be consistent with \code{simd} (\ref{sec:simd.syn}).
\item Add instructions to add \code{<simd>} to the table of headers in [headers].
\item Add instructions to add a new subclause to the table in [numerics.general].
\item Add instructions to add \code{<simd>} [diff.23.library].
\item Add \simdsizev to the wording and replace \code{simd_size_v} to actually implement “Make \code{simd_size} exposition-only.”
\item Restored precondition (and removed \code{noexcept}) on
  \code{reduce_min_index} and \code{reduce_max_index} as directed by LEWG.
\end{revision}

\begin{revision}
\item Strike through wording removed by P3275 (non-const \code{operator[]}).
\item Remove “exposition only” from detailed prose, it's already marked as such in the synopsis.
\item Reorder defintion of \emph{vectorizable type} above its first use.
\item Commas, de-duplication, word order, \code{s/may/can/} in a note.
\item Use text font for “[)” when defining a range of integers.
\item Several small changes from LWG review on 2024-06-26.
\item Reword \code{rebind_simd} and \code{resize_simd}.
\item Remove mention of implementation-defined load/store flags.
\item Remove paragraph about default initialization of \simd.
\item Reword all constructor \emph{Effects} from “Constructs an object \ldots”
  to “Initializes \ldots”.
\item Instead of writing “satisfies X” in \emph{Constraints} and “models X” in
  \emph{Preconditions}, say only “models X” in \emph{Constraints}.
\item Replace \code{is_trivial_v} with “is trivially copyable”.
\item First shot at improving generator function constraints.
\item Reword constraints on unary and binary operators.
\item Add missing/inconsistent \code{explicit} on load constructors.
\item Fix preconditions of subscript operators.
\item Reword effects of compound assignment operators.
\item Add that \code{BinaryOperation} may not modify input \simd.
\item Fix definition of GENERALIZED_SUMs.
\end{revision}

\begin{revision}
\item Say “\textit{op}” instead of “the indicated operator”
\item Fix constraints on shift operators with \simdsizetype{} on the right operand.
\item Remove wording removed by P3275 (non-const \code{operator[]}).
\item Make intrinsics conversion recommended practice.
\item Make \code{simd_flags} template arguments exposition-only.
\item Make \code{simd_alignment} \emph{not} implementation-defined.
\item Reword “supported” to “enabled or disabled”.
\item Apply improved wording from \ref{sec:simd.overview} to \ref{sec:simd.mask.overview}.
\item Add comments for LWG to address to broadcast ctor (\ref{sec:simd.ctor}).
\item Respecify generator ctor to not reuse broadcast constraint (\ref{sec:simd.ctor}).
\item Use \code{to_address} on contiguous iterators (\ref{sec:simd.ctor} and \ref{sec:simd.copy}).
  This is more explicit about allowing memcpy on the complete range rather than
  having to iterate the range per element.
\end{revision}

\begin{revision}
\item Fix default size of \code{simd} and \code{simd_mask} aliases
  (\ref{sec:simd.syn}, necessary for
  \std\code{destructible<\MayBreak{}\std{}simd<\MayBreak\std{}string>>} to be well-formed).
\item Extend value-preserving to encompass conversions from all arithmetic
  types. Use this new freedom in \ref{sec:simd.ctor} to fully constrain the
  generator constructor and to plug a specification hole in the broadcast
  constructor.
\item Fix broadcast constructor wording by constraining \constexprwrapperlike
  arguments to arithmetic types.
  %\todo Reorder \code{simd} and \code{simd_mask} specification in the wording (mask first).
\end{revision}

\section{Straw Polls}


\section{Introduction}

(This is an extended/modified copy of CWG2752.)

Consider:

\begin{lstlisting}
int main()
{
  constexpr auto x = 3.14f;
  assert( x == 3.14f );         // can fail?
  static_assert( x == 3.14f );  // can fail?
}
\end{lstlisting}

Can a conforming implementation represent a floating-point literal with excess
precision, causing the comparisons to fail?

Subclause 5.13.4 [lex.fcon] paragraph 3 specifies:

\begin{wgText}[\CC{} {[lex.fcon]}]
  \setcounter{Paras}{2}\pnum
  If the scaled value is not in the range of representable values for its type,
  the program is ill-formed. Otherwise, the value of a floating-point-literal
  is the scaled value if representable, else the larger or smaller
  representable value nearest the scaled value, chosen in an
  implementation-defined manner.
\end{wgText}

This phrasing leaves little leeway for excess precision.
In contrast, C23 specifies:

\begin{wgText}[ISO/IEC 9899:2024 6.4.4.3 Floating constants]
  \setcounter{Paras}{5}\pnum
  The values of floating constants may be represented in greater range and
  precision than that required by the type (determined by the suffix); the
  types are not changed thereby. See 5.2.5.3.3 regarding evaluation
  formats.\footnote{Hexadecimal floating constants can be used to obtain exact
    values in the semantic type that are independent of the evaluation format.
    Casts produce values in the semantic type, though depend on the rounding
    mode and may raise the inexact floating-point exception.}
\end{wgText}

Subclause 7.1 [expr.pre] paragraph 6 uses very similar wording to allow excess
precision for floating-point computations (including their operands):

\begin{wgText}[\CC{} {[expr.pre]}]
  \setcounter{Paras}{5}\pnum
  The values of the floating-point operands and the results of floating-point
  expressions may be represented in greater precision and range than that
  required by the type; the types are not changed thereby.%
  \footnote{The cast and assignment operators must still perform their specific
    conversions as described in 7.6.1.4 [expr.type.conv], 7.6.3 [expr.cast],
    7.6.1.9 [expr.static.cast] and 7.6.19 [expr.ass].}
\end{wgText}

Taken together, that means that \code{314.f / 100.f} can be computed and
represented more precisely than \code{3.14f}, which is hard to justify.
The footnote appears to imply that \code{(float)3.14f} is required to yield a
value with \float precision, but that conversion (eventually) ends up at 9.4.1
[dcl.init.general] bullet 16.9:

\begin{wgText}[\CC{} {[dcl.init.general]}]
  \setcounter{Paras}{0}
  [\ldots]
  Otherwise, the initial value of the object being initialized is the (possibly
  converted) value of the initializer expression.
  [\ldots]
\end{wgText}

This phrasing leaves no permission to discard excess precision when converting
from a \float value to type \float ("[\ldots] is the value [\ldots]").

However, if initialization is intended to drop excess precision, then an
overloaded operator returning \float can never behave like a built-in operation
with excess precision, because returning a value means initializing the return
value.

The \CC{} standard library inherits the \code{FLT_EVAL_METHOD} macro from the C
standard library. C23 specifies it as follows:

\begin{wgText}[ISO/IEC 9899:2024 5.2.5.3.3 Characteristics of floating types \code{<float.h>}]
  \setcounter{Paras}{25}\pnum
  The values of floating type yielded by operators subject to the usual
  arithmetic conversions, including the values yielded by the implicit
  conversion of operands, and the values of floating constants are evaluated to
  a format whose range and precision may be greater than required by the type.
  Such a format is called an evaluation format.
  In all cases, assignment and cast operators yield values in the format of the
  type.
  The extent to which evaluation formats are used is characterized by the value
  of \code{FLT_EVAL_METHOD}:
  \begin{itemize}
    \item [-1] indeterminable;

    \item [0] evaluate all operations and constants just to the range and
      precision of the type;

    \item [1] evaluate operations and constants of type \float and \double to
      the range and precision of the \double type, evaluate \code{long double}
      operations and constants to the range and precision of the \code{long
      double} type;

    \item [2] evaluate all operations and constants to the range and precision
      of the \code{long double} type.
  \end{itemize}
  All other negative values for \code{FLT_EVAL_METHOD} characterize
  implementation-defined behavior.
  The value of \code{FLT_EVAL_METHOD} does not characterize values returned by
  function calls (see 6.8.7.5, F.6).
\end{wgText}

Taken together, a conforming \CC{} implementation cannot define
\code{FLT_EVAL_METHOD} to 1 or 2, because literals (= "constants") cannot be
represented with excess precision in \CC{}.

\subsection{Annex H of C23}

Annex H of C23 “specifies extension types for programming language C that have
the arithmetic interchange and extended floating-point formats specified in
ISO/IEC 60559”.

This annex modifies \code{FLT_EVAL_METHOD} and is relevant with regard to
discussion around evaluation of e.g. \code{std::float16_t} operations:
\begin{wgText}[ISO/IEC 9899:2024 H.3 Characteristics in \code{<float.h>}]
  \setcounter{Paras}{1}\pnum
  If \code{FLT_RADIX} is \code{2}, the value of \code{FLT_EVAL_METHOD}
  (5.2.5.3.3) characterizes the use of evaluation formats for standard floating
  types and for binary floating types:
  \begin{itemize}
    \item[\code{-1}] indeterminable;
    \item [\code 0] evaluate all operations and constants, whose semantic type
      comprises a set of values that is a strict subset of the values of
      \float, to the range and precision of \float; evaluate all other
      operations and constants to the range and precision of the semantic type;
    \item [\code 1] evaluate operations and constants, whose semantic type comprises
      a set of values that is a strict subset of the values of \double, to the
      range and precision of \double; evaluate all other operations and
      constants to the range and precision of the semantic type;
    \item [\code 2] evaluate operations and constants, whose semantic type comprises
      a set of values that is a strict subset of the values of \code{long
      double}, to the range and precision of \code{long double}; evaluate all
      other operations and constants to the range and precision of the semantic
      type;
    \item [$N$] where \code{_Float$N$} is a supported interchange floating
      type, evaluate operations and constants, whose semantic type comprises a
      set of values that is a strict subset of the values of \code{_Float$N$},
      to the range and precision of \code{_Float$N$}; evaluate all other
      operations and constants to the range and precision of the semantic type;
    \item [$N + \code{1}$] where \code{_Float$N$x} is a supported extended
      floating type, evaluate operations and constants, whose semantic type
      comprises a set of values that is a strict subset of the values of
      \code{_Float$N$x}, to the range and precision of \code{_Float$N$x};
      evaluate all other operations and constants to the range and precision of
      the semantic type.
  \end{itemize}
\end{wgText}

\subsection{Relevance of this issue}

This issue should be irrelevant for all environments where
\code{FLT_EVAL_METHOD} is \code{0}.
An example environment where \code{FLT_EVAL_METHOD} is non-zero is GCC
compiling with \code{-m32} or \code{-mfpmath=387}.
With GCC 13 or later and one of the mentioned compiler flags and e.g.
\code{-std=c++23} the above code example fails both the \code{static_assert}
and the runtime \code{assert}\footnote{\url{https://compiler-explorer.com/z/vrYoT5cer}}.

An example that exhibits different behavior for constant propagation /
expressions can also be constructed%
\footnote{\url{https://compiler-explorer.com/z/5KGoebo75}}:
\begin{lstlisting}
constexpr float a = 0x1.000003p0f; // this rounds to nearest
static_assert(a ==  0x1.000004p0f); // as expected

constexpr float b = 0x2.000005p0f; // this rounds to nearest
static_assert(b ==  0x2.000004p0f); // as expected

constexpr float b0 = 0x1.000002p0f + 0x1.000003p0f;
// -> without intermediate rounding: 0x2.000005p0f
//           -> subsequent rounding: 0x2.000004p0f (A)
// ->    with intermediate rounding: 0x2.000006p0f (B')
//           -> subsequent rounding: 0x2.000008p0f (B)
static_assert(b0 != 0x2.000004p0f); // (A)
static_assert(b0 == 0x2.000006p0f); // (B')
static_assert(b0 == 0x2.000008p0f); // (B)

constexpr float b1 = 0x1.000002p0f + a;
// same constants as 'b0' except rounding for 'a' is required
// -> 0x2.000006p0f -> subsequent rounding: 0x2.000008p0f
static_assert(b1 == 0x2.000008p0f);

constexpr float b2 = 0x1.000002p0f + a - 1.f;
// 0x2.000006p0f - 1 -> 0x1.000006p0f (C)
// 0x2.000006p0f rounds to 0x2.000008p0f -> subtract 1 -> 0x1.000008p0f (D)

static_assert(b2 != 0x1.000006p0f); // (C)
static_assert(b2 == 0x1.000008p0f); // (D)

constexpr float third = 1 / 3.f;
constexpr float five_third = 5 * third;
constexpr float five_third_ = 5 * (1 / 3.f);
static_assert(five_third == five_third_); // (E)
\end{lstlisting}
All of these static assertions hold on GCC, Clang, and MSVC as far as I tested
them, except when compiling with GCC 13 (and up) and the \code{-m32} flag
(targeting 32-bit x86).
There, the assertions marked \code{(A)}, \code{(B')} \code{(B)}, \code{(C)},
\code{(D)}, and \code{(E)} fail.
This is due to \code{FLT_EVAL_METHOD == 2} which GCC interprets as allowing /
requesting constants in \code{long double} precision.

\section{A plan on how to reach a conclusion}

Three steps:
\begin{enumerate}
  \item SG6 documents the possible original intent and their implications for
    resolving CWG2752.
    The group then makes a recommendation on how the issue should be resolved.
    Irrespective of whether a consensus is reached, the paper then progresses
    to EWG.

  \item EWG does what it does.
    Most importantly EWG is the group that has the final say in how this issue
    is resolved.

  \item CWG.
\end{enumerate}

\section{Choose a design intent}

This section only explains the options.
In other words, we want to be able to choose one of these and say “this is the
design intent”.
A discussion of the options follows in the next section.

\subsection{strictest: Disallow all excess precision}\label{o:1}

\begin{itemize}
  \item {[expr.pre]} must disallow greater precision / range in \fp expressions. (But not disallow floating-point contraction. See \sect{sec:fma}.)

  \item Consequently, \code{FLT_EVAL_METHOD} must always be \code{0}.
\end{itemize}
(This had strong SG6 support.)\\
\discussionref{1}

\subsection{compatible: Do exactly the same as C}\label{o:2}

\begin{itemize}
  \item \iref{lex.fcon} must allow representing \fp constants in greater range and
    precision.

  \item Evaluation of constant expressions and compile-time evaluation of
    expressions may use excess precision.

  \item Intermediate rounding in runtime and compile-time evaluation is
    reflected by \code{FLT_EVAL_METHOD}.
\end{itemize}
(This had no SG6 support.)\\
\discussionref{2}

\pagebreak
\subsection{like C but only for runtime evaluation}\label{o:3}

\begin{itemize}
  \item The value of a \fp literal is always rounded to the precision of its
    type (status quo of [lex.fcon]).

  \item Evaluation of floating-point expressions in constant expressions is not
    allowed to use excess precision.

  \item \code{FLT_EVAL_METHOD} only reflects on runtime evaluation of \fp
    expressions.

  \item Constant folding exhibits the same behavior as runtime evaluation.

  \item \Fp evaluation at runtime can use greater precision and range of
    a different \fp type and is only required to round to the precision and
    range of the \fp type on cast and assignment.
    The intermediate precision is exposed to the program via
    \code{FLT_EVAL_METHOD} (for now).

  \item We should consider adding a note to [expr.pre] saying that while excess
    precision in evaluation is allowed, it is only allowed for performance
    reasons and it is preferred that intermediate precision and range match the
    \fp type.
\end{itemize}
(This had some support in SG6.)\\
\discussionref{3}

\section{Discussion}

A general observation:
A simplification where the implementation were free to use excess precision at
runtime as it deems best would lead to suprising results:
Consider two floating-point values \code{a} and \code{b} where
\code{std::isfinite(b)} is statically known to be \code{true}.
With arbitrary excess precision the optimizer would then be allowed to replace
\code{a + b - b} with \code{a}.

A general consequence of excess precision is that \fp evaluation leads to
double rounding and thus potentially worse errors.
However, for \code{+-*/} and \code{sqrt}, if there are twice as many precision
bits in the intermediate type, then double rounding after each operation does
not lead to additional errors.
Where the second rounding occurs is not fully reproducible and can potentially
change via unrelated code changes in the translation unit\footnote{e.g. because of register allocation}.

Without excess precision \code{std::float16_t} and \code{std::bfloat16_t} can
either use a soft-float implementation, round back after every \code{float32_t}
operation, or dedicated hardware is required.
Using \float (binary32) instructions without rounding back down after every
operation is impossible with the current possible values for
\code{FLT_EVAL_METHOD}.
An implementation that wants to evaluate \std\code{float16_t} /
\std\code{bfloat16_t} in higher intermediate precision needs to set
\code{FLT_EVAL_METHOD} to 1 or 2 (or 32?).

\subsection{strictest: Disallow all excess precision}\label{d:1}

I believe [expr.pre] p6 is fairly clear that it was never the design intent to
exclude all excess precision.

Implications of disallowing all excess precision:
\begin{itemize}
  \item The x87 FPU cannot be used with a single “precision control” value,
    because double rounding is not correct (e.g. FPU configured to 80-bit with
    subsequent rounding to 64/32-bit).
    This implies that the compiler would have to set the x87 floating-point
    control word (FPCW) using the FLDCW instruction whenever it needs to
    execute \fp operations (with different precision).

  \item However, the x87 FPU is not really an important target anymore. Every x86
    CPU since the last 20 years can use SSE instructions instead.
\end{itemize}

\subsection{compatible: Do exactly the same as C}\label{d:2}

It might have been the original intent to do the same as C, but [lex.fcon] p3
suggests otherwise.

Implications of adopting this as resolution:
\begin{itemize}
  \item \code{x + 3.14f;} can require 8, 12, 16, or even more bytes to store
    the constant in the resulting binary.
    (This is the status quo of GCC since version 13.)

  \item \code{float x = 3.14f; assert(x == 3.14f);} is allowed to fail
    depending on implementation, target, and compiler flags.
    (This is the status quo of GCC since version 13.)
\end{itemize}

\subsection{like C but only for runtime evaluation}\label{d:3}

\begin{itemize}
  \item The intent here appears to be that we want to prescribe reproducible
    \fp behavior.

  \item However, since that has potentially dramatic consequences on runtime
    performance, this restriction is only a recomendation for runtime
    evaluation.
    We thus acknowledge the existence of hardware where reproducible \fp
    behavior comes at unreasonable performance cost.
    Because of these cases --- and only for these --- [expr.pre] allows excess
    precision in evaluation, which should be reflected by non-zero
    \code{FLT_EVAL_METHOD}.

  \item We should consider a new type trait along the lines of
    \begin{lstlisting}
template <floating-point T>
struct evaluation_type {
  using type = @\seebelow@;
};

template <floating-point T>
using evaluation_type_t = typename evaluation_type<T>::type;
    \end{lstlisting}
    Where e.g. \code{evaluation_type_t<float16_t>} could be \code{float}.
    This would supersede the use of the \code{FLT_EVAL_METHOD}.
    Implementations could then reasonably set \code{FLT_EVAL_METHOD} to
    \code{-1} and rely solely on the traits for reflection of \fp evaluation
    behavior.
\end{itemize}


\section{Floating-point contraction}

\Fp contraction is the transformation of \code{a * b + c} into
\code{std::fma(a, b, c)}.
This effectively increases the intermediate precision of the multiplication
result.
Thus, \fp contraction is related to this discussion.
[expr.pre] p6 appears to allow \fp contraction.

ISO/IEC 60559:2020 specifies
\begin{wgText}[ISO/IEC 60559:2020 10.4 Literal meaning and value-changing optimizations]
  A language standard should also define, and require implementations to
  provide, attributes that allow and disallow value-changing optimizations,
  separately or collectively, for a block.
  These optimizations might include, but are not limited to:
  \begin{itemize}
    \item Applying the associative or distributive laws.

    \item Synthesis of a \textbf{fusedMultiplyAdd} operation from a \textbf{multiplication} and
      an \textbf{addition}.

    \item Synthesis of a \textit{formatOf} operation from an operation and a conversion
      of the result of the operation.

    \item Use of wider intermediate results in expression evaluation.
  \end{itemize}
\end{wgText}

The fourth item is what this paper has been discussing so far.

The second item is considered a different optimization in the 60559 standard.
Therefore, we should also consider \fp contraction separately from
\code{FLT_EVAL_METHOD}.
It is unclear what the original intent for \fp contraction for \CC{} had been.
Existing practice is to default to \fp contraction as an optimization
independent of \code{FLT_EVAL_METHOD}.
Therefore, I suggest we ensure the wording matches existing practice.

Note that the 60559 wording talks about “attributes that allow and disallow
value-changing optimizations”.
\CC{} does not provide such attributes.
However, implementations typically provide them (e.g. as compiler flags
treating one complete translation unit as a “block”, but also as vendor
attributes that can be applied to functions).
This appears to follow the guidance in 60559 which says that if a language
standard doesn't define something it is implementation defined.

Consequently, I'd be wary of making \fp contraction non-conforming.
Rather we want to keep it as a conforming optimization and (for now) continue
to trust the implementations to provide the necessary “attributes” to control
\fp contraction.
Adding such an “attribute” to \CC{} itself is material for another paper, but
should not be done in a resolution to a core issue.

\subsection{Guaranteed opt-out of \fp contraction}

It appears that accoding to the footnote of [expr.pre] p6 the expression
\lstinline@a * b + c@ can be transformed into an FMA, whereas
\lstinline@auto(a * b) + c@ cannot.
Likewise \lstinline@auto ab = a * b; ab * c@ would not lead to \fp contraction.

It is unclear whether a simple \fp wrapper class would inhibit \fp contraction:
\medskip
\begin{lstlisting}
class Float
{
  float x;

public:
  Float(float xx) : x(xx) {}

  friend Float operator+(Float a, Float b) { return a.x. + b.x; }
  friend Float operator*(Float a, Float b) { return a.x. * b.x; }
};

Float test(Float a, Float b, Float c)
{ return a * b + c; } // is contraction allowed or not?
\end{lstlisting}

The copy constructor of \code{Float} implicitly assigns to the data member \code{x}.
But there is no assignment or cast expression.
The return statements in the binary operators of \code{Float} call the
\code{Float(float)} constructor which copies the \code{float} into \code{xx}
and subsequently into \code{x}.
Both copies are neither using a cast not assignment expression.
Consequently this wrapper class would still allow \fp contraction, correct?

With a minor change to the \code{Float(float)} constructor to
\medskip
\begin{lstlisting}
  Float(float xx) : x(float(xx)) {}
\end{lstlisting}
\fp contractions would be inhibited.

I believe we need to clarify whether this matches the intent and at least
add a note in the wording to explain this subtlety.



\section{Wording}

TBD.
But here's at least a sketch if we agree on adopting \ref{o:3}:

\begin{enumerate}
  \item Clarify [expr.pre] that it only provides this freedom for runtime
    evaluation.

  \item Clarify [expr.pre] that \fp contraction is a conforming transformation
    (but not required)

  \item Add the above \code{Float} class example to [expr.pre]?

  \item Stop inheriting \code{FLT_EVAL_METHOD} verbatim from C.
    We need to write our own wording that clarifies \code{FLT_EVAL_METHOD} only
    applies to runtime evaluation and not to constants.
    Also we need to consider adopting and adjusting the wording from Annex H,
    which is important for \code{std::float16_t} and \code{std::bfloat16_t}.
\end{enumerate}

\end{document}
% vim: sw=2 sts=2 ai et tw=0
