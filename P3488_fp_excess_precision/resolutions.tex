\section{Choose a design intent}

This section only explains the options.
In other words, we want to be able to choose one of these and say “this is the
design intent”.
A discussion of the options follows in the next section.

\subsection{strictest: Disallow all excess precision}\label{o:1}

\begin{itemize}
  \item {[expr.pre]} must disallow greater precision / range in \fp expressions. (But not disallow floating-point contraction. See \sect{sec:fma}.)

  \item Consequently, \code{FLT_EVAL_METHOD} must always be \code{0}.
\end{itemize}
(This had strong SG6 support.)\\
\discussionref{1}

\subsection{compatible: Do exactly the same as C}\label{o:2}

\begin{itemize}
  \item \iref{lex.fcon} must allow representing \fp constants in greater range and
    precision.

  \item Evaluation of constant expressions and compile-time evaluation of
    expressions may use excess precision.

  \item Intermediate rounding in runtime and compile-time evaluation is
    reflected by \code{FLT_EVAL_METHOD}.
\end{itemize}
(This had no SG6 support.)\\
\discussionref{2}

\pagebreak
\subsection{like C but only for runtime evaluation}\label{o:3}

\begin{itemize}
  \item The value of a \fp literal is always rounded to the precision of its
    type (status quo of [lex.fcon]).

  \item Evaluation of floating-point expressions in constant expressions is not
    allowed to use excess precision.

  \item \code{FLT_EVAL_METHOD} only reflects on runtime evaluation of \fp
    expressions.

  \item Constant folding exhibits the same behavior as runtime evaluation.

  \item \Fp evaluation at runtime can use greater precision and range of
    a different \fp type and is only required to round to the precision and
    range of the \fp type on cast and assignment.
    The intermediate precision is exposed to the program via
    \code{FLT_EVAL_METHOD} (for now).

  \item We should consider adding a note to [expr.pre] saying that while excess
    precision in evaluation is allowed, it is only allowed for performance
    reasons and it is preferred that intermediate precision and range match the
    \fp type.
\end{itemize}
(This had some support in SG6.)\\
\discussionref{3}
