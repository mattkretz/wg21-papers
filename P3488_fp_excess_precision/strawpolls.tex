\section{Straw Polls}
\subsection{SG6 at Wrocław 2024}
\wgPoll{Constant expressions (not constant folding) may not use greater 
precision and range than the type. $\Rightarrow$ Reproducibility in constant expressions. 
If you need more precision and range use \code{std::float128_t} in constant 
expressions.}
{ 4 & 2 & 0 & 0 & 0 }

\wgPoll{Constant folding behavior follows runtime evaluation behavior.}
{ 5 & 1 & 0 & 0 & 0 }

\wgPoll{Literals are always rounded to their types. (Never can a higher 
precision and range value be used in subsequent operations. \code{1.1f + x} where 
\code{x} is a \code{double} cannot be evaluated as \code{1.1 + x}.)}
{ 3 & 3 & 0 & 0 & 0 }

\wgPoll{We should consider not allowing excess precision in runtime 
evaluation at all (except if we introduce an "attribute" that controls this, as 
documented in 60559). Every binary operation on floating-point is rounded once 
to the precision and range of its type (\code{decltype(expr)}).}
{ 2 & 3 & 0 & 1 & 0 }

\wgPoll{Excess precision in runtime evaluation should be a permissible 
optimization for implementations. The implementation choice of what 
intermediate type was used is reflected by \code{FLT_EVAL_METHOD} (potentially as a 
different trait or with the extended values as defined in C23 Annex H).}
{ 1 & 2 & 1 & 2 & 0 }

\wgPoll{Comparison operators should always round their operands to the 
precision and range of their types in case excess precision is used. In 
addition to operators we expect the \code{<cmath>} comparison functions to have the 
same behavior (\code{std::isless(a, b)} etc...)}
{ 3 & 0 & 2 & 0 & 1 }

SA: Not worth differing from C in this. x87 does comparison on long double 
types.

