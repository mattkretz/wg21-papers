\section{Discussion}

A simplification where the implementation were free to use excess precision at
runtime as it deems best would lead to suprising results:
Consider two floating-point values \code{a} and \code{b} where
\code{std::isfinite(b)} is statically known to be \code{true}.
With arbitrary excess precision the optimizer would then be allowed to replace
\code{a + b - b} with \code{a}.

\subsection{strictest: Disallow all excess precision}\label{d:1}

I believe [expr.pre] p6 is fairly clear that it was never the design intent to
exclude all excess precision.

Implications:
\begin{itemize}
  \item \Fp contraction into FMAs is non-conforming.

  \item The x87 FPU cannot be used with a single “precision control” value,
    because double rounding is not correct (e.g. FPU configured to 80-bit with
    subsequent rounding to 64/32-bit).
    This implies that the compiler would have to set the x87 floating-point
    control word (FPCW) using the FLDCW instruction whenever it needs to
    execute \fp operations (with different precision).

  \item This is likely an ABI break and unnacceptable for existing
    implementations.
\end{itemize}

\subsection{compatible: Do exactly the same as C}\label{d:2}

This may have been the original intent, but [lex.fcon] p3 suggests otherwise.

\begin{itemize}
  \item \code{float x = 3.14f;} can require 8, 12, 16, or even more bytes to be
    stored in the resulting binary.

  \item \code{float x = 3.14f; assert(x == 3.14f);} is allowed to fail
    depending on implementation, target, and compiler flags.

  \item \code{std::float16_t} and \code{std::bfloat16_t} can either use a
    soft-float implementation or requires dedicated hardware.
    Double rounding, by using binary32 instructions is impossible with
    \code{FLT_EVAL_METHOD == 0}.
    An implementation that wants to evaluate
    \code{std::float16_t}/\code{std::bfloat16_t} in higher intermediate
    precision needs to set \code{FLT_EVAL_METHOD} to 1 or 2 (or 32?).
\end{itemize}

\subsection{like C but only for run-time evaluation}\label{d:3}

\begin{itemize}
  \item The intent here appears to be that we want to prescribe reproducible
    \fp behavior.
    In other words, all floating-point code that is \emph{not} evaluated at
    run-time is reproducible.

  \item However, we acknowledge the existence of hardware where this comes at
    unreasonable performance cost. Because of these cases --- and only for
    these --- the non-zero \code{FLT_EVAL_METHOD} modes exist.

  \item Consequence: evaluation at higher precision leads to double rounding
    and thus potentially worse errors.
\end{itemize}
