\newcommand\wgTitle{\code{constant\_wrapper} is the only tool needed for passing constant expressions
  via function arguments}
\newcommand\wgName{Matthias Kretz <m.kretz@gsi.de>}
\newcommand\wgDocumentNumber{P3948R0}
\newcommand\wgGroup{LEWG}
\newcommand\wgTarget{\CC{}26}
%\newcommand\wgAcknowledgements{Daniel Towner and Ruslan Arutyunyan contributed to this paper via discussions / reviews. Thanks also to Jeff Garland for reviewing.}

\usepackage{mymacros}
\usepackage{wg21}
\setcounter{tocdepth}{2} % show sections and subsections in TOC
\hypersetup{bookmarksdepth=5}
\usepackage{changelog}
\usepackage{underscore}
\usepackage{multirow}

\addbibresource{extra.bib}

\newcommand\wglink[1]{\href{https://wg21.link/#1}{#1}}

\renewcommand{\lst}[1]{Listing~\ref{#1}}
\renewcommand{\sect}[1]{Section~\ref{#1}}
\renewcommand{\ttref}[1]{Tony~Table~\ref{#1}}
\renewcommand{\tabref}[1]{Table~\ref{#1}}

\begin{document}
\selectlanguage{american}
\begin{wgTitlepage}
  We added \code{std::constant_wrapper<value>} to \CC{}26 to enable “passing constant expressions as
  function arguments”.
  This is especially important for constructors where we cannot give explicit template arguments.
  However, \code{constant_wrapper} also overloads all operators in such a way that values stay in
  the “type space”.
  This was initially motivated by user-defined literals, which require a unary minus operator.
  For consistency, LEWG voted to overload all operators.
  The existence of \code{constant_wrapper::operator()} lead to fear, uncertainty, and
  doubt\footnote{not intentional FUD, though}, whether wrapping a callable with
  \code{constant_wrapper} leads to inconsistent semantics.
  This paper shows that there is no real problem, and provides a simple solution to resolve the
  contrived problems that can be constructed.
\end{wgTitlepage}

\pagestyle{scrheadings}

\section{Changelog}
(placeholder)
\begin{revision}
\item Add a simple example to the motivation section.
\item Expand the “Generalization” section to clearly define the feature rather
  than just sketching it.
  Also add a discussion of initial value and step.
\item Discuss why reusing the existing \code{iota} algorithm/view does not
  work/suffice for the \code{simd} use case.
\item Discuss why \code{iota_v} is the right name.
%  \todo
\end{revision}

\section{Straw Polls}
\subsection{SG1 at Kona 2022}
\wgPoll{After significant experience with the TS, we recommend that the next
version (the TS version with improvements) of \code{std::simd} target the IS (\CC{}26)}
{10&8&0&0&0}

\wgUnanimous{We like all of the recommended changes to \code{std::simd} proposed in p1928r1
(Includes making all of \code{std::simd} \code{constexpr}, and dropping an ABI stable type)}

\wgPoll{Future papers and future revisions of existing papers that target
\code{std::simd} should go directly to LEWG.
(We do not believe there are SG1 issues with \code{std::simd} today.)}
{9&8&0&0&0}


%\pagebreak
\section{Motivation}

The primary design intent of \std\code{constant_wrapper<x>} is the ability to pass \code{x} as a
template argument via a function (esp. constructor) argument.
The presence of a second type \std\code{constant_arg} that has the same design intent but with extra
semantics and a narrow focus on \std\code{function_ref} is just inconsistent and hard to teach.
After reviewing the reasons why LEWG decided to not use \std\code{constant_wrapper} I got only more
convinced that we should use \std\code{constant_wrapper} for \std\code{function_ref}.

The status quo of \CC{}26 can be understood with this
example:\footnote{\url{https://compiler-explorer.com/z/8z7M6q7PM}}
\medskip\begin{lstlisting}
#include <functional>
#include <type_traits>

int nullary();
int unary(int);
int binary(int, int);

int f0(std::function_ref<int()>);
int f1(std::function_ref<int(int)>);
int f2(std::function_ref<int(int, int)>);
int g0(std::move_only_function<int()>);
int g1(std::move_only_function<int(int)>);
int g2(std::move_only_function<int(int, int)>);

void test()
{
  f0(nullary);          // @\ref{en:nullary}@
  f0(std::cw<nullary>); // @\ref{en:cw:nullary}@
  f1(std::cw<unary>);   // @\ref{en:cw:unary}@
  f2(std::cw<binary>);  // @\ref{en:cw:binary}@
  f0(std::cw<[]() { return 1; }>); // @\ref{en:cw:lambda}@
  g0(nullary);          // @\ref{en:g}@
  g0(std::cw<nullary>); // @\ref{en:g}@
  g1(std::cw<unary>);   // @\ref{en:g}@
  g2(std::cw<binary>);  // @\ref{en:g}@
  g0(std::cw<[]() { return 1; }>); // @\ref{en:g}@

  // these can't work because of a language limitation:
  //f1(std::cw<[](int) { return 1; }>);
  //f2(std::cw<[](int, int) { return 1; }>);
  //g1(std::cw<[](int) { return 1; }>);
  //g2(std::cw<[](int, int) { return 1; }>);
}
\end{lstlisting}

The commented calls are ill-formed because of a language inconsistency:
A call expression on an object that is implicitly convertible to a function pointer is valid
(calling through the function pointer).
However, if such an object is implicitly convertible to a type implementing a call operator, that
call is not considered.
This is similar to operator lookup for fundamental types vs. user-defined types: the former requires
only a conversion operator, the latter is required to be in an associated namespace (which
\code{constant_wrapper} does) and the operator needs to be defined as a hidden friend (which the
language doesn't allow for call operators).

This is unfortunate, but not really important for this discussion.
This discussion wants to focus on what \code{cw<fun>} already means \emph{right now} without adopting any
other paper.

\begin{enumerate}

  \item\label{en:nullary}
    \code{f0(nullary)} is a simple type-erased call through a function pointer into \code{nullary}.

  \item\label{en:cw:nullary}
    \code{f0(cw<nullary>)} is translated into a \emph{direct} jump to \code{nullary}; i.e., this
    doesn't need an indirect call.\footnote{This assumes enabled optimizations; with \code{-O0} both
    \code{cw<nullary>} and \code{constant_arg<nullary>} are compiled into an indirect call.}
    This is possible because the function pointer is passed as a template argument into the place
    where the \code{function_ref} implementation constructs the call.
    (Note that \code{constant_wrapper::operator()} is not invoked, because this operator is not
    viable in a constant expression.)
    Also, note that \code{f0(constant_arg<nullary>)} is equivalent.
    The only difference is that with \code{cw} a non-null value is stored for the function pointer,
    even though that value is never needed.

  \item\label{en:cw:unary}\label{en:cw:binary}
    \code{f1(cw<unary>)} and \code{f2(cw<binary>)} are the same as \code{f0(cw<nullary>)} with one
    or two function arguments.
    Both also result in a direct jump when invoking the \code{function_ref}.

  \item\label{en:cw:lambda}
    \code{f0(cw<[]() \{ return 1; \})} shows that a nullary lambda also works (using
    \code{-fno-inline}: \url{https://compiler-explorer.com/z/vP5YMea8v}).
    In this case the expression \code{cw<[]() \{ re\-turn 1; \}>()} returns \code{cw<1>} because it is
    a valid constant expression invoking \code{constant_wrapper::\MayBreak{}operator()}.
    Contrary to \code{nullary}, the lambda is \code{constexpr} and can thus be used to initialize
    the \code{cw} template argument.
    \code{cw<1>} is subsequently converted to \code{int(1)} when the \code{function_ref} is invoked.
    This is functionally equivalent to \code{f0(constant_arg<[]() \{ return 1; \}>)}, which also
    returns constant \code{int(1)}, except that the implementation doesn't need to go through
    \textit{INVOKE}.

    Important here is the observation that if \code{constant_wrapper::operator()} is well-formed,
    then its \emph{semantics are equivalent} to unwrapping the \code{constant_wrapper} and then
    calling the function.
    The difference is equivalent to whether \code{function_ref::operator()} is implemented as
    \medskip\begin{lstlisting}[style=Vc]
    int operator()() { constexpr int tmp = wrapped(); return tmp; }
    \end{lstlisting}
    or
    \medskip\begin{lstlisting}[style=Vc]
    int operator()() { return wrapped(); }
    \end{lstlisting}

  \item\label{en:g}
    All of these conversions from \code{cw<…>} to \code{function_ref} also work for
    \code{move_only_function} (and also \code{copyable_function} and \code{function}).
    Consider the difference of \code{g0(nullary)} vs. \code{g0(cw<nullary>)}
    (\url{https://compiler-explorer.com/z/8nYhYP7G5}):
    \medskip\begin{lstlisting}[language=asm]
"int std::__polyfunc::_Base_invoker<false, int>::_S_call_storage< …"
    jmp     [QWORD PTR [rdi]]
    \end{lstlisting}
    \medskip\begin{lstlisting}[language=asm]
"int std::__polyfunc::_Base_invoker<false, int>::_S_call_storage< …"
    mov     eax, 2
    ret
    \end{lstlisting}
    Using \code{constant_wrapper}, the indirect call/jump is replaced by an inlined call to
    \code{nullary()}.
    \code{constant_wrapper} thus already enables an optimization similar to the \code{constant_arg}
    overload in \code{function_ref}.
    The same code-gen difference exists for \code{copyable_function} and \code{function}.
    (If \code{nullary()} cannot be inlined, then the \code{cw<nullary>} case turns into a direct
    call/jump to \code{nullary()}.)

\end{enumerate}

In short: \textbf{\code{cw<func>} already works like \code{constant_arg} for all function wrappers.}
Special casing \code{constant_arg} for the non-bind constructor is thus only a minor optimization
for non-optimized builds.

There was a concern that the function pointer to \code{nullary} would be dangling when
\code{cw<nullary>} is used.
I do not know where this concern stems from.
To the contrary, \code{constant_wrapper} is carefully designed in such a way that its
\code{value} member is a “never dangling” object, since it is a const-ref to a template parameter
object.
Thus, indirect calls with \code{-O0} are also valid.

At this point we should be able to agree that using \code{std::cw} to wrap functions works correctly
and can have a run-time performance benefit (at a compile-time cost).
Except \ldots{} what about the failure that was presented in \cite{P3792R0}?

\subsection{Constructing an inconsistency}

Consider the following code\footnote{\url{https://compiler-explorer.com/z/59EMracKT}} as introduced
by \cite{P3792R0}, but with a minor adjustment to the assertion.
The assertion is modified because \cite{P3792R0} argues about broken consistency between the
different function wrappers and the assertion used in this paper actually shows the issue:
\medskip\begin{lstlisting}
#include <functional>
#include <type_traits>
#include <cassert>

static constexpr struct foo_t final
{
  constexpr auto operator()(auto &&...args) const -> int // #1
      requires(std::integral<std::remove_cvref_t<decltype(args)>> && ...)
  {
    return (0 + ... + args);
  }

  constexpr auto operator()(auto &&...args) const -> int // #2
    requires(std::integral<decltype(std::remove_cvref_t<decltype(args)>::value)> && ...)
  {
    return sizeof...(args);
  }

} foo = {};

static constexpr struct baz_t final
{
  static constexpr int value = 42;
} baz = {};

auto main() -> int
{
  std::function_ref<int(baz_t)> f0(std::cw<foo>);
  std::move_only_function<int(baz_t)> f1(std::cw<foo>);
  assert(f0(baz) == f1(baz));
}
\end{lstlisting}

This is currently (\CC{}26 status quo) consistent in that both \code{f0} and \code{f1} return
\code{42}.
Since it isn't obvious how the above code works, note that \code{\#2} is an unnecessary overload at
this point.
What \code{cw<callable>} requires in order to work in this context is a \code{callable} that is
either a function pointer/reference or an object with call operator whose arguments can be unwrapped
(\code{value} member), passed to the call operator, and the result can be wrapped in
\code{constant_wrapper} again.
(\code{con\-stant_wrapper<f>::operator()(Args...)} returns
\code{constant_wrapper<f(Args::value...)>})
So it must be a constant expression; and currently the language requires all types to be structural.
Therefore, \code{std::cw<foo>(baz)} is \code{std::cw<0 + 42>}.
In other words, the code \code{function_ref<int(baz_t)>(cw<foo>)} is what I would call a useless
obfuscation for saying \code{42}.
%The combination of \code{foo} and the function signature imply that the result is \code{42}.

This is more apparent in a simpler example such as:
\medskip\begin{lstlisting}[style=Vc]
int f(std::function_ref<int(std::constant_wrapper<3.f>)>; // #1
int func(std::constant_wrapper<3.f>); // could also use 'auto' as function parameter type
                                      // if we were still able to name it
int test1() { return f(std::cw<func>); }
\end{lstlisting}
This is pointless.
The point of passing a callable with function parameters is to call it with *unknown values* for
those parameters.
But here we had to define the value that it gets called with as \emph{part of the type}.
In line \code{\#1} we could have just used a nullary \code{int()} signature.

Nevertheless, let us consider the possibility of a type with a data member that also derives from a
trait with a \code{static constexpr} \code{value} member (this is the closest I can imagine to such
an issue turning up in real code).
Now, if we change \code{function_ref} to unwrap \code{constant_wrapper}, thereby replacing
\code{nontype}/\code{constant_arg}, then the assertion in the above example fails.
That's because if \code{constant_wrapper} is unconditionally unwrapped by \code{function_ref} it now
calls \code{foo(baz)} rather than \code{cw<foo>(baz)}.
And since the overload set of \code{foo_t} is constructed such that this works and returns something
different (\code{\#2}), it returns \code{1}.
However, since \code{move_only_function} does not unwrap \code{constant_wrapper} the assertion comes
down to \code{1 == 42} and fails.

\section{Resolving the inconsistency}

The actual problem in the \cite{P3792R0} example is an ambiguity/dual semantics: Did the user mean
to call \code{cw<foo>(baz)} or \code{foo(baz)}.
It is fairly simply to detect this potential mismatch in \code{function_ref} and reject the program.
Rejecting is consistent with overload resolution where ambiguities are rejected.

I implemented \code{function_ref} unwrapping \code{constant_wrapper} with such a detection in a
libstdc++ fork.\footnote{%
\url{https://forge.sourceware.org/mkretz/gcc/commit/3277bbbc4e4aaad60294a6cda491ff5cf41c62d8}}
The relevant code for rejecting the ambiguous case is straightforward:
\medskip\begin{lstlisting}[style=Vc]
template <auto __fn, typename... _Args>
  concept __constant_wrapped_invocable
    = sizeof...(_Args) >= 1 && requires(_Args... __args) {
      { cw<__fn>(__args...) } -> _ConstantWrapped;
    };

template<_ConstantWrapped __fn>
  requires __is_invocable_using<const typename __fn::value_type&>
  constexpr
  function_ref(__fn) noexcept
  {
    static_assert(!__constant_wrapped_invocable<__fn::value, _ArgTypes...>,
                  "The given cw<fn> object has ambiguous call semantics.");
    // […]
\end{lstlisting}

The corresponding wording change for this constructor overload looks like this:

\begin{wgText}
  \setcounter{Paras}{7}
\begin{itemdecl}
@template<\wgChange{auto f}{class F0}> constexpr function_ref(\wgChange{nontype_t<f>}{F0 f0})@ noexcept;
\end{itemdecl}

\begin{itemdescr}
\pnum
\wgAdd{Let \tcode{f} be \tcode{f0.value}.}
Let \tcode{F} be \tcode{decltype(f)}.

\pnum
\constraints
\begin{itemize}
  \item \tcode{\exposid{is-invocable-using}<F>} is \tcode{true}\wgAdd{, and}
  \wgItemAdd[\tcode{F0} is a specialization of \code{constant_wrapper}].
\end{itemize}

\pnum
\mandates
\begin{itemize}
  \item If \tcode{is_pointer_v<F> || is_member_pointer_v<F>} is \tcode{true},
    then \tcode{f != nullptr} is \tcode{true}.

  \wgItemAdd[\tcode{sizeof...(ArgTypes)} is $0$ or
    the expression \tcode{f0(declval<ArgTypes>()...)} is either not well-formed or its type is not a
  specialization of \tcode{constant_wrapper}.]
\end{itemize}

\pnum
\effects
Initializes
\exposid{bound-entity} with a pointer to an unspecified object or
null pointer value, and
\exposid{thunk-ptr} with the address of a function \tcode{\placeholder{thunk}}
such that
\tcode{\placeholder{thunk}(\exposid{bound-entity}, \placeholder{call-args}...)}
is expression-equivalent\iref{defns.expression.equivalent} to
\tcode{invoke_r<R>(f, \placeholder{call-args}...)}.
\end{itemdescr}
\end{wgText}

With this change one cannot construct inconsistent \emph{behavior} anymore.
If it would be inconsistent it rather does not compile.
This resolves the concern raised in §3.5 of \cite{P3792R0}:
\begin{quote}{}
  “
  \begin{itemize}
    \item if nontype was replaced by \code{constant_wrapper} as a construction parameter to
      \code{function_ref} and
    \item a user performed a (seemingly) simple refactoring by swapping a standard function wrapper
      with another standard function wrapper (e.g. to benefit from the low cost of
      \code{function_ref} or to use the data storage in other wrappers) where the constructor
        happens to rely on \code{constant_wrapper}
  \end{itemize}
… any of the following might happen:
  \begin{itemize}
    \item the program will continue to work as designed
    \item the program will fail to compile
    \item \st{the program will continue to compile and “work”, but with subtly changed behaviour}
  \end{itemize}
  ”
\end{quote}

\section{Summary}
Status quo inconsistencies:
\begin{itemize}
  \item \code{constant_arg<fun>} works with \code{function_ref} but not with any other function wrapper.

  \item \code{cw<fun>} works with all function wrappers to an equivalent effect of
    \code{constant_arg<fun>} in \code{function_ref}, duplicating the \code{function_ref} constructor.

  \item \code{constant_wrapper} and \code{constant_arg} are tools (a.k.a. workarounds) for passing a
    “constexpr value” via a function/constructor argument.
    Why are there two?
    How do their names tell me when to use one or the other?
\end{itemize}

\section{Proposal}
\begin{itemize}
  \item Reword \code{function_ref} from \code{nontype}/\code{constant_arg} to
    \code{constant_wrapper} (disallowing ambiguous \code{constant_wrapper} use).

  \item Drop \code{nontype}/\code{constant_arg}.

  \item Optional: Mandate no \code{constant_wrapper} arguments to the other function wrappers for
    concern of future breaking changes. (not recommended: so far I am convinced potential changes
    can be non-breaking)
\end{itemize}

\section{Further/related work}
The current inconsistencies for “transparent” wrappers due to the language rules are not helpful.
However, changes in this area can easily be breaking changes for existing code.
Still, it should be possible to opt in to defining a truly transparent wrapper type (where
operator lookup of the wrapped type is fully transparent).
This needs more exploratory work.
An initial patch to GCC to call \code{operator()} for wrappers like \code{constant_wrapper} already
feels more consistent (to me).

\section{Wording}

\begin{wgText}
  \setcounter{Paras}{4}
\begin{itemdecl}
template<class F> constexpr function_ref(F&& f) noexcept;
\end{itemdecl}

\begin{itemdescr}
\pnum
Let \tcode{T} be \tcode{remove_reference_t<F>}.

\pnum
\constraints
\begin{itemize}
\item \tcode{remove_cvref_t<F>} is not the same type as \tcode{function_ref},
\item \tcode{is_member_pointer_v<T>} is \tcode{false},\wgRem{ and}
\item \tcode{\exposid{is-invocable-using}<\cv{} T\&>} is \tcode{true}\wgAdd{, and}
\wgItemAdd[\tcode{remove_cvref_t<F>} is not a specialization of \tcode{constant_wrapper}].
\end{itemize}

\pnum
\effects
Initializes
\exposid{bound-entity} with \tcode{addressof(f)}, and
\exposid{thunk-ptr} with the address of a function \tcode{\placeholder{thunk}}
such that
\tcode{\placeholder{thunk}(\exposid{bound-entity}, \placeholder{call-args}...)}
is expression-equivalent\iref{defns.expression.equivalent} to
\tcode{invoke_r<R>(static_cast<\cv{} T\&>(f), \placeholder{call-args}...)}.
\end{itemdescr}

\begin{itemdecl}
@template<\wgChange{auto f}{class F0}> constexpr function_ref(\wgChange{nontype_t<f>}{F0 f0})@ noexcept;
\end{itemdecl}

\begin{itemdescr}
\pnum
\wgAdd{Let \tcode{f} be \tcode{f0.value} and}
\wgRem{Let} \tcode{F} be \tcode{decltype(f)}.

\pnum
\constraints
\begin{itemize}
  \item \tcode{\exposid{is-invocable-using}<F>} is \tcode{true}\wgAdd{, and}
  \wgItemAdd[\tcode{F0} is a specialization of \code{constant_wrapper}].
\end{itemize}

\pnum
\mandates
\begin{itemize}
  \item If \tcode{is_pointer_v<F> || is_member_pointer_v<F>} is \tcode{true},
    then \tcode{f != nullptr} is \tcode{true}.

  \wgItemAdd[\tcode{sizeof...(ArgTypes)} is $0$ or
    the expression \tcode{f0(declval<ArgTypes>()...)} is either not well-formed or its type is not a
  specialization of \tcode{constant_wrapper}.]
\end{itemize}

\pnum
\effects
Initializes
\exposid{bound-entity} with a pointer to an unspecified object or
null pointer value, and
\exposid{thunk-ptr} with the address of a function \tcode{\placeholder{thunk}}
such that
\tcode{\placeholder{thunk}(\exposid{bound-entity}, \placeholder{call-args}...)}
is expression-equivalent\iref{defns.expression.equivalent} to
\tcode{invoke_r<R>(f, \placeholder{call-args}...)}.
\end{itemdescr}

\begin{itemdecl}
template<@\wgChange{auto f}{class F0}@, class U>
  constexpr function_ref(@\wgChange{nontype_t<f>}{F0 f0}@, U&& obj) noexcept;
\end{itemdecl}

\begin{itemdescr}
\pnum
Let \tcode{T} be \tcode{remove_reference_t<U>}\wgAdd{,
\tcode{f} be \tcode{f0.value},} and
\tcode{F} be \tcode{decltype(f)}.

\pnum
\constraints
\begin{itemize}
\item \tcode{is_rvalue_reference_v<U\&\&>} is \tcode{false},\wgRem{ and}
\item \tcode{\exposid{is-invocable-using}<F, \cv{} T\&>} is \tcode{true}\wgAdd{, and}
\wgItemAdd[\tcode{F0} is a specialization of \code{constant_wrapper}].
\end{itemize}

\pnum
\mandates
If \tcode{is_pointer_v<F> || is_member_pointer_v<F>} is \tcode{true},
then \tcode{f != nullptr} is \tcode{true}.

\pnum
\effects
Initializes
\exposid{bound-entity} with \tcode{addressof(obj)}, and
\exposid{thunk-ptr} with the address of a function \tcode{\placeholder{thunk}}
such that
\tcode{\placeholder{thunk}(\exposid{bound-entity}, \placeholder{call-args}...)}
is expression-equivalent\iref{defns.expression.equivalent} to
\tcode{invoke_r<R>(f, static_cast<\cv{} T\&>(obj), \placeholder{call-args}...)}.
\end{itemdescr}

\begin{itemdecl}
template<@\wgChange{auto f}{class F0}@, class T>
  constexpr function_ref(@\wgChange{nontype_t<f>}{F0 f0}@, @\cv{}@ T* obj) noexcept;
\end{itemdecl}

\begin{itemdescr}
\pnum
\wgAdd{Let \tcode{f} be \tcode{f0.value} and}
\wgRem{Let} \tcode{F} be \tcode{decltype(f)}.

\pnum
\constraints
\begin{itemize}
\item \tcode{\exposid{is-invocable-using}<F, \cv{} T*>} is \tcode{true}\wgAdd{, and}
\wgItemAdd[\tcode{F0} is a specialization of \code{constant_wrapper}].
\end{itemize}

\pnum
\mandates
If \tcode{is_pointer_v<F> || is_member_pointer_v<F>} is \tcode{true},
then \tcode{f != nullptr} is \tcode{true}.

\pnum
\expects
If \tcode{is_member_pointer_v<F>} is \tcode{true},
\tcode{obj} is not a null pointer.

\pnum
\effects
Initializes
\exposid{bound-entity} with \tcode{obj}, and
\exposid{thunk-ptr} with the address of a function \tcode{\placeholder{thunk}}
such that
\tcode{\placeholder{thunk}(\exposid{bound-entity}, \placeholder{call-args}...)}
is expression-equivalent\iref{defns.expression.equivalent} to
\tcode{invoke_r<R>(f, obj, \placeholder{call-args}...)}.
\end{itemdescr}

\begin{itemdecl}
template<class T> function_ref& operator=(T) = delete;
\end{itemdecl}

\begin{itemdescr}
\pnum
\constraints
\begin{itemize}
\item \tcode{T} is not the same type as \tcode{function_ref},
\item \tcode{is_pointer_v<T>} is \tcode{false}, and
\item \tcode{T} is not a specialization of \wgChange{\tcode{nontype_t}}{\tcode{constant_wrapper}}.
\end{itemize}
\end{itemdescr}
\end{wgText}

\end{document}
% vim: sw=2 sts=2 ai et tw=100
