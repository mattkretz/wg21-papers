\newcommand\wgTitle{simd issues: explicit, unsequenced, identity-element position, and members of
disabled simd}
\newcommand\wgName{Matthias Kretz <m.kretz@gsi.de>}
\newcommand\wgDocumentNumber{D3430R1}
\newcommand\wgGroup{LEWG}
\newcommand\wgTarget{\CC{}26}
%\newcommand\wgAcknowledgements{Daniel Towner and Ruslan Arutyunyan contributed to this paper via discussions / reviews. Thanks also to Jeff Garland for reviewing.}

\usepackage{mymacros}
\usepackage{wg21}
\setcounter{tocdepth}{2} % show sections and subsections in TOC
\hypersetup{bookmarksdepth=5}
\usepackage{changelog}
\usepackage{underscore}
\usepackage{multirow}

\addbibresource{extra.bib}

\newcommand\simd[1][]{\type{ba\-sic\_simd#1}\xspace}
\newcommand\simdT{\type{ba\-sic\_simd\MayBreak<\MayBreak{}T>}\xspace}
\newcommand\valuetype{\type{val\-ue\_type}\xspace}
\newcommand\referencetype{\type{ref\-er\-ence}\xspace}
\newcommand\mask[1][]{\type{ba\-sic\_simd\_mask#1}\xspace}
\newcommand\maskT{\type{ba\-sic\_simd\_mask\MayBreak<\MayBreak{}T>}\xspace}
\newcommand\wglink[1]{\href{https://wg21.link/#1}{#1}}

\newcommand\nativeabi{\UNSP{native-abi}}
\newcommand\deducet{\UNSP{deduce-t}}
\newcommand\simdsizev{\UNSP{simd-size-v}}
\newcommand\simdsizetype{\UNSP{simd-size-type}}
\newcommand\simdselect{\UNSP{simd-select-impl}}
\newcommand\maskelementsize{\UNSP{mask-element-size}}
\newcommand\integerfrom{\UNSP{integer-from}}
\newcommand\constexprwrapperlike{\UNSP{constexpr-wrapper-like}}
\newcommand\convertflag{\UNSP{convert-flag}}
\newcommand\alignedflag{\UNSP{aligned-flag}}
\newcommand\overalignedflag{\UNSP{overaligned-flag}}
\newcommand\reductionoperation{\UNSP{reduction-binary-operation}}
\newcommand\simdfloatingpoint{\UNSP{simd-floating-point}}
\newcommand\multisimdfloatingpoint{\UNSP{multi-arg-simd-floating-point}}

\renewcommand{\lst}[1]{Listing~\ref{#1}}
\renewcommand{\sect}[1]{Section~\ref{#1}}
\renewcommand{\ttref}[1]{Tony~Table~\ref{#1}}
\renewcommand{\tabref}[1]{Table~\ref{#1}}

\begin{document}
\selectlanguage{american}
\begin{wgTitlepage}
  This paper collects all issues that came up in LWG review of P1928 (merge
  \code{std::simd}), which require LEWG approval.
\end{wgTitlepage}

\pagestyle{scrheadings}

\section{Changelog}
\begin{revision}
\item Target \CC{}26, addressing SG1 and LEWG.
\item Call for a merge of the (improved \& adjusted) TS specification to the IS.
\item Discuss changes to the ABI tags as consequence of TS experience; calls for polls to change the status quo.
\item Add template parameter \code{T} to \code{simd_abi::fixed_size}.
\item Remove \code{simd_abi::compatible}.
\item Add (but ask for removal) \code{simd_abi::abi_stable}.
\item Mention TS implementation in GCC releases.
\item Add more references to related papers.
\item Adjust the clause number for [numbers] to latest draft.
\item Add open question: what is the correct clause for [simd]?
\item Add open question: integration with ranges.
\item Add \code{simd_mask} generator constructor.
\item Consistently add simd and simd_mask to headings.
\item Remove experimental and parallelism_v2 namespaces.
\item Present the wording twice: with and without diff against N4808 (Parallelism TS 2).
\item Default load/store flags to \code{element_aligned}.
\item Generalize casts: conditionally \code{explicit} converting constructors.
\item Remove named cast functions.
\end{revision}

\begin{revision}
\item Add floating-point conversion rank to condition of \code{explicit} for converting constructors.
\item Call out different or equal semantics of the new ABI tags.
\item Update introductory paragraph of \sect{sec:changes}; R1 incorrectly kept the text from R0.
\item Define simd::size as a \code{constexpr} static data-member of type \code{integral_constant<size_t, N>}. This simplifies passing the size via function arguments and still be useable as a constant expression in the function body.
\item Document addition of \code{constexpr} to the API.
\item Add \code{constexpr} to the wording.
\item Removed ABI tag for passing \code{simd} over ABI boundaries.
\item Apply cast interface changes to the wording.
\item Explain the plan: what this paper wants to merge vs. subsequent papers for additional features. With an aim of minimal removal/changes of wording after this paper.
\item Document rationale and design intent for \code{where} replacement.
\end{revision}

\begin{revision}
\item Propose alternative to \code{hmin} and \code{hmax}.
\item Discuss \code{simd_mask} reductions wrt. consistency with \code{<bit>}. Propose better names to avoid ambiguity.
\item Remove \code{some_of}.
\item Add unary \code{\~{}} to \code{simd_mask}.
\item Discuss and ask for confirmation of masked ``overloads'' names and argument order.
\item Resolve inconsistencies wrt. \code{int} and \code{size_t}: Change \code{fixed_size} and \code{resize_simd} NTTPs from \code{int} to \code{size_t} (for consistency).
\item Discuss conversions on loads and stores.
\item Point to \cite{P2509R0} as related paper.
\item Generalize load and store from pointer to \code{contiguous_iterator}. (\sect{sec:contiguousItLoadStore})
\item Moved ``\code{element_reference} is overspecified'' to ``Open questions''.
\end{revision}

\begin{revision}
\item Remove wording diff.
\item Add std::simd to the paper title.
\item Update ranges integration discussion and mention formatting support via
  ranges (\sect{sec:formatting}).
\item Fix: pass iterators by value not const-ref.
\item Add lvalue-ref qualifier to subscript operators (\sect{sec:lvalue-subscript}).
\item Constrain \code{simd} operators: require operator to be well-formed on objects of \code{value_type} (\ref{sec:simd.unary}, \ref{sec:simd.binary}).
\item Rename mask reductions as decided in Issaquah.
\item Remove R3 ABI discussion and add follow-up question.
\item Add open question on first template parameter of \code{simd_mask} (\sect{sec:basicsimdmask}).
\item Overload loads and stores with mask argument (\ref{sec:simd.ctor}, \ref{sec:simd.copy}, \ref{sec:simd.mask.ctor}, \ref{sec:simd.mask.copy}).
\item Respecify \simd reductions to use a \mask argument instead of \code{const_where_expression} (\ref{sec:simd.reductions}).
\item Add \mask operators returning a \simd (\ref{sec:simd.mask.unary}, \ref{sec:simd.mask.conv})
\item Add conditional operator overloads as hidden friends to \simd and \mask
  (\ref{sec:simd.cond}, \ref{sec:simd.mask.cond}).
\item Discuss \std\code{hash} for \simd (\sect{sec:hash}).
\item Constrain some functions (e.g., min, max, clamp) to be \code{totally_ordered} (\ref{sec:simd.reductions}, \ref{sec:simd.alg}).
\item Asking for reconsideration of conversion rules.
\item Rename load/store flags (\sect{sec:renameandextendflags}).
\item Extend load/store flags with a new flag for conversions on load/store. (\sect{sec:renameandextendflags}).
\item Update \code{hmin}/\code{hmax} discussion with more extensive naming discussion (\sect{sec:hminhmax}).
\item Discuss freestanding \simd (\sect{sec:freestanding}).
\item Discuss \code{split} and \code{concat} (\sect{sec:splitandconcat}).
\item Apply the new library specification style from P0788R3.
\end{revision}

\begin{revision}
\item Added \code{simd_select} discussion.
\end{revision}

\begin{revision}
\item Updated the wording for changes discussed in and requested by LEWG in Varna.
\item Rename to \code{simd_cat} and \code{simd_split}.
\item Drop \code{simd_cat(array)} overload.
\item Replace \code{simd_split} by \code{simd_split} as proposed in P1928R4.
\item Use \code{indirectly_writable} instead of \code{output_iterator}.
\item Replace most \code{size_t} and \code{int} uses by \code{\textit{simd-size-type}} signed integer type.
\item Remove everything in \code{simd_abi} and the namespace itself.
\item Reword section on ABI tags using exposition-only ABI tag aliases.
\item Guarantee generator ctor calls callable exactly once per index.
\item Remove \code{int}/\code{unsigned int} exception from conversion rules of broadcast ctor.
\item Rename \code{loadstore_flags} to \code{simd_flags}.
\item Make \code{simd_flags::operator|} \code{consteval}.
\item Remove \code{simd_flags::operator\&} and \code{simd_flags::operator\^}.
\item Increase minimum SIMD width to 64.
\item Rename \code{hmin}/\code{hmax} to \code{reduce_min} and \code{reduce_max}.
\item Refactor \code{simd_mask<T, Abi>} to \code{basic_simd_mask<Bytes, Abi>} and replace all occurrences accordingly.
\item Rename \code{simd<T, Abi>} to \code{basic_simd<Bytes, Abi>} and replace all occurrences accordingly.
\item Remove \code{long double} from the set of vectorizable types.
\item Remove \code{is_abi_tag}, \code{is_simd}, and \code{is_simd_mask} traits.
\item Make \code{simd_size} exposition-only.
\end{revision}

\begin{revision}
\item Remove mask reduction precondition but ask LEWG for reversal of that decision (\sect{sec:removemaskreductionprecondition}).
\item Fix return type of \mask unary operators.
\item Fix \code{bool} overload of \simdselect (\sect{sec:simdselectwording}).
\item Remove unnecessary implementation freedom in \code{simd_split} (\sect{sec:bettersimdsplitwording}).
\item Use \code{class} instead of \code{typename} in template heads.
\item Implement LEWG decision to SFINAE on \emph{values} of
  constexpr-wrapper-like arguments to the broadcast ctor (\ref{sec:simd.ctor}).
\item Add relational operators to \mask as directed by LEWG (\ref{sec:simd.mask.comparison}).
\item Update section on \code{size_t} vs. \code{int} usage (\sect{sec:simdsizetype}).
\item Remove all open design questions, leaving LWG / wording questions.
\item Add LWG question on implementation note (\sect{sec:implnote}).
\item Add constraint for \code{BinaryOperation} to \code{reduce} overloads (\ref{sec:simd.reductions}).
%  \todo Add \code{numeric_limits} / numeric traits specializations since behavior of e.g. \code{simd<float>} and \code{float} may differ for reasonable implementations.
\end{revision}

\begin{revision}
\item Include \code{std::optional} return value from \code{reduce_min_index} and \code{reduce_max_index} in the exploration.
\item Fix \LaTeX{} markup errors.
\item Remove repetitive mention of “exposition-only” before \deducet.
\item Replace “TU” with “translation unit”.
\item Reorder first paragraphs in the wording, especially reducing the note on compiling down to SIMD instructions.
\item Replace cv-unqualified arithmetic types with a more precise list of types.
\item Move the place where “supported” is defined.
\end{revision}

\begin{revision}
\item Improve wording that includes the \CC{}23 extended floating-point types in the set of vectorizable types (\ref{wording.vectorizable.types}).
\item Improve wording that defines “selected indices” and “selected elements” (\ref{wording.selected.indices}).
\item Remove superfluous introduction paragraph.
\item Improve wording introducing the intent of ABI tags (\ref{wording.ABI.tag})
\item Consistently use \code{size} as a callable in the wording.
\item Add missing \code{type_identity_t} for \code{reduce} (\ref{sec:simd.syn}, \ref{sec:simd.reductions}).
\item Spell out “iff” (\ref{wording.deducet}).
\item Fixed template argument to \nativeabi\ in the default template argument of \code{basic_simd_mask} (\ref{sec:simd.syn}).
\item Fixed default template argument to \code{simd_mask} to be consistent with \code{simd} (\ref{sec:simd.syn}).
\item Add instructions to add \code{<simd>} to the table of headers in [headers].
\item Add instructions to add a new subclause to the table in [numerics.general].
\item Add instructions to add \code{<simd>} [diff.23.library].
\item Add \simdsizev to the wording and replace \code{simd_size_v} to actually implement “Make \code{simd_size} exposition-only.”
\item Restored precondition (and removed \code{noexcept}) on
  \code{reduce_min_index} and \code{reduce_max_index} as directed by LEWG.
\end{revision}

\begin{revision}
\item Strike through wording removed by P3275 (non-const \code{operator[]}).
\item Remove “exposition only” from detailed prose, it's already marked as such in the synopsis.
\item Reorder defintion of \emph{vectorizable type} above its first use.
\item Commas, de-duplication, word order, \code{s/may/can/} in a note.
\item Use text font for “[)” when defining a range of integers.
\item Several small changes from LWG review on 2024-06-26.
\item Reword \code{rebind_simd} and \code{resize_simd}.
\item Remove mention of implementation-defined load/store flags.
\item Remove paragraph about default initialization of \simd.
\item Reword all constructor \emph{Effects} from “Constructs an object \ldots”
  to “Initializes \ldots”.
\item Instead of writing “satisfies X” in \emph{Constraints} and “models X” in
  \emph{Preconditions}, say only “models X” in \emph{Constraints}.
\item Replace \code{is_trivial_v} with “is trivially copyable”.
\item First shot at improving generator function constraints.
\item Reword constraints on unary and binary operators.
\item Add missing/inconsistent \code{explicit} on load constructors.
\item Fix preconditions of subscript operators.
\item Reword effects of compound assignment operators.
\item Add that \code{BinaryOperation} may not modify input \simd.
\item Fix definition of GENERALIZED_SUMs.
\end{revision}

\begin{revision}
\item Say “\textit{op}” instead of “the indicated operator”
\item Fix constraints on shift operators with \simdsizetype{} on the right operand.
\item Remove wording removed by P3275 (non-const \code{operator[]}).
\item Make intrinsics conversion recommended practice.
\item Make \code{simd_flags} template arguments exposition-only.
\item Make \code{simd_alignment} \emph{not} implementation-defined.
\item Reword “supported” to “enabled or disabled”.
\item Apply improved wording from \ref{sec:simd.overview} to \ref{sec:simd.mask.overview}.
\item Add comments for LWG to address to broadcast ctor (\ref{sec:simd.ctor}).
\item Respecify generator ctor to not reuse broadcast constraint (\ref{sec:simd.ctor}).
\item Use \code{to_address} on contiguous iterators (\ref{sec:simd.ctor} and \ref{sec:simd.copy}).
  This is more explicit about allowing memcpy on the complete range rather than
  having to iterate the range per element.
\end{revision}

\begin{revision}
\item Fix default size of \code{simd} and \code{simd_mask} aliases
  (\ref{sec:simd.syn}, necessary for
  \std\code{destructible<\MayBreak{}\std{}simd<\MayBreak\std{}string>>} to be well-formed).
\item Extend value-preserving to encompass conversions from all arithmetic
  types. Use this new freedom in \ref{sec:simd.ctor} to fully constrain the
  generator constructor and to plug a specification hole in the broadcast
  constructor.
\item Fix broadcast constructor wording by constraining \constexprwrapperlike
  arguments to arithmetic types.
  %\todo Reorder \code{simd} and \code{simd_mask} specification in the wording (mask first).
\end{revision}

\section{Straw Polls}


\pagebreak
\section{Issue 1: \code{explicit}}

\newcommand\statusquo[1]{#1}
\newcommand\reconsider{
  {\color{Maroon}$\leftarrow$ reconsider!}
}

\code{simd} has 7 constructors and one conversion operator:

\medskip
\noindent
\begin{tabularx}{\linewidth}{XXl}
  \toprule
  default constructor & \statusquo{not \code{explicit}} & \\\midrule

  copy constructor & \statusquo{not \code{explicit}} & \\\midrule

  broadcast constructor &
    \statusquo{not \code{explicit}, ill-formed when not value-preserving} &
    \reconsider{} \\\midrule

  conversion constructor &
    \statusquo{conditionally \code{explicit}: depends on participating value types} & \\\midrule

  generator construtor &
    \statusquo{\code{explicit}} & \\\midrule

  load constructors &
    \statusquo{\code{explicit}} & \\\midrule

  \textit{Recommended practice:} conversion constructor from \emph{implementation-defined} set
    of types (intrinsics / vector builtin) &
    \statusquo{\code{explicit}} &
    \reconsider{} \\\midrule

  \textit{Recommended practice:} conversion operator to \emph{implementation-defined} set of
    types (intrinsics / vector builtin) &
    \statusquo{\code{explicit}} &
    \reconsider{} \\\bottomrule
\end{tabularx}

\subsection{Broadcast constructor}\label{sec:broadcast}

The authors do not recall that moving the constraint of the broadcast constructor to a conditional
\code{explicit} was considered in LEWG.
The behavior of broadcast and \simd conversion constructors is currently inconsistent.
One allows conversions that are not value-preserving, via explicit constructor / \code{static_cast}.
The other does not.
We recommend that the broadcast constructor is changed to be conditionally \code{explicit}:
\begin{wgText}
\begin{itemdecl}
template<class U>
  constexpr @\wgAdd{explicit(\mbox{\seebelow}) }@basic_simd(U&&@\wgAdd{ x}@) noexcept;
\end{itemdecl}

\begin{itemdescr}
  \pnum Let \tcode{From} denote the type \tcode{remove_cvref_t<U>}.

  \pnum\constraints
  \wgAdd{\tcode{value_type} satisfies \tcode{constructible_from<U>}.}
  \wgRem{\tcode{From} satisfies \tcode{convertible_to<value_type>}, and either}
  \begin{itemize}
    \wgItemRem[ \tcode{From} is an arithmetic type and the conversion from
      \tcode{From} to \tcode{value_type} is value-preserving
    (\mbox{[simd.general]}), or]

    \wgItemRem[ \tcode{From} is not an arithmetic type and does not satisfy
    \mbox{\tcode{\constexprwrapperlike}}, or]

    \wgItemRem[ \tcode{From} satisfies \mbox{\tcode{\constexprwrapperlike}} (\mbox{[simd.syn]})
      \tcode{remove_const_t<decltype(From::value)>} is an arithmetic type, and \tcode{From::value}
      is representable by \tcode{value_type}.]
  \end{itemize}

  \pnum\effects
  Initializes each element to \wgChange{the value of the argument after conversion to
  \tcode{value_type}}{\tcode{value_type(forward<U>(x))}}.

  \pnum\wgAdd{\remarks
  The expression inside \tcode{explicit} evaluates to \tcode{false} if and only if \tcode{From}
  satisfies \tcode{convertible_to<value_type>}, and either}
  \begin{itemize}
    \wgItemAdd[ \tcode{From} is an arithmetic type and the conversion from
      \tcode{From} to \tcode{value_type} is value-preserving
    (\mbox{[simd.general]}), or]

    \wgItemAdd[ \tcode{From} is not an arithmetic type and does not satisfy
    \mbox{\tcode{\constexprwrapperlike}}, or]

    \wgItemAdd[ \tcode{From} satisfies \mbox{\tcode{\constexprwrapperlike}} (\mbox{[simd.syn]}),
      \tcode{remove_const_t<decltype(From::value)>} is an arithmetic type, and \tcode{From::value}
    is representable by \tcode{value_type}.]
  \end{itemize}
\end{itemdescr}

\end{wgText}

\begin{tonytable}{Make explicit conversions more consistent}
  \begin{lstlisting}
using floatv = std::simd<float>;

void f(floatv x)
{
  x + 2; // ill-formed
  x + float(2); // OK
  x + floatv(2); // ill-formed

  x = 2 // ill-formed
  x = float(2) // OK
  x = floatv(2) // ill-formed
}
  \end{lstlisting}
  &
  \begin{lstlisting}
using floatv = std::simd<float>;

void f(floatv x)
{
  x + 2; // ill-formed
  x + float(2); // OK
  x + floatv(2); // OK

  x = 2 // ill-formed
  x = float(2) // OK
  x = floatv(2) // OK
}
  \end{lstlisting}
\end{tonytable}

\subsection{Conversion from/to intrinsic}\label{sec:intrinsics-conversion}

The policy draft on \code{explicit} says “Implicit conversions should exist only between types that
are fundamentally the same”.
The intrinsic types and vector builtin types implemented as extensions in basically every compiler
are “fundamentally the same” as the \code{simd} types of equal value type and width.
Consequently, we should consider implicit conversions.
The reason for the current wording to say \code{explicit} still stems from the TS design which
deliberately wanted to err on the “too strict” side\footnote{that wasn't my preference, but guidance
from WG21 at the time}.
This choice was never reconsidered while merging the TS wording to the IS.

\begin{wgText}
\setcounter{Paras}{2}%
\pnum\recommended
Implementations should enable \wgChange{explicit}{implicit} conversion from and to
implementation-defined types. This adds one or more of the following
declarations to class \tcode{basic_simd}:

\begin{codeblock}
constexpr @\wgRem{explicit }@operator @\impdef@() const;
constexpr @\wgRem{explicit }@basic_simd(const @\impdef@& init);
\end{codeblock}

\begin{example}
  Consider an implementation that supports the type \tcode{__vec4f} and the function \tcode{__vec4f
  _vec4f_addsub(__vec4f, __vec4f)} for the architecture of the execution environment.
  A user may require the use of \tcode{_vec4f_addsub} for maximum performance and thus writes:
  \begin{codeblock}
    using V = basic_simd<float, simd_abi::__simd128>;
    V addsub(V a, V b) {
      return @\wgRem{static_cast<V>(}@_vec4f_addsub(@\wgRem{static_cast<__vec4f>(}a\wgRem{)}, \wgRem{static_cast<__vec4f>(}b\wgRem{))}@);
    }
  \end{codeblock}
\end{example}
\end{wgText}

\begin{tonytable}{Calling an SSE intrinsic}\label{tt:intrinsic}
  \begin{lstlisting}
void f(std::simd<int, 4> x)
{
  x = static_cast<std::simd<int, 4>>(
    _mm_add_epi32(static_cast<__m128i>(x),
                  static_cast<__m128i>(x)));
}
  \end{lstlisting}
  &
  \begin{lstlisting}
void f(std::simd<int, 4> x)
{
  x = _mm_add_epi32(x, x);


}
  \end{lstlisting}
\end{tonytable}%

\subsection{Suggested Polls}
\wgPoll{Make the broadcast constructor conditionally \code{explicit} (\wgDocumentNumber{}
\sect{sec:broadcast})}{&&&&}

\wgPoll{Make conversions to/from implementation-defined vector types implicit (strike
\code{explicit}) (\wgDocumentNumber{} \sect{sec:intrinsics-conversion})}{&&&&}

\section{Issue 2: drop “unsequenced” from generator ctor}\label{sec:generatorctor}

The current wording for the generator constructors (\simd and \mask) says:
\begin{wgText}
    The calls to \tcode{gen} are unsequenced with respect to each other.
    Vectorization-unsafe (\iref{algorithms.parallel.defns}) standard library
    functions may not be invoked by \tcode{gen}.
\end{wgText}

To the authors knowledge this has never been explicitly implemented.
Yes, compilers can relatively easily vectorize generator constructor calls, but that doesn't require
this wording.
In other words, there is no need to restrict user code for the cases where we expect vectorization.

On the other hand, this requirement on user code is likely to be violated in practice.
However, as long as implementations implement the broadcast constructor as an unrolled loop over all
calls, the UB will never materialize.
Unless, at some point in the future an implementation can annotate its unrolled loop with the
necessary “unsequenced” property.
Suddenly latent bugs would materialize.

Furthermore, the current restriction disallows legitimate use cases, such as calling a random number
generator/distribution, performing potentially blocking/synchronizing calls, throwing an exception,
or \std\code{print} debugging.

Therefore, we propose to remove the requirement on the user code and at the same time drop
\code{noexcept} (because throwing from the callable is a valid strategy for error handling).

If we ever find the need for a function that generates \code{simd} objects from unsequenced calls to
scalar functions we can add a named function to do so.
The name of such a function could help to indicate unsequenced execution, which helps in code
reviews to catch potential issues.

\begin{wgText}[{[simd.ctor]}]\setcounter{Paras}{6}
\begin{itemdecl}
template<class G> constexpr explicit basic_simd(G&& gen)@\wgRem{ noexcept}@;
\end{itemdecl}

\begin{itemdescr}
  \pnum Let \tcode{From}$_i$ denote the type
  \tcode{decltype(gen(integral_constant<\simdsizetype, $i$>()))}.

  \pnum\constraints
  \tcode{From}$_i$ satisfies \tcode{convertible_to<value_type>} \foralli.
  In addition, \foralli, if \tcode{From}$_i$ is an arithmetic type, conversion from
  \tcode{From}$_i$ to \tcode{value_type} is value-preserving.

  \pnum\effects
  Initializes the $i^\text{th}$ element with
  \tcode{static_cast<value_type>(gen(integral_constant<\simdsizetype, i>()))} \foralli.

  \pnum
    \wgRem{The calls to \tcode{gen} are unsequenced with respect to each other.
      Vectorization-unsafe (\mbox{\iref{algorithms.parallel.defns}}) standard library
    functions may not be invoked by \tcode{gen}.}
    \tcode{gen} is invoked exactly once for each $i$.
\end{itemdescr}
\end{wgText}

\begin{wgText}[{[simd.mask.ctor]}]\setcounter{Paras}{3}
\begin{itemdecl}
template<class G> constexpr explicit basic_simd_mask(G&& gen)@\wgRem{ noexcept}@;
\end{itemdecl}

\begin{itemdescr}
  \pnum\constraints
  \tcode{static_cast<bool>(gen(integral_constant<\simdsizetype, i>()))} is
  well-formed \foralli.

  \pnum\effects
  Initializes the $i^\text{th}$ element with
  \tcode{gen(integral_constant<\simdsizetype, i>())} \foralli.

  \pnum
    \wgRem{The calls to \tcode{gen} are unsequenced with respect to each other.
      Vectorization-unsafe (\mbox{\iref{algorithms.parallel.defns}}) standard library
    functions may not be invoked by \tcode{gen}.}
    \tcode{gen} is invoked exactly once for each $i$.
\end{itemdescr}
\end{wgText}

\subsection{Suggested Poll}
\wgPoll{Remove wording that unconditionally allows calls to \code{gen} from the generator
  constructors to be unsequenced with respect to each other. At the same time, remove
\code{noexcept} from the constructors. (\wgDocumentNumber{} \sect{sec:generatorctor})}{&&&&}

\section{Issue 3: reorder \code{identity_element} and \code{binary_op} on \code{reduce}}

The masked \std\code{reduce} overloads for \code{simd} require an identity element (for efficient
implementation\footnote{The basic idea is to fill all masked elements of the given \code{simd}
object with the identity element and then perform a tree reduction over all elements of the
\code{simd}.}).
The value of the identity element is know for all vectorizable types and if the
\code{BinaryOperation} is one of \std\code{plus<>}, \std\code{multiplies<>}, \std\code{bit_and<>},
\std\code{bit_or<>}, or \std\code{bit_xor<>}.
For every other user-defined binary operation, the caller must provide a value for the identity
element:

\begin{wgText}[P1928R11]
  \setcounter{Paras}{0}
  \begin{codeblock}
  template<class T, class Abi, class BinaryOperation = plus<>>
    constexpr T reduce(
      const basic_simd<T, Abi>& x, const typename basic_simd<T, Abi>::mask_type& mask,
      type_identity_t<T> identity_element, BinaryOperation binary_op)
  \end{codeblock}
\end{wgText}

The original \code{reduce} overload for the TS was modeled after the overloads that provide an
\emph{initial value}: \code{reduce(InputIt first, InputIt last, T init, BinaryOp op)}.
For these functions the \code{init} parameter precedes the \code{BinaryOp} parameter.

However, the initial value is a very different parameter: It provides an additional value that is
included in the reduction together with the given range.
This is not the case for the \code{simd} overload, where the identity element is included
0--\code{simd::size()} times in the reduction.
More importantly, the value must be such that it doesn't influence the result, otherwise it violates
a precondition of \code{reduce}.

Because of this different nature of the parameter, and because we can provide a default for known
binary operations, the \code{identity_element} parameter can and should be after the
\code{BinaryOp}.
Then the 6 overloads for masked reductions are reduced to a single overload of the form:

\begin{wgText}[P1928R12]
  \setcounter{Paras}{5}
\begin{itemdecl}
template<class T, class Abi, class BinaryOperation = plus<>>
  constexpr T reduce(
    const basic_simd<T, Abi>& x, const typename basic_simd<T, Abi>::mask_type& mask,
    BinaryOperation binary_op = {}, type_identity_t<T> identity_element = @\seebelow@);
\end{itemdecl}

\begin{itemdescr}
  \pnum\constraints
  \begin{itemize}
    \item \tcode{BinaryOperation} models \tcode{\reductionoperation<T>}.

    \item An argument for \tcode{identity_element} is provided for the invocation, unless
      \tcode{BinaryOperation} is one of \code{plus<>}, \code{multiplies<>}, \code{bit_and<>},
      \code{bit_or<>}, or \code{bit_xor<>}.
  \end{itemize}

  \pnum\expects
  \begin{itemize}
    \item \tcode{binary_op} does not modify \tcode{x}.

    \item For all finite values \tcode{y} representable by \tcode{T}, the results of
      \tcode{y == binary_op(simd<T, 1>(identity_element), simd<T, 1>(y))[0]} and
      \tcode{y == binary_op(simd<T, 1>(y), simd<T, 1>(identity_element))[0]} are \tcode{true}.
  \end{itemize}

  \pnum\returns
  If \tcode{none_of(mask)} is \tcode{true}, returns \tcode{identity_element}.
  Otherwise, returns \tcode{\placeholdernc{GENERALIZED_SUM}(binary_op, simd<T, 1>(x[$k_0$]),
  $\ldots$, simd<T, 1>(x[$k_n$]))[0]} where $k_0, \ldots, k_n$ are the selected indices of
  \tcode{mask}.

  \pnum\throws
  Any exception thrown from \tcode{binary_op}.

  \pnum\remarks
  The default argument for \code{identity_element} is equal to
  \begin{itemize}
    \item \tcode{T()} if \code{BinaryOperation} is \code{plus<>},
    \item \tcode{T(1)} if \code{BinaryOperation} is \code{multiplies<>},
    \item \tcode{T(\~{}T())} if \code{BinaryOperation} is \code{bit_and<>},
    \item \tcode{T()} if \code{BinaryOperation} is \code{bit_or<>}, or
    \item \tcode{T()} if \code{BinaryOperation} is \code{bit_xor<>}.
  \end{itemize}
\end{itemdescr}
\end{wgText}

Note that the latest revision of P1928, already contains this new signature / wording, as this was
preferred by LWG.
LEWG still needs to re-confirm that change, otherwise I will have to roll it back.

\subsection{Suggested Poll}

\wgPoll{Reorder \code{binary_op} and \code{identity_element} as suggested by LWG and implemented in
P1928R12.}{&&&&}

\section{Issue 4: Undo removal of members of disabled \simd{}}

(postponed)

\pagebreak
\section{Issue 5: Hidden friend compound assignment operators}

\begin{lstlisting}
std::simd<int, 4> s1, s2;
auto r = std::ref(s1); // r is a std::reference_wrapper

r += s2; // modifies s1: apply += to element wise to s1,sw3
r  = s2; // rebinds r to point to s2
r += s2; // modifies s2
\end{lstlisting}

This is due to \code{r} being convertible to \code{basic_simd\&} and thus binding to:
\smallskip\begin{lstlisting}
template<class T, class Abi> class basic_simd {
  // …
  friend constexpr basic_simd& operator+=(basic_simd&, const basic_simd&) noexcept;
};
\end{lstlisting}

However, if compound assignment is specified as a member then name lookup doesn't find the operator
(no member function lookup via ADL) and the example above becomes ill-formed:
\smallskip\begin{lstlisting}
template<class T, class Abi> class basic_simd {
  // …
  constexpr basic_simd& operator+=(const basic_simd&) noexcept;
};
\end{lstlisting}

Note, however, that the following is already well-formed for scalars with the exact same behavior as
for \code{simd} with hidden friend compound assignment:
\smallskip\begin{lstlisting}
int s1, s2;
auto r = std::ref(s1); // r is a std::reference_wrapper

r += s2; // modifies s1: apply += to element wise to s1,sw3
r  = s2; // rebinds r to point to s2
r += s2; // modifies s2
\end{lstlisting}

Consequently, changing compound assignment for \code{simd} to member operators creates an
inconsistency between \code{simd<T>} and \code{T}.
Also, consider that not every \code{reference_wrapper}-like type implements \code{operator=} as
rebind.
Other types with a conversion operator to lvalue-reference might implement it as assign-through.
(e.g., proxy reference types similar to what we had for \code{simd::operator[]})

\subsection{Suggested Poll}
\wgPoll{Turn [simd.cassign] and [simd.mask.cassign] in P1928R12 into members, as implemented in
P1928R13.}{&&&&}

\end{document}
% vim: sw=2 sts=2 ai et tw=100
