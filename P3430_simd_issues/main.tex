\newcommand\wgTitle{simd issues: explicit, unsequenced, identity-element position, and members of
disabled simd}
\newcommand\wgName{Matthias Kretz <m.kretz@gsi.de>}
\newcommand\wgDocumentNumber{D3430R1}
\newcommand\wgGroup{LEWG}
\newcommand\wgTarget{\CC{}26}
%\newcommand\wgAcknowledgements{Daniel Towner and Ruslan Arutyunyan contributed to this paper via discussions / reviews. Thanks also to Jeff Garland for reviewing.}

\usepackage{mymacros}
\usepackage{wg21}
\setcounter{tocdepth}{2} % show sections and subsections in TOC
\hypersetup{bookmarksdepth=5}
\usepackage{changelog}
\usepackage{underscore}
\usepackage{multirow}

\addbibresource{extra.bib}

\newcommand\simd[1][]{\type{ba\-sic\_simd#1}\xspace}
\newcommand\simdT{\type{ba\-sic\_simd\MayBreak<\MayBreak{}T>}\xspace}
\newcommand\valuetype{\type{val\-ue\_type}\xspace}
\newcommand\referencetype{\type{ref\-er\-ence}\xspace}
\newcommand\mask[1][]{\type{ba\-sic\_simd\_mask#1}\xspace}
\newcommand\maskT{\type{ba\-sic\_simd\_mask\MayBreak<\MayBreak{}T>}\xspace}
\newcommand\wglink[1]{\href{https://wg21.link/#1}{#1}}

\newcommand\nativeabi{\UNSP{native-abi}}
\newcommand\deducet{\UNSP{deduce-t}}
\newcommand\simdsizev{\UNSP{simd-size-v}}
\newcommand\simdsizetype{\UNSP{simd-size-type}}
\newcommand\simdselect{\UNSP{simd-select-impl}}
\newcommand\maskelementsize{\UNSP{mask-element-size}}
\newcommand\integerfrom{\UNSP{integer-from}}
\newcommand\constexprwrapperlike{\UNSP{constexpr-wrapper-like}}
\newcommand\convertflag{\UNSP{convert-flag}}
\newcommand\alignedflag{\UNSP{aligned-flag}}
\newcommand\overalignedflag{\UNSP{overaligned-flag}}
\newcommand\reductionoperation{\UNSP{reduction-binary-operation}}
\newcommand\simdfloatingpoint{\UNSP{simd-floating-point}}
\newcommand\multisimdfloatingpoint{\UNSP{multi-arg-simd-floating-point}}

\renewcommand{\lst}[1]{Listing~\ref{#1}}
\renewcommand{\sect}[1]{Section~\ref{#1}}
\renewcommand{\ttref}[1]{Tony~Table~\ref{#1}}
\renewcommand{\tabref}[1]{Table~\ref{#1}}

\begin{document}
\selectlanguage{american}
\begin{wgTitlepage}
  This paper collects all issues that came up in LWG review of P1928 (merge
  \code{std::simd}), which require LEWG approval.
\end{wgTitlepage}

\pagestyle{scrheadings}

\section{Changelog}
(placeholder)
\begin{revision}
\item Add a simple example to the motivation section.
\item Expand the “Generalization” section to clearly define the feature rather
  than just sketching it.
  Also add a discussion of initial value and step.
\item Discuss why reusing the existing \code{iota} algorithm/view does not
  work/suffice for the \code{simd} use case.
\item Discuss why \code{iota_v} is the right name.
%  \todo
\end{revision}

\section{Straw Polls}
\subsection{SG1 at Kona 2022}
\wgPoll{After significant experience with the TS, we recommend that the next
version (the TS version with improvements) of \code{std::simd} target the IS (\CC{}26)}
{10&8&0&0&0}

\wgUnanimous{We like all of the recommended changes to \code{std::simd} proposed in p1928r1
(Includes making all of \code{std::simd} \code{constexpr}, and dropping an ABI stable type)}

\wgPoll{Future papers and future revisions of existing papers that target
\code{std::simd} should go directly to LEWG.
(We do not believe there are SG1 issues with \code{std::simd} today.)}
{9&8&0&0&0}


\pagebreak
\section{Issue 1: \code{explicit}}

\newcommand\statusquo[1]{#1}
\newcommand\reconsider{
  {\color{Maroon}$\leftarrow$ reconsider!}
}

\code{simd} has 7 constructors and one conversion operator:

\medskip
\noindent
\begin{tabularx}{\linewidth}{XXl}
  \toprule
  default constructor & \statusquo{not \code{explicit}} & \\\midrule

  copy constructor & \statusquo{not \code{explicit}} & \\\midrule

  broadcast constructor &
    \statusquo{not \code{explicit}, ill-formed when not value-preserving} &
    \reconsider{} \\\midrule

  conversion constructor &
    \statusquo{conditionally \code{explicit}: depends on participating value types} & \\\midrule

  generator construtor &
    \statusquo{\code{explicit}} & \\\midrule

  load constructors &
    \statusquo{\code{explicit}} & \\\midrule

  \textit{Recommended practice:} conversion constructor from \emph{implementation-defined} set
    of types (intrinsics / vector builtin) &
    \statusquo{\code{explicit}} &
    \reconsider{} \\\midrule

  \textit{Recommended practice:} conversion operator to \emph{implementation-defined} set of
    types (intrinsics / vector builtin) &
    \statusquo{\code{explicit}} &
    \reconsider{} \\\bottomrule
\end{tabularx}

\subsection{Broadcast constructor}\label{sec:broadcast}

The authors do not recall that moving the constraint of the broadcast constructor to a conditional
\code{explicit} was considered in LEWG.
The behavior of broadcast and \simd conversion constructors is currently inconsistent.
One allows conversions that are not value-preserving, via explicit constructor / \code{static_cast}.
The other does not.
We recommend that the broadcast constructor is changed to be conditionally \code{explicit}:
\begin{wgText}
\begin{itemdecl}
template<class U>
  constexpr @\wgAdd{explicit(\mbox{\seebelow}) }@basic_simd(U&&@\wgAdd{ x}@) noexcept;
\end{itemdecl}

\begin{itemdescr}
  \pnum Let \tcode{From} denote the type \tcode{remove_cvref_t<U>}.

  \pnum\constraints
  \wgAdd{\tcode{value_type} satisfies \tcode{constructible_from<U>}.}
  \wgRem{\tcode{From} satisfies \tcode{convertible_to<value_type>}, and either}
  \begin{itemize}
    \wgItemRem[ \tcode{From} is an arithmetic type and the conversion from
      \tcode{From} to \tcode{value_type} is value-preserving
    (\mbox{[simd.general]}), or]

    \wgItemRem[ \tcode{From} is not an arithmetic type and does not satisfy
    \mbox{\tcode{\constexprwrapperlike}}, or]

    \wgItemRem[ \tcode{From} satisfies \mbox{\tcode{\constexprwrapperlike}} (\mbox{[simd.syn]})
      \tcode{remove_const_t<decltype(From::value)>} is an arithmetic type, and \tcode{From::value}
      is representable by \tcode{value_type}.]
  \end{itemize}

  \pnum\effects
  Initializes each element to \wgChange{the value of the argument after conversion to
  \tcode{value_type}}{\tcode{value_type(forward<U>(x))}}.

  \pnum\wgAdd{\remarks
  The expression inside \tcode{explicit} evaluates to \tcode{false} if and only if \tcode{From}
  satisfies \tcode{convertible_to<value_type>}, and either}
  \begin{itemize}
    \wgItemAdd[ \tcode{From} is an arithmetic type and the conversion from
      \tcode{From} to \tcode{value_type} is value-preserving
    (\mbox{[simd.general]}), or]

    \wgItemAdd[ \tcode{From} is not an arithmetic type and does not satisfy
    \mbox{\tcode{\constexprwrapperlike}}, or]

    \wgItemAdd[ \tcode{From} satisfies \mbox{\tcode{\constexprwrapperlike}} (\mbox{[simd.syn]}),
      \tcode{remove_const_t<decltype(From::value)>} is an arithmetic type, and \tcode{From::value}
    is representable by \tcode{value_type}.]
  \end{itemize}
\end{itemdescr}

\end{wgText}

\begin{tonytable}{Make explicit conversions more consistent}
  \begin{lstlisting}
using floatv = std::simd<float>;

void f(floatv x)
{
  x + 2; // ill-formed
  x + float(2); // OK
  x + floatv(2); // ill-formed

  x = 2 // ill-formed
  x = float(2) // OK
  x = floatv(2) // ill-formed
}
  \end{lstlisting}
  &
  \begin{lstlisting}
using floatv = std::simd<float>;

void f(floatv x)
{
  x + 2; // ill-formed
  x + float(2); // OK
  x + floatv(2); // OK

  x = 2 // ill-formed
  x = float(2) // OK
  x = floatv(2) // OK
}
  \end{lstlisting}
\end{tonytable}

\subsection{Conversion from/to intrinsic}\label{sec:intrinsics-conversion}

The policy draft on \code{explicit} says “Implicit conversions should exist only between types that
are fundamentally the same”.
The intrinsic types and vector builtin types implemented as extensions in basically every compiler
are “fundamentally the same” as the \code{simd} types of equal value type and width.
Consequently, we should consider implicit conversions.
The reason for the current wording to say \code{explicit} still stems from the TS design which
deliberately wanted to err on the “too strict” side\footnote{that wasn't my preference, but guidance
from WG21 at the time}.
This choice was never reconsidered while merging the TS wording to the IS.

\begin{wgText}
\setcounter{Paras}{2}%
\pnum\recommended
Implementations should enable \wgChange{explicit}{implicit} conversion from and to
implementation-defined types. This adds one or more of the following
declarations to class \tcode{basic_simd}:

\begin{codeblock}
constexpr @\wgRem{explicit }@operator @\impdef@() const;
constexpr @\wgRem{explicit }@basic_simd(const @\impdef@& init);
\end{codeblock}

\begin{example}
  Consider an implementation that supports the type \tcode{__vec4f} and the function \tcode{__vec4f
  _vec4f_addsub(__vec4f, __vec4f)} for the architecture of the execution environment.
  A user may require the use of \tcode{_vec4f_addsub} for maximum performance and thus writes:
  \begin{codeblock}
    using V = basic_simd<float, simd_abi::__simd128>;
    V addsub(V a, V b) {
      return @\wgRem{static_cast<V>(}@_vec4f_addsub(@\wgRem{static_cast<__vec4f>(}a\wgRem{)}, \wgRem{static_cast<__vec4f>(}b\wgRem{))}@);
    }
  \end{codeblock}
\end{example}
\end{wgText}

\begin{tonytable}{Calling an SSE intrinsic}\label{tt:intrinsic}
  \begin{lstlisting}
void f(std::simd<int, 4> x)
{
  x = static_cast<std::simd<int, 4>>(
    _mm_add_epi32(static_cast<__m128i>(x),
                  static_cast<__m128i>(x)));
}
  \end{lstlisting}
  &
  \begin{lstlisting}
void f(std::simd<int, 4> x)
{
  x = _mm_add_epi32(x, x);


}
  \end{lstlisting}
\end{tonytable}%

\subsection{Suggested Polls}
\wgPoll{Make the broadcast constructor conditionally \code{explicit} (\wgDocumentNumber{}
\sect{sec:broadcast})}{&&&&}

\wgPoll{Make conversions to/from implementation-defined vector types implicit (strike
\code{explicit}) (\wgDocumentNumber{} \sect{sec:intrinsics-conversion})}{&&&&}

\section{Issue 2: drop “unsequenced” from generator ctor}\label{sec:generatorctor}

The current wording for the generator constructors (\simd and \mask) says:
\begin{wgText}
    The calls to \tcode{gen} are unsequenced with respect to each other.
    Vectorization-unsafe (\iref{algorithms.parallel.defns}) standard library
    functions may not be invoked by \tcode{gen}.
\end{wgText}

To the authors knowledge this has never been explicitly implemented.
Yes, compilers can relatively easily vectorize generator constructor calls, but that doesn't require
this wording.
In other words, there is no need to restrict user code for the cases where we expect vectorization.

On the other hand, this requirement on user code is likely to be violated in practice.
However, as long as implementations implement the broadcast constructor as an unrolled loop over all
calls, the UB will never materialize.
Unless, at some point in the future an implementation can annotate its unrolled loop with the
necessary “unsequenced” property.
Suddenly latent bugs would materialize.

Furthermore, the current restriction disallows legitimate use cases, such as calling a random number
generator/distribution, performing potentially blocking/synchronizing calls, throwing an exception,
or \std\code{print} debugging.

Therefore, we propose to remove the requirement on the user code and at the same time drop
\code{noexcept} (because throwing from the callable is a valid strategy for error handling).

If we ever find the need for a function that generates \code{simd} objects from unsequenced calls to
scalar functions we can add a named function to do so.
The name of such a function could help to indicate unsequenced execution, which helps in code
reviews to catch potential issues.

\begin{wgText}[{[simd.ctor]}]\setcounter{Paras}{6}
\begin{itemdecl}
template<class G> constexpr explicit basic_simd(G&& gen)@\wgRem{ noexcept}@;
\end{itemdecl}

\begin{itemdescr}
  \pnum Let \tcode{From}$_i$ denote the type
  \tcode{decltype(gen(integral_constant<\simdsizetype, $i$>()))}.

  \pnum\constraints
  \tcode{From}$_i$ satisfies \tcode{convertible_to<value_type>} \foralli.
  In addition, \foralli, if \tcode{From}$_i$ is an arithmetic type, conversion from
  \tcode{From}$_i$ to \tcode{value_type} is value-preserving.

  \pnum\effects
  Initializes the $i^\text{th}$ element with
  \tcode{static_cast<value_type>(gen(integral_constant<\simdsizetype, i>()))} \foralli.

  \pnum
    \wgRem{The calls to \tcode{gen} are unsequenced with respect to each other.
      Vectorization-unsafe (\mbox{\iref{algorithms.parallel.defns}}) standard library
    functions may not be invoked by \tcode{gen}.}
    \tcode{gen} is invoked exactly once for each $i$.
\end{itemdescr}
\end{wgText}

\begin{wgText}[{[simd.mask.ctor]}]\setcounter{Paras}{3}
\begin{itemdecl}
template<class G> constexpr explicit basic_simd_mask(G&& gen)@\wgRem{ noexcept}@;
\end{itemdecl}

\begin{itemdescr}
  \pnum\constraints
  \tcode{static_cast<bool>(gen(integral_constant<\simdsizetype, i>()))} is
  well-formed \foralli.

  \pnum\effects
  Initializes the $i^\text{th}$ element with
  \tcode{gen(integral_constant<\simdsizetype, i>())} \foralli.

  \pnum
    \wgRem{The calls to \tcode{gen} are unsequenced with respect to each other.
      Vectorization-unsafe (\mbox{\iref{algorithms.parallel.defns}}) standard library
    functions may not be invoked by \tcode{gen}.}
    \tcode{gen} is invoked exactly once for each $i$.
\end{itemdescr}
\end{wgText}

\subsection{Suggested Poll}
\wgPoll{Remove wording that unconditionally allows calls to \code{gen} from the generator
  constructors to be unsequenced with respect to each other. At the same time, remove
\code{noexcept} from the constructors. (\wgDocumentNumber{} \sect{sec:generatorctor})}{&&&&}

\section{Issue 3: reorder \code{identity_element} and \code{binary_op} on \code{reduce}}

The masked \std\code{reduce} overloads for \code{simd} require an identity element (for efficient
implementation\footnote{The basic idea is to fill all masked elements of the given \code{simd}
object with the identity element and then perform a tree reduction over all elements of the
\code{simd}.}).
The value of the identity element is know for all vectorizable types and if the
\code{BinaryOperation} is one of \std\code{plus<>}, \std\code{multiplies<>}, \std\code{bit_and<>},
\std\code{bit_or<>}, or \std\code{bit_xor<>}.
For every other user-defined binary operation, the caller must provide a value for the identity
element:

\begin{wgText}[P1928R11]
  \setcounter{Paras}{0}
  \begin{codeblock}
  template<class T, class Abi, class BinaryOperation = plus<>>
    constexpr T reduce(
      const basic_simd<T, Abi>& x, const typename basic_simd<T, Abi>::mask_type& mask,
      type_identity_t<T> identity_element, BinaryOperation binary_op)
  \end{codeblock}
\end{wgText}

The original \code{reduce} overload for the TS was modeled after the overloads that provide an
\emph{initial value}: \code{reduce(InputIt first, InputIt last, T init, BinaryOp op)}.
For these functions the \code{init} parameter precedes the \code{BinaryOp} parameter.

However, the initial value is a very different parameter: It provides an additional value that is
included in the reduction together with the given range.
This is not the case for the \code{simd} overload, where the identity element is included
0--\code{simd::size()} times in the reduction.
More importantly, the value must be such that it doesn't influence the result, otherwise it violates
a precondition of \code{reduce}.

Because of this different nature of the parameter, and because we can provide a default for known
binary operations, the \code{identity_element} parameter can and should be after the
\code{BinaryOp}.
Then the 6 overloads for masked reductions are reduced to a single overload of the form:

\begin{wgText}[P1928R12]
  \setcounter{Paras}{5}
\begin{itemdecl}
template<class T, class Abi, class BinaryOperation = plus<>>
  constexpr T reduce(
    const basic_simd<T, Abi>& x, const typename basic_simd<T, Abi>::mask_type& mask,
    BinaryOperation binary_op = {}, type_identity_t<T> identity_element = @\seebelow@);
\end{itemdecl}

\begin{itemdescr}
  \pnum\constraints
  \begin{itemize}
    \item \tcode{BinaryOperation} models \tcode{\reductionoperation<T>}.

    \item An argument for \tcode{identity_element} is provided for the invocation, unless
      \tcode{BinaryOperation} is one of \code{plus<>}, \code{multiplies<>}, \code{bit_and<>},
      \code{bit_or<>}, or \code{bit_xor<>}.
  \end{itemize}

  \pnum\expects
  \begin{itemize}
    \item \tcode{binary_op} does not modify \tcode{x}.

    \item For all finite values \tcode{y} representable by \tcode{T}, the results of
      \tcode{y == binary_op(simd<T, 1>(identity_element), simd<T, 1>(y))[0]} and
      \tcode{y == binary_op(simd<T, 1>(y), simd<T, 1>(identity_element))[0]} are \tcode{true}.
  \end{itemize}

  \pnum\returns
  If \tcode{none_of(mask)} is \tcode{true}, returns \tcode{identity_element}.
  Otherwise, returns \tcode{\placeholdernc{GENERALIZED_SUM}(binary_op, simd<T, 1>(x[$k_0$]),
  $\ldots$, simd<T, 1>(x[$k_n$]))[0]} where $k_0, \ldots, k_n$ are the selected indices of
  \tcode{mask}.

  \pnum\throws
  Any exception thrown from \tcode{binary_op}.

  \pnum\remarks
  The default argument for \code{identity_element} is equal to
  \begin{itemize}
    \item \tcode{T()} if \code{BinaryOperation} is \code{plus<>},
    \item \tcode{T(1)} if \code{BinaryOperation} is \code{multiplies<>},
    \item \tcode{T(\~{}T())} if \code{BinaryOperation} is \code{bit_and<>},
    \item \tcode{T()} if \code{BinaryOperation} is \code{bit_or<>}, or
    \item \tcode{T()} if \code{BinaryOperation} is \code{bit_xor<>}.
  \end{itemize}
\end{itemdescr}
\end{wgText}

Note that the latest revision of P1928, already contains this new signature / wording, as this was
preferred by LWG.
LEWG still needs to re-confirm that change, otherwise I will have to roll it back.

\subsection{Suggested Poll}

\wgPoll{Reorder \code{binary_op} and \code{identity_element} as suggested by LWG and implemented in
P1928R12.}{&&&&}

\section{Issue 4: Undo removal of members of disabled \simd{}}

(postponed)

\pagebreak
\section{Issue 5: Hidden friend compound assignment operators}

\begin{lstlisting}
std::simd<int, 4> s1, s2;
auto r = std::ref(s1); // r is a std::reference_wrapper

r += s2; // modifies s1: apply += to element wise to s1,sw3
r  = s2; // rebinds r to point to s2
r += s2; // modifies s2
\end{lstlisting}

This is due to \code{r} being convertible to \code{basic_simd\&} and thus binding to:
\smallskip\begin{lstlisting}
template<class T, class Abi> class basic_simd {
  // …
  friend constexpr basic_simd& operator+=(basic_simd&, const basic_simd&) noexcept;
};
\end{lstlisting}

However, if compound assignment is specified as a member then name lookup doesn't find the operator
(no member function lookup via ADL) and the example above becomes ill-formed:
\smallskip\begin{lstlisting}
template<class T, class Abi> class basic_simd {
  // …
  constexpr basic_simd& operator+=(const basic_simd&) noexcept;
};
\end{lstlisting}

Note, however, that the following is already well-formed for scalars with the exact same behavior as
for \code{simd} with hidden friend compound assignment:
\smallskip\begin{lstlisting}
int s1, s2;
auto r = std::ref(s1); // r is a std::reference_wrapper

r += s2; // modifies s1: apply += to element wise to s1,sw3
r  = s2; // rebinds r to point to s2
r += s2; // modifies s2
\end{lstlisting}

Consequently, changing compound assignment for \code{simd} to member operators creates an
inconsistency between \code{simd<T>} and \code{T}.
Also, consider that not every \code{reference_wrapper}-like type implements \code{operator=} as
rebind.
Other types with a conversion operator to lvalue-reference might implement it as assign-through.
(e.g., proxy reference types similar to what we had for \code{simd::operator[]})

\subsection{Suggested Poll}
\wgPoll{Turn [simd.cassign] and [simd.mask.cassign] in P1928R12 into members, as implemented in
P1928R13.}{&&&&}

\end{document}
% vim: sw=2 sts=2 ai et tw=100
