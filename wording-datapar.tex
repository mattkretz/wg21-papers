\wgSubsection{Class template \type{datapar}}{datapar}
\wgSubsubsection{Class template \datapar overview}{datapar.overview}
\lstinputlisting[]{datapar.cpp}

\pnum The class template \datapar{}\type{<T, Abi>} is a one-dimensional smart array.
In contrast to \type{valarray} (26.6), the number of elements in the array is determined at compile time, according to the \type{Abi} template parameter.

\pnum The first template argument \type T must be an integral or floating-point fundamental type.
The type \bool is not allowed.

\pnum The second template argument \type{Abi} must be a tag type from the \code{datapar_abi} namespace.

\begin{itemdecl}
typedef implementation_defined native_handle_type;
\end{itemdecl}
\begin{itemdescr}
  \pnum
  The \type{native_handle_type} member type is an alias for the \code{native_handle()} member function return type.
  It is used to expose an implementation-defined handle for implementation- and target-specific extensions.
\end{itemdescr}

%  \begin{itemdecl}
%typedef implementation_defined register_value_type;
%  \end{itemdecl}
%  \begin{itemdescr}
%  \end{itemdescr}

\begin{itemdecl}
static constexpr size_type size();
\end{itemdecl}
\begin{itemdescr}
  \pnum\returns the number of elements stored in objects of the given \datapar[<T, Abi>] type.
\end{itemdescr}

\wgSubsubsection{\datapar constructors}{datapar.ctor}
\begin{itemdecl}
datapar() = default;
\end{itemdecl}
\begin{itemdescr}
  \pnum
  \effects
  Constructs an object with all elements initialized to \code{T()}.
  \wgNote{This zero-initializes the object.}
\end{itemdescr}

\begin{itemdecl}
datapar(value_type);
\end{itemdecl}
\begin{itemdescr}
  \pnum
  \effects
  Constructs an object with each element initialized to the value of the argument.
\end{itemdescr}

\comment{Should I add a generator ctor, taking a lambda to initialize the elements?}

\begin{itemdecl}
template <class U, class Abi2> datapar(datapar<U, Abi2> x);
\end{itemdecl}
\begin{itemdescr}
  \pnum\remarks This constructor shall not participate in overload resolution unless either
  \begin{itemize}
    \item \type{Abi} and \type{Abi2} are equal and
      \type U and \type T are different integral types and
      \code{make_signed<U>::type} equals \code{make_signed<T>::type}, or
    \item at least one of \type{Abi} or \type{Abi2} is an instantiation of \fixedsizescoped and \code{size() == x.size()} and \type U is implicitly convertible to \type T.
  \end{itemize}
  \pnum\effects Constructs an object where the $i$-th element equals \code{static_cast<T>(x[i])} \foralli.
\end{itemdescr}

\wgSubsubsection{\datapar load functions}{datapar.load}
\comment{Move to non-member. Ask LEWG about load expression vs. template param.}
\begin{itemdecl}
static datapar load(const value_type *x);
\end{itemdecl}
\begin{itemdescr}
  \pnum \returns A new \datapar object with each element $i$ initialized to \code{x[i]} \foralli.
  \pnum \remarks If \datapar{}\code{::size()} is greater than the number of values pointed to by the argument, the behavior is undefined.
\end{itemdescr}

\begin{itemdecl}
template <class Flags> static datapar load(const value_type *x, Flags);
\end{itemdecl}
\begin{itemdescr}
  \pnum\returns A new \datapar object with each element $i$ initialized to \code{x[i]} \foralli.
  \pnum\remarks If \datapar{}\code{::size()} is greater than the number of values pointed to by the first argument, the behavior is undefined.
  \pnum\remarks If the template parameter is of type \type{aligned_tag} and the pointer value is not a multiple of \code{memory_alignment<\datapar{}>}, the behavior is undefined.
\end{itemdescr}

\begin{itemdecl}
template <class U, class Flags = unaligned_tag> static datapar load(const U *x, Flags = Flags());
\end{itemdecl}
\begin{itemdescr}
  \pnum\returns A new \datapar object with each element $i$ initialized to \code{static_cast<T>(x[i])} \foralli.
  \pnum\remarks If \datapar{}\code{::size()} is greater than the number of values pointed to by the first argument, the behavior is undefined.
  \pnum\remarks If the second template parameter is of type \type{aligned_tag} and the pointer value is not a multiple of \code{memory_alignment<\datapar, \type U>}, the behavior is undefined.
\end{itemdescr}

\wgSubsubsection{\datapar store functions}{datapar.store}
\begin{itemdecl}
void store(value_type *x);
\end{itemdecl}
\begin{itemdescr}
  \pnum\effects Copies each element such that the $i$-th element is stored to \code{x[i]}.
  \pnum\remarks If \datapar{}\code{::size()} is greater than the number of values pointed to by the first argument, the behavior is undefined.
\end{itemdescr}

\begin{itemdecl}
template <class Flags> void store(value_type *x, Flags);
\end{itemdecl}
\begin{itemdescr}
  \pnum\effects Copies each element such that the $i$-th element is stored to \code{x[i]}.
  \pnum\remarks If \datapar{}\code{::size()} is greater than the number of values pointed to by the first argument, the behavior is undefined.
  \pnum\remarks If the template parameter is of type \type{aligned_tag} and the pointer value is not a multiple of \code{memory_alignment<\datapar{}>}, the behavior is undefined.
\end{itemdescr}

\begin{itemdecl}
template <class U, class Flags = unaligned_tag> void store(U *x, Flags = Flags());
\end{itemdecl}
\begin{itemdescr}
  \pnum\effects Copies each element such that the $i$-th element is first converted to \type U and then stored to \code{x[i]}.
  \pnum\remarks If \datapar{}\code{::size()} is greater than the number of values pointed to by the first argument, the behavior is undefined.
  \pnum\remarks If the second template parameter is of type \type{aligned_tag} and the pointer value is not a multiple of \code{memory_alignment<\datapar, \type U>}, the behavior is undefined.
\end{itemdescr}

\begin{itemdecl}
void store(value_type *x, mask_type);
\end{itemdecl}
\begin{itemdescr}
  \pnum\effects Copies each element where the corresponding element in the second argument is \true such that the $i$-th element is stored to \code{x[i]}.
  \pnum\remarks If the largest $i$ where the second argument is \true is greater than the number of values pointed to by the first argument, the behavior is undefined.
\end{itemdescr}

\begin{itemdecl}
template <class Flags> void store(value_type *x, mask_type, Flags);
\end{itemdecl}
\begin{itemdescr}
  \pnum\effects Copies each element where the corresponding element in the second argument is \true such that the $i$-th element is stored to \code{x[i]}.
  \pnum\remarks If the largest $i$ where the second argument is \true is greater than the number of values pointed to by the first argument, the behavior is undefined.
  \pnum\remarks If the template parameter is of type \type{aligned_tag} and the pointer value is not a multiple of \code{memory_alignment<\datapar{}>}, the behavior is undefined.
\end{itemdescr}

\begin{itemdecl}
template <class U, class Flags = unaligned_tag> void store(U *x, mask_type, Flags = Flags());
\end{itemdecl}
\begin{itemdescr}
  \pnum\effects Copies each element where the corresponding element in the second argument is \true such that the $i$-th element is first converted to \type U and then stored to \code{x[i]}.
  \pnum\remarks If the largest $i$ where the second argument is \true is greater than the number of values pointed to by the first argument, the behavior is undefined.
  \pnum\remarks If the template parameter is of type \type{aligned_tag} and the pointer value is not a multiple of \code{memory_alignment<\datapar, \type U>}, the behavior is undefined.
\end{itemdescr}

\wgSubsubsection{\datapar subscript operators}{datapar.subscr}
\begin{itemdecl}
reference operator[](size_type i);
\end{itemdecl}
\begin{itemdescr}
  \pnum\returns An lvalue reference to the $i$-th element.
  \pnum\postconditions Assignment of objects of type \type T modify the $i$-th element without aliasing violations.
  \pnum                Modification of \code{*this} does not invalidate references held to the return value.
  Subsequent reads from such references yield the new value of the $i$-th element.
\end{itemdescr}

\begin{itemdecl}
const_reference operator[](size_type) const;
\end{itemdecl}
\begin{itemdescr}
  \pnum\returns A \const lvalue reference to the $i$-th element.
  \pnum\postconditions Modification of \code{*this} does not invalidate references held to the return value.
  Subsequent reads from such references yield the new value of the $i$-th element.
\end{itemdescr}

\wgSubsubsection{\datapar unary operators}{datapar.unary}
\begin{itemdecl}
datapar &operator++();
\end{itemdecl}
\begin{itemdescr}
  \pnum\effects Increments every element of \code{*this} by one.
  \pnum\returns An lvalue reference to \code{*this} after incrementing.
  \pnum\remarks Overflow semantics follow the same semantics as for \type T.
\end{itemdescr}

\begin{itemdecl}
datapar operator++(int);
\end{itemdecl}
\begin{itemdescr}
  \pnum\effects Increments every element of \code{*this} by one.
  \pnum\returns A copy of \code{*this} before incrementing.
  \pnum\remarks Overflow semantics follow the same semantics as for \type T.
\end{itemdescr}

\begin{itemdecl}
datapar &operator--();
\end{itemdecl}
\begin{itemdescr}
  \pnum\effects Decrements every element of \code{*this} by one.
  \pnum\returns An lvalue reference to \code{*this} after decrementing.
  \pnum\remarks Underflow semantics follow the same semantics as for \type T.
\end{itemdescr}

\begin{itemdecl}
datapar operator--(int);
\end{itemdecl}
\begin{itemdescr}
  \pnum\effects Decrements every element of \code{*this} by one.
  \pnum\returns A copy of \code{*this} before decrementing.
  \pnum\remarks Underflow semantics follow the same semantics as for \type T.
\end{itemdescr}

\begin{itemdecl}
mask_type operator!() const;
\end{itemdecl}
\begin{itemdescr}
  \pnum\returns A mask object with the $i$-th element set to \code{!operator[](i)} \foralli.
\end{itemdescr}

\begin{itemdecl}
datapar operator~() const;
\end{itemdecl}
\begin{itemdescr}
  \pnum\requires The first template argument \type T to \datapar must be an integral type.
  \pnum\returns A new \datapar object where each bit is the inverse of the corresponding bit in \code{*this}.
  \pnum\remarks \datapar{}\code{::operator\textasciitilde{}()} shall not participate in overload resolution if \type T is a floating-point type.
\end{itemdescr}

\begin{itemdecl}
datapar operator+() const;
\end{itemdecl}
\begin{itemdescr}
  \pnum \returns A copy of \code{*this}
\end{itemdescr}

\begin{itemdecl}
datapar operator-() const;
\end{itemdecl}
\begin{itemdescr}
  \pnum\returns A new \datapar object where the $i$-th element is initialized to \code{-operator[](i)} \foralli.
\end{itemdescr}

\wgSubsubsection{\datapar native handles}{datapar.native}
\begin{itemdecl}
native_handle_type &native_handle();
\end{itemdecl}
\begin{itemdescr}
  \pnum\returns An lvalue reference to the implementation-specific object implementing the data-parallel semantics.
\end{itemdescr}

\begin{itemdecl}
const native_handle_type &native_handle() const;
\end{itemdecl}
\begin{itemdescr}
  \pnum\returns A \const lvalue reference to the implementation-specific object implementing the data-parallel semantics.
\end{itemdescr}

\wgSubsection{\type{datapar} non-member operations}{datapar.nonmembers}
\wgSubsubsection{\datapar binary operators}{datapar.binary}
\begin{itemdecl}
template <class L, class R> using datapar_return_type = ...;  // exposition only
template <class T, class Abi, class U>
datapar_return_type<datapar<T, Abi>, U> operator+ (datapar<T, Abi>, const U &);
template <class T, class Abi, class U>
datapar_return_type<datapar<T, Abi>, U> operator- (datapar<T, Abi>, const U &);
template <class T, class Abi, class U>
datapar_return_type<datapar<T, Abi>, U> operator* (datapar<T, Abi>, const U &);
template <class T, class Abi, class U>
datapar_return_type<datapar<T, Abi>, U> operator/ (datapar<T, Abi>, const U &);
template <class T, class Abi, class U>
datapar_return_type<datapar<T, Abi>, U> operator% (datapar<T, Abi>, const U &);
template <class T, class Abi, class U>
datapar_return_type<datapar<T, Abi>, U> operator& (datapar<T, Abi>, const U &);
template <class T, class Abi, class U>
datapar_return_type<datapar<T, Abi>, U> operator| (datapar<T, Abi>, const U &);
template <class T, class Abi, class U>
datapar_return_type<datapar<T, Abi>, U> operator^ (datapar<T, Abi>, const U &);
template <class T, class Abi, class U>
datapar_return_type<datapar<T, Abi>, U> operator<<(datapar<T, Abi>, const U &);
template <class T, class Abi, class U>
datapar_return_type<datapar<T, Abi>, U> operator>>(datapar<T, Abi>, const U &);
template <class T, class Abi, class U>
datapar_return_type<datapar<T, Abi>, U> operator+ (const U &, datapar<T, Abi>);
template <class T, class Abi, class U>
datapar_return_type<datapar<T, Abi>, U> operator- (const U &, datapar<T, Abi>);
template <class T, class Abi, class U>
datapar_return_type<datapar<T, Abi>, U> operator* (const U &, datapar<T, Abi>);
template <class T, class Abi, class U>
datapar_return_type<datapar<T, Abi>, U> operator/ (const U &, datapar<T, Abi>);
template <class T, class Abi, class U>
datapar_return_type<datapar<T, Abi>, U> operator% (const U &, datapar<T, Abi>);
template <class T, class Abi, class U>
datapar_return_type<datapar<T, Abi>, U> operator& (const U &, datapar<T, Abi>);
template <class T, class Abi, class U>
datapar_return_type<datapar<T, Abi>, U> operator| (const U &, datapar<T, Abi>);
template <class T, class Abi, class U>
datapar_return_type<datapar<T, Abi>, U> operator^ (const U &, datapar<T, Abi>);
template <class T, class Abi, class U>
datapar_return_type<datapar<T, Abi>, U> operator<<(const U &, datapar<T, Abi>);
template <class T, class Abi, class U>
datapar_return_type<datapar<T, Abi>, U> operator>>(const U &, datapar<T, Abi>);
\end{itemdecl}
\begin{itemdescr}
  \newcommand\common[2]{\code{\textit{common}(\type{#1}, \type{#2})}}
  \newcommand\commonabi[3]{\code{\textit{commonabi}(\type{#1}, \type{#2}, \type{#3})}}
  \newcommand\dataparreturntype{\type{datapar_return_type<datapar<T, Abi>, U>}\xspace}
  \pnum\remarks The return type of these operators (\dataparreturntype) shall be deduced according to the following rules:
  \begin{itemize}
    \item Let \common{A}{B} identify the type:
      \comment{\textit{unusual arithmetic conversions} ;-)}
      \begin{itemize}
        \item \type A if \type A equals \type B.
        \item Otherwise, \type A if \type B is not a fundamental arithmetic type.
        \item Otherwise, \type B if \type A is not a fundamental arithmetic type.
        \item Otherwise, \code{decltype(A() + B())} if either one of the types \type A or \type B is a floating-point type.
        \item Otherwise, \type A if \code{sizeof(A) > sizeof(B)}.
        \item Otherwise, \type B if \code{sizeof(A) < sizeof(B)}.
        \item Otherwise, \type C shall identify the type with greater integer conversion rank of the types \type A and \type B and:
          \begin{itemize}
            \item \type C is used if \code{is_signed_v<A> == is_signed_v<B>}, and
            \item \code{make_unsigned_t<C>} otherwise.
          \end{itemize}
      \end{itemize}
    \item Let \commonabi{A0}{A1}{T} identify the type:
      \begin{itemize}
        \item \type{A0} if \type{A0} equals \type{A1}. % the remaining cases therefore only cover one native ABI and one fixed_size<N>
        \item Otherwise, \code{abi_for_size_t<T, datapar_size_v<T, A0>>} if it is equal to either \type{A0} or \type{A1}.
        \item Otherwise, \fixedsizescoped{}\code{<datapar_size_v<T, A0>>}.
      \end{itemize}
    \item If \code{is_datapar_v<U> == true}
      then the return type is \datapar[<\common{T}{U::value_type}, \commonabi{Abi}{U::abi_type}{\common{T}{U::value_type}}>].
      \wgNote{
        This rule also matches if \code{datapar_size_v<T, Abi> != U::size()}.
        The overload resolution participation condition in the next paragraph discards the operator.
      }
    \item Otherwise, if \type T is integral and \type U is \intt
      the return type shall be \datapar{}\type{<T, Abi>}.
    \item Otherwise, if \type T is integral and \type U is \uint
      the return type shall be \datapar{}\code{<make_unsigned_t<T>, Abi>}.
    \item Otherwise, if \type U is a fundamental arithmetic type or \type U is convertible to \intt
      then the return type shall be \datapar[<\common{T}{U}, \commonabi{Abi}{\fixedsizescoped{}\code{<datapar_size_v<T, Abi>>}}{\common{T}{U}}>].
    \item Otherwise, if \type U is implicitly convertible to \datapar{}\type{<V, A>}, where \type V and \type A are determined according to standard template type deduction,
      then the return type shall be \datapar[<\common{T}{V}, \commonabi{Abi}{A}{\common{T}{V}}>].
    \item Otherwise, if \type U is implicitly convertible to \datapar{}\type{<T, Abi>},
      the return type shall be \datapar{}\type{<T, Abi>}.
    \item Otherwise the operator does not participate in overload resolution.%
      \comment{
        This seems a strange place to put this.
        Alternatively, modify the above rule to unconditionally use \datapar{}\type{<T, Abi>}.
        The paragraph below would lead to the same effect.
      }
  \end{itemize}

  \pnum\remarks Each of these operators only participate in overload resolution if all of the following hold:
  \begin{itemize}
    \item The indicated operator can be applied to objects of type \dataparreturntype{}\code{::}\type{value_type}.
    \item \datapar{}\type{<T, Abi>} is implicitly convertible to \dataparreturntype.
    \item \type U is implicitly convertible to \dataparreturntype.
  \end{itemize}
  \comment{
    I think this allows \code{datapar<float, fixed_size<4>>() + double()} and returns \datapar[<double, fixed_size<4>>].
    I also think that's what we want.
  }

  \pnum\remarks The operators with \type{const U \&} as first parameter shall not participate in overload resolution if \code{is_datapar_v<U> == true}.

  \pnum\returns A new \datapar object initialized with the results of the component-wise application of the indicated operator after both operands have been converted to the return type.
\end{itemdescr}

\wgSubsubsection{\datapar compound assignment}{datapar.cassign}
\begin{itemdecl}
template <class T, class Abi, class U> datapar<T, Abi> &operator+= (datapar<T, Abi> &, const U &);
template <class T, class Abi, class U> datapar<T, Abi> &operator-= (datapar<T, Abi> &, const U &);
template <class T, class Abi, class U> datapar<T, Abi> &operator*= (datapar<T, Abi> &, const U &);
template <class T, class Abi, class U> datapar<T, Abi> &operator/= (datapar<T, Abi> &, const U &);
template <class T, class Abi, class U> datapar<T, Abi> &operator%= (datapar<T, Abi> &, const U &);
template <class T, class Abi, class U> datapar<T, Abi> &operator&= (datapar<T, Abi> &, const U &);
template <class T, class Abi, class U> datapar<T, Abi> &operator|= (datapar<T, Abi> &, const U &);
template <class T, class Abi, class U> datapar<T, Abi> &operator^= (datapar<T, Abi> &, const U &);
template <class T, class Abi, class U> datapar<T, Abi> &operator<<=(datapar<T, Abi> &, const U &);
template <class T, class Abi, class U> datapar<T, Abi> &operator>>=(datapar<T, Abi> &, const U &);
\end{itemdecl}
\begin{itemdescr}
  \pnum\remarks Each of these operators only participates in overload resolution if all of the following hold:
  \begin{itemize}
    \item The indicated operator can be applied to objects of type \type{datapar_return_type<datapar<T, Abi>, U>::value_type}.
    \item \datapar{}\type{<T, Abi>} is implicitly convertible to \type{datapar_return_type<datapar<T, Abi>, U>}.
    \item \type U is implicitly convertible to \type{datapar_return_type<datapar<T, Abi>, U>}.
    \item \type{datapar_return_type<datapar<T, Abi>, U>} is implicitly convertible to \datapar{}\type{<T, Abi>}.
  \end{itemize}
  \pnum\effects Each of these operators performs the indicated operation component-wise on each of the elements of the first argument and the corresponding element of the second argument after conversion to \datapar{}\code{<T, Abi>}.
  \pnum\returns A reference to the first argument.
\end{itemdescr}

\wgSubsubsection{\datapar logical operators}{datapar.logical}
\realnote The omission of logical operators is deliberate.

\wgSubsubsection{\datapar compare operators}{datapar.comparison}
\begin{itemdecl}
template <class T, class Abi, class U>
typename datapar_return_type<datapar<T, Abi>, U>::mask_type operator==(datapar<T, Abi>, const U &);
template <class T, class Abi, class U>
typename datapar_return_type<datapar<T, Abi>, U>::mask_type operator!=(datapar<T, Abi>, const U &);
template <class T, class Abi, class U>
typename datapar_return_type<datapar<T, Abi>, U>::mask_type operator>=(datapar<T, Abi>, const U &);
template <class T, class Abi, class U>
typename datapar_return_type<datapar<T, Abi>, U>::mask_type operator<=(datapar<T, Abi>, const U &);
template <class T, class Abi, class U>
typename datapar_return_type<datapar<T, Abi>, U>::mask_type operator> (datapar<T, Abi>, const U &);
template <class T, class Abi, class U>
typename datapar_return_type<datapar<T, Abi>, U>::mask_type operator< (datapar<T, Abi>, const U &);
template <class T, class Abi, class U>
typename datapar_return_type<datapar<T, Abi>, U>::mask_type operator==(const U &, datapar<T, Abi>);
template <class T, class Abi, class U>
typename datapar_return_type<datapar<T, Abi>, U>::mask_type operator!=(const U &, datapar<T, Abi>);
template <class T, class Abi, class U>
typename datapar_return_type<datapar<T, Abi>, U>::mask_type operator>=(const U &, datapar<T, Abi>);
template <class T, class Abi, class U>
typename datapar_return_type<datapar<T, Abi>, U>::mask_type operator<=(const U &, datapar<T, Abi>);
template <class T, class Abi, class U>
typename datapar_return_type<datapar<T, Abi>, U>::mask_type operator> (const U &, datapar<T, Abi>);
template <class T, class Abi, class U>
typename datapar_return_type<datapar<T, Abi>, U>::mask_type operator< (const U &, datapar<T, Abi>);
\end{itemdecl}
\begin{itemdescr}
  \pnum\remarks The return type of these operators shall be the \type{mask_type} member type of the type deduced according to the rules defined in [datapar.binary].

  \pnum\remarks Each of these operators only participates in overload resolution if all of the following hold:
  \begin{itemize}
    \item \datapar{}\type{<T, Abi>} is implicitly convertible to \type{datapar_return_type<datapar<T, Abi>, U>}.
    \item \type U is implicitly convertible to \type{datapar_return_type<datapar<T, Abi>, U>}.
  \end{itemize}

  \pnum\remarks The operators with \type{const U \&} as first parameter shall not participate in overload resolution if \code{is_datapar_v<U> == true}.

  \pnum\returns A new \mask object initialized with the results of the component-wise application of the indicated operator after both operands have been converted to \type{datapar_return_type<datapar<T, Abi>, U>}.
\end{itemdescr}

\wgSubsubsection{\datapar casts}{datapar.casts}
\begin{itemdecl}
template <class T, class U, class... Us>
conditional_t<(T::size() == (U::size() + Us::size()...)), T,
            array<T, (U::size() + Us::size()...) / T::size()>> datapar_cast(U, Us...);
\end{itemdecl}
\begin{itemdescr}
  \pnum\remarks The \code{datapar_cast} function only participates in overload resolution if all of the following hold:
  \begin{itemize}
    \item \code{is_datapar_v<T>}
    \item \code{is_datapar_v<U>}
    \item All types in the template parameter pack \type{Us} are equal to \type U.
    \item \code{U::size() + Us::size()...} is an integral multiple of \code{T::size()}.
  \end{itemize}

  \pnum\returns A new \datapar object initialized with the converted values as one object of \type T or an array of \type T.
  All scalar elements \code{x\textsubscript{i}} of the function argument(s) are converted as if
  \code{y\textsubscript{i} = static_cast<typename T::value_type>(x\textsubscript{i})} is executed.
  The resulting \code{y\textsubscript{i}} intialize the return object(s) of type \type T.
  \wgNote{%
    For \code{T::size() == 2 * U::size()} the following holds:
    \code{datapar_cast<T>(x0, x1)[i] == static_cast<typename T::value_type>(array<U, 2>\{x0, x1\}[i / U::size()][i \% U::size()])}.
    For \code{2 * T::size() == U::size()} the following holds:
    \code{datapar_cast<T>(x)[i][j] == static_cast<typename T::value_type>(x[i * T\MayBreak::size() + j])}.
  }
\end{itemdescr}

\wgSubsubsection{\datapar transcendentals}{datapar.transcend}
\textit{TODO}

