\section{Generalization}
By defining a variable template \std\code{simd_iota<T>} where \code{T} must
be a \simd type, we're simply initializing a sequence of values at compile time.
We can create such an object for more types.
This is especially interesting for the degenerate case in SIMD-generic
programming, where \code{T} could \eg be an \code{int}.
An \std\code{simd_iota<int>} is nothing other than an object \code{int} with value
\code{0}.

We can easily generalize to \std\code{iota_v<std::array<T, N>>} and
\std\code{iota_v<T[N]>}.
And the next step then is to allow any type that
\begin{itemize}
  \item has a static extent,
  \item has a \code{value_type} member type,
  \item can be list-initialized with \code{N} numbers of type
    \code{value_type}, where \code{N} equals the static extent of the type, and
  \item where \code{value_type() + 1} is an constant expression and convertible to \code{value_type}.
\end{itemize}

But there are more types (in the standard library and beyond) where we can
create such an object.
All we need is a type
\begin{enumerate}
  \item with valid \code{ranges::range_value_t<T>} type (this could be weakened
    to also allow \code{std::\MayBreak{}tuple<\MayBreak{}int, int>}),
  \item with static extent (\code{T::size()}, \code{T::extent},
    \code{std::extent_v<T>}, or \code{std::tuple_size_v<T>}),
  \item and that can be list-initialized from a sequence of \code{N}
    integers (cast to \code{range_value_t<T>}), where \code{N} equals
    the static extent of the type.
\end{enumerate}
For the scalar case, a very general constraint requires \code{T} to be
\begin{itemize}
  \item a regular type
  \item that can be list-initialized from a single value
  \item and that compares equal to that value after construction.
\end{itemize}

Consequently you could write
\medskip\begin{lstlisting}[style=Vc]
auto x = std::iota_v<float[5]>;
auto y = std::iota_v<std::array<my_fixed_point, 8>>;
// ...
\end{lstlisting}

A second generalization could allow different sequences other than only
\code{0, 1, 2, 3, 4, \ldots}.
\std\code{iota} and \std\code{ranges::iota} take a \code{value} argument to
define the first value in the sequence.
They do not allow any different step other than applying the pre-increment
operator.

For \code{simd}, I would typically just write \eg
\medskip\begin{lstlisting}[style=Vc]
constexpr auto vec = std::iota_v<std::simd<int>> * 3 + 5; // 5, 8, 11, ...
\end{lstlisting}
To construct the same sequence for an array, \code{iota_v} would require a
“first” and a “step” argument:
\medskip\begin{lstlisting}[style=Vc]
constexpr auto arr = std::iota_v<std::array<int, 4>, 5, 3>; // 5, 8, 11, 14
\end{lstlisting}

Providing a (defaulted) “step” argument is simple and more general.
The only reason, that I can think of, for not adding it is that \std\code{iota}
/ \std\code{ranges::iota} don't have it.
