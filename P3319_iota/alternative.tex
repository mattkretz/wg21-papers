\section{Alternative: reuse existing \code{iota}}

We already have \code{std::iota} and \code{std::ranges::iota}.
Why isn't that sufficient to create a solution that composes?

One motivation for \code{iota_v<simd<int>>} instead of \code{simd<int>::iota}
is that \code{iota_v<int>} works while \code{int::iota} cannot work.
The same is true for \code{simd<int>(views::iota(0))} vs.
\code{int(views::\MayBreak{}iota(0))}.
Supporting the degenerate case is very helpful for SIMD-generic programming.

\medskip\begin{lstlisting}[style=Vc]
// scalar loop:
for (int i = 0; i < 1024; ++i) {
  ...
}

// simd loop:
for (auto i = iota_v<simd<int>>; all_of(i < 1024); i += simd<int>::size) {
  ...
}

// simd-generic loop:
for (auto i = iota_v<T>; all_of(i < 1024); i += simd_size_v<T>) {
  ...
}

// alternative:
for (int ii = 0; ii < 1024; ii += simd_size_v<T>) {
  T i = ii + iota_v<T>;
  ...
}
\end{lstlisting}

In addition, what \code{simd(range)} does depends on the outcome of P3299
“range ctors”.
In any case we are requiring contiguous_range.
So \code{simd(random_access_range)} needs another paper altogether (while
  convenient, this is rarely what the user wanted; making non-contiguous loads
ill-formed helps against “performance errors”).
So we could overload for specific non-contiguous ranges, where we know that we
can restore good performance.
But that's going to be a closed set, rather than a general concept.
Why then would \code{simd(std::views:iota(0))} work but
\code{simd(boost::\MayBreak{}views::\MayBreak{}iota(0))} is ill-formed?

Our desired outcome of P3299 is that \code{simd(range)} requires a statically
sized contiguous range with exactly matching size.
Thus we would need to call
\code{std::simd::load<simd<int>>(std::\MayBreak{}views::\MayBreak{}iota(0))} instead.
This is now even more verbose than the current solution \code{simd<int>([](int
i) { return i; })}.
It completely fails at the goal to make the code more readable.

Then what about \code{std::views::iota(0) | std::ranges::to<basic_simd>()}?
It's still too long for a rather basic constant.
And why should this work if both
\begin{itemize}
  \item \code{std::views::iota(0) | std::ranges::to<std::array>();} and
  \item \code{std::views::iota(0) | std::ranges::to<std::array<int, 4>();}
\end{itemize}
don't work?

