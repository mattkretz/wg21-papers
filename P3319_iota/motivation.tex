\section{Motivation}
The 90\%\footnote{Sorry, that number is completely made up.} use case for simd
generator constructors is a simd with values 0, 1, 2, 3, \ldots{} potentially
with scaling and offset applied.
However, often it would be easier and more readable to use an “iota”
\code{simd} object instead.

\begingroup
  \smaller[1]
\begin{tabular}{p{.45\textwidth}|p{.45\textwidth}}
  generator ctor & iota \\
  \hline
  \begin{lstlisting}
std::simd<int> a([](int i) { return i; };

std::simd<int> b([](int i) { return 2 + 3 * i; };
  \end{lstlisting}
  &
  \begin{lstlisting}
auto a = std::iota_v<std::simd<int>>;

auto b = 2 + 3 * std::iota_v<std::simd<int>>;
  \end{lstlisting}
\end{tabular}
\endgroup

\pagebreak
An example where an \code{iota_v<simd>} comes up is the calculation of the
Mandelbrot set.
The program needs to iterate over all visible pixels and calculate the
corresponding value in the complex plane.
Thus a loop like
\medskip\begin{lstlisting}[style=Vc]
for (int x = 0; x < 1024; ++x) {
  float real = float(x) * scale + offset;
\end{lstlisting}
turns into
\medskip\begin{lstlisting}[style=Vc]
using floatv = simd<float>;
using intv = rebind_simd_t<int, floatv>;
for (intv x = iota_v<intv>; any_of(x < 1024); x += intv::size()) {
  floatv real = floatv(x) * scale + offset;
\end{lstlisting}

The minimal definition I propose for \simd can look like this:
\medskip\begin{lstlisting}[style=Vc]
namespace std {
  template <class T>
    requires std::is_arithmetic_v<T> or detail::simd_type<T>
    inline constexpr T
    simd_iota_v = T();

  template <detail::simd_type T>
    inline constexpr T
    simd_iota_v<T> = T([](auto i) { return static_cast<typename T::value_type>(i); });
}
\end{lstlisting}

If \cite{P3287R0} is adopted to introduce a \std\code{simd} namespace, it would
be called \std\code{simd::\MayBreak{}iota_v} (and
\std\code{simd_generic::\MayBreak{}iota_v}) instead.

