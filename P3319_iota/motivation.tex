\section{Motivation}
The 90\%\footnote{Sorry, that number is completely made up.} use case for simd
generator constructors is a simd with values 0, 1, 2, 3, \ldots{} potentially
with scaling and offset applied.
However, often it would be easier and more readable to use an “iota”
\code{simd} object instead.

\begingroup
  \smaller[1]
\begin{tabular}{p{.45\textwidth}|p{.45\textwidth}}
  generator ctor & iota \\
  \hline
  \begin{lstlisting}
namespace dp = std::datapar;

dp::simd<int> a([](int i) { return i; };

dp::simd<int> b([](int i) { return 2 + 3 * i; };
  \end{lstlisting}
  &
  \begin{lstlisting}
namespace dp = std::datapar;

auto a = dp::iota<dp::simd<int>>;

auto b = 2 + 3 * dp::iota<dp::simd<int>>;
  \end{lstlisting}
\end{tabular}
\endgroup

\pagebreak
An example where an \code{datapar::iota<simd>} comes up is the calculation of the
Mandelbrot set.
The program needs to iterate over all visible pixels and calculate the
corresponding value in the complex plane.
Thus a loop like
\medskip\begin{lstlisting}[style=Vc]
for (int x = 0; x < 1024; ++x) {
  float real = float(x) * scale + offset;
\end{lstlisting}
turns into
\medskip\begin{lstlisting}[style=Vc]
using floatv = dp::simd<float>;
using intv = dp::rebind_t<int, floatv>;
for (intv x = dp::iota<intv>; any_of(x < 1024); x += intv::size()) {
  floatv real = floatv(x) * scale + offset;
\end{lstlisting}

The minimal definition proposed can be implemented like this:
\medskip\begin{lstlisting}[style=Vc]
namespace std::datapar {
  template <class T>
    requires is_arithmetic_v<T>
      || (@\exposconcept{simd-type}<T>@ && is_arithmetic_v<typename T::value_type>)
    constexpr T iota = T();

  template <class T, class Abi>
    constexpr basic_simd<T, Abi>
    iota<basic_simd<T, Abi>>([](T i) {
      static_assert(basic_simd<T, Abi>::size() - 1 <= numeric_limits<T>::max());
      return i;
    });
}
\end{lstlisting}
