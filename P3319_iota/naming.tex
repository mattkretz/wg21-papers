\section{Naming: Is reuse of the term “iota” confusing or helpful?}

In the Vc library, the library behind the initial proposal back in 2013,
there's a \code{Vc::\MayBreak{}Vector<T>::\MayBreak{}Indexes\MayBreak{}From\MayBreak{}Zero()} constant.
Back then SG1/WG21 wanted to reduce the scope for the TS to a minimum and the
constant was never considered any further.
In any case, \code{IndexesFromZero} is a fairly descriptive/elaborate name.
But in the standard library we already have a term for a sequence like this.
And it's “iota”.
Using a different term for something that isn't different (concept) is
confusing and incoherent.

\code{std::iota} has an existing meaning, as an algorithm that initializes a
given existing range.
What this paper proposes is still sufficiently different that we don't want to
overload that exact name.
Instead, since we're defining an “iota value”, we propose the name \code{iota_v}.

If we don't move \code{simd} into a \std\code{simd} subnamespace and if we
don't want to generalize the “iota value” beyond \code{simd}, then we should be
considering \code{std::simd_iota_v} over \code{std::iota_v}.
