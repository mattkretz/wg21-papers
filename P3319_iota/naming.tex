\section{Naming: Is reuse of the term “iota” confusing or helpful?}

In the Vc library, the library behind the initial proposal back in 2013,
there's a \code{Vc::\MayBreak{}Vector<T>::\MayBreak{}Indexes\MayBreak{}From\MayBreak{}Zero()} constant.
Back then SG1/WG21 wanted to reduce the scope for the TS to a minimum and the
constant was never considered any further.
In any case, \code{IndexesFromZero} is a fairly descriptive/elaborate name.
But in the standard library we already have a term for a sequence like this.
And it's “iota”.
Using a different term for something that isn't different (concept) is
confusing and incoherent.

\code{std::iota} has an existing meaning, as an algorithm that initializes a
given existing range.
What this paper proposes is sufficiently different that we don't want to
overload that exact name.
In addition, with \code{std::iota} being a function and this proposal adding a
variable template it is technically impossible to overload the same name.

If we decide not to generalize the facility then \std\code{simd_iota} /
\stdsimd\code{iota} is the preferred name.
If we do want to generalize, we propose the name \std\code{iota_v}, since we're
defining an “iota value”.
If LEWG considers the \code{_v} suffix to be reserved for traits then we should
consider \std\code{iota_value} instead.

