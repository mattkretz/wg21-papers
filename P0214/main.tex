\newcommand\wgTitle{Data-Parallel Vector Types \& Operations}
\newcommand\wgName{Matthias Kretz <m.kretz@gsi.de>}
\newcommand\wgDocumentNumber{D0214R9}
\newcommand\wgGroup{LWG}
\newcommand\wgTarget{Parallelism TS 2}
\newcommand\wgAcknowledgements{
  \begin{itemize}
    \item This work was supported by GSI Helmholtzzentrum für Schwerionenforschung
      and the Hessian LOEWE initiative through the Helmholtz International Center
      for FAIR (HIC for FAIR).

    \item Jens Maurer contributed important feedback and suggestions on the API.
      Thanks also for presenting the paper in Kona 2017 and Toronto 2017.

    \item Thanks to Hartmut Kaiser for presenting in Issaquah 2016.

    \item Geoffrey Romer did a very helpful review of the wording.
  \end{itemize}
}

\usepackage{mymacros}
\usepackage{wg21}
\usepackage{underscore}
\useStdStablenameRef

\addbibresource{extra.bib}

\newcommand\simd[1][]{\type{simd#1}\xspace}
\newcommand\valuetype{\type{value\_type}\xspace}
\newcommand\referencetype{\type{reference}\xspace}
\newcommand\whereexpression{\type{where\_expression}\xspace}
\newcommand\simdcast{\code{simd\_cast}\xspace}
\newcommand\mask[1][]{\type{simd\_mask#1}\xspace}
\newcommand\fixedsizeN{\type{simd\_abi::fixed\_size<N>}\xspace}
\newcommand\fixedsizescoped{\type{simd\_abi::fixed\_size}\xspace}
\newcommand\fixedsize{\type{fixed\_size}\xspace}
\newcommand\simdEP{\code{execution::}\type{simd}\xspace}
\newcommand\seqEP{\code{execution::}\type{seq}\xspace}
\newcommand\dataparalleltype{data-parallel type\xspace}
\newcommand\dataparalleltypes{data-parallel types\xspace}

\newcommand\flagsRequires[1]{%
  \pnum\requires
  If the template parameter \type{Flags} is \type{vector\_aligned\_tag}, \code{mem} shall point to storage aligned by \code{memory\_alignment\_v<#1>}.
  If the template parameter \type{Flags} is \type{overaligned\_tag<N>}, \code{mem} shall point to storage aligned by \code N.
  If the template parameter \type{Flags} is \type{element\_aligned\_tag}, \code{mem} shall point to storage aligned by \code{alignof(value\_type)}.
}

\newcommand\disallowUserSpecialization[1]{The behavior of a program that adds specializations for \code{#1} is undefined.}

\newcommand\targetArch{target architecture\xspace}
\newcommand\targetArchs{target architectures\xspace}
\newcommand\currentTarget{currently targeted system\xspace}

%\newcommand\specialsfinae{shall either not be declared or not participate in overload resolution}
\newcommand\specialsfinae{shall not participate in overload resolution\xspace}

\newcommand\realArithmeticType{vectorizable type\xspace}

\usepackage{pifont}

\newcommand\foralli[1][]{for all \code i $\in$ \code{[0, #1size())}\xspace}
\newcommand\forallmaskedi[2][]{%
  for all \code i
  $\in \{j \in \mathbb{N}_0 | j < \code{#1size()} ⋀ \code{#2[}j\code{]}\}$%
  \xspace%
}
\newcommand\chck{\item[\color{black}\ensuremath{\checkmark}]}
\newcommand\todo{\item[\color{black}\ding{46}] \color{gray}}
\newcommand\itemheader[1]{\item[] \hfill \textcolor{gray}{\textsc{#1}}}

\begin{document}
\selectlanguage{american}
\begin{wgTitlepage}
  This paper describes class templates for portable data-parallel (e.g. SIMD) programming via vector types.
\end{wgTitlepage}

\pagestyle{scrheadings}
\addtocounter{section}{-1}
\section{Remarks}
\begin{itemize}
  \item This documents talks about “vector” types/objects.
    In general this will not refer to the \std\type{vector} class template.
    References to the container type will explicitly call out the \code{std} prefix to avoid confusion.
  \item In the following, \VSize{T} denotes the number of scalar values (width) in a vector of type \type T (sometimes also called the number of SIMD lanes)
  \item \parencite{N4184}, \parencite{N4185}, and \parencite{N4395} provide more information on the rationale and design decisions.
    \parencite{N4454} discusses a matrix multiplication example.
    My PhD thesis \parencite{Kretz2015} contains a very thorough discussion of the topic.
  \item This paper is not supposed to specify a complete API for data-parallel types and operations.
    It is meant as a useful starting point.
    Once the foundation is settled on, higher level APIs will be proposed.
\end{itemize}

\section{Changelog}
\subsection{Changes from R8}
Previous revision: \parencite{P0214R8}.
\begin{itemize}
  \item Use the word-of-power “width” in the specification of \code{size()}.

  \item Aligned wording in \code{copy_from} and \code{copy_to} restricting pointer alignment with the wording in [ptr.align].
    Also added wording for the meaning of \code{element_aligned_tag}.

  \item Fixed missing SFINAE constraints on \code{Flags} in masked load.
    Streamlined the wording to use itemization for listing multiple constraints.

  \item Removed \code{const} from the second template argument to \code{const_where_expression}.
    Consequently, removed \code{remove_const_t} from the defintions and wording in \code{const_where_expression}.
    (I verified the change in my implementation.)

  \item Use "extended ABI tag" instead of "implementation-defined ABI tag". The corresponding word of power is defined in \ref{extended ABI tag}.

  \item Renamed \code{abi_for_size<T, N>} to \code{simd_abi::deduce<T, N>}.

  \item Improved wording in case multiple ABI tags match the constraints passed to \code{simd_abi::deduce<T, N>}.

  \item Clarify valid values for \code N for \code{is_simd_flag_type<overaligned_tag<N>>} (\ref{is_simd_flag_type}).

  \item Call out that user specialization of traits is UB.

  \item Simplified \code{memory_alignment} SFINAE logic.

  \item Improved wording for selected elements/indexes.

  \item Rewrite the logic for what template arguments \simd and \mask are supported.

  \item Add an example to \ref{impl-conversion-example}.

  \item Handle cv-qualification in broadcast ctor (\ref{sec:simd.ctor}).

  \item Call out constraints on the generator callable in the generator ctor (\ref{sec:simd.ctor}).

  \item Add Throws specification to \code{reduce} overloads that a user-defined \code{binary_op}.
\end{itemize}

\subsection{Changes from R7}
Previous revision: \parencite{P0214R7}.
\begin{itemize}
  \item Removed the explicit signatures of the special math functions in favor of a rule to translate the \code{<cmath>} signatures to \simd types.
    This resolves an overspecification, which disallowed certain valid implementation strategies. (see \ref{cl:cmath-spec})

  \item Added margin-note questioning non-normative note about special math function preconditions. (see \ref{cl:precond-violation})

  \item Fixed and finalized note on the intent of the data-parallel library. (see \ref{cl:intent-note})

  \item Renamed \code{neutral_element} to \code{identity_element} and specified the requirement on its value in relation to \code{binary_op}. Removed the note explaining an implementation detail. (see \ref{sec:simd.reductions})
\end{itemize}

\subsection{Changes from R6}
Previous revision: \parencite{P0214R6}.
\begin{itemize}
  \item Fixed return type of \code{isinf}. (\ref{sec:simd.math})
  \chck Changed namespace to \code{std::experimental::parallelism_v2}.
  \chck Shortened intro sentence in \ref{sec:simd.syn}.
  \chck Added missing \code{const_where_expression} overload for \code{where(bool, const T\&}. (\ref{sec:simd.syn}, \ref{sec:simd.mask.where}) (The overload is present in my reference implementation.)
  \chck Removed \code{const} from \code{const_where_expression} return type. (\ref{sec:simd.syn})
  \item Fixed incorrect \code{const where_expression} return types in \ref{sec:simd.mask.where}.
  \chck Removed \code{flags} namespace \ref{sec:simd.syn}.
  \chck all of your global constexpr variables should be inline
  \chck Rename template parameters \code{A} to \code{Abis} and \code{As...} to \code{Abis...}.
  \chck Stable names follow a hierarchy; use [simd.class] instead of [simd]; rename [simd_mask] to [simd.mask]
  \chck reduce: Moved defaulted \code{BinaryOperation} argument to the end.
  \chck reduce: Replaced \code{default_natural_element} with \emph{implementation-defined}
  \chck reduce: Explained what the default \code{natural_element} is supposed to do. (\ref{sec:simd.reductions})
  \chck Consistently initialze tags.
  \chck Put the “see belows” in italics instead of comments.
  \chck Consistently pass by const-ref.
  (LWG: Why are the arguments by value?
  Be consistent.
  Resolved to consistently pass by const-ref. Rationale: if an implementation wants a non-inline, pass-by-value function it can inline the public const-ref function and call an internal function.
  If the public function uses by value parameters it may happen that the implied copy is not optimized away.)
  \chck Simplify casts using aliases or default arguments.
  \chck Split second \code{reduce} overload in six functions, one without default \code{neutral_element} and five explicitly overloaded for \code{plus<>}, \code{multiplies<>}, \code{bit_and<>}, \code{bit_or}, and \code{bit_xor}.
  \chck Improved requires clause on \code{reduce} to mention return type and generalize ABI tag type requirement.
  \chck Require integral \code{value_type} for bitwise reductions.
  \chck Removed impossible condition from \code{simd_cast} remark.
  \chck Fixed masked \code{hmin} and \code{hmax} specification to use “forallmaskedi”.
  \chck Forward \code{min}, \code{max}, \code{minmax}, and \code{clamp} to \code{std::min/max/minmax/clamp}.
  \chck \code{scalar} \emph{is} not an alias for \code{fixed_size<1>}.
  \chck “an implementation \emph{shall} support at least \emph{all} \ldots” (\ref{sec:simd.abi})
  \chck Replaced “exact-bool” arguments from \emph{implementation-defined} to \emph{see below}.
  \chck Reveresed “if” to “unless” logic.
  \chck Fixed \code{is_mask} to \code{is_simd_mask}.
  \chck Fixed incorrect \code{simd_mask} exposition-only member name to \code{mask} (\ref{sec:simd.whereexpr}).
  \chck Fixed constraints on generator ctor to require generator be callable with all element indexes.
  \chck Fixed wording to allow vectorized execution of the generator.
  \chck Moved all wording about “target” or “architecture” into non-normative notes.
  \chck Add \code{is_simd_flag_type} trait and use it for all loads and stores.
  \chck Define and use the term \emph{\realArithmeticType} to simplify the wording.
  \chck Define “selected elements” in \code{where_expression} to use it instead of \code{data[i]} where \code{mask[i]} is \true.
  \chck \ref{sec:simd.whereexpr} reword what the members of \code{where_expression} mean and where they come from
  \chck Replaced “floating-point and integral” with “arithmetic”.
  \chck Consistently use “element” instead of “component”.
  \chck Consistently place “and”/“or” and the end of bullet points instead of the front.
  \item Allow shifts on signed integers as was requested by LEWG.
  \chck Added introduction in \ref{sec:simd.general}.
  \chck Added wording in \ref{sec:simd.general} to clarify that the application of operations/operators on elements in \simd and \mask are unsequenced with respect to each other.
  \chck Fix Order: requires, effects, sync, post-cond, returns, throws, complexity, remarks, error-cond
  \chck Removed repetitions of clause names.
  \chck “Let X be foo” doesn't need a “Remarks” clause.
  \chck Defined “vectorizable types” as a term for “arithmetic types other than bool”.
  \item Removed immutable masked load.
    Requested in LWG review session because it's too clever and may block/hinder acceptance.
    LEWG queried/informed via reflector.
    Quick review of the issue at the start of Jacksonville meeting requested.
  \item Introduced exposition-only \code{nodeduce_t} to replace \code{remove_const_t} (\ref{sec:simd.mask.where}).
  \item Removed \code{noexcept} from \code{hmin} and \code{hmax} in the synopsis, to match the declarations below.

  %\todo Find better wording for removing \code{operator~} and bitwise operators from \simd.
  %\todo look for prior art to improve the \returns clause in increment and decrement
  %\todo (optional: put function argument names in the synopsis)
\end{itemize}

\subsection{Changes from R5}
Previous revision: \parencite{P0214R5}.
\begin{itemize}
  \item Renamed \code{memload} to \code{copy_from} and \code{memstore} to \code{copy_to}. (\ref{sec:simd.copy}, \ref{sec:simd.mask.copy})
  \item Fixed \code{split} to never convert the \valuetype. (\ref{sec:simd.casts})
  \item Added missing \type{long double} overload of \code{ceil}. (\ref{sec:simd.math})
\end{itemize}

\subsection{Changes from R4}
Previous revision: \parencite{P0214R4}.
\begin{itemize}
  \item Changed \type{align_val_t} argument of \type{overaligned<N>} and \type{overaligned_tag<N>} to \type{size_t}.
    (Usage is otherwise too cumbersome.)
  \itemheader{changes after LEWG Review}
  \item Remove section on naming (after the topic was discussed and decided in LEWG).
  \item Rename \code{datapar} to \code{simd} and \code{mask} to \code{simd_mask}.
  \item Remove incorrect template parameters to scalar boolean reductions.
  \item Merge proposed \code{simd_cast} into the wording (using Option 3) and extend \code{static_simd_cast} accordingly.
  \item Merge proposed \code{split} and \code{concat} functions into the wording.
  \item Since the target is a TS, place headers into \code{experimental/} directory.
  %\itemheader{changes after Toronto}
  %\todo New section on design decisions; discuss decisions made by SG1, LEWG, and through discussion with Jens; especially discuss the ABI parameter:
  %\begin{itemize}
  %  \item it provides the intended lowest level access to SIMD programming
  %  \item it allows easy high-level abstraction
  %  \item how the default was chosen, and why having a \emph{compatible} ABI be the default is the nicer default
  %  \item an ABI parameter influences the choice of function parameter passing, not necessarily the choice of instructions or even number of elements (e.g. an implementation can provide an ABI that passes via 128-bit registers but otherwise works with 256-bit registers)
  %\end{itemize}
\end{itemize}

\subsection{Changes from R3}
Previous revision: \parencite{P0214R3}.
\begin{itemize}
  \itemheader{changes before Kona}
  \item Add special math overloads for signed char and short.
        They are important to avoid widening to multiple SIMD registers and since no integer promotion is applied for \simd types.
  \item Editorial: Prefer \code{using} over \code{typedef}.
  \item Overload shift operators with \intt argument for the right hand side.
        This enables more efficient implementations.
        This signature is present in the Vc library, and was forgotten in the wording.
  \item Remove empty section about the omission of logical operators.
  \item Modify \mask compares to return a \mask instead of \bool (\ref{sec:simd.mask.comparison}).
        This resolves an inconsistency with all the other binary operators.
  \item Editorial: Improve \referencetype member specification (\ref{sec:simd.overview}).
  \item Require \code{swap(v[0], v[1])} to be valid (\ref{sec:simd.overview}).
  \item Fixed inconsisteny of masked load constructor after move of \code{memload} to \type{where_expression} (\ref{sec:simd.whereexpr}).
  \item Editorial: Use Requires clause instead of Remarks to require the memory argument to loads and stores to be large enough (\ref{sec:simd.whereexpr}, \ref{sec:simd.copy}, \ref{sec:simd.mask.copy}).
  \item Add a note to special math functions that precondition violation is UB (\ref{sec:simd.math}).
  \item Bugfix: Binary operators involving two \type{simd::reference} objects must work (\ref{sec:simd.overview}).
  \item Editorial: Replace Note clauses in favor of \wgNote{}.
  \item Editorial: Replace UB Remarks on load/store alignment requirements with Requires clauses.
  \item Add an example section (\ref{sec:Examples}).
  \itemheader{changes after Kona}
  \item[---] design related:
  \item Readd \bool overloads of mask reductions and ensure that implicit conversions to \bool are ill-formed.
  \item Clarify effects of using an ABI parameter that is not available on the target (\ref{simd.deleted}, \ref{simd_mask.deleted}, \ref{simd_size}).
  \item Split \whereexpression into \const and non-\const class templates.
  \item Add section on naming.
  \item Discuss the questions/issues raised on \code{max_fixed_size} in Kona (\autoref{sec:maxfixedsize}).
  \item Make \code{max_fixed_size} dependent on \type{T}.
  \item Clarify that converting loads and stores only work with arrays of non-bool arithmetic type (\ref{sec:simd.copy}).
  \item Discuss \mask and \type{bitset} reduction interface differences.
  \item Relax requirements on return type of generator function for the generator constructor (\ref{sec:simd.ctor}).
  \item Remove overly generic \code{simd_cast} function.
  \item Add proposal for a widening cast function.
  \item Add proposal for \code{split} and \code{concat} cast functions.
  \item Add \code{noexcept} or “Throws: Nothing.” to most functions.

  \item[---] wording fixes \& improvements:
  \item Remove non-normative noise about ABI tag types (\ref{sec:simd.abi}).
  \item Remove most of the text about vendor-extensions for ABI tag types, since it's QoI (\ref{sec:simd.abi}).
  \item Clarify the differences and intent of \type{compatible<T>} vs. \type{native<T>} (\ref{sec:simd.abi}).
  \item Move definition of \whereexpression out of the synopsis (\ref{sec:simd.whereexpr}).
  \item Editorial: Improve \code{is_simd} and \code{is_mask} wording (\ref{sec:simd.traits}).
  \item Make \emph{ABI tag} a consistent term and add \code{is_abi_tag} trait (\ref{sec:simd.traits}, \ref{sec:simd.abi}).
  \item Clarify that \fixedsizeN must support all \code{N} matching all possible extended ABI tags (\ref{sec:simd.abi}).
  \item Clarify \code{abi_for_size} wording (\ref{sec:simd.traits}).
  \item Turn \code{memory_alignment} into a trait with a corresponding \code{memory_alignment_v} variable template.
  \item Clarify \code{memory_alignment} wording; when it has no \code{value} member; and imply its value through a reference to the load and store functions (\ref{sec:simd.traits}).
  \item Remove exposition-only \type{where_expression} constructor and make exposition-only data members private (\ref{sec:simd.whereexpr}).
  \item Editorial: use “shall not participate in overload resolution unless” consistently.
  \item Add a note about variability of \code{max_fixed_size} (\ref{sec:simd.abi}).
  \item Editorial: use “\targetArch{}” and “\currentTarget{}” consistently.
  \item Add margin notes presenting a wording alternative that avoids “target system” and “target architecture” in normative text.
  \item Specify result of masked reduce with empty mask (\ref{sec:simd.reductions}).
  \item Editorial: clean up the use of “supported” and resolve contradictions resulting from incorrect use of conventions in the rest of the standard text (\ref{simd.type requirements}, \ref{simd_mask.type requirements}, \ref{sec:simd.traits}).
  \item Add \autoref{sec:feature-detection} \nameref{sec:feature-detection}.
\end{itemize}

\subsection{Changes from R2}
Previous revision: \parencite{P0214R2}.
\begin{itemize}
    \item Fixed return type of masked \code{reduce} (\ref{sec:simd.reductions}).
    \item Added binary \code{min}, \code{max}, \code{minmax}, and \code{clamp} (\ref{sec:simd.alg}).
    \item Moved member \code{min} and \code{max} to non-member \code{hmin} and \code{hmax}, which cannot easily be optimized from \code{reduce}, since no function object such as \code{std::plus} exists (\ref{sec:simd.reductions}).
    \item Fixed neutral element of masked \code{hmin}/\code{hmax} and drop UB (\ref{sec:simd.reductions}).
    \item Removed remaining reduction member functions in favor of non-member \code{reduce} (as requested by LEWG).
    \item Replaced \code{init} parameter of masked \code{reduce} with \code{neutral_element} (\ref{sec:simd.reductions}).
    \item Extend \type{where_expression} to support \const \simd objects (\ref{sec:simd.mask.where}).
    \item Fixed missing \code{explicit} keyword on \code{simd_mask(bool)} constructor (\ref{sec:simd.mask.ctor}).
    \item Made binary operators for \simd and \mask friend functions of \simd and \mask, simplifying the SFINAE requirements considerably (\ref{sec:simd.binary}, \ref{sec:simd.mask.binary}).
    \item Restricted broadcasts to only allow non-narrowing conversions (\ref{sec:simd.ctor}).
    \item Restricted simd to simd conversions to only allow non-narrowing conversions with \type{fixed_size} ABI (\ref{sec:simd.ctor}).
    \item Added generator constructor (as discussed in LEWG in Issaquah) (\ref{sec:simd.ctor}).
    \item Renamed \code{copy_from} to \code{memload} and \code{copy_to} to \code{memstore}. (\ref{sec:simd.copy}, \ref{sec:simd.mask.copy})
    \item Documented effect of \type{overaligned_tag<N>} as \type{Flags} parameter to load/store. (\ref{sec:simd.copy}, \ref{sec:simd.mask.copy})
    \item Clarified cv requirements on \type T parameter of \simd and \mask.
    \item Allowed all implicit \mask conversions with \fixedsize ABI and equal size (\ref{sec:simd.mask.ctor}).
    \item Made increment and decrement of \type{where_expression} return \type{void}.
    \item Added \code{static_simd_cast} for simple casts (\ref{sec:simd.casts}).
    \item Clarified default constructor (\ref{sec:simd.overview}, \ref{sec:simd.overview}).
    \item Clarified \simd and \mask with invalid template parameters to be complete types with deleted constructors, destructor, and assignment (\ref{sec:simd.overview}, \ref{sec:simd.overview}).
    \item Wrote a new subsection for a detailed description of \type{where_expression} (\ref{sec:simd.whereexpr}).
    \item Moved masked loads and stores from \simd and \mask to \type{where_expression} (\ref{sec:simd.whereexpr}).
          This required two more overloads of \code{where} to support value objects of type \mask (\ref{sec:simd.mask.where}).
    \item Removed \type{where_expression}\code{::operator!} (\ref{sec:simd.whereexpr}).
    \item Added aliases \type{native_simd}, \type{native_mask}, \type{fixed_size_simd}, \type{fixed_size_mask} (\ref{sec:simd.syn}).
    \item Removed \bool overloads of mask reductions awaiting a better solution (\ref{sec:simd.mask.reductions}).
    \item Removed special math functions with \code f and \code l suffix and \code l and \code{ll} prefix (\ref{sec:simd.math}).
    \item Modified special math functions with mixed types to use \type{fixed_size} instead of \code{abi_for_size} (\ref{sec:simd.math}).
    \item Added simple ABI cast functions \code{to_fixed_size}, \code{to_native}, and \code{to_compatible} (\ref{sec:simd.casts}).
\end{itemize}

\subsection{Changes from R1}
Previous revision: \parencite{P0214R1}.
\begin{itemize}
    \item Fixed converting constructor synopsis of \simd and \mask to also allow varying Abi types.
    \item Modified the wording of \code{\mask{}::native_handle()} to make the existence of the functions implementation-defined.
    \item Updated the discussion of member types to reflect the changes in R1.
    \item Added all previous SG1 straw poll results.
    \item Fixed \code{\textit{commonabi}} to not invent native Abi that makes the operator ill-formed.
    \item Dropped table of math functions.
    \item Be more explicit about the implementation-defined Abi types.
    \item Discussed resolution of the \fixedsizeN design (\ref{sec:fixedsize progress}).
    \item Made the \type{compatible} and \type{native} ABI aliases depend on \type T (\ref{sec:simd.abi}).
    \item Added \code{max_fixed_size} constant (\ref{simd.maxfixedsize.def}).
    \item Added masked loads.
    \item Added rationale for return type of \simd[::operator-()] (\ref{sec:unary minus}).
  \color{black}\item[---] SG1 guidance:
    \item Dropped the default load / store flags.
    \item Renamed the (un)aligned flags to \code{element_aligned} and \code{vector_aligned}.
    \item Added an \code{overaligned<N>} load / store flag.
    \item Dropped the ampersand on \code{native_handle} (no strong preference).
    \item Completed the set of math functions (i.e. add trig, log, and exp).
  \color{black}\item[---] LEWG (small group) guidance:
    \item Dropped \code{native_handle} and add non-normative wording for supporting \code{static_cast} to implementation-defined SIMD extensions.
    \item Dropped non-member load and store functions.
    Instead have \code{copy_from} and \code{copy_to} member functions for loads and stores. (\ref{sec:simd.copy}, \ref{sec:simd.mask.copy})
    (Did not use the \code{load} and \code{store} names because of the unfortunate inconsistency with \std\type{atomic}.)
    \item Added algorithm overloads for \simd reductions.
    Integrate with \code{where} to enable masked reductions. (\ref{sec:simd.reductions})
    This made it necessary to spell out the class \type{where_expression}.
\end{itemize}
\subsection{Changes from R0}
Previous revision: \parencite{P0214R0}.
\begin{itemize}
  \item Extended the \code{simd_abi} tag types with a \code{fixed_size<N>} tag to handle arbitrarily sized vectors (\ref{sec:simd.abi}).
  \item Converted \code{memory_alignment} into a non-member trait (\ref{sec:simd.traits}).
  \item Extended implicit conversions to handle \fixedsizeN (\ref{sec:simd.ctor}).
  \item Extended binary operators to convert correctly with \fixedsizeN (\ref{sec:simd.binary}).
  \item Dropped the section on “\simd logical operators”. Added a note that the omission is deliberate.
  \item Added logical and bitwise operators to \mask (\ref{sec:simd.mask.binary}).
  \item Modified \mask compares to work better with implicit conversions (\ref{sec:simd.mask.comparison}).
  \item Modified \code{where} to support different Abi tags on the \mask and \simd arguments (\ref{sec:simd.mask.where}).
  \item Converted the load functions to non-member functions.
    SG1 asked for guidance from LEWG whether a load-expression or a template parameter to load is more appropriate.
  \item Converted the store functions to non-member functions to be consistent with the load functions.
  \item Added a note about masked stores not invoking out-of-bounds accesses for masked-off elements of the vector.
  \item Converted the return type of \simd{}\code{::operator[]} to return a smart reference instead of an lvalue reference.
  \item Modified the wording of \mask{}\code{::operator[]} to match the reference type returned from \simd{}\code{::operator[]}.
  \item Added non-trig/pow/exp/log math functions on \simd.
  \item Added discussion on defaulting load/store flags.
  \item Added sum, product, min, and max reductions for \simd.
  \item Added load constructor.
  \item Modified the wording of \code{native_handle()} to make the existence of the functions implementation-defined, instead of only the return type.
    Added a section in the discussion (cf. Section \ref{sec:native}).
  \item Fixed missing flag objects.
\end{itemize}
  %\todo Fix missing flag objects and specify the freedom for implementations to add additional flags and that \code{operator|} must work.
  %\todo Added usage examples

\section{Straw Polls}
\subsection{SG1 at Kona 2022}
\wgPoll{After significant experience with the TS, we recommend that the next
version (the TS version with improvements) of \code{std::simd} target the IS (\CC{}26)}
{10&8&0&0&0}

\wgUnanimous{We like all of the recommended changes to \code{std::simd} proposed in p1928r1
(Includes making all of \code{std::simd} \code{constexpr}, and dropping an ABI stable type)}

\wgPoll{Future papers and future revisions of existing papers that target
\code{std::simd} should go directly to LEWG.
(We do not believe there are SG1 issues with \code{std::simd} today.)}
{9&8&0&0&0}

\section{Introduction}

Parallel Algorithms enable implementations of the existing STL algorithms to use non-sequential semantics when executing the user-supplied code (explicit callable or implicit operator call).
The first argument to the algorithm function determines this change in execution semantics via an \emph{execution policy}.
This paper introduces a new execution policy, called \dataparEP.
\dataparEP requires user-provided function objects to be callable with \datapar[<T, Abi>] arguments instead of the \type T arguments the \seqEP variant would use.
The algorithm therefore processes chunks of \datapar[<T, Abi>::size()] objects concurrently.
The execution order of the chunks retains the sequential semantics of the non-parallel algorithms.

As a consequence, the applicability of the execution policy is limited to iterators where \datapar[<Iterator::value_type>] is a valid template instantiation of \datapar.
A future extension of \datapar may lift this restriction by allowing certain (or all) user-defined types as first template argument to \datapar.


\section{Examples}\label{sec:Examples}

\subsection{Loop Vectorization}
This shows a low-level approach of manual loop chunking + epilogue for vectorization (“Leave no room for a lower-level language below \CC{} (except assembler).” \parencite{str99}).
It also shows SIMD loads, operations, write-masking (blending), and stores.
\medskip
\lstinputlisting[linerange={1-18}]{examples.cpp}

%\subsection{}


% ft=tex tw=0 et sw=2 spell

\section{Wording}\label{sec:wording}
\subsection{Feature test macro}

In [version.syn] bump the \code{__cpp_lib_simd} version.

\subsection{Changes to {[simd]}}
\def\rSec#1[#2]#3{%
  \ifcase#1\wgSubsection[subsection]{#3}{#2}
  \or\wgSubsubsection[subsubsection]{#3}{#2}
  \or\wgSubsubsubsection[paragraph]{#3}{#2}
  \or\error
\fi}

Add the following to \iref{simd.syn}, after the declaration of \code{cat}:
\begin{wgText}[{[simd.syn]}]
\begin{codeblock}
  template<size_t Bs, class... Abis>
    constexpr basic_simd_mask<Bs, @\deducet@<@\integerfrom@<Bs>,
                              (basic_simd_mask<Bs, Abis>::size() + ...)>>
      cat(const basic_simd_mask<Bs, Abis>&...) noexcept;

  @\wgAdd{template<class T> inline constexpr T iota = \mbox{\seebelow};}@

  // [simd.mask.reductions], \tcode{basic_simd_mask} reductions
\end{codeblock}
\end{wgText}

Add the following at the end of \iref{simd.creation}:
\begin{wgText}[{[simd.creation]}]
  \setcounter{WGClause}{29}
  \setcounter{WGSubSection}{10}
  \setcounter{WGSubSubSection}{7}
  \setcounter{WGSubSubSubSection}{6}
  \setcounter{Paras}{4}
\begin{itemdescr}
  \pnum\returns
  A data-parallel object initialized with the concatenated values in the \tcode{xs} pack of
  data-parallel objects: The $i^\text{th}$ \tcode{basic_simd}/\tcode{basic_simd_mask} element of the
  $j^\text{th}$ parameter in the \tcode{xs} pack is copied to the return value's element with index
  $i$ + the sum of the width of the first $j$ parameters in the \tcode{xs} pack.
\end{itemdescr}

\begin{wgBAdd}
\begin{itemdecl}
@\wgAdd{template<class T> inline constexpr T iota = \mbox{\seebelow};}@
\end{itemdecl}

\begin{itemdescr}
\pnum
\wgAdd{\constraints \tcode{is_arithmetic_v<T>} is \tcode{true} or \tcode{T}
is an enabled specialization of \code{basic_simd}.}

\pnum
\wgAdd{\mandates \tcode{is_arithmetic_v<T>} is \tcode{true} or
\tcode{T::size() - 1} $\le$ \tcode{numeric_limits<typename T::value_type>:: max()}.}

  \pnum
  \wgAdd{\effects
    If \tcode{is_arithmetic_v<T>} is \tcode{true} the value of
    \tcode{iota<T>} is equal to \tcode{T()}.
    Otherwise, the value of \tcode{iota<T>} is equal to \tcode{T([](typename
    T::value_type i) \{ return i; \})}.
  }
\end{itemdescr}
\end{wgBAdd}

\wgSubsubsubsection[paragraph]{Algorithms}{simd.alg}
\end{wgText}

\section{Discussion}

\subsection{Member Types}
The member types may not seem obvious.
Rationales:
\begin{typelist*}
  \item[value_type]
    In the spirit of the \valuetype member of STL containers, this type denotes the \emph{logical} type of the values in the vector.

  \item[register_value_type]
    On some targets it may be beneficial to implement \datapar instantiations of some \type T with a different type \type{register_value_type}, which has higher precision than \type T.
    This is mostly an implementation detail, but can be important to know in some situations, especially whenever \type{native_handle_type} is involved.

    \guidance{A better name might be \type{native_value_type}.}

  \item[native_handle_type]
    The type used for enabling access to an implementation-defined member object (via the \code{native_handle()} function).

  \item[reference]
    Used as the return type of the non-const scalar subscript operator.
    This may use implementation-defined means to solve possible type aliasing issues.

  \item[const_reference]
    Used as the return type of the const scalar subscript operator.
    From my experience with Vc, it is safest to actually not use a const lvalue reference here, but a temporary.

  \item[mask_type]
    The natural mask type for this \datapar instantiation.
    This type is used as return type of compares and write-mask on assignments.

  \item[size_type]
    Standard member type used for \code{size()} and \code{operator[]}.

  \item[target_type]
    The \type{Abi} template parameter to \datapar.

\end{typelist*}

\subsection{Conversions}
The \datapar conversion constructor only allows implicit conversion from \datapar template instantiations with the same \type{Abi} type and compatible \valuetype.
Discussion in SG1 showed clear preference for only allowing implicit conversion between integral types that only differ in signedness.
All other conversions could be implemented via an explicit conversion constructor.
The alternative (preferred) is to use \simdcast consistently for all other conversions.

\subsection{Broadcast Constructor}
The broadcast constructor is not declared as \code{explicit} to ease the use of scalar prvalues in expressions involving data-parallel operations.
The operations where such a conversion should not be implicit consequently need to use SFINAE / concepts to inhibit the conversion.

\subsection{Compound Assignment}
The semantics of compound assignment would allow less strict implicit conversion rules.
Consider \code{datapar<int>() *= double()}: the corresponding multiplication operator would not compile because the implicit conversion to \datapar[<float>] is non-portable.
Compound assignment, on the other hand, implies an implicit conversion back to the type of the expression on the left of the assignment operator.
Thus, it is possible to define compound operators that execute the operation correctly on the promoted type without sacrificing portability.
There are two arguments for not relaxing the rules for compound assignment, though:
\begin{enumerate}
  \item Consistency: The conversion of an expression with compound assignment to a binary operator suddenly would not compile anymore.
  \item The implicit conversion in the \code{int * double} case could be expensive and unintended.
    This is already a problem for builtin types where many developers multiply \float variables with \double prvalues.
\end{enumerate}

\subsection{Fundamental SIMD Type or Not?}
\subsubsection{The Issue}
There has been renewed discussion on the reflectors over the question whether \CC{} should define a fundamental, native SIMD type (let us call it \type{fundamental<T>}) and a generic data-parallel type on top which supports an arbitrary number of elements (call it \type{arbitrary<T, N>}).
The alternative to defining both types is to only define \type{arbitrary<T, N = default_size<T>>}, since it encompasses the \type{fundamental<T>} type.

With regard to this proposal this second approach would add a third template parameter to \datapar and \mask as shown in \lst{datapar N}.
\begin{lstlisting}[style=Vc,numbers=left,float,label=lst:datapar N,caption={
  Possible declaration of the class template parameters of a \datapar class with arbitrary width.
}]
template <class T, size_t N = datapar_size_v<T, datapar_abi::compatible>,
          class Abi = datapar_abi::compatible>
class datapar;
\end{lstlisting}

\subsubsection{Standpoints}
The controversy is about how the flexibility of a type with arbitrary \code N is presented to the users.
Is there a (clear) distinction between a “fundamental” type with target-dependent (i.e. fixed) \code N and a higher-level abstraction with arbitrary \code N which can potentially compile to inefficient machine code.
Or should the \CC{} standard only define \type{arbitrary} and set it to a default \code N value that corresponds to the target-dependent \code N.
Thus, the default \code N, of \type{arbitrary} would correspond to \type{fundamental}.

It is interesting to note that \type{arbitrary<T, 1>} is the class variant of \type T.
Consequently, if we say there is no need for a \type{fundamental} type then we could argue for the deprecation of the builtin arithmetic types, in favor of \type{arbitrary<T, 1>}. \wgNote{This is an academic discussion, of course.}

The author has implemented a library where a clear distinction is made between \type{fundamental<T, Abi>} and \type{arbitrary<T, N>}.
The documentation and all teaching material says that the user should program with \type{fundamental}.
The \type{arbitrary} type should be used in special circumstances, or wherever \type{fundamental} works with the \type{arbitrary} type in its interfaces (e.g. for gather \& scatter or the \code{ldexp} \& \code{frexp} functions).

\subsubsection{Issues}
The definition of two separate class templates can alleviate some source compatibility issues resulting from different \code N on different target systems.
Consider the simplest example of a multiplication of an \intt vector with a \float vector:
\smallskip\begin{lstlisting}[style=Vc]
arbitrary<float>() * arbitrary<int>();  // compiles for some targets, fails for others
fundamental<float>() * fundamental<int>();  // never compiles, requires explicit cast
\end{lstlisting}
The \datapar[<T>] operators specified in such a way that source compatibility is ensured.
For a type with user definable \code N, the binary operators should work slightly different with regard to implicit conversions.
Most importantly, \type{arbitrary<T, N>} solves the issue of portable code containing mixed integral and floating-point values.
A user would typically create aliases such as:
\smallskip\begin{lstlisting}[style=Vc]
using floatvec = datapar<float>;
using intvec = arbitrary<int, floatvec::size()>;
using doublevec = arbitrary<int, floatvec::size()>;
\end{lstlisting}
Objects of types \type{floatvec}, \type{intvec}, and \type{doublevec} will work together independent of the target system.

Obviously these type aliases are basically the same if the \code N parameter of \type{arbitrary} has a default value:
\smallskip\begin{lstlisting}[style=Vc]
using floatvec = arbitrary<float>;
using intvec = arbitrary<int, floatvec::size()>;
using doublevec = arbitrary<int, floatvec::size()>;
\end{lstlisting}
The ability to create these aliases is not the issue.
Seeing the need for using such a pattern is the issue.
Typically a developer will think no more of it if his code compiles on his machine.
If \code{arbitrary<float>() * arbitrary<int>()} just happens to compile (which is likely) then this is the code that will get checked in to the repository.
Note that with the existence of the \type{fundamental} class template, the \code N parameter of the \type{arbitrary} class would not have a default value and thus force the user to think a second longer about portability.

%Consider the \code{ldexp} function in \lst{ldexp}.
%\begin{lstlisting}[style=Vc,numbers=left,float,label=lst:ldexp,caption={
%  Declaration of \code{ldexp} for \type{fundamental} and \type{arbitrary} (only single-precision \float).
%}]
%template <class Abi>
%fundamental<float, Abi> ldexp(fundamental<float, Abi> x,
%                              arbitrary<int, fundamental<float, Abi>::size()> exp);
%template <int N>
%arbitrary<float, N> ldexp(arbitrary<float, N> x, arbitrary<int, N> exp);
%\end{lstlisting}
%%\lst{fundamental-ldexp}
%\begin{lstlisting}[style=Vc,numbers=left,float,label=lst:fundamental-ldexp,caption={
%  Calls to \code{ldexp} compile or fail to compile independent of the target.
%}]
%fundamental<float> a = ...;
%fundamental<int> exp = ...;
%a = ldexp(a, exp);  // compiles nowhere
%\end{lstlisting}
%%\lst{arbitrary-ldexp}
%\begin{lstlisting}[style=Vc,numbers=left,float,label=lst:arbitrary-ldexp,caption={
%  Calls to \code{ldexp} compile or fail to compile depending on the target.
%}]
%arbitrary<float> a = ...;
%arbitrary<int> exp = ...;
%a = ldexp(a, exp);  // compiles where a.size() == exp.size(), fails otherwise
%\end{lstlisting}
%
%
%The \type{fundamental} and \type{arbitrary} types have mostly the same interface.
%It is interesting to make \type{arbitrary} behave slightly differently, catering to the special use cases where \type{fundamental} is designed for safety.
%Consider implicit conversions (cf. \lst{implicit conversions}): \type{fundamental} must be very restrictive with regard to implicit conversions.
%\begin{lstlisting}[style=Vc,numbers=left,float,label=lst:implicit conversion,caption={
%  Implicit conversions behave differently for \type{fundamental} and \type{arbitrary}.
%}]
%auto x0 = fundamental<int>() * 1.f; // fails to compile since fundamental<int>::size()
%                                    // might be different to fundamental<float>::size()
%auto x1 = arbitrary<int>() * 1.f;   // x1 is of type arbitrary<float,
%                                    // arbitrary<int>::size()>
%\end{lstlisting}
%Implicit conversions are only allowed where \code{size()} is equal for the two types for each conceivable target systems.\footnote{
%  The safest approach is to disallow all implicit conversions.
%  Vc allows conversion between signed and unsigned integral types.
%}
%The \type{arbitrary} type, on the other hand, can allow implicit conversions to work much more alike to the fundamental arithmetic types.

\section{Feature Detection Macros}\label{sec:feature-detection}

For the purposes of SD-6, feature detection initially will be provided through the shipping vehicle (TS) itself.
For a standalone feature detection macro, I recommend \code{__cpp_lib_simd}.
If LEWG decides to rename the \simd class template, the feature detection macro needs to be renamed accordingly.

% vim: tw=0 sw=2 ft=tex

\end{document}
% vim: sw=2 sts=2 ai et tw=0
