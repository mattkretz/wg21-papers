\wgSubsection{Class template \type{simd_mask}}{simd.mask.class}
\wgSubsubsection{Class template \mask overview}{simd.mask.overview}
\lstinputlisting[]{mask.cpp}

\pnum The class template \mask is a \dataparalleltype with the element type \bool.
The \width of a given \mask specialization is a constant expression, determined by the template parameters.
Specifically, \code{simd_mask<T, Abi>::size() == simd<T, Abi>::size()}.

\simdTypeRequirements{simd_mask}

\pnum Default initialization performs no initialization of the elements; value-initialization initializes each element with \false.
\wgNote{Thus, default initialization leaves the elements in an indeterminate state.}

\begin{itemdecl}
static constexpr size_t size() noexcept;
\end{itemdecl}
\begin{itemdescr}
  \pnum\returns The width of \simd[<T, Abi>].
\end{itemdescr}

\pnum Implementations should enable explicit conversion from and to implementation-defined types.
This adds one or more of the following declarations to class \mask:
\begin{itemdecl}
explicit operator implementation-defined() const;
explicit simd_mask(const implementation-defined& init);
\end{itemdecl}

\pnum The member type \referencetype has the same interface as \simd[<T, Abi>::reference], except its \valuetype is \bool.
(\ref{sec:reference type})

\wgSubsubsection{\mask constructors}{simd.mask.ctor}
\begin{itemdecl}
explicit simd_mask(value_type x) noexcept;
\end{itemdecl}
\begin{itemdescr}
  \pnum\effects Constructs an object with each element initialized to \code x.
\end{itemdescr}

\begin{itemdecl}
template <class U> simd_mask(const simd_mask<U, simd_abi::fixed_size<size()>>& x) noexcept;
\end{itemdecl}
\begin{itemdescr}
  \pnum\effects Constructs an object of type \mask where the $i$-th element equals \code{x[i]} \foralli.

  \pnum\remarks This constructor shall not participate in overload resolution unless \type{abi_type} is \fixedsizescoped{}\code{<size()>}.
\end{itemdescr}

\begin{itemdecl}
template <class Flags> simd_mask(const value_type* mem, Flags);
\end{itemdecl}
\begin{itemdescr}
  \flagsRequires{\mask{}}
  %\pnum\requires
  \code{[mem, mem + size())} is a valid range.

  \pnum\effects Constructs an object where the $i$-th element is initialized to \code{mem[i]} \foralli.

  \pnum\remarks This constructor shall not participate in overload resolution unless \code{is_simd_flag_type_v<Flags>} is \true.
\end{itemdescr}

\wgSubsubsection{\mask copy functions}{simd.mask.copy}
\begin{itemdecl}
template <class Flags> void copy_from(const value_type* mem, Flags);
\end{itemdecl}
\begin{itemdescr}
  \flagsRequires{\mask{}}
  %\pnum\requires
  \code{[mem, mem + size())} is a valid range.

  \pnum\effects Replaces the elements of the \mask object such that the $i$-th element is replaced with \code{mem[i]} \foralli.

  \pnum\remarks The program is ill-formed unless \code{is_simd_flag_type_v<Flags>} is \true.
\end{itemdescr}

\begin{itemdecl}
template <class Flags> void copy_to(value_type* mem, Flags) const;
\end{itemdecl}
\begin{itemdescr}
  \flagsRequires{\mask{}}
  %\pnum\requires
  \code{[mem, mem + size())} is a valid range.

  \pnum\effects Copies all \mask elements as if \code{mem[i] = operator[](i)} \foralli.

  \pnum\remarks The program is ill-formed unless \code{is_simd_flag_type_v<Flags>} is \true.
\end{itemdescr}

\wgSubsubsection{\mask{} subscript operators}{simd.mask.subscr}
\begin{itemdecl}
reference operator[](size_t i);
\end{itemdecl}
\begin{itemdescr}
  \simdElementReference{\mask{}}
\end{itemdescr}

\begin{itemdecl}
value_type operator[](size_t i) const;
\end{itemdecl}
\begin{itemdescr}
  \pnum\requires \code{i < size()}.

  \pnum\returns The value of the $i$-th element.

  \pnum\throws Nothing.
\end{itemdescr}

\wgSubsubsection{\mask unary operators}{simd.mask.unary}
\begin{itemdecl}
simd_mask operator!() const noexcept;
\end{itemdecl}
\begin{itemdescr}
  \pnum\returns The result of element-wise application of \code{operator!}.
  %A \mask object with the $i$-th element set to the logical negation \foralli.
\end{itemdescr}

\wgSubsection{\type{simd_mask} non-member operations}{simd.mask.nonmembers}

\wgSubsubsection{\mask binary operators}{simd.mask.binary}
\begin{itemdecl}
friend simd_mask operator&&(const simd_mask&, const simd_mask&) noexcept;
friend simd_mask operator||(const simd_mask&, const simd_mask&) noexcept;
friend simd_mask operator& (const simd_mask&, const simd_mask&) noexcept;
friend simd_mask operator| (const simd_mask&, const simd_mask&) noexcept;
friend simd_mask operator^ (const simd_mask&, const simd_mask&) noexcept;
\end{itemdecl}
\begin{itemdescr}
  \pnum\returns A \mask object initialized with the results of the element-wise application of the indicated operator.
\end{itemdescr}

\wgSubsubsection{\mask compound assignment}{simd.mask.cassign}
\begin{itemdecl}
friend simd_mask& operator&=(simd_mask& lhs, const simd_mask& rhs) noexcept;
friend simd_mask& operator|=(simd_mask& lhs, const simd_mask& rhs) noexcept;
friend simd_mask& operator^=(simd_mask& lhs, const simd_mask& rhs) noexcept;
\end{itemdecl}
\begin{itemdescr}
  \pnum\effects These operators perform the indicated binary element-wise operation.

  \pnum\returns \code{lhs}.
\end{itemdescr}

\wgSubsubsection{\mask comparisons}{simd.mask.comparison}
\begin{itemdecl}
friend simd_mask operator==(const simd_mask&, const simd_mask&) noexcept;
friend simd_mask operator!=(const simd_mask&, const simd_mask&) noexcept;
\end{itemdecl}
\begin{itemdescr}
  \pnum\returns An object initialized with the results of the element-wise application of the indicated operator.
\end{itemdescr}

\wgSubsubsection{\mask reductions}{simd.mask.reductions}
\begin{itemdecl}
template <class T, class Abi> bool  all_of(const simd_mask<T, Abi>& k) noexcept;
\end{itemdecl}
\begin{itemdescr}
  \pnum\returns \true if all boolean elements in \code{k} are \true, \false otherwise.
\end{itemdescr}

\begin{itemdecl}
template <class T, class Abi> bool  any_of(const simd_mask<T, Abi>& k) noexcept;
\end{itemdecl}
\begin{itemdescr}
  \pnum\returns \true if at least one boolean element in \code{k} is \true, \false otherwise.
\end{itemdescr}

\begin{itemdecl}
template <class T, class Abi> bool none_of(const simd_mask<T, Abi>& k) noexcept;
\end{itemdecl}
\begin{itemdescr}
  \pnum\returns \true if none of the boolean elements in \code{k} is \true, \false otherwise.
\end{itemdescr}

\begin{itemdecl}
template <class T, class Abi> bool some_of(const simd_mask<T, Abi>& k) noexcept;
\end{itemdecl}
\begin{itemdescr}
  \pnum\returns \true if at least one of the boolean elements in \code{k} is \true and at least one of the boolean elements \code{k} is \false, \false otherwise.
\end{itemdescr}

\begin{itemdecl}
template <class T, class Abi> int popcount(const simd_mask<T, Abi>& k) noexcept;
\end{itemdecl}
\begin{itemdescr}
  \pnum\returns The number of boolean elements in \code{k} that are \true.
\end{itemdescr}

\begin{itemdecl}
template <class T, class Abi> int find_first_set(const simd_mask<T, Abi>& k);
\end{itemdecl}
\begin{itemdescr}
  \pnum\requires \code{any_of(k)} returns \true.

  \pnum\returns The lowest element index \code i where \code{k[i]} is \true.
\end{itemdescr}

\begin{itemdecl}
template <class T, class Abi> int find_last_set(const simd_mask<T, Abi>& k);
\end{itemdecl}
\begin{itemdescr}
  \pnum\requires \code{any_of(k)} returns \true.

  \pnum\returns The greatest element index \code i where \code{k[i]} is \true.
\end{itemdescr}

\begin{itemdecl}
bool  all_of(@\emph{see below}@) noexcept;
bool  any_of(@\emph{see below}@) noexcept;
bool none_of(@\emph{see below}@) noexcept;
bool some_of(@\emph{see below}@) noexcept;
int popcount(@\emph{see below}@) noexcept;
\end{itemdecl}
\begin{itemdescr}
  \pnum\returns \code{all_of} and \code{any_of} return their arguments; \code{none_of} returns the negation of its argument; \code{some_of} returns \false; \code{popcount} returns the integral representation of its argument; \code{find_first_set} and \code{find_last_set} return 0.

  \pnum\remarks The functions shall not participate in overload resolution unless the argument is of type \bool.
\end{itemdescr}

\begin{itemdecl}
int find_first_set(@\emph{see below}@) noexcept;
int find_last_set(@\emph{see below}@) noexcept;
\end{itemdecl}
\begin{itemdescr}
  \pnum\requires The value of the argument is \true.

  \pnum\returns \code 0.

  \pnum\remarks The functions shall not participate in overload resolution unless the argument is of type \bool.
\end{itemdescr}

\wgSubsubsection{Where functions}{simd.mask.where}
\begin{itemdecl}
template <class T, class Abi>
where_expression<simd_mask<T, Abi>, simd<T, Abi>> where(const typename simd<T, Abi>::mask_type& k,
                                                        simd<T, Abi>& v) noexcept;
template <class T, class Abi>
const_where_expression<simd_mask<T, Abi>, simd<T, Abi>> where(
    const typename simd<T, Abi>::mask_type& k, const simd<T, Abi>& v) noexcept;

template <class T, class Abi>
where_expression<simd_mask<T, Abi>, simd_mask<T, Abi>> where(const nodeduce_t<simd_mask<T, Abi>>& k,
                                                             simd_mask<T, Abi>& v) noexcept;
template <class T, class Abi>
const_where_expression<simd_mask<T, Abi>, simd_mask<T, Abi>> where(
    const nodeduce_t<simd_mask<T, Abi>>& k, const simd_mask<T, Abi>& v) noexcept;
\end{itemdecl}
\begin{itemdescr}
  \pnum\returns An object \ref{sec:simd.whereexpr} with \code{mask} and \code{data} initialized with \code{k} and \code{v} respectively.
\end{itemdescr}

\begin{itemdecl}
template <class T> where_expression<bool, T> where(@\emph{see below}@ k, T& v) noexcept;
template <class T>
const_where_expression<bool, T> where(@\emph{see below}@ k, const T& v) noexcept;
\end{itemdecl}
\begin{itemdescr}
  \pnum\remarks The functions shall not participate in overload resolution unless
  \begin{itemize}
    \item \type T is neither a \simd nor a \mask specialization, and
    \item the first argument is of type \bool.
  \end{itemize}
  \pnum\returns An object \ref{sec:simd.whereexpr} with \code{mask} and \code{data} initialized with \code{k} and \code{v} respectively.
\end{itemdescr}

% vim: tw=0 spell sw=2
