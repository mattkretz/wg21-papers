\section{Split \& Concat}\label{sec:split and concat}
The following presents an option for extending the above wording with two cast functions.
The \code{split} function allows to turn one \datapar or \mask object into two or more \datapar/\mask objects with smaller element counts.
The \code{concat} function allows to combine multiple \datapar or \mask objects into a single \datapar/\mask object consisting of all the input elements.

 Here is a simple example for \code{split} and \code{concat}:
 \smallskip
\begin{lstlisting}[style=Vc]
fixed_size_datapar<float, 12> x = ...;
auto [a, b] = split<8, 4>(x);
// e.g. on x86 you'd get: decltype(a) == datapar<float, avx>
//                   and: decltype(b) == datapar<float, sse>
x = concat(a + 1, b + 2);
\end{lstlisting}

The \type{abi_for_size_t} choice below could also be changed to use the \fixedsize ABI tag unconditionally.
Since a \fixedsize \datapar is implicitly convertible to a non-\fixedsize \datapar type with equal \code{size()}, this may be the more generic solution.
I have a slight preference for \type{abi_for_size_t}, since it more naturally supports the pattern of splitting a \fixedsize object into several native \datapar objects.
That pattern is not fully covered by the second \code{split} variant (e.g. consider the example above).

The same discussion of \type{abi_for_size_t} vs. \fixedsize is valid for the return type of \code{concat}.

Add to the synopsis in \ref{sec:datapar.syn}:
\begin{wgText}
  \begin{lstlisting}[style=Vc]
    template <size_t... Sizes, class T, class A>
    tuple<datapar<T, abi_for_size_t<Sizes>>...> split(const datapar<T, A>&);
    template <size_t... Sizes, class T, class A>
    tuple<mask<T, abi_for_size_t<Sizes>>...> split(const mask<T, A>&);

    template <class V, class T, class A>
    array<V, datapar_size_v<T, A> / V::size()> split(const datapar<T, A>&);
    template <class V, class T, class A>
    array<V, datapar_size_v<T, A> / V::size()> split(const mask<T, A>&);

    template <class T, class... As>
    datapar<T, abi_for_size_t<T, (datapar_size_v<T, As> + ...)>> concat(const datapar<T, As>&...);
    template <class T, class... As>
    mask<T, abi_for_size_t<T, (datapar_size_v<T, As> + ...)>> concat(const mask<T, As>&...);
  \end{lstlisting}
\end{wgText}

Append to \ref{sec:datapar.casts}:
\begin{wgText}

  \begin{itemdecl}
    template <size_t... Sizes, class T, class A>
    tuple<datapar<T, abi_for_size_t<Sizes>>...> split(const datapar<T, A>& x);
    template <size_t... Sizes, class T, class A>
    tuple<mask<T, abi_for_size_t<Sizes>>...> split(const mask<T, A>& x);
  \end{itemdecl}
  \begin{itemdescr}
    \pnum\remarks These functions shall not participate in overload resolution unless the sum of all values in the \code{Sizes} pack is equal to \code{datapar_size_v<T, A>}.
    \pnum\returns A \type{tuple} of \datapar/\mask objects with the $i$-th \datapar/\mask element of the $j$-th \type{tuple} element initialized to the value of the element in \code x with index $i$ + partial sum of the first $j$ values in the \code{Sizes} pack.
  \end{itemdescr}

  \begin{itemdecl}
    template <class V, class T, class A>
    array<V, datapar_size_v<T, A> / V::size()> split(const datapar<T, A>& x);
    template <class V, class T, class A>
    array<V, datapar_size_v<T, A> / V::size()> split(const mask<T, A>& x);
  \end{itemdecl}
  \begin{itemdescr}
    \pnum\remarks These functions shall not participate in overload resolution unless
    \begin{itemize}
      \item \code{is_datapar_v<V>} for the first signature / \code{is_mask_v<V>} for the second signature,
      \item and \code{datapar_size_v<T, A>} is an integral multiple of \code{V::size()}.
    \end{itemize}

    \pnum\returns An \type{array} of \datapar/\mask objects with the $i$-th \datapar/\mask element of the $j$-th \type{array} element initialized to the value of the element in \code x with index $i + j \cdot $\code{V::size()}.
  \end{itemdescr}

  \begin{itemdecl}
    template <class T, class... As>
    datapar<T, abi_for_size_t<T, (datapar_size_v<T, As> + ...)>> concat(const datapar<T, As>&... xs);
    template <class T, class... As>
    mask<T, abi_for_size_t<T, (datapar_size_v<T, As> + ...)>> concat(const mask<T, As>&... xs);
  \end{itemdecl}
  \begin{itemdescr}
    \pnum\returns A \datapar/\mask object initialized with the concatenated values in the \code{xs} pack of \datapar/\mask objects.
    The $i$-th \datapar/\mask element of the $j$-th parameter in the \code{xs} pack is copied to the return value's element with index $i$ + partial sum of the \code{size()} of the first $j$ parameters in the \code{xs} pack.
  \end{itemdescr}

\end{wgText}

% vim: tw=0 sw=2
