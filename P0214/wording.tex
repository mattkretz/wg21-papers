\section{Wording}

The following is a draft targeting inclusion into the Parallelism TS 2.
It defines a basic set of data-parallel types and operations.

\newcommand\clause{Clause\xspace}
\newcommand\width{width\xspace}

\begin{wgText}
  \setcounter{WGClause}{7}
  \wgSection{Data-Parallel Types}{simd.types}
  \wgSubsection{General}{simd.general}
  \pnum
  The data-parallel library consists of data-parallel types and operations on these types.
  A data-parallel type consists of elements of an underlying arithmetic type, called the \emph{element type}.
  The number of elements is a constant for each data-parallel type and called the \emph{\width} of that type.

  \pnum
  Throughout this \clause, the term \emph{data-parallel type} refers to all \emph{supported} \ref{sec:simd.overview} specializations of the \simd and \mask class templates.
  A \emph{data-parallel object} is an object of \emph{data-parallel type}.

  \pnum
  An \emph{element-wise operation} applies a specified operation to the elements of one or more data-parallel objects.
  Each such application is unsequenced with respect to the others.
  A \emph{unary element-wise operation} is an element-wise operation that applies a unary operation to each element of a data-parallel object.
  A \emph{binary element-wise operation} is an element-wise operation that applies a binary operation to corresponding elements of two data-parallel objects.

  \pnum Throughout this \clause, the set of \emph{vectorizable types} for a data-parallel type comprises all cv-unqualified arithmetic types other than \bool.

  \pnum \label{cl:intent-note}\wgNote{
      The intent is to support acceleration through data-parallel execution resources, such as SIMD registers and instructions or execution units driven by a common instruction decoder.
      If such execution resources are unavailable, the interfaces support a transparent fallback to sequential execution.
  }

  \wgSubsection{Header \code{<experimental/simd>} synopsis}{simd.syn}
\lstinputlisting[]{synopsis.cpp}

\pnum
The header \code{<experimental/simd>} defines class templates (\simd, \mask, \type{const_where_expression}, and \type{where_expression}), tag types, trait types, and function templates for concurrent manipulation of the values in \simd and \mask objects.

\wgSubsubsection{\simd ABI tags}{simd.abi}

\begin{itemdecl}
namespace simd_abi {
  struct scalar {};
  template <int N> struct fixed_size {};
  template <typename T> constexpr int max_fixed_size = implementation-defined;
  template <typename T> using compatible = implementation-defined;
  template <typename T> using native = implementation-defined;
}
\end{itemdecl}
\begin{itemdescr}
  \pnum
  An \emph{ABI tag} type indicates a choice of \targetArch dependent size and binary representation for \simd and \mask objects.
  The ABI tag, together with a given element type implies a number of elements.
  ABI tag types are used as the second template argument to \simd and \mask.
  \wgNote{
    The ABI tag is orthogonal to selecting the machine instruction set.
    The selected machine instruction set limits the usable ABI tag types, though (see \ref{simd.type requirements}).
    The ABI tags enable users to safely pass \simd and \mask objects between translation unit boundaries (e.g. function calls or I/O).
  }

  \pnum
  Use of the \type{scalar} tag type forces \simd and \mask to store a single component (i.e. \simd{}\type{<T, simd_abi::scalar>::size()} returns \code 1).
  \wgNote{\type{scalar} shall not be an alias for \type{fixed_size<1>}.}

  \pnum\label{simd.fixedsize.def}%
  Use of the \fixedsizeN tag type forces \simd and \mask to store and manipulate \code N components (i.e. \simd{}\type{<T, \fixedsizeN{}>::size()} returns \code N).
  An implementation must support at least any \code N $\in [1\ldots 32]$.
  Additionally, for every supported \simd[<T, A>] (see \ref{simd.type requirements}), where \type A is an implementation-defined ABI tag, \code N $=$ \simd[<T, A>::size()] must be supported.

  \wgNote{
    An implementation may choose to forego ABI compatibility between differently compiled translation units for \simd and \mask instantiations using the same \fixedsizeN tag.
    Otherwise, the efficiency of \simd[<T, Abi>] is likely to be better than for \simd[<T, fixed_size<simd_size_v<T, Abi>>>] (with \type{Abi} not a instance of \fixedsizescoped).
  }

  \pnum\label{simd.maxfixedsize.def}%
  The value of \code{max_fixed_size<T>} declares that an instance of \simd[<T, fixed_size<N>>] with \code{N <= max_fixed_size<T>} is supported by the implementation.
  \wgNote{
    It is unspecified whether an implementation supports \simd[<T, fixed_size<N>>] with \code{N > max_fixed_size<T>}.
    The value of \code{max_fixed_size<T>} may depend on compiler flags and may change between different compiler versions.
  }

  \pnum
  An implementation may define additional ABI tag types in the simd_abi namespace, to support other forms of data-parallel computation.

  \pnum
  \type{simd_abi::compatible<T>} is an alias for the ABI tag with the most efficient data parallel execution for the element type \type T that ensures ABI compatibility on the \targetArch.%
  \comment[Alternative]{
    \type{compatible<T>} is an implementation-defined alias for an ABI tag.
    \wgNote{
      The intent is to use ABI tag producing the most efficient data parallel execution for the element type \type T that ensures ABI compatibility between translation units on the \targetArch.
    }
  }

  \wgExample{
    Consider a \targetArch supporting the implementation-defined ABI tags \type{simd128} and \type{simd256}, where the \type{simd256} type requires an optional ISA extension on said \targetArch.
    Also, the \targetArch does not support \type{long double} with either ABI tag.
    The implementation therefore defines
    \begin{itemize}
      \item \type{compatible<T>} as an alias for \type{simd128} for all arithmetic \type T, except \type{long double},
      \item and \type{compatible<long double>} as an alias for \type{scalar}.
    \end{itemize}
  }

  \pnum
  \type{simd_abi::native<T>} is an alias for the ABI tag with the most efficient data parallel execution for the element type \type T that is supported on the \currentTarget.%
  \comment[Alternative]{
    \type{native<T>} is an implementation-defined alias for an ABI tag.
    \wgNote{
      The intent is to use an ABI tag producing the most efficient data parallel execution for the element type \type T that is supported on the \currentTarget.
    }
  }
  \wgNote{
    For \targetArchs without ISA extensions, the \type{native<T>} and \type{compatible<T>} aliases will likely be the same.
    For \targetArchs with ISA extensions, compiler flags may influence the \type{native<T>} alias while \type{compatible<T>} will be the same independent of such flags.
  }

  \wgExample{
      Consider a \currentTarget supporting the implementation-defined ABI tags \type{simd128} and \type{simd256}, where hardware support for \type{simd256} only exists for floating-point types.
    The implementation therefore defines \type{native<T>} as an alias for
    \begin{itemize}
      \item \type{simd256} if \type T is a floating-point type,
      \item and \type{simd128} otherwise.
    \end{itemize}
  }
\end{itemdescr}

\wgSubsubsection{\simd type traits}{simd.traits}
\begin{itemdecl}
template <class T> struct is_abi_tag;
\end{itemdecl}
\begin{itemdescr}
  \pnum The type \type{is_abi_tag<T>} is a \UnaryTypeTrait with a \BaseCharacteristic of \type{true_type} if \type T is the type of a standard or implementation-defined ABI tag, and \type{false_type} otherwise.
\end{itemdescr}

\begin{itemdecl}
template <class T> struct is_simd;
\end{itemdecl}
\begin{itemdescr}
  \pnum The type \type{is_simd<T>} is a \UnaryTypeTrait with a \BaseCharacteristic of \type{true_type} if \type T is an instance of the \simd class template, and \type{false_type} otherwise.
\end{itemdescr}

\begin{itemdecl}
template <class T> struct is_mask;
\end{itemdecl}
\begin{itemdescr}
  \pnum The type \type{is_mask<T>} is a \UnaryTypeTrait with a \BaseCharacteristic of \type{true_type} if \type T is an instance of the \mask class template, and \type{false_type} otherwise.
\end{itemdescr}

\begin{itemdecl}
template <class T, size_t N> struct abi_for_size { using type = implementation-defined; };
\end{itemdecl}
\begin{itemdescr}
  \pnum The member \type{type} shall be omitted if
  \begin{itemize}
    \item \type T is not a cv-unqualified floating-point or integral type except \bool.
    \item or if \fixedsizeN is not supported (see \ref{simd.fixedsize.def}).
  \end{itemize}

  \pnum Otherwise, the member typedef \type{type} shall name an ABI tag type that satisfies
  \begin{itemize}
    \item \code{simd_size_v<T, type> == N},
    \item \simd[<T, type>] is default constructible (see \ref{simd.type requirements}),
  \end{itemize}
  \code{simd_abi::scalar} takes precedence over \fixedsize\code{<1>}.
  The precedence of implementation-defined ABI tags over \fixedsizeN is implementation-defined.
  \wgNote{
    It is expected that implementation-defined ABI tags can produce better optimizations and thus take precedence over \fixedsizeN.
  }
\end{itemdescr}

\begin{itemdecl}
template <class T, class Abi = simd_abi::compatible<T>> struct simd_size;
\end{itemdecl}
\begin{itemdescr}
  \pnum\label{simd_size}%
  \type{simd_size<T, Abi>} shall have no member \code{value} if either
  \begin{itemize}
    \item \type T is not a cv-unqualified floating-point or integral type except \bool,
    \item or \code{is_abi_tag_v<Abi>} is \false.
  \end{itemize}
  \wgNote{The rules are different from \ref{simd.deleted}}

  \pnum
  Otherwise, the type \type{simd_size<T, Abi>} is a \BinaryTypeTrait with a \BaseCharacteristic of \type{integral_constant<size_t, N>} with \code{N} equal to the number of elements in a \simd[<T, Abi>] object.
  \wgNote{
    If \simd[<T, Abi>] is not supported for the \currentTarget, \type{simd_size<T, Abi>::value} produces the value \simd[<T, Abi>::size()] would return if it were supported.
  }

\end{itemdescr}

\begin{itemdecl}
template <class T, class U = typename T::value_type> struct memory_alignment;
\end{itemdecl}
\begin{itemdescr}
  \pnum
  \type{memory_alignment<T, U>} shall have no member \code{value} if either
  \begin{itemize}
    \item \type T is cv-qualified,
    \item or \type U is cv-qualified,
    \item or \code{!is_simd_v<T> \&\& !is_mask_v<T>},
    \item or \code{is_simd_v<T>} and \type U is not an arithmetic type or \type U is \bool,
    \item or \code{is_mask_v<T>} and \type U is not \bool.
  \end{itemize}

  \pnum
  Otherwise, the type \type{memory_alignment<T, U>} is a \BinaryTypeTrait with a \BaseCharacteristic of \type{integral_constant<size_t, N>} for some implementation-defined \code{N}.
  \wgNote{
    \code{value} identifies the alignment restrictions on pointers used for (converting) loads and stores for the given type \type T on arrays of type \type U (see \ref{sec:simd.load}, \ref{sec:simd.store}, \ref{sec:simd_mask.load}, \ref{sec:simd_mask.store}).
  }
\end{itemdescr}

\wgSubsubsection{Class templates \code{const_where_expression} and \code{where_expression}}{simd.whereexpr}
\lstinputlisting[]{whereexpression.cpp}

\pnum The class templates \code{const_where_expression} and \code{where_expression<M, T>} combine a predicate and a value object to implement an interface that restricts assignments and/or operations on the value object to the elements selected via the predicate.

\pnum The first template argument \type M must be cv-unqualified \bool or a cv-unqualified \mask instantiation.

\pnum The second template argument \type T must be a cv-unqualified or \const qualified type \type{U}.
If \type M is \bool, \type{U} must be an arithmetic type.
Otherwise, \type{U} must either be \type M or \type{M::simd_type}.

\begin{itemdecl}
const M& mask;  // exposition only
T& data;        // exposition only
\end{itemdecl}
\begin{itemdescr}
  \pnum\wgNote{
  The implementation initializes a \type{where_expression<M, T>} object with a predicate of type \type M and a reference to a value object of type \type T.
  The predicate object and a const qualified value object may be copied by the constructor implementation.}

  \pnum\wgNote{
  The following declarations refer to the predicate as data member \code{mask} and to the value reference as data member \code{data}.
  }
\end{itemdescr}

\begin{itemdecl}
remove_const_t<T> operator-() const &&;
\end{itemdecl}
\begin{itemdescr}
  \pnum\returns If \type M is \bool, \code{-data} if \code{mask} is \true, \code{data} otherwise.
  If \type M is not \bool, returns an object with the $i$-th element initialized to \code{-data[i]} if \code{mask[i]} is \true and \code{data[i]} otherwise \foralli[M::].
\end{itemdescr}

\begin{itemdecl}
template <class U, class Flags>
[[nodiscard]] remove_const_t<T> copy_from(const U *mem, Flags) const &&;
\end{itemdecl}
\begin{itemdescr}
  \pnum\remarks If \code{remove_const_t<T>} is \bool or \code{is_mask_v<remove_const_t<T>>}, the function shall not participate in overload resolution unless \type U is \bool.
  Otherwise, the function shall not participate in overload resolution unless \type U is an \realArithmeticType.

  \pnum\returns If \type M is \bool, return \code{mem[0]} if \code{mask} equals \true and return \code{data} otherwise.
  If \type M is not \bool, return an object with the $i$-th element initialized to the $i$-th element of \code{data} if \code{mask[i]} is \false and \code{static_cast<T::value_type>(mem[i])} if \code{mask[i]} is \true \foralli[M::].

  \pnum\requires If \type M is not \bool, the largest $i$ where \code{mask[i]} is \true is less than the number of values pointed to by \code{mem}.

  \flagsRemarks{\type T, \type U}
\end{itemdescr}

\begin{itemdecl}
template <class U, class Flags> void copy_to(U *mem, Flags) const &&;
\end{itemdecl}
\begin{itemdescr}
  \pnum\remarks If \code{remove_const_t<T>} is \bool or \code{is_mask_v<remove_const_t<T>>}, the function shall not participate in overload resolution unless \type U is \bool.
  Otherwise, the function shall not participate in overload resolution unless \type U is an \realArithmeticType.

  \pnum\effects If \type M is \bool, assigns \code{data} to \code{mem[0]} unless \code{mask} is \false.
  If \type M is not \bool, copies the elements \code{data[i]} where \code{mask[i]} is \true as if \code{mem[i] = static_cast<U>(data[i])} \foralli[M::].

  \pnum\requires If \type M is not \bool, the largest $i$ where \code{mask[i]} is \true is less than the number of values pointed to by \code{mem}.

  \flagsRemarks{\type remove_const_t<T>, \type U}
\end{itemdescr}

\begin{itemdecl}
template <class U> void operator=(U&& x);
template <class U> void operator+=(U&& x);
template <class U> void operator-=(U&& x);
template <class U> void operator*=(U&& x);
template <class U> void operator/=(U&& x);
template <class U> void operator%=(U&& x);
template <class U> void operator&=(U&& x);
template <class U> void operator|=(U&& x);
template <class U> void operator^=(U&& x);
template <class U> void operator<<=(U&& x);
template <class U> void operator>>=(U&& x);
\end{itemdecl}
\begin{itemdescr}
  \pnum\remarks Each of these operators shall not participate in overload resolution unless the indicated operator can be applied to objects of type \type T.
  \pnum\effects
  If \type M is \bool, applies the indicated operator on \code{data} and \code{forward<U>(x)} unless \code{mask} is \false.
  If \type M is not \bool, applies the indicated operator on \code{data} and \code{forward<U>(x)} without modifying the elements \code{data[i]} where \code{mask[i]} is \false \foralli[M::].
  \pnum\remarks It is unspecified whether the arithmetic/bitwise operation, which is implied by a compound assignment operator, is executed on all elements or only on the ones written back.
\end{itemdescr}

\begin{itemdecl}
void operator++();
void operator++(int);
void operator--();
void operator--(int);
\end{itemdecl}
\begin{itemdescr}
  \pnum\remarks Each of these operators shall not participate in overload resolution unless the indicated operator can be applied to objects of type \type T.
  \pnum\effects
  If \type M is \bool, applies the indicated operator on \code{data} unless \code{mask} is \false.
  If \type M is not \bool, applies the indicated operator on \code{data} without modifying the elements \code{data[i]} where \code{mask[i]} is \false \foralli[M::].
  \wgNote{It is unspecified whether the inc-/decrement operation is executed on all elements or only on the ones written back.}
\end{itemdescr}

\begin{itemdecl}
template <class U, class Flags> void copy_from(const U *mem, Flags);
\end{itemdecl}
\begin{itemdescr}
  \pnum\remarks If \type T is \bool or \code{is_mask_v<T>}, the function shall not participate in overload resolution unless \type U is \bool.

  \pnum\effects If \type M is \bool, assign \code{mem[0]} to \code{data} unless \code{mask} is \false.
  If \type M is not \bool, replace the elements of \code{data} where \code{mask[i]} is \true such that the $i$-th element is assigned with \code{static_cast<T::value_type>(mem[i])} \foralli[M::].

  \pnum\requires If \type M is not \bool, the largest $i$ where \code{mask[i]} is \true is less than the number of values pointed to by \code{mem}.

  \flagsRemarks{\type T, \type U}
\end{itemdescr}

% vim: tw=0

  \wgSubsection{Class template \type{simd}}{simd.class}
\wgSubsubsection{Class template \simd overview}{simd.overview}
\lstinputlisting[]{simd.cpp}

\pnum The class template \simd{} is a \dataparalleltype.
The \width of a given \simd instantiation is a constant expression, determined by the template parameters.

\newcommand\simdTypeRequirements[1]{
\pnum\label{#1.type requirements}\label{#1.deleted}%
Each instantiation of \type{#1} shall be a complete type with deleted default constructor, deleted destructor, deleted copy constructor, and deleted copy assignment unless all of the following hold:
\begin{itemize}
  \item The first template argument \type T is a cv-unqualified \realArithmeticType.
  \item The second template argument \type{Abi} is an ABI tag.
  \item The \type{Abi} type is a supported ABI tag.
    It is supported if
    \begin{itemize}
      \item \type{Abi} is \type{simd_abi::scalar}, or
      \item \type{Abi} is \fixedsizeN with \code N $\le 32$ or implementation-defined additional valid values for \code N (see \ref{simd.fixedsize.def}).
    \end{itemize}
    It is implementation-defined whether a given combination of \type T and an implementation-defined ABI tag is supported.
    \wgNote{The intent is for implementations to decide on the basis of the \currentTarget.}
\end{itemize}
}
\simdTypeRequirements{simd}

\wgExample{
  Consider an implementation that defines the implementation-defined ABI tags \type{simd_x} and \type{gpu_y}.
  When the compiler is invoked to translate to a machine that has support for the \type{simd_x} ABI tag for all arithmetic types other than \type{long double} and no support for the \type{gpu_y} ABI tag, then:
  \begin{itemize}
    \item \simd[<T, simd_abi::gpu_y>] is not supported for any \type T and results in a type with deleted constructor
    \item \simd[<long double, simd_abi::simd_x>] is not supported and results in a type with deleted constructor
    \item \simd[<double, simd_abi::simd_x>] is supported
    \item \simd[<long double, simd_abi::scalar>] is supported
  \end{itemize}
}

\pnum Default initialization performs no initialization of the elements; value-initialization initializes each element with \code{T()}.
\wgNote{Thus, default initialization leaves the elements in an indeterminate state.}

\pnum The member type \referencetype is an unspecified type acting as a reference to an element of a data-parallel type with the following properties:
\label{sec:reference type}
\begin{itemize}
  \item The type has a deleted default constructor, copy constructor, and copy assignment operator.

  \item Assignment, compound assignment, increment, and decrement operators shall not participate in overload resolution unless the \referencetype object is an rvalue and the corresponding operator for \valuetype is usable.

  \item Application of an assignment, compound assignment, increment, or decrement operator on a \referencetype object is applied to the referenced element.

  \item Objects of type \referencetype are implicitly convertible to \valuetype returning the value of the referenced element.

  \item If a binary operator is applied to an object of type \referencetype, the operator is only applied after converting the \referencetype object to \valuetype.

  \item Calls to \code{swap(\referencetype \&\&, \valuetype \&)} and \code{swap(\valuetype \&, \referencetype \&\&)} exchange the values referred to by the \referencetype object and the \valuetype reference.
  Calls to \code{swap(\referencetype \&\&, \referencetype \&\&)} exchange the values referred to by the \referencetype objects.
\end{itemize}

\begin{itemdecl}
static constexpr size_t size() noexcept;
\end{itemdecl}
\begin{itemdescr}
  \pnum\returns the number of elements stored in objects of the given \simd[<T, Abi>] type.
\end{itemdescr}

\pnum\begin{noteEnv} Implementations are encouraged to enable \code{static_cast}ing from and to implementation-defined types.
This would add one or more of the following declarations to class \simd:
\begin{itemdecl}
explicit operator implementation-defined() const;
explicit simd(const implementation-defined& init);
\end{itemdecl}
\end{noteEnv}

\wgSubsubsection{\simd constructors}{simd.ctor}
\begin{itemdecl}
template <class U> simd(U&&);
\end{itemdecl}
\begin{itemdescr}
  \pnum\effects Constructs an object with each element initialized to the value of the argument after conversion to \valuetype.

  \pnum\throws Any exception thrown while converting the argument to \valuetype.

  \pnum\remarks This constructor shall not participate in overload resolution unless:
  \comment[Q]{Mention forwarding on conversion to \valuetype?}%
  \comment[Q]{\type U is cv- and ref-qualified, is the wording below OK?}
  \begin{itemize}
    \item \type U is a \realArithmeticType and every possible value of type \type U can be represented with type \valuetype, or
    \item \type U is not an arithmetic type and is implicitly convertible to \valuetype, or
    \item \type U is \intt, or
    \item \type U is \uint and \valuetype is an unsigned integral type.
  \end{itemize}
\end{itemdescr}

\begin{itemdecl}
template <class U> simd(const simd<U, simd_abi::fixed_size<size()>>& x);
\end{itemdecl}
\begin{itemdescr}
  \pnum\effects Constructs an object where the $i$-th element equals \code{static_cast<T>(x[i])} \foralli.

  \pnum\remarks This constructor shall not participate in overload resolution unless
  \begin{itemize}
    \item \type{abi_type} is \fixedsizescoped{}\code{<size()>}, and
    \item every possible value of \type U can be represented with type \valuetype, and
    \item if both \type U and \valuetype are integral, the integer conversion rank [conv.rank] of \valuetype is greater than the integer conversion rank of \type U.
  \end{itemize}
\end{itemdescr}

\begin{itemdecl}
template <class G> simd(G&& gen);
\end{itemdecl}
\begin{itemdescr}
  \pnum\effects Constructs an object where the $i$-th element is initialized to \code{gen(integral_constant<size_t, i>())}.

  \pnum\remarks This constructor shall not participate in overload resolution unless \code{simd(gen(integral_constant<size_t, i>()))} is well-formed \foralli.
  %\pnum\remarks
  The calls to \code{gen} are unsequenced with respect to each other.
  \wgNote{This allows vectorized execution of the \code{gen} calls.}
\end{itemdescr}

\begin{itemdecl}
template <class U, class Flags> simd(const U *mem, Flags);
\end{itemdecl}
\begin{itemdescr}
  \flagsRequires{\simd, \type U}
  %\pnum\requires
  \code{size()} is less than or equal to the number of values pointed to by \code{mem}.

  \pnum\effects Constructs an object where the $i$-th element is initialized to \code{static_cast<T>(mem[i])} \foralli.

  \pnum\remarks This constructor shall not participate in overload resolution unless \type U is a \realArithmeticType and \code{is_simd_flag_type_v<Flags>} is \true.
\end{itemdescr}

\wgSubsubsection{\simd copy functions}{simd.copy}
\begin{itemdecl}
template <class U, class Flags> void copy_from(const U *mem, Flags);
\end{itemdecl}
\begin{itemdescr}
  \flagsRequires{\simd, \type U}
  %\pnum\requires
  \code{size()} is less than or equal to the number of values pointed to by \code{mem}.

  \pnum\effects Replaces the elements of the \simd object such that the $i$-th element is assigned with \code{static_cast<T>(mem[i])} \foralli.

  \pnum\remarks This function shall not participate in overload resolution unless \type U is a \realArithmeticType and \code{is_simd_flag_type_v<Flags>} is \true.
\end{itemdescr}

\begin{itemdecl}
template <class U, class Flags> void copy_to(U *mem, Flags);
\end{itemdecl}
\begin{itemdescr}
  \flagsRequires{\simd, \type U}
  %\pnum\requires
  \code{size()} is less than or equal to the number of values pointed to by \code{mem}.

  \pnum\effects Copies all \simd elements as if \code{mem[i] = static_cast<U>(operator[](i))} \foralli.

  \pnum\remarks This function shall not participate in overload resolution unless \type U is a \realArithmeticType and \code{is_simd_flag_type_v<Flags>} is \true.
\end{itemdescr}

\wgSubsubsection{\simd subscript operators}{simd.subscr}
\newcommand\simdElementReference[1]{
  \pnum\requires \code{i < size()}

  \pnum\returns A temporary object of type \referencetype (see \ref{sec:reference type}) that references the $i$-th element.
  %with the following effects:
  %\begin{itemize}
    %\item The assignment, compound assignment, increment, and decrement operators of \referencetype execute the indicated operation on the $i$-th element of the #1 object.
%
    %\item Conversion to \valuetype returns a copy of the $i$-th element.
  %\end{itemize}

  \pnum\throws Nothing.
}
\begin{itemdecl}
reference operator[](size_t i);
\end{itemdecl}
\begin{itemdescr}
  \simdElementReference{\simd{}}
\end{itemdescr}

\begin{itemdecl}
value_type operator[](size_t i) const;
\end{itemdecl}
\begin{itemdescr}
  \pnum\requires \code{i < size()}

  \pnum\returns A copy of the $i$-th element.

  \pnum\throws Nothing.
\end{itemdescr}

\wgSubsubsection{\simd unary operators}{simd.unary}
\begin{itemdecl}
simd& operator++();
\end{itemdecl}
\begin{itemdescr}
  \pnum\effects Is a unary element-wise operation that applies \code{operator++}.

  \pnum\returns \code{*this}

  \pnum\throws Nothing.
\end{itemdescr}

\begin{itemdecl}
simd operator++(int);
\end{itemdecl}
\begin{itemdescr}
  \pnum\effects Is a unary element-wise operation that applies \code{operator++}.

  \pnum\returns A copy of \code{*this} before incrementing.

  \pnum\throws Nothing.
\end{itemdescr}

\begin{itemdecl}
simd& operator--();
\end{itemdecl}
\begin{itemdescr}
  \pnum\effects Is a unary element-wise operation that applies \code{operator--}.

  \pnum\returns \code{*this}

  \pnum\throws Nothing.
\end{itemdescr}

\begin{itemdecl}
simd operator--(int);
\end{itemdecl}
\begin{itemdescr}
  \pnum\effects Is a unary element-wise operation that applies \code{operator--}.

  \pnum\returns A copy of \code{*this} before decrementing.

  \pnum\throws Nothing.
\end{itemdescr}

\begin{itemdecl}
mask_type operator!() const;
\end{itemdecl}
\begin{itemdescr}
  \pnum\returns A \mask object with the $i$-th element set to \code{!operator[](i)} \foralli.

  \pnum\throws Nothing.
\end{itemdescr}

\begin{itemdecl}
simd operator~() const;
\end{itemdecl}
\begin{itemdescr}
  \pnum\returns A \simd object where each bit is the inverse of the corresponding bit in \code{*this}.

  \pnum\throws Nothing.

  \pnum\remarks \simd{}\code{::operator\textasciitilde{}()} \specialsfinae unless \type T is an integral type.
\end{itemdescr}

\begin{itemdecl}
simd operator+() const;
\end{itemdecl}
\begin{itemdescr}
  \pnum\returns \code{*this}

  \pnum\throws Nothing.
\end{itemdescr}

\begin{itemdecl}
simd operator-() const;
\end{itemdecl}
\begin{itemdescr}
  \pnum\returns A \simd object where the $i$-th element is initialized to \code{-operator[](i)} \foralli.

  \pnum\throws Nothing.
\end{itemdescr}

\wgSubsection{\type{simd} non-member operations}{simd.nonmembers}

\wgSubsubsection{\simd binary operators}{simd.binary}
\begin{itemdecl}
friend simd operator+ (const simd& lhs, const simd& rhs);
friend simd operator- (const simd& lhs, const simd& rhs);
friend simd operator* (const simd& lhs, const simd& rhs);
friend simd operator/ (const simd& lhs, const simd& rhs);
friend simd operator% (const simd& lhs, const simd& rhs);
friend simd operator& (const simd& lhs, const simd& rhs);
friend simd operator| (const simd& lhs, const simd& rhs);
friend simd operator^ (const simd& lhs, const simd& rhs);
friend simd operator<<(const simd& lhs, const simd& rhs);
friend simd operator>>(const simd& lhs, const simd& rhs);
\end{itemdecl}
\begin{itemdescr}
  \pnum\returns A \simd object initialized with the results of the element-wise application of the indicated operator.

  \pnum\throws Nothing.

  \pnum\remarks Each of these operators \specialsfinae unless the indicated operator can be applied to objects of type \type{value_type}.
\end{itemdescr}

\begin{itemdecl}
friend simd operator<<(const simd& v, int n);
friend simd operator>>(const simd& v, int n);
\end{itemdecl}
\begin{itemdescr}
  \pnum\returns A \simd object where the $i$-th element is initialized to the result of applying the indicated operator to \code{v[i]} and \code n \foralli.

  \pnum\throws Nothing.

  \pnum\remarks Both operators \specialsfinae unless the indicated operator can be applied to objects of type \type{value_type}.
\end{itemdescr}

\wgSubsubsection{\simd compound assignment}{simd.cassign}
\begin{itemdecl}
friend simd& operator+= (simd& lhs, const simd& rhs);
friend simd& operator-= (simd& lhs, const simd& rhs);
friend simd& operator*= (simd& lhs, const simd& rhs);
friend simd& operator/= (simd& lhs, const simd& rhs);
friend simd& operator%= (simd& lhs, const simd& rhs);
friend simd& operator&= (simd& lhs, const simd& rhs);
friend simd& operator|= (simd& lhs, const simd& rhs);
friend simd& operator^= (simd& lhs, const simd& rhs);
friend simd& operator<<=(simd& lhs, const simd& rhs);
friend simd& operator>>=(simd& lhs, const simd& rhs);
\end{itemdecl}
\begin{itemdescr}
  \pnum\effects Each of these operators performs the indicated operator element-wise on each of the corresponding elements of the arguments.

  \pnum\returns \code{lhs}.

  \pnum\throws Nothing.

  \pnum\remarks Each of these operators \specialsfinae unless the indicated operator can be applied to objects of type \type{value_type}.
\end{itemdescr}

\begin{itemdecl}
friend simd& operator<<=(simd& v, int n);
friend simd& operator>>=(simd& v, int n);
\end{itemdecl}
\begin{itemdescr}
  \pnum\effects Performs the indicated shift by \code n operation on the $i$-th element of \code v \foralli.

  \pnum\returns \code v.

  \pnum\throws Nothing.

  \pnum\remarks Both operators \specialsfinae unless the indicated operator can be applied to objects of type \valuetype.
\end{itemdescr}

\wgSubsubsection{\simd compare operators}{simd.comparison}
\begin{itemdecl}
friend mask_type operator==(const simd&, const simd&);
friend mask_type operator!=(const simd&, const simd&);
friend mask_type operator>=(const simd&, const simd&);
friend mask_type operator<=(const simd&, const simd&);
friend mask_type operator> (const simd&, const simd&);
friend mask_type operator< (const simd&, const simd&);
\end{itemdecl}
\begin{itemdescr}
  \pnum\returns A \mask object initialized with the results of the element-wise application of the indicated operator.

  \pnum\throws Nothing.
\end{itemdescr}

\wgSubsubsection{\simd reductions}{simd.reductions}
\begin{itemdecl}
template <class T, class Abi, class BinaryOperation = std::plus<>>
T reduce(const simd<T, Abi>& x, BinaryOperation binary_op = BinaryOperation());
\end{itemdecl}
\begin{itemdescr}
  \pnum\requires \code{binary_op} shall be callable with two arguments of type \type T returning \type T, or callable with two arguments of type \simd[<T, A1>] returning \simd[<T, A1>] for every \type{A1} that is an ABI tag type.

  \pnum\returns \code{\textit{GENERALIZED_SUM}(binary_op, x.data[i], \ldots)} \foralli.

  \pnum\wgNote{This overload of \code{reduce} does not require an initial value because \code x is guaranteed to be non-empty.}
\end{itemdescr}

\begin{itemdecl}
template <class M, class V, class BinaryOperation>
typename V::value_type reduce(const const_where_expression<M, V>& x, typename V::value_type neutral_element,
                              BinaryOperation binary_op);
\end{itemdecl}
\begin{itemdescr}
  \pnum\requires \code{binary_op} shall be callable with two arguments of type \type T returning \type T, or callable with two arguments of type \simd[<T, A1>] returning \simd[<T, A1>] for every \type{A1} that is an ABI tag type.

  \pnum\returns
  If \code{none_of(x.mask)}, returns \code{neutral_element}.
  Otherwise, returns \code{\textit{GENERALIZED_SUM}(binary_op, x.data[i], \ldots)} \forallmaskedi{x.mask}.

  \pnum\wgNote{This overload of \code{reduce} requires a neutral value to enable a parallelized implementation:
  A temporary \simd object initialized with \code{neutral_element} is conditionally assigned from \code{x.data} using \code{x.mask}.
  Subsequently, the parallelized reduction is applied to the temporary object.}
\end{itemdescr}

\begin{itemdecl}
template <class M, class V>
typename V::value_type reduce(const const_where_expression<M, V>& x, plus<> binary_op = plus<>());
\end{itemdecl}
\begin{itemdescr}
  \pnum\returns
  If \code{none_of(x.mask)}, returns 0.
  Otherwise, returns \code{\textit{GENERALIZED_SUM}(binary_op, x.data[i], \ldots)} \forallmaskedi{x.mask}.

  \pnum\throws Nothing.
\end{itemdescr}

\begin{itemdecl}
template <class M, class V>
typename V::value_type reduce(const const_where_expression<M, V>& x, multiplies<> binary_op);
\end{itemdecl}
\begin{itemdescr}
  \pnum\returns
  If \code{none_of(x.mask)}, returns 1.
  Otherwise, returns \code{\textit{GENERALIZED_SUM}(binary_op, x.data[i], \ldots)} \forallmaskedi{x.mask}.

  \pnum\throws Nothing.
\end{itemdescr}

\begin{itemdecl}
template <class M, class V>
typename V::value_type reduce(const const_where_expression<M, V>& x, bit_and<> binary_op);
\end{itemdecl}
\begin{itemdescr}
  \pnum\requires \code{is_integral_v<V::value_type>} is \true.

  \pnum\returns
  If \code{none_of(x.mask)}, returns ~(V::value_type()).
  Otherwise, returns \code{\textit{GENERALIZED_SUM}(binary_op, x.data[i], \ldots)} \forallmaskedi{x.mask}.

  \pnum\throws Nothing.
\end{itemdescr}

\begin{itemdecl}
template <class M, class V>
typename V::value_type reduce(const const_where_expression<M, V>& x, bit_or<> binary_op);
template <class M, class V>
typename V::value_type reduce(const const_where_expression<M, V>& x, bit_xor<> binary_op);
\end{itemdecl}
\begin{itemdescr}
  \pnum\requires \code{is_integral_v<V::value_type>} is \true.

  \pnum\returns
  If \code{none_of(x.mask)}, returns 0.
  Otherwise, returns \code{\textit{GENERALIZED_SUM}(binary_op, x.data[i], \ldots)} \forallmaskedi{x.mask}.

  \pnum\throws Nothing.
\end{itemdescr}

\begin{itemdecl}
template <class T, class Abi> T hmin(const simd<T, Abi>& x);
\end{itemdecl}
\begin{itemdescr}
  \pnum\returns The value of an element \code{x[j]} for which \code{x[j] <= x[i]} \foralli.

  \pnum\throws Nothing.
\end{itemdescr}

\begin{itemdecl}
template <class M, class V> T hmin(const const_where_expression<M, V>& x);
\end{itemdecl}
\begin{itemdescr}
  \pnum\returns If \code{none_of(x.mask)}, the return value is \code{numeric_limits<V::value_type>::max()}.
  Otherwise, returns the value of an element \code{x.data[j]} for which \code{x.mask[j] == true} and \code{x.data[j] <= x.data[i]} \forallmaskedi{x.mask}.

  \pnum\throws Nothing.
\end{itemdescr}

\begin{itemdecl}
template <class T, class Abi> T hmax(const simd<T, Abi>& x);
\end{itemdecl}
\begin{itemdescr}
  \pnum\returns The value of an element \code{x[j]} for which \code{x[j] >= x[i]} \foralli.

  \pnum\throws Nothing.
\end{itemdescr}

\begin{itemdecl}
template <class M, class V> T hmax(const const_where_expression<M, V>& x);
\end{itemdecl}
\begin{itemdescr}
  \pnum\returns If \code{none_of(x.mask)}, the return value is \code{numeric_limits<V::value_type>::min()}.
  Otherwise, returns the value of an element \code{x.data[j]} for which \code{x.mask[j] == true} and \code{x.data[j] >= x.data[i]} \forallmaskedi{x.mask}.

  \pnum\throws Nothing.
\end{itemdescr}


\wgSubsubsection{\simd casts}{simd.casts}
\begin{itemdecl}
  template <class T, class U, class Abi> @\emph{see below}@ simd_cast(const simd<U, Abi>& x);
\end{itemdecl}
\begin{itemdescr}
  \pnum Let \type{To} identify \type{T::\valuetype} if \code{is_simd_v<T>} is \true, or \type T otherwise.

  \pnum\returns A \simd object with the $i$-th element initialized to \code{static_cast<To>(x[i])}.

  \pnum\throws Nothing.

  \pnum\remarks The function shall not participate in overload resolution unless
  \begin{itemize}
    \item every possible value of type \type U can be represented with type \type{To}, and
    \item either \code{is_simd_v<T>} is \false, or \code{T::size() == simd<U, Abi>::size()} is \true.
  \end{itemize}
  %\pnum\remarks
  If \code{is_simd_v<T>} is \true, the return type is \type T.
  Otherwise, if \type U is \type T, the return type is \simd[<T, Abi>].
  Otherwise, the return type is \simd[<T, \fixedsizescoped{}<\simd{}<U, Abi>::size()>>].

\end{itemdescr}

\begin{itemdecl}
template <class T, class U, class Abi> @\emph{see below}@ static_simd_cast(const simd<U, Abi>& x);
\end{itemdecl}
\begin{itemdescr}
  \pnum Let \type{To} identify \type{T::\valuetype} if \code{is_simd_v<T>} or \type T otherwise.

  \pnum\returns A \simd object with the $i$-th element initialized to \code{static_cast<To>(x[i])}.

  \pnum\throws Nothing.

  \pnum\remarks The function shall not participate in overload resolution unless either \code{is_simd_v<T>} is \false or \code{T::size() == simd<U, Abi>::size()} is \true.
  %\pnum\remarks
  If \code{is_simd_v<T>} is \true, the return type is \type T.
  Otherwise, if either \type U is \type T or \type U and \type T are integral types that only differ in signedness, the return type is \simd[<T, Abi>].
  Otherwise, the return type is \simd[<T, \fixedsizescoped{}<\simd{}<U, Abi>::size()>>].
\end{itemdescr}

\begin{itemdecl}
template <class T, class Abi>
fixed_size_simd<T, simd_size_v<T, Abi>> to_fixed_size(const simd<T, Abi>& x) noexcept;
template <class T, class Abi>
fixed_size_simd_mask<T, simd_size_v<T, Abi>> to_fixed_size(const simd_mask<T, Abi>& x) noexcept;
\end{itemdecl}
\begin{itemdescr}
  \pnum\returns An object of the return type with the $i$-th element initialized to \code{x[i]}.
\end{itemdescr}

\begin{itemdecl}
template <class T, size_t N> native_simd<T> to_native(const fixed_size_simd<T, N>& x) noexcept;
template <class T, size_t N> native_simd_mask<T> to_native(const fixed_size_simd_mask<T, N>> &x) noexcept;
\end{itemdecl}
\begin{itemdescr}
  \pnum\returns An object of the return type with the $i$-th element initialized to \code{x[i]}.

  \pnum\remarks These functions shall not participate in overload resolution unless \code{simd_size_v<T, simd_abi::native<T>> == N} is \true.
\end{itemdescr}

\begin{itemdecl}
template <class T, size_t N> simd<T> to_compatible(const fixed_size_simd<T, N>& x) noexcept;
template <class T, size_t N> simd_mask<T> to_compatible(const fixed_size_simd_mask<T, N>& x) noexcept;
\end{itemdecl}
\begin{itemdescr}
  \pnum\returns An object of the return type with the $i$-th element initialized to \code{x[i]}.

  \pnum\remarks These functions shall not participate in overload resolution unless \code{simd_size_v<T, simd_abi::compatible<T>> == N} is \true.
\end{itemdescr}

\begin{itemdecl}
template <size_t... Sizes, class T, class Abi>
tuple<simd<T, abi_for_size_t<Sizes>>...> split(const simd<T, Abi>& x);
template <size_t... Sizes, class T, class Abi>
tuple<simd_mask<T, abi_for_size_t<Sizes>>...> split(const simd_mask<T, Abi>& x);
\end{itemdecl}
\begin{itemdescr}
  \pnum\returns A \type{tuple} of data-parallel objects with the $i$-th \simd/\mask element of the $j$-th \type{tuple} element initialized to the value of the element in \code x with index $i$ + partial sum of the first $j$ values in the \code{Sizes} pack.

  \pnum\remarks These functions shall not participate in overload resolution unless the sum of all values in the \code{Sizes} pack is equal to \code{simd_size_v<T, Abi>}.
\end{itemdescr}

\begin{itemdecl}
template <class V, class Abi>
array<V, simd_size_v<typename V::value_type, Abi> / V::size()> split(
    const simd<typename V::value_type, Abi>&);
template <class V, class Abi>
array<V, simd_size_v<typename V::value_type, Abi> / V::size()> split(
    const simd_mask<typename V::value_type, Abi>&);
\end{itemdecl}
\begin{itemdescr}
  \pnum\returns An \type{array} of data-parallel objects with the $i$-th \simd/\mask element of the $j$-th \type{array} element initialized to the value of the element in \code x with index $i + j \cdot $\code{V::size()}.

  \pnum\remarks These functions shall not participate in overload resolution unless
  \begin{itemize}
    \item \code{is_simd_v<V>} is \true for the first signature / \code{is_mask_v<V>} is \true for the second signature, and
    \item \code{simd_size_v<typename V::value_type, Abi>} is an integral multiple of \code{V::size()}.
  \end{itemize}
\end{itemdescr}

\begin{itemdecl}
template <class T, class... Abis>
simd<T, abi_for_size_t<T, (simd_size_v<T, Abis> + ...)>> concat(const simd<T, Abis>&... xs);
template <class T, class... Abis>
simd_mask<T, abi_for_size_t<T, (simd_size_v<T, Abis> + ...)>> concat(const simd_mask<T, Abis>&... xs);
\end{itemdecl}
\begin{itemdescr}
  \pnum\returns A data-parallel object initialized with the concatenated values in the \code{xs} pack of data-parallel objects:
  The $i$-th \simd/\mask element of the $j$-th parameter in the \code{xs} pack is copied to the return value's element with index $i$ + partial sum of the \code{size()} of the first $j$ parameters in the \code{xs} pack.
\end{itemdescr}

\wgSubsubsection{\simd algorithms}{simd.alg}
\begin{itemdecl}
template <class T, class Abi> simd<T, Abi> min(const simd<T, Abi>& a, const simd<T, Abi>& b) noexcept;
\end{itemdecl}
\begin{itemdescr}
  \pnum\returns The result of binary element-wise application of \code{std::min(a[i], b[i])} \foralli.
  %An object with the $i$-th element initialized to the value of \code{std::min(a[i], b[i])} \foralli.
\end{itemdescr}

\begin{itemdecl}
template <class T, class Abi> simd<T, Abi> max(const simd<T, Abi>&, const simd<T, Abi>&) noexcept;
\end{itemdecl}
\begin{itemdescr}
  \pnum\returns The result of binary element-wise application of \code{std::max(a[i], b[i])} \foralli.
  %\pnum\returns An object with the $i$-th element initialized to the value of \code{std::max(a[i], b[i])} \foralli.
\end{itemdescr}

\begin{itemdecl}
template <class T, class Abi>
std::pair<simd<T, Abi>, simd<T, Abi>> minmax(const simd<T, Abi>&, const simd<T, Abi>&) noexcept;
\end{itemdecl}
\begin{itemdescr}
  \pnum\returns A pair initialized with
  \begin{itemize}
    \item the result of binary element-wise application of \code{std::min(a[i], b[i])} \foralli in the \code{first} member, and
    \item the result of binary element-wise application of \code{std::max(a[i], b[i])} \foralli in the \code{second} member.
  \end{itemize}
\end{itemdescr}

\begin{itemdecl}
template <class T, class Abi>
simd<T, Abi> clamp(const simd<T, Abi>& v, const simd<T, Abi>& lo, const simd<T, Abi>& hi);
\end{itemdecl}
\begin{itemdescr}
  \pnum\requires No element in \code{lo} shall be greater than the corresponding element in \code{hi}.

  \pnum\returns The result of element-wise\comment{do we really need a definition of \emph{ternary element-wise}?} application of \code{std::clamp(a[i], lo[i], hi[i])} \foralli.
\end{itemdescr}

\wgSubsubsection{\simd math library}{simd.math}
\lstinputlisting[]{math.cpp}

\pnum Each listed function concurrently applies the indicated mathematical function element-wise.
The results per element are not required to be bitwise equal to the application of the function which is overloaded for the element type.
\comment{Neither the C nor the \CC{} standard say anything about expected error/precision.
It seems returning 0 from all functions is a conforming implementation --- just bad QoI.}
\wgNote{
  If a precondition of the indicated mathematical function is violated, the behavior is undefined.
}

\pnum If \code{abs} is called with an argument of type \simd[<X, Abi>] for which \code{is_unsigned<X>::value} is \true, the program is ill-formed.

% vim: tw=0 spell sw=2

  \wgSubsection{Class template \type{mask}}{mask}
\wgSubsubsection{Class template \mask overview}{mask.overview}
\lstinputlisting[]{mask.cpp}

\pnum The class template \mask[<T, Abi>] is a one-dimensional smart array of booleans.
The number of elements in the array is determined at compile time, equal to the number of elements in \datapar[<T, Abi>].

\pnum The first template argument \type T must be an integral or floating-point fundamental type.
The type \bool is not allowed.

\pnum The second template argument \type{Abi} must be a tag type from the \code{datapar_abi} namespace.

\begin{itemdecl}
static constexpr size_type size();
\end{itemdecl}
\begin{itemdescr}
  \pnum\returns the number of boolean elements stored in objects of the given \mask[<T, Abi>] type.
\end{itemdescr}

\pnum\realnote Implementations are encouraged to enable \code{static_cast}ing from/to (an) implementation-defined SIMD mask type(s).
This would add one or more of the following declarations to class \mask:
\begin{itemdecl}
explicit operator implementation_defined() const;
explicit datapar(const implementation_defined &init);
\end{itemdecl}

\wgSubsubsection{\mask constructors}{mask.ctor}
\begin{itemdecl}
mask() = default;
\end{itemdecl}
\begin{itemdescr}
  \pnum\effects Constructs an object with all elements initialized to \code{bool()}.
  \wgNote{This zero-initializes the object.}
\end{itemdescr}

\begin{itemdecl}
explicit mask(value_type);
\end{itemdecl}
\begin{itemdescr}
  \pnum\effects Constructs an object with each element initialized to the value of the argument.
\end{itemdescr}

\begin{itemdecl}
template <class U, class Abi2> mask(mask<U, Abi2> x);
\end{itemdecl}
\begin{itemdescr}
  \pnum\remarks This constructor shall not participate in overload resolution unless
    \datapar[<U, Abi2>] is implicitly convertible to \datapar[<T, Abi>].
  \pnum\effects Constructs an object of type \mask where the $i$-th element equals \code{x[i]} \foralli.
\end{itemdescr}

\begin{itemdecl}
template <class Flags> mask(const value_type *mem, Flags);
\end{itemdecl}
\begin{itemdescr}
  \pnum\effects Constructs an object where the $i$-th element is initialized to \code{mem[i]} \foralli.
  \pnum\remarks If \code{size()} returns a value greater than the number of values pointed to by the first argument, the behavior is undefined.
  \pnum\remarks If the \type{Flags} template parameter is of type \type{flags::vector_aligned_tag} and the pointer value is not a multiple of \code{memory_alignment<\mask{}>}, the behavior is undefined.
\end{itemdescr}

\begin{itemdecl}
template <class Flags> mask(const value_type *mem, mask k, Flags);
\end{itemdecl}
\begin{itemdescr}
  \pnum\effects Constructs an object where the $i$-th element is initialized to \code{k[i] ? mem[i] : false} \foralli.
  \pnum\remarks If the largest $i$ where \code{k[i]} is \true is greater than the number of values pointed to by the first argument, the behavior is undefined.
  \pnum\remarks If the \type{Flags} template parameter is of type \type{flags::vector_aligned_tag} and the pointer value is not a multiple of \code{memory_alignment<\mask{}>}, the behavior is undefined.
\end{itemdescr}

\wgSubsubsection{\mask load function}{mask.load}
\begin{itemdecl}
template <class Flags> void copy_from(const value_type *mem, Flags);
\end{itemdecl}
\begin{itemdescr}
  \pnum\effects Replaces the elements of the \mask object such that the $i$-th element is assigned with \code{mem[i]} \foralli.
  \pnum\remarks If \code{size()} returns a value greater than the number of values pointed to by the first argument, the behavior is undefined.
  \pnum\remarks If the \type{Flags} template parameter is of type \type{flags::vector_aligned_tag} and the pointer value is not a multiple of \code{memory_alignment<\mask{}>}, the behavior is undefined.
\end{itemdescr}

\begin{itemdecl}
template <class Flags> void copy_from(const value_type *mem, mask k, Flags);
\end{itemdecl}
\begin{itemdescr}
  \pnum\effects Replaces all elements of the \mask object where $k[i]$ is \true such that the $i$-th element is assigned with \code{mem[i]} \foralli.
  \pnum\remarks If the largest $i$ where \code{k[i]} is \true is greater than the number of values pointed to by the first argument, the behavior is undefined.
  \pnum\remarks If the \type{Flags} template parameter is of type \type{flags::vector_aligned_tag} and the pointer value is not a multiple of \code{memory_alignment<\mask{}>}, the behavior is undefined.
\end{itemdescr}

\wgSubsubsection{\mask store functions}{mask.store}
\begin{itemdecl}
template <class Flags> void copy_to(value_type *mem, Flags);
\end{itemdecl}
\begin{itemdescr}
  \pnum\effects Copies all \mask elements as if \code{mem[i] = operator[](i)} \foralli.
  \pnum\remarks If \code{size()} returns a value greater than the number of values pointed to by \code{mem}, the behavior is undefined.
  \pnum\remarks If the \type{Flags} template parameter is of type \type{flags::vector_aligned_tag} and the pointer value is not a multiple of \code{memory_alignment<\mask{}>}, the behavior is undefined.
\end{itemdescr}

\begin{itemdecl}
template <class Flags> void copy_to(value_type *mem, mask k, Flags);
\end{itemdecl}
\begin{itemdescr}
  \pnum\effects Copies each \mask element $i$ where \code{k[i]} is \true as if \code{mem[i] = operator[](i)} \foralli.
  \pnum\remarks If the largest $i$ where \code{k[i]} is \true is greater than the number of values pointed to by \code{mem}, the behavior is undefined.
  \wgNote{
    Masked stores only write to the bytes in memory selected by the \code k argument.
    This prohibits implementations that load, blend, and store the complete vector.
  }
  \pnum\remarks If the \type{Flags} template parameter is of type \type{flags::vector_aligned_tag} and the pointer value is not a multiple of \code{memory_alignment<\mask{}>}, the behavior is undefined.
\end{itemdescr}

\wgSubsubsection{\mask{} subscript operators}{mask.subscr}
\begin{itemdecl}
reference operator[](size_type i);
\end{itemdecl}
\begin{itemdescr}
  \dataparElementReference
\end{itemdescr}

\begin{itemdecl}
value_type operator[](size_type) const;
\end{itemdecl}
\begin{itemdescr}
  \pnum\returns A copy of the $i$-th element.
\end{itemdescr}

\wgSubsubsection{\mask unary operators}{mask.unary}
\begin{itemdecl}
mask operator!() const;
\end{itemdecl}
\begin{itemdescr}
  \pnum\returns A mask object with the $i$-th element set to the logical negation \foralli.
\end{itemdescr}

\wgSubsection{\type{mask} non-member operations}{mask.nonmembers}

\wgSubsubsection{\mask binary operators}{mask.binary}
\begin{itemdecl}
friend mask operator&&(const mask &, const mask &);
friend mask operator||(const mask &, const mask &);
friend mask operator& (const mask &, const mask &);
friend mask operator| (const mask &, const mask &);
friend mask operator^ (const mask &, const mask &);
\end{itemdecl}
\begin{itemdescr}
  \pnum\returns A \mask object initialized with the results of the component-wise application of the indicated operator.
\end{itemdescr}

\wgSubsubsection{\mask compares}{mask.comparison}
\begin{itemdecl}
friend bool operator==(const mask &, const mask &);
\end{itemdecl}
\begin{itemdescr}
  \pnum\returns \true if all boolean elements of the first argument equal the corresponding element of the second argument.
  It returns \false otherwise.
\end{itemdescr}

\begin{itemdecl}
friend bool operator!=(const mask &, const mask &);
\end{itemdecl}
\begin{itemdescr}
  \pnum\returns \code{!operator==(a, b)}.
\end{itemdescr}

\wgSubsubsection{\mask reductions}{mask.reductions}
\begin{itemdecl}
template <class T, class Abi> bool  all_of(mask<T, Abi>);
constexpr bool  all_of(bool);
\end{itemdecl}
\begin{itemdescr}
  \pnum\returns \true if all boolean elements in the function argument equal \true, \false otherwise.
\end{itemdescr}

\begin{itemdecl}
template <class T, class Abi> bool  any_of(mask<T, Abi>);
constexpr bool  any_of(bool);
\end{itemdecl}
\begin{itemdescr}
  \pnum\returns \true if at least one boolean element in the function argument equals \true, \false otherwise.
\end{itemdescr}

\begin{itemdecl}
template <class T, class Abi> bool none_of(mask<T, Abi>);
constexpr bool none_of(bool);
\end{itemdecl}
\begin{itemdescr}
  \pnum\returns \true if none of the boolean element in the function argument equals \true, \false otherwise.
\end{itemdescr}

\begin{itemdecl}
template <class T, class Abi> bool some_of(mask<T, Abi>);
constexpr bool some_of(bool);
\end{itemdecl}
\begin{itemdescr}
  \pnum\returns \true if at least one of the boolean elements in the function argument equals \true and at least one of the boolean elements in the function argument equals \false, \false otherwise.
  \pnum\realnote \code{some_of(bool)} unconditionally returns \false.
\end{itemdescr}

\begin{itemdecl}
template <class T, class Abi> int popcount(mask<T, Abi>);
constexpr int popcount(bool);
\end{itemdecl}
\begin{itemdescr}
  \pnum\returns The number of boolean elements that are \true.
\end{itemdescr}

\begin{itemdecl}
template <class T, class Abi> int find_first_set(mask<T, Abi> m);
\end{itemdecl}
\begin{itemdescr}
  \pnum\returns The lowest element index \code i where \code{m[i] == true}.
  \pnum\remarks If \code{none_of(m) == true} the behavior is undefined.
\end{itemdescr}

\begin{itemdecl}
template <class T, class Abi> int find_last_set(mask<T, Abi> m);
\end{itemdecl}
\begin{itemdescr}
  \pnum\returns The highest element index \code i where \code{m[i] == true}.
  \pnum\remarks If \code{none_of(m) == true} the behavior is undefined.
\end{itemdescr}

\begin{itemdecl}
constexpr int find_first_set(bool);
constexpr int find_last_set(bool);
\end{itemdecl}
\begin{itemdescr}
  \pnum\returns 0 if the argument is \true.
\end{itemdescr}

\wgSubsubsection{Masked assigment}{mask.where}
\begin{itemdecl}
template <class T, class A>
where_expression<const mask<T, A> &, datapar<T, A>> where(
    const typename datapar<T, A>::mask_type &m, datapar<T, A> &v);
template <class T, class A>
const where_expression<const mask<T, A> &, const datapar<T, A>> where(
    const typename datapar<T, A>::mask_type &m, const datapar<T, A> &v);
\end{itemdecl}
\begin{itemdescr}
  \pnum\returns A temporary object with the following properties:
  \begin{itemize}
    \item The object is not \textit{CopyConstructible}.
    \item Assignment and compound assignment operators only participate in overload resolution if the corresponding operator for \datapar[<T, A>] is usable.
    \item \effects Assignment and compound assignment implement the same semantics as the corresponding operator for \datapar[<T, A>] with the exception that elements of \code v stay unmodified if the corresponding boolean element in \code m is \false.
    \item The assignment and compound assignment operators return \void.
  \end{itemize}
  \pnum\realnote The \const overload is only useful for functions that should be applied to a masked \datapar input.
\end{itemdescr}

\begin{itemdecl}
template <class T> where_expression<bool, T> where(bool k, T &d);
\end{itemdecl}
\begin{itemdescr}
  \pnum\remarks The function only participates in overload resolution if \type T is not a \datapar instantiation.
  \pnum\returns A temporary object with the following properties:
  \begin{itemize}
    \item The object is not \textit{CopyConstructible}.
    \item Assignment and compound assignment operators only participate in overload resolution if the corresponding operator for \type T is usable.
    \item \effects If the first argument is \false, the assignment operators do nothing.
      If the first argument is \true, the assignment operators forward to the corresponding builtin assignment operator.
    \item The assignment and compound assignment operators return \void.
  \end{itemize}
\end{itemdescr}

% vim: tw=0 spell sw=2

\end{wgText}

% vim: tw=0
