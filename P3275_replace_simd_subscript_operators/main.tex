\newcommand\wgTitle{Replace simd operator[] with getter and setter functions --- or not}
\newcommand\wgName{Matthias Kretz <m.kretz@gsi.de>}
\newcommand\wgDocumentNumber{D3275R0}
\newcommand\wgGroup{LEWG}
\newcommand\wgTarget{\CC{}26}
%\newcommand\wgAcknowledgements{Thanks to Daniel Towner, Ruslan Arutyunyan, Jonathan Müller, Jeff Garland, and Nicolas Morales for discussions and/or pull requests on this/previous paper(s).}

\usepackage{mymacros}
\usepackage{wg21}
\setcounter{tocdepth}{2} % show sections and subsections in TOC
\hypersetup{bookmarksdepth=5}
\usepackage{changelog}
\usepackage{underscore}
\usepackage{multirow}

\addbibresource{extra.bib}

\newcommand\simd[1][]{\type{ba\-sic\_simd#1}\xspace}
\newcommand\simdT{\type{ba\-sic\_simd\MayBreak<\MayBreak{}T>}\xspace}
\newcommand\valuetype{\type{val\-ue\_type}\xspace}
\newcommand\referencetype{\type{ref\-er\-ence}\xspace}
\newcommand\mask[1][]{\type{ba\-sic\_simd\_mask#1}\xspace}
\newcommand\maskT{\type{ba\-sic\_simd\_mask\MayBreak<\MayBreak{}T>}\xspace}
\newcommand\wglink[1]{\href{https://wg21.link/#1}{#1}}

\newcommand\nativeabi{\UNSP{native-abi}}
\newcommand\deducet{\UNSP{deduce-t}}
\newcommand\simdsizev{\UNSP{simd-size-v}}
\newcommand\simdsizetype{\UNSP{simd-size-type}}
\newcommand\simdselect{\UNSP{simd-select-impl}}
\newcommand\maskelementsize{\UNSP{mask-element-size}}
\newcommand\integerfrom{\UNSP{integer-from}}
\newcommand\constexprwrapperlike{\UNSP{constexpr-wrapper-like}}

\renewcommand{\lst}[1]{Listing~\ref{#1}}
\renewcommand{\sect}[1]{Section~\ref{#1}}
\renewcommand{\ttref}[1]{Tony~Table~\ref{#1}}
\renewcommand{\tabref}[1]{Table~\ref{#1}}

\begin{document}
\selectlanguage{american}
\begin{wgTitlepage}
  There was discussion in LEWG in Kona '23 whether \code{operator[]} is the
  right interface for reading and writing individual elements of a \simd or
  \mask.
  This paper discusses the underlying issue and explores alternatives.
\end{wgTitlepage}

\pagestyle{scrheadings}

\section{Changelog}
\begin{revision}
\item Target \CC{}26, addressing SG1 and LEWG.
\item Call for a merge of the (improved \& adjusted) TS specification to the IS.
\item Discuss changes to the ABI tags as consequence of TS experience; calls for polls to change the status quo.
\item Add template parameter \code{T} to \code{simd_abi::fixed_size}.
\item Remove \code{simd_abi::compatible}.
\item Add (but ask for removal) \code{simd_abi::abi_stable}.
\item Mention TS implementation in GCC releases.
\item Add more references to related papers.
\item Adjust the clause number for [numbers] to latest draft.
\item Add open question: what is the correct clause for [simd]?
\item Add open question: integration with ranges.
\item Add \code{simd_mask} generator constructor.
\item Consistently add simd and simd_mask to headings.
\item Remove experimental and parallelism_v2 namespaces.
\item Present the wording twice: with and without diff against N4808 (Parallelism TS 2).
\item Default load/store flags to \code{element_aligned}.
\item Generalize casts: conditionally \code{explicit} converting constructors.
\item Remove named cast functions.
\end{revision}

\begin{revision}
\item Add floating-point conversion rank to condition of \code{explicit} for converting constructors.
\item Call out different or equal semantics of the new ABI tags.
\item Update introductory paragraph of \sect{sec:changes}; R1 incorrectly kept the text from R0.
\item Define simd::size as a \code{constexpr} static data-member of type \code{integral_constant<size_t, N>}. This simplifies passing the size via function arguments and still be useable as a constant expression in the function body.
\item Document addition of \code{constexpr} to the API.
\item Add \code{constexpr} to the wording.
\item Removed ABI tag for passing \code{simd} over ABI boundaries.
\item Apply cast interface changes to the wording.
\item Explain the plan: what this paper wants to merge vs. subsequent papers for additional features. With an aim of minimal removal/changes of wording after this paper.
\item Document rationale and design intent for \code{where} replacement.
\end{revision}

\begin{revision}
\item Propose alternative to \code{hmin} and \code{hmax}.
\item Discuss \code{simd_mask} reductions wrt. consistency with \code{<bit>}. Propose better names to avoid ambiguity.
\item Remove \code{some_of}.
\item Add unary \code{\~{}} to \code{simd_mask}.
\item Discuss and ask for confirmation of masked ``overloads'' names and argument order.
\item Resolve inconsistencies wrt. \code{int} and \code{size_t}: Change \code{fixed_size} and \code{resize_simd} NTTPs from \code{int} to \code{size_t} (for consistency).
\item Discuss conversions on loads and stores.
\item Point to \cite{P2509R0} as related paper.
\item Generalize load and store from pointer to \code{contiguous_iterator}. (\sect{sec:contiguousItLoadStore})
\item Moved ``\code{element_reference} is overspecified'' to ``Open questions''.
\end{revision}

\begin{revision}
\item Remove wording diff.
\item Add std::simd to the paper title.
\item Update ranges integration discussion and mention formatting support via
  ranges (\sect{sec:formatting}).
\item Fix: pass iterators by value not const-ref.
\item Add lvalue-ref qualifier to subscript operators (\sect{sec:lvalue-subscript}).
\item Constrain \code{simd} operators: require operator to be well-formed on objects of \code{value_type} (\ref{sec:simd.unary}, \ref{sec:simd.binary}).
\item Rename mask reductions as decided in Issaquah.
\item Remove R3 ABI discussion and add follow-up question.
\item Add open question on first template parameter of \code{simd_mask} (\sect{sec:basicsimdmask}).
\item Overload loads and stores with mask argument (\ref{sec:simd.ctor}, \ref{sec:simd.copy}, \ref{sec:simd.mask.ctor}, \ref{sec:simd.mask.copy}).
\item Respecify \simd reductions to use a \mask argument instead of \code{const_where_expression} (\ref{sec:simd.reductions}).
\item Add \mask operators returning a \simd (\ref{sec:simd.mask.unary}, \ref{sec:simd.mask.conv})
\item Add conditional operator overloads as hidden friends to \simd and \mask
  (\ref{sec:simd.cond}, \ref{sec:simd.mask.cond}).
\item Discuss \std\code{hash} for \simd (\sect{sec:hash}).
\item Constrain some functions (e.g., min, max, clamp) to be \code{totally_ordered} (\ref{sec:simd.reductions}, \ref{sec:simd.alg}).
\item Asking for reconsideration of conversion rules.
\item Rename load/store flags (\sect{sec:renameandextendflags}).
\item Extend load/store flags with a new flag for conversions on load/store. (\sect{sec:renameandextendflags}).
\item Update \code{hmin}/\code{hmax} discussion with more extensive naming discussion (\sect{sec:hminhmax}).
\item Discuss freestanding \simd (\sect{sec:freestanding}).
\item Discuss \code{split} and \code{concat} (\sect{sec:splitandconcat}).
\item Apply the new library specification style from P0788R3.
\end{revision}

\begin{revision}
\item Added \code{simd_select} discussion.
\end{revision}

\begin{revision}
\item Updated the wording for changes discussed in and requested by LEWG in Varna.
\item Rename to \code{simd_cat} and \code{simd_split}.
\item Drop \code{simd_cat(array)} overload.
\item Replace \code{simd_split} by \code{simd_split} as proposed in P1928R4.
\item Use \code{indirectly_writable} instead of \code{output_iterator}.
\item Replace most \code{size_t} and \code{int} uses by \code{\textit{simd-size-type}} signed integer type.
\item Remove everything in \code{simd_abi} and the namespace itself.
\item Reword section on ABI tags using exposition-only ABI tag aliases.
\item Guarantee generator ctor calls callable exactly once per index.
\item Remove \code{int}/\code{unsigned int} exception from conversion rules of broadcast ctor.
\item Rename \code{loadstore_flags} to \code{simd_flags}.
\item Make \code{simd_flags::operator|} \code{consteval}.
\item Remove \code{simd_flags::operator\&} and \code{simd_flags::operator\^}.
\item Increase minimum SIMD width to 64.
\item Rename \code{hmin}/\code{hmax} to \code{reduce_min} and \code{reduce_max}.
\item Refactor \code{simd_mask<T, Abi>} to \code{basic_simd_mask<Bytes, Abi>} and replace all occurrences accordingly.
\item Rename \code{simd<T, Abi>} to \code{basic_simd<Bytes, Abi>} and replace all occurrences accordingly.
\item Remove \code{long double} from the set of vectorizable types.
\item Remove \code{is_abi_tag}, \code{is_simd}, and \code{is_simd_mask} traits.
\item Make \code{simd_size} exposition-only.
\end{revision}

\begin{revision}
\item Remove mask reduction precondition but ask LEWG for reversal of that decision (\sect{sec:removemaskreductionprecondition}).
\item Fix return type of \mask unary operators.
\item Fix \code{bool} overload of \simdselect (\sect{sec:simdselectwording}).
\item Remove unnecessary implementation freedom in \code{simd_split} (\sect{sec:bettersimdsplitwording}).
\item Use \code{class} instead of \code{typename} in template heads.
\item Implement LEWG decision to SFINAE on \emph{values} of
  constexpr-wrapper-like arguments to the broadcast ctor (\ref{sec:simd.ctor}).
\item Add relational operators to \mask as directed by LEWG (\ref{sec:simd.mask.comparison}).
\item Update section on \code{size_t} vs. \code{int} usage (\sect{sec:simdsizetype}).
\item Remove all open design questions, leaving LWG / wording questions.
\item Add LWG question on implementation note (\sect{sec:implnote}).
\item Add constraint for \code{BinaryOperation} to \code{reduce} overloads (\ref{sec:simd.reductions}).
%  \todo Add \code{numeric_limits} / numeric traits specializations since behavior of e.g. \code{simd<float>} and \code{float} may differ for reasonable implementations.
\end{revision}

\begin{revision}
\item Include \code{std::optional} return value from \code{reduce_min_index} and \code{reduce_max_index} in the exploration.
\item Fix \LaTeX{} markup errors.
\item Remove repetitive mention of “exposition-only” before \deducet.
\item Replace “TU” with “translation unit”.
\item Reorder first paragraphs in the wording, especially reducing the note on compiling down to SIMD instructions.
\item Replace cv-unqualified arithmetic types with a more precise list of types.
\item Move the place where “supported” is defined.
\end{revision}

\begin{revision}
\item Improve wording that includes the \CC{}23 extended floating-point types in the set of vectorizable types (\ref{wording.vectorizable.types}).
\item Improve wording that defines “selected indices” and “selected elements” (\ref{wording.selected.indices}).
\item Remove superfluous introduction paragraph.
\item Improve wording introducing the intent of ABI tags (\ref{wording.ABI.tag})
\item Consistently use \code{size} as a callable in the wording.
\item Add missing \code{type_identity_t} for \code{reduce} (\ref{sec:simd.syn}, \ref{sec:simd.reductions}).
\item Spell out “iff” (\ref{wording.deducet}).
\item Fixed template argument to \nativeabi\ in the default template argument of \code{basic_simd_mask} (\ref{sec:simd.syn}).
\item Fixed default template argument to \code{simd_mask} to be consistent with \code{simd} (\ref{sec:simd.syn}).
\item Add instructions to add \code{<simd>} to the table of headers in [headers].
\item Add instructions to add a new subclause to the table in [numerics.general].
\item Add instructions to add \code{<simd>} [diff.23.library].
\item Add \simdsizev to the wording and replace \code{simd_size_v} to actually implement “Make \code{simd_size} exposition-only.”
\item Restored precondition (and removed \code{noexcept}) on
  \code{reduce_min_index} and \code{reduce_max_index} as directed by LEWG.
\end{revision}

\begin{revision}
\item Strike through wording removed by P3275 (non-const \code{operator[]}).
\item Remove “exposition only” from detailed prose, it's already marked as such in the synopsis.
\item Reorder defintion of \emph{vectorizable type} above its first use.
\item Commas, de-duplication, word order, \code{s/may/can/} in a note.
\item Use text font for “[)” when defining a range of integers.
\item Several small changes from LWG review on 2024-06-26.
\item Reword \code{rebind_simd} and \code{resize_simd}.
\item Remove mention of implementation-defined load/store flags.
\item Remove paragraph about default initialization of \simd.
\item Reword all constructor \emph{Effects} from “Constructs an object \ldots”
  to “Initializes \ldots”.
\item Instead of writing “satisfies X” in \emph{Constraints} and “models X” in
  \emph{Preconditions}, say only “models X” in \emph{Constraints}.
\item Replace \code{is_trivial_v} with “is trivially copyable”.
\item First shot at improving generator function constraints.
\item Reword constraints on unary and binary operators.
\item Add missing/inconsistent \code{explicit} on load constructors.
\item Fix preconditions of subscript operators.
\item Reword effects of compound assignment operators.
\item Add that \code{BinaryOperation} may not modify input \simd.
\item Fix definition of GENERALIZED_SUMs.
\end{revision}

\begin{revision}
\item Say “\textit{op}” instead of “the indicated operator”
\item Fix constraints on shift operators with \simdsizetype{} on the right operand.
\item Remove wording removed by P3275 (non-const \code{operator[]}).
\item Make intrinsics conversion recommended practice.
\item Make \code{simd_flags} template arguments exposition-only.
\item Make \code{simd_alignment} \emph{not} implementation-defined.
\item Reword “supported” to “enabled or disabled”.
\item Apply improved wording from \ref{sec:simd.overview} to \ref{sec:simd.mask.overview}.
\item Add comments for LWG to address to broadcast ctor (\ref{sec:simd.ctor}).
\item Respecify generator ctor to not reuse broadcast constraint (\ref{sec:simd.ctor}).
\item Use \code{to_address} on contiguous iterators (\ref{sec:simd.ctor} and \ref{sec:simd.copy}).
  This is more explicit about allowing memcpy on the complete range rather than
  having to iterate the range per element.
\end{revision}

\begin{revision}
\item Fix default size of \code{simd} and \code{simd_mask} aliases
  (\ref{sec:simd.syn}, necessary for
  \std\code{destructible<\MayBreak{}\std{}simd<\MayBreak\std{}string>>} to be well-formed).
\item Extend value-preserving to encompass conversions from all arithmetic
  types. Use this new freedom in \ref{sec:simd.ctor} to fully constrain the
  generator constructor and to plug a specification hole in the broadcast
  constructor.
\item Fix broadcast constructor wording by constraining \constexprwrapperlike
  arguments to arithmetic types.
  %\todo Reorder \code{simd} and \code{simd_mask} specification in the wording (mask first).
\end{revision}

\section{Straw Polls}


\section{Motivation}
\subsection{proxy references in \CC{} make me :'(}
\begin{wgText}[{[simd.general]}]
  A data-parallel type consists of one or more elements of an underlying
  vectorizable type, called the element type.
  [\ldots]
  The elements in a data-parallel type are indexed from 0 to $\mathrm{width} - 1$.
\end{wgText}

Since that's a given, we sure want to be able to access individual elements.
\simd and \mask implement \code{operator[]} for element access:

\medskip\begin{lstlisting}[style=Vc]
std::simd<int> x = 0;
x[0] = 1;      // OK
int y = x[0];  // OK
x[0] = y;      // OK
auto z = x[0]; // OK, but:   #1
z = 2;         // ill-formed #2
\end{lstlisting}

The \simd and \mask types hold values of their \valuetype, but they typically
don't hold \emph{objects} of \valuetype.
Consequently, both \code{operator[]} overloads cannot return an
lvalue-reference.
The \code{const} overload therefore returns a prvalue and the non-\code{const}
overload returns a proxy reference.
The proxy reference implements assignment and compound-assignment operators for
assigning through to the selected element of the \simd/\mask.
Thus, line \code{\#1} above deduces the type of \code{z} to be that proxy
reference type.
The declaration of \code{z} does not look like a reference at all.
Therefore, assignment in line \code{\#2} is ill-formed, in order to avoid
the surprising behavior of modifying \code{x}.

In any case, the fact that a proxy reference is used instead of an
lvalue-reference, makes subscripting into a \code{simd} error-prone.
Whenever the subscript expression is used in a function argument with deduced
type, bad things are likely to happen.\footnote{Just recall
\code{vector<bool>::reference}.}
If we had a language feature to decay the proxy reference type to \valuetype on
deduction, then a lot of problems could be avoided.
But we don't have that feature and there's no reasonable chance to get it for
\CC{}26.

\subsection{Why, though? lvalue ref is fine, no?}

With GCC today you can write
\medskip\begin{lstlisting}[style=Vc]
using simd [[gnu::vector_size(16)]] = int;

simd x = {};
int& ref = x[0];
x += 1;
ref = 2;
\end{lstlisting}
If that's possible, then \code{std::simd<int>}'s subscript operator can simply
return an lvalue reference, no?
While that's true for this example, it's not true in general.
For one, Clang is fairly strict about not handing out lvalue references.
I.e. the above code does not compile.
But more importantly, for some targets or implementations an intrinsic type
might need to be used, which doesn't allow forming lvalue references to its
elements either.
Also \simd does not prohibit an implementation to use a different element type
internally for its SIMD registers.
E.g. an efficient implementation of 8-bit and 16-bit integers on the (outdated)
Intel Knights architecture required the use of 32-bit integer SIMD registers
and instructions.
It is also conceivable that implementations will implement
\code{simd<std::float16_t>} using 32-bit \code{float} SIMD registers for
targets without hardware support.

The situation for \mask is much clearer.
There are three typical storage formats for masks in current hardware:
\begin{enumerate}
  \item Full SIMD registers where either all bits are 0 or all bits are 1 for
    the complete number of value bits.
  \item Bitmasks use one bit per mask element.
    This is analogue to \code{vector<bool>} and \code{bitset} not being able to
    return lvalue references to \code{bool}.
  \item Mask registers that use one bit per value type byte.
    This is similar to the above, where we would need to return a reference to
    a single bit (just at a different position).
\end{enumerate}
Therefore, even if Clang would implement GCC's behavior with regard to forming
lvalue references to vector elements, that doesn't help for \mask.

\subsection{simd<UDT>}\label{sec:simd<array>}
If we want to extend \simd's vectorizable types to user-defined types, we need
to consider a consistency issue:
\code{simd<T>} applies every operator and operation element-wise (unless the
name clearly hints at a horizontal operation).

While I don't think e.g. \code{simd<array<short, 4>, 2>} is a useful thing,
it's also not completely crazy.
However, its only interesting semantics in a \simd is data-parallel
subscripting (apply \code{operator[]} element-wise):
\medskip\begin{lstlisting}[style=Vc]
using xyzw = std::array<short, 4>;
std::simd<xyzw, 2> a = {};
std::simd<short, 2> w = a[3]; // yes, simd not array: element-wise subscript!
a[3] = w + std::integral_constant<short, 1>();
\end{lstlisting}

(This is neither going to be great for performance, nor is it clear whether we
should implement such a “data-parallel subscript”, which requires a proxy
reference again.)

This example is not meant to motivate an element-wise \code{operator[]} for \code{simd}.
It's meant to show that the current \code{simd::operator[]} is inconsistent
with the “apply operators element-wise” rule.
Applying \code{operator[]} element-wise on a \code{simd<int>} is obviously
ill-formed since \code{int} doesn't have a subscript operator.
Consequently, maybe the current P1928 \simd and \mask shouldn't overload
\code{operator[]}?



\section{Replacement exploration}

If we only want to get rid of the proxy reference but are not concerned about
the consistency argument in \sect{sec:simd<array>}, then we could consider a
read-only subscript operator.
We still have two choices:\\
\begingroup
  \smaller[1]
\begin{tabular}{p{.45\textwidth}|p{.45\textwidth}}
  read-only forever & keep design space open \\
  \hline
  \begin{lstlisting}
class simd {
  value_type operator[](@\simdsizetype@ i) const;

};
  \end{lstlisting}

  \medskip
  \begin{lstlisting}
std::simd<int> v;
int x = v[0];
  \end{lstlisting}
  &
  \begin{lstlisting}
class simd {
  value_type operator[](@\simdsizetype@ i) const;
  void operator[](@\simdsizetype@) = delete;
};
  \end{lstlisting}

  \medskip
  \begin{lstlisting}
std::simd<int> v;
int x = std::as_const(v)[0];
  \end{lstlisting}
\end{tabular}
\endgroup

\subsection{Making the case for: a read-only subscript is sufficient}

A common use case for the subscript operator arises through the generator
constructors of \simd and \mask.
With P1928 you would write a permutation like this:
\medskip\begin{lstlisting}[style=Vc]
simd<int> v;
simd<int> reversed([&](int i) { return v[v.size - i - 1]; };
\end{lstlisting}
The generator constructor often reads from another \simd and since it needs to
compute a scalar, it typically only reads one element at a time.
And it never updates a value via subscripting; the update happens by
constructing a whole new \simd (with a good chance of the compiler producing
vector instructions).
Making such code any harder to write is not necessarily helping users.
The above example is intuitively understandable (well, the subscripting part,
the generator constructor maybe less so).

Therefore, it seems like the simplest and still fairly usable “fix” is to
remove the non-const subscript overload.

There is some curious existing practice in GCC supporting this approach:
\medskip\begin{lstlisting}[style=Vc]
using simd [[gnu::vector_size(16)]] = int;

constexpr simd f(simd x) {
  x[0] = 1;
  return x;
}

constexpr simd test0 = f(simd{}); // ill-formed: x[0] = 1 is not a constant expression

constexpr simd g(simd x) {
  x = simd{1, x[1], x[2], x[3]};
  return x;
}

constexpr simd test1 = f(simd{}); // OK
\end{lstlisting}
I.e. assignment through vector subscripts cannot be used in constant expressions.
Instead a complete new vector must be constructed.
If the non-const subscript operator is removed from \simd and \mask, then GCC's
restriction for constant expressions becomes \std\simd's behavior.


\subsection{Is simd as a read-only range a sufficient replacement?}

If \simd and \mask have a \code{begin()} and \code{end()} iterator, making
them read-only random-access ranges, then accessing an element is
equivalent to accessing a scalar from an \code{ini\-til\-iz\-er_list}:
\medskip\begin{lstlisting}[style=Vc]
std::simd<int, 4> v;
auto v0 = v.begin()[0];
auto v3 = v.begin()[3];
\end{lstlisting}

In a very similar approach, making \simd convertible to \code{array} allows
subscripting through the array:
\medskip\begin{lstlisting}[style=Vc]
std::simd<int, 4> v;
std::array a = v;
a[1] += 1;
v = a;
\end{lstlisting}

\subsection{Allowing for writable subscript after \CC{}26}

If we want to keep the design space open while still overloading
\simd\code{::operator[]}, then subscripting would become even more awkward to
use.
Consequently, \simd and \mask should rather have no subscript operator at all
for \CC{}26.
In the following exploratory examples, I will use the function names \code{get}
and \code{set} as placeholder names.
I also added a line to every example, considering the same syntax for the
degenerate case of an \code{int} instead of a \code{simd<int>}.

\begin{enumerate}
  \item P1928 status quo:\label{explP1928}
\medskip\begin{lstlisting}[style=Vc]
std::simd<int> v;
v[0] += 1;

int x;
x[0] += 1; // nope
\end{lstlisting}

\item \code{set(index, value)} member function:\label{memberset}
\medskip\begin{lstlisting}[style=Vc]
std::simd<int> v;
v.set(0, 1 + v.get(0));

int x;
x.set(0, 1 + x.get(0)); // nope
\end{lstlisting}

\item \code{set(object, index, value)} non-member function:\label{nonmemberset}
\medskip\begin{lstlisting}[style=Vc]
std::simd<int> v;
set(v, 0, 1 + get(v, 0));

int x;
set(x, 0, 1 + get(x, 0)); // not impossible
\end{lstlisting}

\item explicit proxy reference without assignment and conversion operators:\label{proxyset}
\medskip\begin{lstlisting}[style=Vc]
std::simd<int> v;
element_reference(v, 0).set(1 + element_reference(v, 0).get());

int x;
element_reference(x, 0).set(1 + element_reference(x, 0).get()); // not impossible
\end{lstlisting}

\item explicit proxy reference with operators:\label{proxyoperators}
\medskip\begin{lstlisting}[style=Vc]
std::simd<int> v;
element_reference(v, 0) += 1;

int x;
element_reference(x, 0) += 1; // not impossible
\end{lstlisting}

\item make degenerate size 1 \simdT convertible to/from \code{T} (and \mask to/from \code{bool})
  \medskip\begin{lstlisting}[style=Vc]
std::simd<int> v;
int v0 = permute<1>(v, [](int) { return 0; });
v0 += 1;
v = permute<v.size>(simd_cat(v, v0), [](int i) { return i == 0 ? v.size : i; });

// not impossible:
int x;
int x0 = permute<1>(x, [](int) { return 0; });
x0 += 1;
x = permute<1>(simd_cat(x, x0), [](int i) { return i == 0 ? 1 : i; });
  \end{lstlisting}

\end{enumerate}

\subsection{Discussion}
In the example above I chose the problem of updating the value of a single
element of a \simd, to showcase how much compound assignment can aid in
readability.
In my opinion, the missing compound read-modify-write syntax in examples
\ref{memberset}, \ref{nonmemberset}, and \ref{proxyset} is a huge downside.

Further observations on the above examples:
\begin{itemize}
  \item \code{set(x, y, z)} is not intuitive whereas \code{x[y] = z} cleary
    states the intended operation.

  \item \code{x.set(y, z)} is better than \code{set(x, y, z)} in terms of “what
    is set where?”, but ideally a “set” function would only take a single
    argument: the new value.

  \item This is achieved by example \ref{proxyset}, which creates an object
    that identifies a single element, thus allowing \code{set} to only take the
    new value as function argument.

  \item We can pass lvalue-references around, (e.g. \code{int\& x = data[0];}).
    Examples \ref{memberset} and \ref{nonmemberset} don't allow an equivalent
    for \simd elements.
    \ref{proxyset} and \ref{proxyoperators} however would act as a drop-in for
    lvalue references and thus would allow modifying a single \simd element
    “from a distance”.\footnote{Typically not a good idea, though.}
\end{itemize}

The ability to write simd-generic element access is not super important, but
certainly aids against code duplication in some situations.


\subsection{Recommendation}
I still believe the use of the subscript operator for \simd and \mask is fairly
intuitive and natural.
From experience I would guess that read-only subscript is 90\% if not 99\% of
the typical current use of subscripting.
I may be biased from writing many unit tests, and nobody actually uses
assignment through subscripts (or if they do, a generator constructor would
have been the better solution anyway).
Therefore, I recommend to simply remove the non-const subscript operator from
\simd and \mask.

If that's not an acceptable outcome, my next recommendation would be the
addition of an \code{element_reference} type that implements all (compound)
assignment operators (but without restricting them to rvalue, like the current
implicit proxy reference type does).
Basically make example \ref{proxyoperators} work.


\section{Proposed polls}

All of these polls are phrased against the status-quo (P1928).
Thus no concensus on all polls implies we keep the \simd and \mask subscript
operators with proxy-reference on non-const subscripts.

\wgPoll{Remove non-const \code{operator[]} from \simd and \mask. ($\Rightarrow$
Subscripting will stay read-only forever.)}{&&&&}

\wgPoll{Remove all subscript operators if we make \simd and \mask random-access
ranges (TBD). ($\Rightarrow$ status-quo until paper making \simd and \mask a
range lands.)}{&&&&}

\wgPoll{Replace subscript operators by member \code{get} and \code{set}
functions (names TBD).}{&&&&}

\wgPoll{Replace subscript operators by non-member \code{get} and \code{set}
functions (names TBD).}{&&&&}

\wgPoll{Replace subscript operators by \code{element_reference} and \code{set}
functions (names TBD).}{&&&&}


\section{Wording}\label{sec:wording}

TBD after deciding on the preferred solution.

\end{document}
% vim: sw=2 sts=2 ai et tw=0
