\newcommand\wgTitle{Replace simd operator[] with getter and setter functions --- or not}
\newcommand\wgName{Matthias Kretz <m.kretz@gsi.de>}
\newcommand\wgDocumentNumber{D3275R1}
\newcommand\wgGroup{LEWG}
\newcommand\wgTarget{\CC{}26}
\newcommand\wgAcknowledgements{Daniel Towner and Ruslan Arutyunyan contributed
to this paper via discussions / reviews. Thanks also to Jeff Garland for reviewing.}

\usepackage{mymacros}
\usepackage{wg21}
\setcounter{tocdepth}{2} % show sections and subsections in TOC
\hypersetup{bookmarksdepth=5}
\usepackage{changelog}
\usepackage{underscore}
\usepackage{multirow}

\addbibresource{extra.bib}

\newcommand\simd[1][]{\type{ba\-sic\_simd#1}\xspace}
\newcommand\simdT{\type{ba\-sic\_simd\MayBreak<\MayBreak{}T>}\xspace}
\newcommand\valuetype{\type{val\-ue\_type}\xspace}
\newcommand\referencetype{\type{ref\-er\-ence}\xspace}
\newcommand\mask[1][]{\type{ba\-sic\_simd\_mask#1}\xspace}
\newcommand\maskT{\type{ba\-sic\_simd\_mask\MayBreak<\MayBreak{}T>}\xspace}
\newcommand\wglink[1]{\href{https://wg21.link/#1}{#1}}

\newcommand\nativeabi{\UNSP{native-abi}}
\newcommand\deducet{\UNSP{deduce-t}}
\newcommand\simdsizev{\UNSP{simd-size-v}}
\newcommand\simdsizetype{\UNSP{simd-size-type}}
\newcommand\simdselect{\UNSP{simd-select-impl}}
\newcommand\maskelementsize{\UNSP{mask-element-size}}
\newcommand\integerfrom{\UNSP{integer-from}}
\newcommand\constexprwrapperlike{\UNSP{constexpr-wrapper-like}}

\renewcommand{\lst}[1]{Listing~\ref{#1}}
\renewcommand{\sect}[1]{Section~\ref{#1}}
\renewcommand{\ttref}[1]{Tony~Table~\ref{#1}}
\renewcommand{\tabref}[1]{Table~\ref{#1}}

\begin{document}
\selectlanguage{american}
\begin{wgTitlepage}
  There was discussion in LEWG in Kona '23 whether \code{operator[]} is the
  right interface for reading and writing individual elements of a \simd or
  \mask.
  This paper discusses the underlying issue and explores alternatives.
\end{wgTitlepage}

\pagestyle{scrheadings}

\section{Changelog}
(placeholder)
\begin{revision}
\item Add a simple example to the motivation section.
\item Expand the “Generalization” section to clearly define the feature rather
  than just sketching it.
  Also add a discussion of initial value and step.
\item Discuss why reusing the existing \code{iota} algorithm/view does not
  work/suffice for the \code{simd} use case.
\item Discuss why \code{iota_v} is the right name.
%  \todo
\end{revision}

\section{Straw Polls}
\subsection{SG1 at Kona 2022}
\wgPoll{After significant experience with the TS, we recommend that the next
version (the TS version with improvements) of \code{std::simd} target the IS (\CC{}26)}
{10&8&0&0&0}

\wgUnanimous{We like all of the recommended changes to \code{std::simd} proposed in p1928r1
(Includes making all of \code{std::simd} \code{constexpr}, and dropping an ABI stable type)}

\wgPoll{Future papers and future revisions of existing papers that target
\code{std::simd} should go directly to LEWG.
(We do not believe there are SG1 issues with \code{std::simd} today.)}
{9&8&0&0&0}


\section{Motivation}
\subsection{proxy references in \CC{} make me :'(}
\begin{wgText}[{[simd.general]}]
  A data-parallel type consists of one or more elements of an underlying
  vectorizable type, called the element type.
  [\ldots]
  The elements in a data-parallel type are indexed from 0 to $\mathrm{width} - 1$.
\end{wgText}

Since that's a given, we sure want to be able to access individual elements.
\simd and \mask implement \code{operator[]} for element access:

\medskip\begin{lstlisting}[style=Vc]
std::simd<int> x = 0;
x[0] = 1;      // OK
int y = x[0];  // OK
x[0] = y;      // OK
auto z = x[0]; // OK, but:   #1
z = 2;         // ill-formed #2
\end{lstlisting}

The \simd and \mask types hold values of their \valuetype, but they typically
don't hold \emph{objects} of \valuetype.
Consequently, both \code{operator[]} overloads cannot return an
lvalue-reference.
The \code{const} overload therefore returns a prvalue and the non-\code{const}
overload returns a proxy reference.
The proxy reference implements assignment and compound-assignment operators for
assigning through to the selected element of the \simd/\mask.
Thus, line \code{\#1} above deduces the type of \code{z} to be that proxy
reference type.
The declaration of \code{z} does not look like a reference at all.
Therefore, assignment in line \code{\#2} is ill-formed, in order to avoid
the surprising behavior of modifying \code{x}.

In any case, the fact that a proxy reference is used instead of an
lvalue-reference, makes subscripting into a \code{simd} error-prone.
Whenever the subscript expression is used in a function argument with deduced
type, bad things are likely to happen.\footnote{Just recall
\code{vector<bool>::reference}.}
If we had a language feature to decay the proxy reference type to \valuetype on
deduction, then a lot of problems could be avoided.
But we don't have that feature and there's no reasonable chance to get it for
\CC{}26.

\subsection{Why, though? lvalue ref is fine, no?}

With GCC today you can write
\medskip\begin{lstlisting}[style=Vc]
using simd [[gnu::vector_size(16)]] = int;

simd x = {};
int& ref = x[0];
x += 1;
ref = 2;
\end{lstlisting}
If that's possible, then \code{std::simd<int>}'s subscript operator can simply
return an lvalue reference, no?
While that's true for this example, it's not true in general.
For one, Clang is fairly strict about not handing out lvalue references.
I.e. the above code does not compile.
But more importantly, for some targets or implementations an intrinsic type
might need to be used, which doesn't allow forming lvalue references to its
elements either.
Also \simd does not prohibit an implementation to use a different element type
internally for its SIMD registers.
E.g. an efficient implementation of 8-bit and 16-bit integers on the (outdated)
Intel Knights architecture required the use of 32-bit integer SIMD registers
and instructions.
It is also conceivable that implementations will implement
\code{simd<std::float16_t>} using 32-bit \code{float} SIMD registers for
targets without hardware support.

The situation for \mask is much clearer.
There are three typical storage formats for masks in current hardware:
\begin{enumerate}
  \item Full SIMD registers where either all bits are 0 or all bits are 1 for
    the complete number of value bits.
  \item Bitmasks use one bit per mask element.
    This is analogue to \code{vector<bool>} and \code{bitset} not being able to
    return lvalue references to \code{bool}.
  \item Mask registers that use one bit per value type byte.
    This is similar to the above, where we would need to return a reference to
    a single bit (just at a different position).
\end{enumerate}
Therefore, even if Clang would implement GCC's behavior with regard to forming
lvalue references to vector elements, that doesn't help for \mask.

\subsection{simd<UDT>}\label{sec:simd<array>}
If we want to extend \simd's vectorizable types to user-defined types, we need
to consider a consistency issue:
\code{simd<T>} applies every operator and operation element-wise (unless the
name clearly hints at a horizontal operation).

While I don't think e.g. \code{simd<array<short, 4>, 2>} is a useful thing,
it's also not completely crazy.
However, its only interesting semantics in a \simd is data-parallel
subscripting (apply \code{operator[]} element-wise):
\medskip\begin{lstlisting}[style=Vc]
using xyzw = std::array<short, 4>;
std::simd<xyzw, 2> a = {};
std::simd<short, 2> w = a[3]; // yes, simd not array: element-wise subscript!
a[3] = w + std::integral_constant<short, 1>();
\end{lstlisting}

(This is neither going to be great for performance, nor is it clear whether we
should implement such a “data-parallel subscript”, which requires a proxy
reference again.)

This example is not meant to motivate an element-wise \code{operator[]} for \code{simd}.
It's meant to show that the current \code{simd::operator[]} is inconsistent
with the “apply operators element-wise” rule.
Applying \code{operator[]} element-wise on a \code{simd<int>} is obviously
ill-formed since \code{int} doesn't have a subscript operator.
Consequently, maybe the current P1928 \simd and \mask shouldn't overload
\code{operator[]}?



\section{Replacement exploration}

If we only want to get rid of the proxy reference but are not concerned about
the consistency argument in \sect{sec:simd<array>}, then we could consider a
read-only subscript operator.
We still have two choices:\\
\begingroup
  \smaller[1]
\begin{tabular}{p{.45\textwidth}|p{.45\textwidth}}
  read-only forever & keep design space open \\
  \hline
  \begin{lstlisting}
class simd {
  value_type operator[](@\simdsizetype@ i) const;

};
  \end{lstlisting}

  \medskip
  \begin{lstlisting}
std::simd<int> v;
int x = v[0];
  \end{lstlisting}
  &
  \begin{lstlisting}
class simd {
  value_type operator[](@\simdsizetype@ i) const;
  void operator[](@\simdsizetype@) = delete;
};
  \end{lstlisting}

  \medskip
  \begin{lstlisting}
std::simd<int> v;
int x = std::as_const(v)[0];
  \end{lstlisting}
\end{tabular}
\endgroup

\subsection{Making the case for: a read-only subscript is sufficient}

A common use case for the subscript operator arises through the generator
constructors of \simd and \mask.
With P1928 you would write a permutation like this:
\medskip\begin{lstlisting}[style=Vc]
simd<int> v;
simd<int> reversed([&](int i) { return v[v.size - i - 1]; };
\end{lstlisting}
The generator constructor often reads from another \simd and since it needs to
compute a scalar, it typically only reads one element at a time.
And it never updates a value via subscripting; the update happens by
constructing a whole new \simd (with a good chance of the compiler producing
vector instructions).
Making such code any harder to write is not necessarily helping users.
The above example is intuitively understandable (well, the subscripting part,
the generator constructor maybe less so).

Therefore, it seems like the simplest and still fairly usable “fix” is to
remove the non-const subscript overload.

There is some curious existing practice in GCC supporting this approach:
\medskip\begin{lstlisting}[style=Vc]
using simd [[gnu::vector_size(16)]] = int;

constexpr simd f(simd x) {
  x[0] = 1;
  return x;
}

constexpr simd test0 = f(simd{}); // ill-formed: x[0] = 1 is not a constant expression

constexpr simd g(simd x) {
  x = simd{1, x[1], x[2], x[3]};
  return x;
}

constexpr simd test1 = f(simd{}); // OK
\end{lstlisting}
I.e. assignment through vector subscripts cannot be used in constant expressions.
Instead a complete new vector must be constructed.
If the non-const subscript operator is removed from \simd and \mask, then GCC's
restriction for constant expressions becomes \std\simd's behavior.


\subsection{Is simd as a read-only range a sufficient replacement?}

If \simd and \mask have a \code{begin()} and \code{end()} iterator, making
them read-only random-access ranges, then accessing an element is
equivalent to accessing a scalar from an \code{ini\-til\-iz\-er_list}:
\medskip\begin{lstlisting}[style=Vc]
std::simd<int, 4> v;
auto v0 = v.begin()[0];
auto v3 = v.begin()[3];
\end{lstlisting}

In a very similar approach, making \simd convertible to \code{array} allows
subscripting through the array:
\medskip\begin{lstlisting}[style=Vc]
std::simd<int, 4> v;
std::array a = v;
a[1] += 1;
v = a;
\end{lstlisting}

\subsection{Allowing for writable subscript after \CC{}26}

If we want to keep the design space open while still overloading
\simd\code{::operator[]}, then subscripting would become even more awkward to
use.
Consequently, \simd and \mask should rather have no subscript operator at all
for \CC{}26.
In the following exploratory examples, I will use the function names \code{get}
and \code{set} as placeholder names.
I also added a line to every example, considering the same syntax for the
degenerate case of an \code{int} instead of a \code{simd<int>}.

\begin{enumerate}
  \item P1928 status quo:\label{explP1928}
\medskip\begin{lstlisting}[style=Vc]
std::simd<int> v;
v[0] += 1;

int x;
x[0] += 1; // nope
\end{lstlisting}

\item \code{set(index, value)} member function:\label{memberset}
\medskip\begin{lstlisting}[style=Vc]
std::simd<int> v;
v.set(0, 1 + v.get(0));

int x;
x.set(0, 1 + x.get(0)); // nope
\end{lstlisting}

\item \code{set(object, index, value)} non-member function:\label{nonmemberset}
\medskip\begin{lstlisting}[style=Vc]
std::simd<int> v;
set(v, 0, 1 + get(v, 0));

int x;
set(x, 0, 1 + get(x, 0)); // not impossible
\end{lstlisting}

\item explicit proxy reference without assignment and conversion operators:\label{proxyset}
\medskip\begin{lstlisting}[style=Vc]
std::simd<int> v;
element_reference(v, 0).set(1 + element_reference(v, 0).get());

int x;
element_reference(x, 0).set(1 + element_reference(x, 0).get()); // not impossible
\end{lstlisting}

\item explicit proxy reference with operators:\label{proxyoperators}
\medskip\begin{lstlisting}[style=Vc]
std::simd<int> v;
element_reference(v, 0) += 1;

int x;
element_reference(x, 0) += 1; // not impossible
\end{lstlisting}

\item make degenerate size 1 \simdT convertible to/from \code{T} (and \mask to/from \code{bool})\footnote{\code{permute<1>} returns a \code{simd<int, 1>}, which could be implicitly convertible to \code{int}.}
  \medskip\begin{lstlisting}[style=Vc]
std::simd<int> v;
int v0 = permute<1>(v, [](int) { return 0; });
v0 += 1;
v = permute<v.size>(simd_cat(v, v0), [](int i) { return i == 0 ? v.size : i; });

// not impossible:
int x;
int x0 = permute<1>(x, [](int) { return 0; });
x0 += 1;
x = permute<1>(simd_cat(x, x0), [](int i) { return i == 0 ? 1 : i; });
  \end{lstlisting}

\end{enumerate}

\subsection{Discussion}
In the example above I chose the problem of updating the value of a single
element of a \simd, to showcase how much compound assignment can aid in
readability.
In my opinion, the missing compound read-modify-write syntax in examples
\ref{memberset}, \ref{nonmemberset}, and \ref{proxyset} is a huge downside.

Further observations on the above examples:
\begin{itemize}
  \item Making \code{simd<T, 1>} convertible to \code{T} seems interesting, but
    not like a solution to this problem.

  \item \code{set(x, y, z)} is not intuitive whereas \code{x[y] = z} cleary
    states the intended operation.

  \item \code{x.set(y, z)} is better than \code{set(x, y, z)} in terms of “what
    is set where?”, but ideally a “set” function would only take a single
    argument: the new value.

  \item This is achieved by example \ref{proxyset}, which creates an object
    that identifies a single element, thus allowing \code{set} to only take the
    new value as function argument.

  \item We can pass lvalue-references around, (e.g. \code{int\& x = data[0];}).
    Examples \ref{memberset} and \ref{nonmemberset} don't allow an equivalent
    for \simd elements.
    \ref{proxyset} and \ref{proxyoperators} however would act as a drop-in for
    lvalue references and thus would allow modifying a single \simd element
    “from a distance”.\footnote{Typically not a good idea, though.}
\end{itemize}

The ability to write simd-generic element access is not super important, but
certainly aids against code duplication in some situations.


\subsection{Recommendation}
I still believe the use of the subscript operator for \simd and \mask is fairly
intuitive and natural.
From experience I would guess that read-only subscript is 90\% if not 99\% of
the typical current use of subscripting.
I may be biased from writing many unit tests, and nobody actually uses
assignment through subscripts (or if they do, a generator constructor would
have been the better solution anyway).
Therefore, I recommend to simply remove the non-const subscript operator from
\simd and \mask.

If that's not an acceptable outcome, my next recommendation would be the
addition of an \code{element_reference} type that implements all (compound)
assignment operators (but without restricting them to rvalue, like the current
implicit proxy reference type does).
Basically make example \ref{proxyoperators} work.


\section{Proposed polls}

All of these polls are phrased against the status-quo (P1928).
Thus no concensus on all polls implies we keep the \simd and \mask subscript
operators with proxy-reference on non-const subscripts.

\wgPoll{Remove non-const \code{operator[]} from \simd and \mask. ($\Rightarrow$
Subscripting will stay read-only forever.)}{&&&&}

\wgPoll{Remove all subscript operators if we make \simd and \mask random-access
ranges (TBD). ($\Rightarrow$ status-quo until paper making \simd and \mask a
range lands.)}{&&&&}

\wgPoll{Replace subscript operators by member \code{get} and \code{set}
functions (names TBD).}{&&&&}

\wgPoll{Replace subscript operators by non-member \code{get} and \code{set}
functions (names TBD).}{&&&&}

\wgPoll{Replace subscript operators by \code{element_reference} and \code{set}
functions (names TBD).}{&&&&}


\section{Wording}\label{sec:wording}

TBD after deciding on the preferred solution.

\end{document}
% vim: sw=2 sts=2 ai et tw=0
