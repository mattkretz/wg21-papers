\newcommand\wgTitle{Deprecate \code{numeric\_limits::has\_denorm}}
\newcommand\wgName{Matthias Kretz <m.kretz@gsi.de>}
\newcommand\wgDocumentNumber{D2614R0}
\newcommand\wgGroup{SG6, LEWG}
\newcommand\wgTarget{\CC{}26}
\newcommand\wgAcknowledgements{
Thanks to WG14 and specifically Fred Tydeman for their work on \code{*\_HAS\_SUBNORM} and 
presenting in SG6. Thanks to Dietmar Kühl, Fred Tydeman, Jens Maurer, John McFarlane, and 
Mark Hoemmen for the discussion in SG6 that motivated this paper. Thanks to Mark Hoemmen 
for pointing out that we should deprecate \code{has_denorm_loss}.
}

\usepackage{mymacros}
\usepackage{wg21}
\usepackage{changelog}
\usepackage{underscore}

\addbibresource{extra.bib}

\newcommand\wglink[1]{\href{https://wg21.link/#1}{#1}}

\begin{document}
\selectlanguage{american}
\begin{wgTitlepage}
  Since C is intent on obsoleting the \code{*_HAS_SUBNORM} macros, we should consider the 
  analogue change in \CC{}: the deprecation of \code{numeric_limits::has_denorm}. In 
  general, compile-time constants that describe floating-point \emph{behavior} are 
  problematic, since behavior might change at run-time.
  As a drive-by change, let's also deprecate \code{numeric_limits::has_denorm_loss}.
\end{wgTitlepage}

\pagestyle{scrheadings}

%\section{Changelog}
(placeholder)
\begin{revision}
\item Add a simple example to the motivation section.
\item Expand the “Generalization” section to clearly define the feature rather
  than just sketching it.
  Also add a discussion of initial value and step.
\item Discuss why reusing the existing \code{iota} algorithm/view does not
  work/suffice for the \code{simd} use case.
\item Discuss why \code{iota_v} is the right name.
%  \todo
\end{revision}


%\section{Straw Polls}
\subsection{SG1 at Kona 2022}
\wgPoll{After significant experience with the TS, we recommend that the next
version (the TS version with improvements) of \code{std::simd} target the IS (\CC{}26)}
{10&8&0&0&0}

\wgUnanimous{We like all of the recommended changes to \code{std::simd} proposed in p1928r1
(Includes making all of \code{std::simd} \code{constexpr}, and dropping an ABI stable type)}

\wgPoll{Future papers and future revisions of existing papers that target
\code{std::simd} should go directly to LEWG.
(We do not believe there are SG1 issues with \code{std::simd} today.)}
{9&8&0&0&0}


\section{Introduction}

\code{numeric_limits} has a member called \code{has_denorm} of type 
\code{float_denorm_style}:
\medskip\begin{lstlisting}[style=Vc]
enum float_denorm_style {
  denorm_indeterminate = -1,
  denorm_absent        = 0,
  denorm_present       = 1
};
\end{lstlisting}

As \textcite{WG14N2993} states:
\begin{quote}{}
There are several ways subnormals are “supported” in the field:

\begin{itemize}
  \item Partial support - hardware has encodings, but operations “fail”.
  \begin{itemize}
    \item Operands are flushed to zero; results are kept.
    \item Operands are kept; results are flushed to zero.
    \item Some operations flush, others do not flush.
    \item Results suffer double rounding.
    \item Support can be changed at runtime (not by means in Stardard C).
  \end{itemize}
  \item Not at all. There are no hardware encodings of subnormals.
  \item Full support as per IEEE-754.
\end{itemize}
\end{quote}

Since hardware can change in future iterations, an implementation that does not want to 
risk an ABI break via \code{numeric_limits} will never set \code{has_denorm} to 
\code{denorm_absent} or \code{denorm_present}. The only ABI-safe sensible value is 
\code{denorm_indeterminate}. I.e. implementations cannot give a compile-time guarantee 
about \emph{run-time behavior}.

The \code{has_denorm} value is not helping \CC{} users. Worst case, it is misleading 
users, resulting in incorrect assumptions and possibly breaking algorithms at some point.

\subsection{\code{has_denorm_loss}}

The subsequent member in \code{numeric_limits}, \code{has_denorm_loss} is also calling for 
deprecation. Who can tell me the meaning of: “true if loss of accuracy is detected as a 
denormalization loss, rather than as an inexact result.\footnote{See ISO/IEC/IEEE 60559.}” 
cppreference.com explains \cite{has_denorm_loss.cppreference}:
\begin{quote}{}
  No implementation of denormalization loss mechanism exists (accuracy loss is detected 
  after rounding, as inexact result), and this option was removed in the 2008 revision of 
  IEEE Std 754.

  libstdc++, libc++, libCstd, and stlport4 define this constant as false for all 
  floating-point types. Microsoft Visual Studio defines it as true for all floating-point 
  types.
\end{quote}

I don't own a IEEE 754 revision older than the 2008 revision, so it's hard to check. But 
at least the 2008 revision has no occurence of the word “loss” and no relevant occurence 
of “accuracy”. The footnote's reference to the IEEE 754 standard is impossible to follow.

\section{Proposed solution}

The \code{has_denorm} and \code{has_denorm_loss} values should not be used.

A shallow code 
search\footnote{\url{https://codesearch.isocpp.org/cgi-bin/cgi_ppsearch?q=has_denorm&search=Search}} 
suggests that no code actually relies on \code{has_denorm}. However, a removal of the 
value would be a major compatiblity break. We can deprecate it, but without an actual 
intent of removal (since it would break too much). As an alternative to deprecation, we 
could change paragraph 46 “Meaningful for all floating-point types.” to state that it's 
not even meaningful for floating-point types. Thus, user-defined types could still define 
a meaning for \code{has_denorm}.

The preference of SG6 after discussing \cite{WG14N2993} was deprecation of 
\code{has_denorm}.

\code{has_denorm_loss} should simply be deprecated (without actual intent of removal, 
though). The reference to IEEE 754 should be removed in any case.

\section{Wording}

TBD.

\section{Suggested Straw Polls}

\wgPoll{Should the use of \code{numeric_limits::has_denorm} be discouraged via a change in 
the standard?}{&&&&}

\wgPoll{Deprecate \code{numeric_limits::has_denorm}?}{&&&&}

\wgPoll{Document \code{numeric_limits::has_denorm} as not meaningful for arithmetic 
types?}{&&&&}

\wgPoll{Deprecate \code{numeric_limits::has_denorm_loss}?}{&&&&}

\end{document}
% vim: sw=2 sts=2 ai et tw=90 formatoptions=trowan2
