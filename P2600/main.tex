\newcommand\wgTitle{A minimal ADL restriction to avoid ill-formed template instantiation}
\newcommand\wgName{Matthias Kretz <m.kretz@gsi.de>}
\newcommand\wgDocumentNumber{D2600R0}
\newcommand\wgGroup{EWGI, EWG}
\newcommand\wgTarget{\CC{}26}
%\newcommand\wgAcknowledgements{ }

\usepackage{mymacros}
\usepackage{wg21}
\usepackage{changelog}
\usepackage{underscore}

\addbibresource{extra.bib}

\newcommand\wglink[1]{\href{https://wg21.link/#1}{#1}}
\newcommand\notyetinstantiated{not\hyp{}yet\hyp{}instantiated\hyp{}template}

\begin{document}
\selectlanguage{american}
\begin{wgTitlepage}
  I researched and tested a minimal change to ADL to avoid ill-formed instantiations of 
  templates through ADL. If ADL ignores \notyetinstantiated{} types (just like ADL ignores 
  incomplete types) little to no functionality is lost while losing a sharp edge of the 
  language. This paper presents the idea, discusses potential code breakage, and presents 
  potential alternatives/extensions of the idea.
\end{wgTitlepage}

\pagestyle{scrheadings}

%\section{Changelog}
(placeholder)
\begin{revision}
\item Add a simple example to the motivation section.
\item Expand the “Generalization” section to clearly define the feature rather
  than just sketching it.
  Also add a discussion of initial value and step.
\item Discuss why reusing the existing \code{iota} algorithm/view does not
  work/suffice for the \code{simd} use case.
\item Discuss why \code{iota_v} is the right name.
%  \todo
\end{revision}


%\section{Straw Polls}
\subsection{SG1 at Kona 2022}
\wgPoll{After significant experience with the TS, we recommend that the next
version (the TS version with improvements) of \code{std::simd} target the IS (\CC{}26)}
{10&8&0&0&0}

\wgUnanimous{We like all of the recommended changes to \code{std::simd} proposed in p1928r1
(Includes making all of \code{std::simd} \code{constexpr}, and dropping an ABI stable type)}

\wgPoll{Future papers and future revisions of existing papers that target
\code{std::simd} should go directly to LEWG.
(We do not believe there are SG1 issues with \code{std::simd} today.)}
{9&8&0&0&0}


\section{Introduction}

Consider the innocent-looking code in \lst{example1} (as posted to the core reflector by 
Jonathan Wakely\footnote{\url{http://lists.isocpp.org/core/2021/06/11161.php} (the basic 
idea of this paper was already hinted at in the resulting discussion thread)}).
\begin{lstlisting}[style=Vc,float={hb},label=lst:example1,caption={
Ill-formed instantiation of \type{Wrap<Incomplete>} because of ADL
}]
struct Incomplete;
template <typename T> struct Wrap { T t; };

template <typename Unused>
struct Testable
{
  explicit operator bool() const { return true; }
};

int main()
{
  Testable<Incomplete> l;
  if (l) // OK
    return 1;
  if (!(bool)l) // OK
    return 0;
  if (!l) // OK
    return 0;

  Testable<Wrap<Incomplete>> l2;
  if (l2) // OK
    return 1;
  if (!(bool)l2) // OK
    return 0;
  if (!l2) // ERROR
    return 0;
}
\end{lstlisting}

The expressions \code{!l} and \code{!l2} lead to name lookup of \code{operator!}. The 
associated entities of \code{l2} are \type{Testable<Wrap<Incomplete>>}, 
\type{Wrap<In\-com\-plete>}, and \type{Incomplete}.
In this example \type{Testable<Wrap<Incomplete>>} must already have been successfully 
instantiated, otherwise the declaration of \code{l2} would have been ill-formed.
The type \type{In\-com\-plete} is incomplete and the failure to look for \code{operator!} 
inside \type{Incomplete} is ignored.
The type \type{Wrap<Incomplete>} is a template specialization which has not been 
instantiated at this point in the translation. But instead of treating the type like the 
incomplete \type{Incomplete} type, the compiler attempts to instantiate the 
specialization, which leads to an ill-formed definition of its data member since it has 
incomplete type.

The expression \code{!l} is well-formed since ADL ignores incomplete types. Consequently, 
a later definition of \type{Incomplete} as
\medskip\begin{lstlisting}[style=Vc]
struct Incomplete {
  friend bool operator!(Testable<Incomplete>) { return true; }
}
\end{lstlisting}
changes the value of the expression \code{!l}.

Why does the standard allow incomplete types (i.e. not require completion) as associated 
entities even though this can lead to ODR violations? Isn't the same reasoning applicable 
to templates that have not be instantiated yet? And what are the use cases for defining a 
hidden friend in \type{Wrap<Incomplete>} which wins in overload resolution when the 
argument is a \type{Testable<Wrap<Incomplete>>}? (The last question is not only a rhetoric 
question, see \sect{reference_wrapper} for a possible answer.)

\subsection{Is this a real problem?}

To determine whether \lst{example1} poses a real problem, consider that 
\codelst@Testable<T>@ is equivalent to \codelst@std::unique_ptr<T>@ wrt. 
\codelst@operator!@. I.e. a few corner cases, but still legitimate uses of 
\codelst@unique_ptr@ lead to puzzling errors\footnote{OTOH, \code{unique_ptr} is not 
\emph{boolean-testable}, so maybe users should simply learn to cast \code{unique_ptr}, 
\code{optional}, etc to \code{bool}?}.

Furthermore, for \codelst@std::unique_ptr<std::array<Incomplete, N>>@, instantiation on 
ADL can occur in many more places; for example on range-based for loops as shown in 
\lst{iterating_uniqptr}.
\begin{lstlisting}[style=Vc,numbers=left,float={hbtp},label=lst:iterating_uniqptr,caption={
Iteration over \code{std::span<std::unique_ptr<std::array<Incomplete, N>>>} is ill-formed
}]
class Incomplete;
using Data = std::array<Incomplete, 3>;
using Ptr = std::unique_ptr<Data>;

void assert_nonnull (const std::span<Ptr> &x)
{
  for (const Ptr &ptr : x)  // ERROR: 'std::array<_Tp, _Nm>::_M_elems' has
    assert (ptr);           //        incomplete type
}
\end{lstlisting}
The reason for the error in \lst{iterating_uniqptr} isn't the lookup of \code{begin} and 
\code{end} (it would be if \code{x} had no \code{begin} or \code{end} members) but rather 
that operators applied to the iterator lead to ADL which instantiates 
\codelst@std::array<Incomplete, 3>@. Thus, to write an iterator that doesn't break on 
legitimate use of incomplete types a library developer has to use the ADL-proofing pattern 
from \lst{adlproofediterator} as explained by \textcite{P2538R0}.
\begin{lstlisting}[style=Vc,numbers=left,float={hbtp},label=lst:adlproofediterator,caption={
ADL-proofed iterator type
}]
template <typename T>
struct IteratorImpl {
  class type {
    // ...
  };
};

template <typename T>
using Iterator = typename IteratorImpl<T>::type;
\end{lstlisting}
We might expect standard library developers to learn and apply this pattern. But we should 
not expect from any other \CC{} library developer to work around ADL surprises.

\subsection{Instantiation via ADL is blocking language evolution}\label{blocking}
With the current behavior of ADL, adding non-member \codelst@operator[]@ or overloadable 
\codelst@operator?:@ \citep{P0917R3} to the language\footnote{Motivation is / will be 
given in their own papers.} is a breaking change. Existing code that currently compiles 
just fine would suddenly instantiate templates through ADL and thus potentially lead to 
ill-formed instantiation.

\subsection{Another example}

To complete the picture, note that the issue is not specific to operators (e.g. 
\lst{example2} as presented by \textcite{P2538R0}). However, for operators there's no 
simple “always fully qualify your calls” rule to avoid ADL.
\begin{lstlisting}[style=Vc,numbers=left,float={hbtp},label=lst:example2,caption={
Minimal example triggering ill-formed instantiation
}]
template<class T> struct Holder { T t; };
struct Incomplete;
Holder<Incomplete> *p;
int f(Holder<Incomplete>*);
int x = f(p);    // error: Holder<Incomplete> is an associated entity
int y = ::f(p);  // ok: no ADL
\end{lstlisting}

\textcite{P2538R0} goes on to show how the original definition of \codelst@std::projected@ 
leads to ill-formed instantiation on ADL. His proposed solution requires ADL-proofing of 
\codelst@std::projected@, which requires a change in how the class template is defined.

The proposed language change in this paper will make this and similar changes unnecessary. 
The language would lose one sharp edge at a minor cost (more complex wording \& new, 
unlikely, and easy to work around corner case for users).

\subsection{History}

To understand how much code might be affected by a breaking change in this area, it is 
helpful to know that template instantiation via ADL had not been implemented in compilers 
for a long time. Template instantiation via ADL works since GCC 4.5.0 (April 2010), Clang 
3.1.0 (May 2012), and since an ICC release between ICC 14 and 16\footnote{I have only been 
  able to test with the compilers available on Compiler Explorer}.

%\pagebreak

\section{Proposed solution}

Let us consider a simple solution to make ADL avoid the above situations:\\
\emph{Don't instantiate templates via ADL} (with one or two exceptions).

\textsc{Rationale:} If a given associated entity has not been instantiated at this point 
in the translation,
\begin{enumerate}
\item the user might have avoided instantiation intentionally since instantiation would be 
ill-formed; and
\item the type was not important enough up to this point that instantiation was necessary 
--- the chances for it to make a semantic difference are small ---.
\end{enumerate}

If I interpret the current wording correctly, the reason compilers instantiate templates 
via ADL is [temp.inst] p2:
\begin{wgText}[{[temp.inst]}]

\setcounter{Paras}{1}
\pnum Unless a class template specialization is a declared specialization, the class 
template specialization is implicitly instantiated when the specialization is referenced 
in a context that requires a completely-defined object type or when the completeness of 
the class type affects the semantics of the program. [\ldots]

\end{wgText}
In basically all cases, the completeness of the class type does not affect the semantics 
of the program, though. But the compiler cannot know that instantiation is unnecessary 
until after it instantiated the template. The subsequent note in [temp.inst] a
clarifies that the knowledge about whether a name exists or not is considered affecting 
semantics. Therefore implementations have no choice of avoiding instantiation, even if 
they kept track of ADL-relevant names (i.e. hidden friends).


\subsection{Necessary instantiation}

The one case where instantiation via ADL is still required is shown in 
\lst{needsInstantiation}. The situation in \lst{needsInstantiation} is unlikely (but 
certainly not impossible\footnote{E.g. \url{https://gcc.gnu.org/PR34870}, which lead to 
GCC instantiating templates via ADL}), since almost every other use of the lvalue \code{x} 
would lead to instantiation.
\begin{lstlisting}[style=Vc,numbers=left,float={hbtp},label=lst:needsInstantiation,caption={
Requires instantiation or reasonable code could break.
}]
template <typename T>
struct A {
  friend void f(const A&);
};

void g(const A<int>& x) {
  f(x);
}
\end{lstlisting}

Therefore, given an argument of reference to \code{T}, ADL should instantiate \code{T} and 
its bases but none of its template arguments.

\subsection{Namespaces of base classes}

Consider \lst{baseNamespace} which defines a base class in a different namespace 
(\code{A}) than the derived class template.
\begin{lstlisting}[style=Vc,numbers=left,float={hbtp},label=lst:baseNamespace,caption={
ADL in namespace of base class
}]
namespace A {
  class B {};
  void f(B*);
}
void f(void*);

template <class T> class C : public A::B {};

void g(C<int>* p) {
  f(p);    // calls  ::f  because C<int> has not been instantiated
  A::B* other_ptr = p; // instantiates C<int> to find its base types
  f(p);    // calls A::f  because C<int> has been instantiated before ADL
}

// analogue situation with incomplete types:
class I;
void g0(I* p) { f(p); } // calls ::f
class I : public A::B {};
void g1(I* p) { f(p); } // calls A::f
\end{lstlisting}
In order for ADL to consider \code{A} as associated namespace when an argument of type 
\code{C<int>*} is used, the base types of the class template need to be know. However, if 
ADL were not to instantiate \code{C<int>} anymore, the behavior would depend on the 
preceding code: whether \code{C<int>} had already been instantiated or not. The analogue 
issue for an incomplete type \code{I} exhibits the same behavior change in ADL after 
defining \code{I}. However, for the incomplete type it is more obvious why the base types' 
namespaces are not considered (“You didn't say that \code{A::B} is a base!”).

Note that overload resolution still has to instantiate class templates to find bases, as 
shown in \lst{instantiateOverloadResolution}.
\begin{lstlisting}[style=Vc,numbers=left,float={hbtp},label=lst:instantiateOverloadResolution,caption={
Class template instantiation on overload resolution if ADL does not instantiate 
\code{C<int>}
}]
class B0 {};
namespace A {
  class B : public B0 {};
  void f(B*);
}
void f(void*);
void f(B0*);

template <class T> class C : public A::B {};

void g(C<int>* p) {
  f(p); // calls ::f(B0*)
        // candidates: ::f(void*) and ::f(B0*), overload resolution instantiates
        // C<int> and picks ::f(B0*), i.e. overload resolution works unchanged
  f(p); // calls A::f(A::B*)
        // because ADL now finds the additional candidate A::f(A::B*)
}
\end{lstlisting}
It is an open question whether template instantiation via ADL on arguments of pointer type 
is enough of a problem to warrant the surprising behavior of 
\lst{instantiateOverloadResolution} (vs. the surprising behavior of \lst{example2}). As a 
more conservative change to ADL, given an argument of pointer to \code{T}, let ADL 
instantiate \code{T} and its bases but none of its template arguments.

Beyond references and pointers --- as far as I have seen --- there is little use in ADL 
instantiating templates. I propose to modify [basic.lookup.argdep] and/or [temp.inst] to 
treat \notyetinstantiated{} types like incomplete types except if the 
\notyetinstantiated{} type is the type of the function argument\footnote{\emph{stops 
handwaving}; this is nowhere near wording \ldots}.

\subsection{\code{std::reference_wrapper} example}\label{sec:reference_wrapper}
\codelst@std::reference_wrapper<X>@, where \codelst@X@ implements hidden friends, is the 
canonical example where this proposal has the potential for breaking existing code.
An idea for avoiding the breaking change is presented in \sect{convopadl}. Note that, 
since \CC{}20, \codelst@std::reference_wrapper<T>@ does not require \codelst@T@ to be 
complete and we can therefore construct valid examples with the type 
\codelst@std::reference_wrapper<Wrap<Incomplete>>@. The
\begin{lstlisting}[numbers=left,float={hbtp},label=lst:transparent_reference_wrapper,caption={
Hidden friends are transparent for \code{std::reference_wrapper}
}]
struct X {
  friend constexpr int f(const X&) { return 1; }
  friend constexpr bool operator!(const X&) { return false; }
};

int g(std::reference_wrapper<X> ref) {
  return !ref ? 0 : f(ref);  // returns 1
}
\end{lstlisting}
relevant class of examples follows the pattern shown in 
\lst{transparent_reference_wrapper}. This pattern relies on:
\begin{enumerate}
\item \codelst@std::reference_wrapper<X>@ is convertible to \codelst@X@, and
\item ADL looks into \codelst@X@ for hidden friends when 
\codelst@std::reference_wrapper<X>@ is a function argument (operand).
\end{enumerate}
Note that \codelst@X@ doesn't need to know about (or even spell out) 
\codelst@std::reference_wrapper@ and can therefore work with any type that has a 
conversion operator to \codelst@X@ \emph{and where \codelst@X@ is an associated entity}. 
Notably, and inconsistently, function arguments of type \codelst@XRef@,
\begin{lstlisting}[numbers=left,float={hbtp},label=lst:intransparent_reference_wrapper,caption={
Hidden friends are not transparent for a non-template reference wrapper
}]
struct XRef {
  X* ptr;
  constexpr operator X&() const noexcept { return *ptr; }
};
\end{lstlisting}
as defined in \lst{intransparent_reference_wrapper}, will not find hidden friends of 
\codelst@X@ on name lookup. See \sect{convopadl} for an idea that would resolve the 
inconsistency.

With the following ingredients we can build an example (see \lst{refwrapperproblem}) that 
would break with the proposed language change:
\begin{itemize}
\item A class template specialization \codelst@X<Y>@ that has not been instantiated yet.
\item A hidden friend in \codelst@X<Y>@.
\item A function call (or operator) which wants to find said hidden friend in name lookup.
\item Wrap everything as \codelst@std::reference_wrapper<X<Y>>@.
\end{itemize}
\begin{lstlisting}[numbers=left,float={hbtp},label=lst:refwrapperproblem,caption={
\code{std::reference_wrapper} example which would fail unless 
line~\ref{lstline:refwrapperproblem0} is uncommented
}]
template <typename T>
struct X
{
  T data;
  friend auto operator<=>(const X&, const X&) = default;
};

// static_assert(std::totally_ordered<X<int>>); @\label{lstline:refwrapperproblem0}@
// the following fails unless X<int> was instantiated already (e.g. with the
//   line above)
static_assert(std::totally_ordered<std::reference_wrapper<X<int>>>);
\end{lstlisting}

As far as I can tell there is no class template in the standard library that fits this 
description. However, the \codelst@std::experimental::simd<T, Abi>@ class template of the 
Parallelism TS 2 specifies its operators as hidden friends. Consequently, the example in 
\lst{refwrappersimd} would not instantiate \codelst@simd<float>@ anymore via ADL and thus 
fail to compile.
\begin{lstlisting}[numbers=left,float={hbtp},label=lst:refwrappersimd,caption={
Regression when combining \code{std::reference_wrapper} with 
\code{std::experimental::simd}.
}]
#include <experimental/simd>
#include <functional>

auto f(std::reference_wrapper<std::experimental::simd<float>> x) {
  return x == x; // would not instantiate simd<float> anymore
}
\end{lstlisting}

An example similar to \lst{refwrappersimd} is still highly unlikely to occur outside of 
code snippets meant to demonstrate the issue. In general, an expression directly involving 
\codelst@simd<float>@ (i.e. the wrapped type) would precede the use of the type with a 
reference wrapper with high probability.
More importantly, the explicit use of \codelst@std::reference_wrapper@ or similar 
reference types is not common. At least \codelst@reference_wrapper@ is unwrapped 
transparently for \codelst@std::bind@ and typically unwraps automatically when a function 
with a reference to the wrapped type is called.
% One example, the Barton–Nackman trick, is discussed by \textcite{ODwyerBartonNackman}.

While the chance of breakage after restricting instantiation via ADL may be close to zero, 
we have no way of making a more substantial statement. It is impossible to prove that no 
breakage will occur, because the absence of certain code patterns cannot be proven.

\section{Alternative solutions / additional ADL ideas}

I believe the above suggestion would be a strict improvement of the \CC{} language and 
better than the alternatives listed in the following. However, in order to reduce the 
potential for breaking existing code and to resolve an inconsistency of ADL with regard to 
reference wrappers, I believe the idea in \sect{convopadl} should be part of the final 
solution.

\subsection{Add conversion operator return types to associated entities}
\label{sec:convopadl}
Observe the inconsistency in \lst{motivateADLconvop}.
\begin{lstlisting}[style=Vc,numbers=left,float,label=lst:motivateADLconvop,caption={
Different name lookup for three reference types that should be equivalent 
(\url{https://godbolt.org/z/83fqT76vn})
}]
struct foo {
  friend bool operator!(const foo&) noexcept;
};

template <class T>
struct ref {
  T* data;
  constexpr operator T&() const noexcept { return *data; }
};

struct int_ref {
  int* data;
  constexpr operator int&() const noexcept { return *data; }
};

struct foo_ref {
  foo* data;
  constexpr operator foo&() const noexcept { return *data; }
};

void f(const ref<foo>& r0, const int_ref& r1, const foo_ref& r2) {
  !r0; // finds foo::operator! because foo is an associated entity
  !r1; // OK
  !r2; // ERROR
}
\end{lstlisting}
Each of the three types \code{ref<foo>}, \code{int_ref}, and \code{foo_ref} (partially) 
model a reference (conversion operator defined as for \code{std::ref\-er\-ence_wrapper}). 
But they behave considerably different because name lookup is different.

To make \lst{motivateADLconvop} consistent, we could add another rule to ADL and add the 
return types of all conversion operators as associated entities. Since these types are 
much more likely to make a semantic difference on name lookup, template instantiation 
should be performed. However, only the conversion operators of the function argument type 
itself must be considered; conversion operators of return types of conversion operators 
are irrelevant, as are conversion operators of template arguments. As a consequence the 
examples \lst{refwrapperproblem} and \lst{refwrappersimd} would work again.

Obviously, more ADL has the potential to add more problems. But if this is introduced 
together with not instantiating templates anymore, the net effect should be positive. 
Counter-examples are welcome to improve the discussion of the idea.

I have implemented this idea in GCC and found no regressions in the GCC and libstdc++ test 
suites. More testers are welcome. A modified GCC 12.1 can be obtained at 
\url{https://github.com/mattkretz/gcc/tree/7895934f858fbb2e6039}.

\subsection{Opt-out/in ADL}

If the committee cannot agree on changing the way ADL and template instantiation work, we 
are basically only left with new opt-in/out syntax, if we want to solve this problem. I am 
not ready to explore this idea. I mention it here to give a complete picture of the 
solution space.

As a straw-man example consider \lst{optinoutadl}.
\begin{lstlisting}[numbers=left,float={hbtp},label=lst:optinoutadl,caption={
Straw-man opt-in/out syntax for ADL
}]
// opts out of considering implicit associated entities; bar and its associated
// entities are the only associated entities:
struct foo : adl bar { /* ... */ };

// no associated entities:
struct meow : adl void { /* ... */ };

// fix the unique_ptr problem from @\lst{example1}@
namespace std {
  template<class T, class D = default_delete<T>>
  class unique_ptr : adl void {
    // ...
  };
}
\end{lstlisting}
Note that an opt-in/out solution would not open the evolution path for 
\codelst@operator[]@ and \codelst@operator?:@.

\subsection{Tentative instantiation}
It seems like an obvious solution to require IFINAE (instantiation failure is not an 
error) on ADL, i.e. to require tentative template instantiation. However, this would place 
a huge burden on implementations: The instantiation depth might be very deep, before the 
condition is found. IFINAE requires a rollback of all unfinished instantiations that lead 
to the issue. It seems like a huge but solvable task for compilers, but without 
implementation experience it is not a realistic path forward.

\subsection{Inhibit instantiation if a template parameter is incomplete}

A minimal solution for solving the problem in \lst{example1} inhibits instantiation on ADL 
if a template parameter is incomplete. This would cover some of the cases where ADL 
appears too eager. However, only minor variations of such examples would instantiate the 
class templates again, leading to the same errors we were trying to fix. The solution 
would thus appear to be rather fragile and potentially more confusing than helpful.

\subsection{Instantiation if and only if there are hidden friends with a matching name}

The current wording already makes instantiation conditional on whether “the completeness 
of the class type affects the semantics of the program”. Can we take this a step further, 
and require instantiation only if the compiler determines name lookup will find something 
it wouldn't find otherwise? Such a condition would come close to tentative instantiation 
but isn't necessarily the same thing. Name lookup requires less information of the 
complete type. Only when determining viability and performing overload resolution is the 
completely instantiated type unavoidable.

An implementation would have to
\begin{enumerate}
\item keep a list of names of all hidden friends, and
\item be able to determine \emph{additional} associated namespaces from base classes.
\end{enumerate}
Instantiation is only necessary if a hidden friend was found via name lookup. For the 
other case instantiation will likely be required for overload resolution, though.

\section{Implementation Experience}
I implemented the presented idea for GCC. The necessary change to the ADL code was 
straightforward: In principle the change involved only making template instantiation 
conditional on whether the type is the argument type itself or an associated entity.
A modified GCC 12.1 can be obtained at 
\url{https://github.com/mattkretz/gcc/tree/7266f1c5f75fc7a970de}.

One test of GCC's testsuite broke (\lst{noexcept41}) with the change.
\begin{lstlisting}[style=Vc,numbers=left,float={hbtp},label=lst:noexcept41,caption={
GCC test that broke after implementation of less eager ADL (\code{declval<a<int>>} without 
parenthesis is no error --- with parenthesis \code{a<int>} isn't an associated entity)}]
template <typename> struct g { static const int h = 0; };
template <typename i> void declval() { static_assert(!g<i>::h,""); }
template <typename> struct a {
  template <typename d, typename c>
  friend auto f(d &&, c &&)
    noexcept(declval<c>) -> decltype(declval<c>); // { dg-error "different exception" }
};
template <typename d, typename c> auto f(d &&, c &&) -> decltype(declval<c>);
struct e {};
static_assert((e{}, declval<a<int>>),""); // { dg-error "no context to resolve type" }
\end{lstlisting}
The test case started failing because the expression \code{(e{}, declval<a<int>>)} 
requires name lookup of \code{operator,} and the type \code{a<int>} is an associated 
entity and therefore was instantiated (i.e. GCC was looking for a hidden friend comma 
operator in \code{a<int>}) which lead to the actual failure this test was looking for. To 
me \lst{noexcept41} is no motivation for holding back on this proposal, rather the 
opposite. For what it's worth, Clang doesn't even instantiate \code{a<int>} for this test 
case.

I found no further code breakage.

\section{Wording}
TBD.

\section{Suggested Straw Polls}
None at this point.
%\wgPoll{}{&&&&}

\section{Acknowledgements}
This paper and my implementation for GCC are a reaction to a discussion on the CWG list 
with input from a Jonathan Wakely, Richard Smith, Peter Dimov, Barry Revzin, Arthur 
O'Dwyer, Ville Voutilainen, and Daveed Vandevoorde.
Arthur O'Dwyer suggested to consider \code{std::reference_wrapper} examples and explore 
inconsistencies wrt. \code{boolean_testable}.

\end{document}
% vim: sw=2 sts=2 ai et tw=90 formatoptions=trowan2
