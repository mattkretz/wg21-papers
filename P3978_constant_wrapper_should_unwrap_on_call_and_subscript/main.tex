\newcommand\wgTitle{\code{constant\_wrapper} should unwrap on call and subscript}
\newcommand\wgName{Matthias Kretz <m.kretz@gsi.de>}
\newcommand\wgDocumentNumber{D3978R0}
\newcommand\wgGroup{LEWG}
\newcommand\wgTarget{DR against \CC{}26, \CC{}29}
%\newcommand\wgAcknowledgements{Daniel Towner and Ruslan Arutyunyan contributed to this paper via discussions / reviews. Thanks also to Jeff Garland for reviewing.}

\usepackage{mymacros}
\usepackage{wg21}
\setcounter{tocdepth}{2} % show sections and subsections in TOC
\hypersetup{bookmarksdepth=5}
\usepackage{changelog}
\usepackage{underscore}
\usepackage{multirow}

\addbibresource{extra.bib}

\newcommand\wglink[1]{\href{https://wg21.link/#1}{#1}}

\renewcommand{\lst}[1]{Listing~\ref{#1}}
\renewcommand{\sect}[1]{Section~\ref{#1}}
\renewcommand{\ttref}[1]{Tony~Table~\ref{#1}}
\renewcommand{\tabref}[1]{Table~\ref{#1}}

\begin{document}
\selectlanguage{american}
\begin{wgTitlepage}
  This paper proposes unwrapping overloads of \code{operator()} and \code{operator[]} to
  \code{constant_wrapper}.
\end{wgTitlepage}

\pagestyle{scrheadings}

\section{Changelog}
(placeholder)
\begin{revision}
\item Add a simple example to the motivation section.
\item Expand the “Generalization” section to clearly define the feature rather
  than just sketching it.
  Also add a discussion of initial value and step.
\item Discuss why reusing the existing \code{iota} algorithm/view does not
  work/suffice for the \code{simd} use case.
\item Discuss why \code{iota_v} is the right name.
%  \todo
\end{revision}

\section{Straw Polls}
\subsection{SG1 at Kona 2022}
\wgPoll{After significant experience with the TS, we recommend that the next
version (the TS version with improvements) of \code{std::simd} target the IS (\CC{}26)}
{10&8&0&0&0}

\wgUnanimous{We like all of the recommended changes to \code{std::simd} proposed in p1928r1
(Includes making all of \code{std::simd} \code{constexpr}, and dropping an ABI stable type)}

\wgPoll{Future papers and future revisions of existing papers that target
\code{std::simd} should go directly to LEWG.
(We do not believe there are SG1 issues with \code{std::simd} today.)}
{9&8&0&0&0}


\section{Motivation}

As discussed in~\cite{P3948R0}, because of language inconsistencies, \std\code{constant_wrapper}
is inconsistently not unwrapping for call and subscript operators whereas all other operators can
be found via ADL and the conversion operator implemented in \code{constant_wrapper}.
Looking at \std\code{reference_wrapper} we see that this same issue has been resolved via an
\code{operator()} overload that unwraps and applies \code{INVOKE}.

I always had these unwrapping overloads in my implementation of \code{constant_wrapper} shipping in
the vir-simd library\footnote{https://github.com/mattkretz/vir-simd}
(\code{vir::constexpr_wrapper}).
While replacing my implementation with \std\code{constant_wrapper} I noticed the mismatch.

\subsection{Examples}

\medskip
\begin{lstlisting}
auto test1()
{
  constexpr int iota[4] = {0, 1, 2, 3};
  auto x = std::cw<iota>;
  return x[1]; // #1 OK
}

auto test2()
{
  auto x = std::cw<std::array<int, 4> {0, 1, 2, 3}>;
  return x[1]; // #2 ill-formed
}
\end{lstlisting}

The subscript in \code{\#1} is fine because \code{x} is convertible to \code{int[4]} and because
operator lookup is different for built-in types, the built-in subscript operator is found and
\code{int(1)} is returned.
The subscript in \code{\#2} does not work because, even though \code{array<int, 4>} is an associated
namespace and \code{x} is convertible to \code{array<int, 4>}, the \code{array::operator[]} member
function is not found.
This is because only hidden friend operators are considered via ADL.
Note how ADL makes the following work for \code{operator+}:
\medskip
\begin{lstlisting}
struct X { friend int operator+(X, int a) { return a + 1; } };

auto test3()
{
  auto x = std::cw<X{}>;
  return x + 1; // #3 OK
}
\end{lstlisting}
The expression \code{x + 1} in line \code{\#3} finds \code{X::operator+} via ADL and thus converts
the left operand to \code{operator+} to \code{X}, returning \code{int(2)}.

The situation for \code{operator()} is equivalent:
\medskip
\begin{lstlisting}
int fun(int x) { return x + 1; }

auto test1()
{
  auto x = std::cw<fun>;
  return x(1); // #4 OK
}

auto test2()
{
  auto x = std::cw<[](int x) { return x + 1; }>;
  return x(1); // #5 ill-formed
}
\end{lstlisting}
The call expression in \code{\#4} relies on special core wording that looks through the conversion
operator to find the function pointer and then call the function.
The expression in \code{\#5}, however, cannot find the \code{operator()} member of the lambda,
because the operator is a member, not a hidden friend.

\section{Discussion}
The most glaring question on this issue is why would we do this for \code{operator()} and
\code{operator[]} but for none of the other operators.
This seems inconsistent.
However, we need to realize that the inconsistency is in the language, forcing the inconsistent
definition of operators in the library.
This makes the \emph{behavior} of the API, where operators unwrap transparently, more consistent.
After all, the conversion operator and the additional type template argument in
\code{constant_wrapper} exist exactly because \code{constant_wrapper} is supposed to transparently
unwrap.

Wouldn't it then be better to fix the language?\\
For what it's worth, I think it would be a hugely helpful change to make the behavior of all
operators as consistent as possible.
Currently, every operator has its own set of restrictions and extras.
If I could, I'd make \emph{every} operator overloadable as non-member (and thus hidden friend) and
implement all operators of standard library types as hidden friends.
But the amount of code we would break \ldots{}
The only possibility that is not a breaking change is to add syntax that opts into new behavior,
which has a high acceptance barrier.

In terms of consistency we also have \std\code{reference_wrapper} to consider.
The similar naming is not accidental.
Consequently, the unwrapping behavior should also be consistent.

\section{Wording}
(rough draft wording)

Bump \code{__cpp_lib_constant_wrapper}.

\begin{wgText}[{[const.wrap.class]}]

\begin{codeblock}
    // call and index
    template<@\exposconcept{constexpr-param}@ T, @\exposconcept{constexpr-param}@... Args>
      constexpr auto operator()(this T, Args...) noexcept
        requires requires { constant_wrapper<T::value(Args::value...)>(); }
          { return constant_wrapper<T::value(Args::value...)>{}; }
    @\wgAdd{template<\exposconcept{constexpr-param} T, class... Args>}@
      @\wgAdd{constexpr \seebelow{} operator()(this T, Args\&\&...) noexcept(\seebelow);}@
    template<@\exposconcept{constexpr-param}@ T, @\exposconcept{constexpr-param}@... Args>
      constexpr auto operator[](this T, Args...) noexcept
        -> constant_wrapper<(T::value[Args::value...])>
          { return {}; }
    @\wgAdd{template<\exposconcept{constexpr-param} T, class... Args>}@
      @\wgAdd{constexpr \seebelow{} operator[](this T, Args\&\&...) noexcept(\seebelow);}@
\end{codeblock}

\begin{itemdecl}
constexpr @\exposid{cw-fixed-value}@(T (&arr)[Extent]) noexcept;
\end{itemdecl}

\setcounter{Paras}{3}

\begin{itemdescr}
\pnum
\effects
Initialize elements of \exposid{data} with corresponding elements of \tcode{arr}.
\end{itemdescr}

\begin{itemdecl}
@\wgAdd{template<\exposconcept{constexpr-param} T, class... Args>}@
  @\wgAdd{constexpr invoke_result_t<decltype(T::value), Args...> operator()(this T, Args\&\&...)}@
    @\wgAdd{noexcept(is_nothrow_invocable_v<decltype(T::value), Args...>);}@
\end{itemdecl}

\begin{itemdescr}
\pnumAdd
\wgAdd{%
\constraints
The type of \tcode{T::value} is not a fundamental type and no type in
\tcode{remove_cvref_t<Args>...} models \exposconcept{constexpr-param}.
}

\pnumAdd
\wgAdd{%
\returns
\tcode{\mbox{\placeholdernc{INVOKE}(T::value, std::forward<Args>(args)...)}}\iref{func.require}.
}
\end{itemdescr}

\begin{itemdecl}
@\wgAdd{template<\exposconcept{constexpr-param} T, class... Args>}@
  @\wgAdd{constexpr \seebelow{} operator[](this T, Args\&\&...) noexcept(\seebelow);}@
\end{itemdecl}

\begin{itemdescr}
\pnumAdd
\wgAdd{%
\constraints
The type of \tcode{T::value} is not a fundamental type and no type in
\tcode{remove_cvref_t<Args>...} models \exposconcept{constexpr-param}.
}

\pnumAdd
\wgAdd{%
\returns
\tcode{T::value[std::forward<Args>(args)...]}.
}
\end{itemdescr}

\end{wgText}

%\pagebreak
\end{document}
% vim: sw=2 sts=2 ai et tw=100
